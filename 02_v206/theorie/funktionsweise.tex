%\subsection{Die Funktionsweise der Wärmepumpe \cite[vgl.][]{man:v206}} % Zitieren soweit notwendig
%\label{sec:funktionsweise}
%Die Wärmepumpe pumpt ein Gas in einem abgeschlossenen Kreislauf durch die beiden Reservoirs.
%In den Reservoirs besteht der Kreislauf aus Kupferrohren, die zu einer Spirale aufgewickelt sind.
%So kann wärme zwischen den Reservoirs und dem Kreislauf fließen.
%Dabei wird durch einen Kompressor und ein Drosselventil ein höherer Druck in der Kupferspirale w erzeugt.
%Das Gas kondensiert,  verliert an Volumen und gibt Wärme an die Umgebung ab.
%Die Temperatur in Reservoir w steigt also. 
%Das Flüssige Gas wird durch das Drosselventil in den zweiten Teil des Kreislaufes gelassen.
%Durch eine technische Vorrichtung wird sichergestellt, dass nur das flüssige Gas den Kreislauf passiert.
%Im zweiten Teil des Kreislaufs wird durch die Pumpe ein Unterdruck im Vergleich zum ersten Teil erzeugt.
%Dabei verdampft das Gas wieder, es nimmt an Volumen zu und nimmt Wärme aus der Umgebung auf.
%Das Reservoir k wird also kälter.
%Eine weitere technische Vorrichtung gewährleistet, dass das Transportmittel nur im gasförmigen Zustand 
%den zweiten Teil des Kreislaufs zur Pumpe hin verlässt.

%Steht jetzt bei Durchführung hihi
