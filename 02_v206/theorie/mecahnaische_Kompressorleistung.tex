 \subsection{Mechanische Kompressorleistung}

Der Kompressor in der Wärmepumpe komprimiert das Volumen des Transportgases nahezu adiabatisch \cite[vgl.][]{man:v206}.
Wenn er ein Gasvolumen $V_a$ auf den Wert $V_b$ verringert gilt allgemein für die Arbeit \cite[vgl.][]{man:v206}:
\begin{align}
    A_m = \int_{V_a}^{V_b} \diff V
\end{align}
Für adiabatische Verformungen gilt zusätzlich die Poisson`sche Gleichung
\begin{align}
    p_a V_{a}^{\kappa} = p_{b} V_{b}^{\kappa} = p V^{\kappa} = \mathrm{const.}
\end{align}
$\kappa$ steht hier für das Verhältnis aus den Wärmekapazitäten $\frac{C_V}{C_p}$.
$C_V$ Steht hierbei für die Wärmekapazität bei konstantem Volumen und $C_p$ für die Wärmekapazität bei konstantem Druck.
% Eine Herleitung findet sich in \cite[][320]{demtroeder}
