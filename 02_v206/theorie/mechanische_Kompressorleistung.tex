 \subsection{Mechanische Kompressorleistung}

Der Kompressor in der Wärmepumpe komprimiert das Volumen des Transportgases nahezu adiabatisch \cite[vgl.][]{man:v206}.
Wenn er ein Gasvolumen $V_a$ auf den Wert $V_b$ verringert gilt allgemein für die Arbeit \cite[vgl.][]{man:v206}:
\begin{align}
    A_m = - \int_{V_a}^{V_b} p \diff V
\end{align}
Für adiabatische Verformungen gilt zusätzlich die Poisson`sche Gleichung
\begin{align}
    p_a V_{a}^{\kappa} = p_{b} V_{b}^{\kappa} = p V^{\kappa} = \mathrm{const.}
\end{align}
$\kappa$ steht hier für das Verhältnis aus den Wärmekapazitäten $\frac{C_V}{C_p}$.
$C_V$ Steht hierbei für die Wärmekapazität bei konstantem Volumen und $C_p$ für die Wärmekapazität bei konstantem Druck.
% Eine Herleitung findet sich in \cite[][320]{demtroeder}
In \cite{man:v206} wird hieraus die eine Formel für die Mechanische Arbeit
\begin{align}
    A_{\text{m}} = \frac{1}{\kappa - 1} \left( p_b \sqrt[\kappa]{\frac{p_a}{p_b}} -p_a \right) V_a
\end{align}
und für die mechanische Kompressorleistung hergeleitet:
\begin{align}
    N_{\text{mech}} = \frac{\Delta A_m}{\Delta t} = \frac{1}{\kappa - 1}%
     \left(p_b \sqrt[\kappa]{\frac{p_a}{p_b}} - p_a \right) \frac{1}{\rho_{\text{k}}}\frac{\Delta m}{\Delta t}
     \label{eq:mech_leistung}
\end{align}
Hierbei ist $\rho_{\text{k}}$ die Dichte des Gases in $V_a$ und kann mittels der idealen Gasgleichung $p V = k_\text{B} N T$
mit der Boltzmannkonstante $ k_\text{B}$ und der Teilchenanzahl $N$ wie folgt genähert werden
\begin{align*}
    \frac{p V}{T} =  k_\text{B} N = \text{const}.
\end{align*}
Somit ergibt sich mit $V = \frac{m}{\rho}$ für die Volumina $V_0$ und $V_\text{k}$ des Transportgases mit $m_0 = m_\text{k} = m$
\begin{align}
    \nonumber   \frac{p_0 m}{\rho_0 T_0} = \frac{p_{\text{k}} m}{\rho_{\text{k}} T_{\text{k}}}
    \intertext{und somit schließlich}
    \rho_{\text{k}} = 
\end{align}