 \subsection{Mechanische Kompressorleistung}

Der Kompressor in der Wärmepumpe komprimiert das Volumen des Transportgases nahezu adiabatisch \cite[vgl.][]{man:v206}.
Das bedeutet, dass bei der Kompression nahezu keine Temperaturänderung stattfindet.
Wenn er ein Gasvolumen $V_\text{k}$ auf den Wert $V_\text{w}$ verringert, gilt allgemein für die Arbeit \cite[vgl.][]{man:v206}
\begin{align}
    A_\text{m} = - \int_{V_\text{k}}^{V_\text{w}} p \diff V.
\end{align}
Für adiabatische Verformungen gilt zusätzlich die Poisson`sche Gleichung
\begin{align}
    p_\text{k} V_\text{k}^{\kappa} = p_\text{w} V_\text{w}^{\kappa} = p V^{\kappa} = \mathrm{const.}
\end{align}
Hier steht $\kappa$ für das Verhältnis aus den Wärmekapazitäten $\frac{C_V}{C_p}$.
Dabei steht $C_V$ für die Wärmekapazität bei konstantem Volumen und $C_p$ für die Wärmekapazität bei konstantem Druck.
% Eine Herleitung findet sich in \cite[][320]{demtroeder}
In \cite{man:v206} wird hieraus die eine Formel für die mechanische Arbeit
\begin{align}
    A_{\text{m}} = \frac{1}{\kappa - 1} \left( p_\text{w} \sqrt[\kappa]{\frac{p_\text{k}}{p_\text{w}}} -p_\text{k} \right) V_\text{k}
\end{align}
und für die mechanische Kompressorleistung 
\begin{align}
    N_{\text{mech}} = \frac{\Delta A_\text{m}}{\Delta t} = \frac{1}{\kappa - 1}%
     \left(p_\text{w} \sqrt[\kappa]{\frac{p_\text{k}}{p_\text{w}}} - p_\text{k} \right) \frac{1}{\rho_{\text{k}}}\frac{\Delta m}{\Delta t}.
     \label{eq:mech_leistung}
\end{align}
hergeleitet.
Hierbei ist $\rho_{\text{k}}$ die Dichte des Gases in $V_\text{k}$ und kann mittels der idealen Gasgleichung $p V = k_\text{w} N T$
mit der Boltzmannkonstante $ k_\text{w}$ und der Teilchenanzahl $N$ wie folgt genähert werden
\begin{align*}
    \frac{p V}{T} =  k_\text{w} N = \text{const}.
\end{align*}
Somit ergibt sich mit $V = \frac{m}{\rho}$ für die Volumina $V_0$ und $V_\text{k}$ des Transportgases mit $m_0 = m_\text{k} = m$
\begin{align}
    \nonumber   \frac{p_0 m}{\rho_0 T_0} = \frac{p_{\text{k}} m}{\rho_{\text{k}} T_{\text{k}}}
    \intertext{und somit schließlich}
    \rho_{\text{k}} = \frac{\rho_0 T_0 p_\text{k}}{p_0 T_\text{k}}.
    \label{eq:rho_k}
\end{align}