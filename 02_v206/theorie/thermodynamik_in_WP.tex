\subsection{Thermodynamik in der Wärmepumpe}
% Die Wärmepumpe macht sich in ihrer Funktionsweise den zweiten Hauptsatz der Thermodynamik zunutze laut dem
% wärmeren zu einem kälteren Reservoir fließt \cite[vgl][321]{demtroeder}.

Der Energie die einem der Reservoire durch Änderung der Temperatur hinzugefügt wird,
heißt Wärmemenge
\begin{align}
    \diff Q = c m \diff T 
    \label{eq:dQ}
\end{align} 
%% Es ist wichtig sich zu entscheiden, wie die Hinzugefügte Wärmemenge heißen soll: Q_1 oder \Delta Q_1
$ c $ ist hier die spezifische Wärmekapazität des Materials,
 die im Allgemeinen von der Temperatur abhängen kann.
Der erste Hauptsatz der Thermodynamik bringt die bewegte Wärmemenge
mit der inneren Energie d.h. der gemittelten Temperatur der von aussen wirkenden Arbeit
ins Verhältnis \cite[vgl][318]{demtroeder}. 
Die Wärmepumpe kann als abgeschlossenes System betrachtet werden. 
\begin{align}
                \diff U &= \diff W + \diff Q \\
    \nonumber   \Delta U &= A + (\Delta Q_{2} - \Delta Q_{1})
\end{align}
Da die Wärmepumpe die Wärmeenergie von einem Reservoir in das andere transportiert
    bliebt die innere Energie konstant $ \Delta U = 0 $ .

\begin{align}
                \Delta Q_{1}    &= A + \Delta Q_{2} % DIe Q_i hängen selber auch mit veränderungen in der Temperatur zusammen...
\end{align}
%
Die Wärmepumpe muss also die Arbeit $\Delta W = A$ aufwenden um die Wärmeenergie aus Reservoir 2 in Reservoir 1 zu überführen.
An den einzelnen Reservoirs wird keine Arbeit in Form von einer Druckänderung gewirkt, da der Luftdruck konstant ist.
\begin{align*}
    \Delta W = - p \Delta V = 0 
\end{align*}
Wenn man jetzt idealisierend annimmt, das in dem Prozess keine Wärme verloren geht, kann man den zweiten Hauptsatz der Thermodynamik
verwenden um die Wärmemengen mit der abgegeben Temperatur in Verbindung zu setzen \cite[vgl.][1]{man:v206}. 
Es gilt:
\begin{align*}
    \int \frac{1}{T} \diff Q = 0 % Warum auch immer das aus dem zweiten Hauptsatz folgt
\end{align*}
Daraus folgt:
\begin{align}
    \frac{Q_{1}}{T_{1}} - \frac{Q_{2}}{T_{2}} &= 0
    %\frac{Q_{1}}{T_{1}} - \frac{Q_{2}}{T_{2}} &= c m \left(\frac{T_{0}}{T_{1}} - 1 \right) -c m \left(\frac{T_{0}}{T_{1}} - 1 \right) \\
    % &= c m T_{0} \left(\frac{1}{T_{1}} - \frac{1}{T_{2}}\right) = 0
\end{align}
Wenn man diese beiden Formeln ineinander einsetzt erhält man für die Wärmemenge $Q_1$ folgende Relation \cite{man:v206}: 
\begin{align}
\nonumber    Q_1 &= A + \frac{T_2}{T_1}Q_1 \\
    Q_1 &= A \frac{T_1}{T_1 - T_2}
\end{align}
