\subsection{Thermodynamik in der Wärmepumpe}
% Die Wärmepumpe macht sich in ihrer Funktionsweise den zweiten Hauptsatz der Thermodynamik zunutze laut dem
% wärmeren zu einem kälteren Reservoir fließt \cite[vgl][321]{demtroeder}.

Der Energie die einem der Reservoire durch Änderung der Temperatur hinzugefügt wird,
heißt Wärmemenge
\begin{align}
    \diff Q = c m \diff T 
    \label{eq:dQ}
\end{align} 
%% Es ist wichtig sich zu entscheiden, wie die Hinzugefügte Wärmemenge heißen soll: Q_\text{w} oder \Delta Q_\text{w}
$ c $ ist hier die spezifische Wärmekapazität des Materials,
 die im Allgemeinen von der Temperatur abhängen kann.
Der erste Hauptsatz der Thermodynamik bringt die bewegte Wärmemenge
mit der inneren Energie d.h. der gemittelten Temperatur der von aussen wirkenden Arbeit
ins Verhältnis \cite[vgl][318]{demtroeder}. 
Die Wärmepumpe kann als abgeschlossenes System betrachtet werden. 
\begin{align}
                \diff U &= \diff W + \diff Q \\
    \nonumber   \Delta U &= A + ( Q_\text{k} - Q_\text{w})
\end{align}
$Q_\text{w}$ steht hierbei für die Wärmemenge im wärmer werdenden Reservoir, $Q_\text{k}$ ist entsprechend die Wärmemenge im kälteren Reservoir.
Da die Wärmepumpe die Wärmeenergie von einem Reservoir in das andere transportiert
    bliebt die innere Energie konstant $ \Delta U = 0 $.

\begin{align}
    Q_\text{w}    &= A +  Q_\text{k} % Die Q_i hängen selber auch mit veränderungen in der Temperatur zusammen...
\end{align}
%
Die Wärmepumpe muss also die Arbeit $A$ aufwenden um die Wärmeenergie aus Reservoir k in Reservoir w zu überführen.
An den einzelnen Reservoirs wird keine Arbeit in Form von einer Druckänderung gewirkt, da der Luftdruck konstant ist.
\begin{align*}
    \Delta W = - p \Delta V = 0 
\end{align*}
Wird jetzt idealisierend angenommen, dass in dem Prozess keine Wärme verloren geht, kann der zweite Hauptsatz der Thermodynamik
verwendet werden um die Wärmemengen mit der abgegebenen Temperatur in Verbindung zu setzen \cite[vgl.][1]{man:v206}. 
Es gilt:
\begin{align*}
    \int \frac{1}{T} \diff Q = 0 % Warum auch immer das aus dem zweiten Hauptsatz folgt
\end{align*}
Daraus folgt:
\begin{align}
    \frac{Q_\text{w}}{T_\text{w}} - \frac{Q_\text{k}}{T_\text{k}} &= 0.% Überprüfen
    %\frac{Q_\text{w}}{T_\text{1}} - \frac{Q_\text{k}}{T_\text{k}} &= c m \left(\frac{T_\text{0}}{T_\text{1}} - 1 \right) -c m \left(\frac{T_\text{0}}{T_\text{1}} - 1 \right) \\
    % &= c m T_\text{0} \left(\frac{1}{T_\text{1}} - \frac{1}{T_\text{k}}\right) = 0
\end{align}
Diese beiden Formeln ineinander eingesetzt erhält ergeben für die Wärmemenge $Q_\text{w}$ folgende Relation \cite{man:v206}
\begin{align}
\nonumber    Q_\text{w} &= A + \frac{T_\text{k}}{T_\text{w}}Q_\text{w} \\
    Q_\text{w} &= A \frac{T_\text{w}}{T_\text{w} - T_\text{k}}.
\end{align}
