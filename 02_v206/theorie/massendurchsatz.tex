\subsection{Der Massendurchsatz \cite[vgl.][]{man:v206}}
Anhand der übertragenen Wärmemenge kann man die Menge des Gases berechnen, das pro Zeiteinheit
im zweiten Teil des Pumpkreislaufes verdampft.
Dies ist der Massendurchsatz $\diff / m\diff t$.
Das ist die Menge an Gas, die in derselben Zeit durch die ganze Wärmepumpe fließt.
Die Menge an verdampftem Gas in Reservoir k hängt folgendermaßen mit der Wärmemenge in Reservoir k zusammen:
\begin{align}
    \frac{\diff Q_\text{k}}{\diff t} = L \frac{\diff m}{\diff t}
    \label{eq:massedurchsatz}
\end{align} 
Die Änderung der Wärmemenge $ Q_\text{k}$ kann über die Messung bestimmt werden:
\begin{align}
    \frac{\Delta Q_\text{k}}{\Delta t} = (m_\text{H2O} c_{\text{H2O}} + m_\text{k} c_\text{k}) \frac{\Delta T_\text{k}}{\Delta t}
    \label{eq:aenderung_q2}
\end{align} 
Um die Verdampfungswärme $L$ des Transportgases herauszufinden machen wir uns folgende in \cite[][5]{man:v203} hergeleitete
Gleichung zunutze:
\begin{align}
    \ln{p} = -\frac{L}{R}\cdot \frac{1}{T} + \ln{p_0}
    \label{eq:Dampfdruckkurve}
\end{align}