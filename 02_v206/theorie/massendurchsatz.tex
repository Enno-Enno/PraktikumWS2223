\subsection{Der Massendurchsatz \cite[vgl.]{man:v206}}
Anhand der übertragenen Wärmemenge kann man die Menge des Gases berechnen, das pro Zeiteinheit
im zweiten Teil des Pumpkreislaufes verdampft.
Das ist die Menge an Gas, die in derselben Zeit durch die ganze Wärmepumpe fließt.
Die Menge an verdampftem Gas in Reservoir 2 hängt folgendermaßen mit der Wärmemenge in Reservoir 2 zusammen:
\begin{align}
    \frac{\Delta Q_2}{\Delta t} = L \frac{\Delta m}{\Delta T}
\end{align} 
Das Reservoir 2 hat eine 