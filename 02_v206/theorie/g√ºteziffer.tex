\subsection{Die Güteziffer}
Die Güteziffer $ \nu $ beschreibt das Verhältnis von Aufgewendeter Arbeit zu transportierter Wärme \cite[vgl.][1]{man:v206}
Sie wird folgendermaßen berechnet.
\begin{align}
\nonumber    \nu_{\text{ideal}} &= \frac{Q_{1}}{A} \\
    \nu_{\text{ideal}} &= \frac{T_1}{T_1 - T_2}
    \label{eq:gueteTheorie}
    \intertext{Da für $ Q_1 $ eine ideale Wärmeübertragung angenommen wurde ist die reale Güteziffer kleiner}
\nonumber    \nu_{\text{real}} &< \frac{Q_{1}}{A} 
\end{align}
% Der folgende Teil sollte vielleicht zur Auswertung wandern: 
Um die reale Güteziffer zu berechnen muss die gemessene Wärmemenge mit der tatsächlich geleisteten
Arbeit der Wärmepumpe in Verhältnis gesetzt werden.
\begin{align}
    \nonumber A &= N \Delta t \\
    \nonumber \frac{\diff Q_1}{\diff t} &= (m_1 c_w + m_k c_k) \frac{\diff T}{\diff t} \\
    \nu_{\text{real}} &= \frac{Q_{1}}{A} 
    \label{eq:guetePraxis}
\end{align}
