\section{Auswertung}
\subsection{Fehlerrechnung}

\subsubsection{Gaußsche Fehlerfortpflanzung}

Wenn zu Messdaten die Standardabweichung bekannt ist, und mit diesen Messdaten weiter gerechnet werden soll,
wird die Gaußsche Fehlerfortpflanzung verwendet. 
Angenommen wir haben $k$ Messwerte $x_i [i \in \mathbb{N}, i \leq k]$ mit den Standardabweichungen $\Delta x_i$
und eine abgeleitete Größe $f(x_i)$ dann ist der Fehler von $f$
\begin{align}
    \Delta f(x_i) = \sqrt{
    \left(\frac{\partial f}{\partial x_1} \Delta x_1\right)^2%
     + \left(\frac{\partial f}{\partial x_2} \Delta x_2\right)^2%
     + \dots%
     + \left(\frac{\partial f}{\partial x_k} \Delta x_k\right)^2%
    }.
    \label{eq:gauspflanz}
\end{align} 
Im Ergebnis gibt es dann den Mittelwert von $f$ mit der errechneten Abweichung $\overline{f} \pm \Delta f $.
Um Rechenfehler zu vermeiden wird das Python Paket \texttt{uncertainties} \cite[][]{uncertainties} verwendet.
Hier wird die Fehlerfortpflanzung automatisch verrechnet, wenn man die Variablen als \texttt{ufloat} definiert.


%\subsubsection{Fehlerrechnung bei der linearen Regression}

\subsection{Die Güteziffer}
Um die Güteziffer $\nu$ zu bestimmen, muss gemäß \eqref{eq:guetePraxis} zunächst
die zeitliche Änderung des Temperaturverlaufs ermittelt werden.
Dies geschieht, indem die die Messdaten der Temperaturen beider Reservoire aus Tabelle \ref{tab:messdaten} geplottet werden.
Anschließend wird mit Hilfe einer nicht-linearen Ausgleichsrechnung eine Funktion $T \left(t\right)$ ermittelt, die den Verlauf 
für das jeweilige Reservoir beschreiben soll.
Diese Funktion wird nach der Zeit differenziert, sodass man $\nu$ für je vier unterschiedliche Temperaturen bestimmen kann.
Für die vier Berechnungen der Güteziffer werden Tabelle \ref{tab:messdaten} jeweils die Temperaturen $T_{\text{k}}$ und $T_{\text{w}}$ nach 
$t_1 = \qty[]{3}{\s}$, $t_2 = \qty[]{9}{\s}$, $t_3 = \qty[]{15}{\s}$ und $t_4 = \qty[]{21}{\s}$ entnommen.


\subsubsection{Plot der Temperaturverläufe}
Plottet man die Temperaturverläufe der beiden Reservoire in Abhängigkeit von der Zeit ergibt sich Abbildung \ref{fig:temperaturverlauf}.

\begin{figure}
    \includegraphics[]{build/plot_temp_verlauf.pdf}
    \caption[]{Zeitabhängige Temperaturverläufe beider Reservoire}
    \label{fig:temperaturverlauf}
\end{figure}


\subsubsection[]{Ermittlung der Ausgleichsfunktionen}
Um die Ausgleichskurve zu approximieren, sind folgende Funktionen möglich:
\begin{align}
    T \left(t\right) &= A t^2 + B t + C \label{eq:ausgleichsfunktion_1} \\
    T \left(t\right) &= \frac{A}{1 + B t^{\alpha}} \label{eq:ausgleichsfunktion_2} \\
    T \left(t\right) &= \frac{A t^{\alpha}}{1 + B t^{\alpha}} + C \label{eq:ausgleichsfunktion_3}
\end{align}
Hierbei sind $A$, $B$, $C$ und $\alpha$ die zu bestimmenden Parameter, wobei $1 \leq \alpha \leq 2$ gelten soll.
Mit Hilfe der Python \cite[]{python} Funktion \texttt{curve\_fit} aus dem Paket \texttt{scipy.optimize} 
\cite[]{scipy} lassen sich die nicht-linearen Ausgleichsrechnungen durchführen.
\begin{figure}
    \includegraphics[]{build/plot_ausgleich_1.pdf}
    \caption[]{Ausgleichskurve zu Funktion \eqref{eq:ausgleichsfunktion_1}}
    \label{fig:ausgleichsplot_1}
\end{figure}
\begin{figure}
    \includegraphics[]{build/plot_ausgleich_2.pdf}
    \caption[]{Ausgleichskurve zu Funktion \eqref{eq:ausgleichsfunktion_2}}
    \label{fig:ausgleichsplot_2}
\end{figure}
\begin{figure}
    \includegraphics[]{build/plot_ausgleich_3.pdf}
    \caption[]{Ausgleichskurve zu Funktion \eqref{eq:ausgleichsfunktion_3}}
    \label{fig:ausgleichsplot_3}
\end{figure}
Vergleicht man die Plots \ref{fig:ausgleichsplot_1}, \ref{fig:ausgleichsplot_2} und \ref{fig:ausgleichsplot_3} mit einander,
erkennt man unmittelbar, dass die Approximation \eqref{eq:ausgleichsfunktion_2} auch nach einstellen der Startparameter am ungenausten ist.
\eqref{eq:ausgleichsfunktion_1} und \eqref{eq:ausgleichsfunktion_3} erscheinen zunächst in etwa den gleichen Exaktheitsgrad zu haben.
Im Folgenden wird mit \eqref{eq:ausgleichsfunktion_3} weiter gerechnet, da diese besonders für kleine Zeiten näher an den eigentlichen Messdaten liegt
als \eqref{eq:ausgleichsfunktion_1}.
Die entsprechenden Parameter betragen dann für das kalte Reservoir
\begin{align*}
    A_{\text{k}} &= -0.9113 \pm 0.0742 & B_{\text{k}} &= 0.0204 \pm 0.0008 \\
    C_{\text{k}} &= 294.6660 \pm 0.1221 & \alpha_{\text{k}} &= 1.1946 \pm 0.0378 \\
    %error kalt:  [0.07422951 0.00082891 0.12214775 0.03782461]
    \intertext{und für das warme Reservoir}
    A_{\text{w}} &= 1.1391 \pm 0.0429 & B_{\text{w}} &= 0.0229  \pm 0.0005 \\
    C_{\text{w}} &= 294.0467 \pm 0.0704 & \alpha_{\text{w}} &= 1.2211 \pm 0.0175,
\end{align*}
%error warm:  [0.04285291 0.0004705  0.07042233 0.01747473]
wobei die Fehler der Kovarianzmatrix entnommen wurden.


\subsubsection[]{Bestimmung der Differentialquotienten}
\label{sec: dif_quot}
Differenziert man \eqref{eq:ausgleichsfunktion_3} nach der Zeit, erhält man mittels Quotientenregel
\begin{align*}
    \frac{\symup{d}T}{\symup{d}t} = \frac{\alpha A t^{\alpha -1}}{\left(1 + B t^{\alpha}\right)^2}.
\end{align*}
%
Setzt man für beide Reservoire jeweils die obigen Parameter und entsprechenden Zeiten ein, erhält man 
\begin{align*}
    \frac{\symup{d}}{\symup{d}t} T_\text{k} \left(t_1\right) &= \left( -1.16 \pm 0.12 \right)\, \unit[]{\kelvin\per\min}  &
    \frac{\symup{d}}{\symup{d}t} T_\text{w} \left(t_1\right) &= \left(  1.50 \pm 0.07 \right)\, \unit[]{\kelvin\per\min}   \\
    \frac{\symup{d}}{\symup{d}t} T_\text{k} \left(t_2\right) &= \left( -1.02 \pm 0.12 \right)\, \unit[]{\kelvin\per\min}  &
    \frac{\symup{d}}{\symup{d}t} T_\text{w} \left(t_2\right) &= \left(  1.27 \pm 0.07 \right)\, \unit[]{\kelvin\per\min} \\
    \frac{\symup{d}}{\symup{d}t} T_\text{k} \left(t_3\right) &= \left( -0.80 \pm 0.09 \right)\, \unit[]{\kelvin\per\min} &
    \frac{\symup{d}}{\symup{d}t} T_\text{w} \left(t_3\right) &= \left(  0.96 \pm 0.05 \right)\, \unit[]{\kelvin\per\min} \\
    \frac{\symup{d}}{\symup{d}t} T_\text{k} \left(t_4\right) &= \left( -0.63 \pm 0.06 \right)\, \unit[]{\kelvin\per\min}  &
    \frac{\symup{d}}{\symup{d}t} T_\text{w} \left(t_4\right) &= \left(  0.72 \pm 0.03 \right)\, \unit[]{\kelvin\per\min}.
\end{align*}
%

\subsubsection[]{Berechnung der Güteziffer}
Gemäß Formel \eqref{eq:gueteTheorie} ergeben sich folgende ideale Güteziffern $\nu_\text{ideal}$ in Tabelle \ref{tab:gueteziffern}.
%
Mit Hilfe der spezifischen Wärmekapazität von Wasser $c_{\text{w}} = \qty{4190}{\joule\per\kg\per\kelvin}$ \cite[]{leifi}, der gegebenen Wärmekapazität
der Kupferspirale $m_{\text{K}} c_{\text{K}} = \qty{750}{\joule\per\kelvin}$ und der zuvor bestimmten Differentialquotienten
kann bei bei einer Wassermenge von $\qty[]{3}{\liter}$ pro Behälter die reale Güteziffer $\nu_{\text{real}}$ gemäß \eqref{eq:guetePraxis} bestimmt werden.
Hierfür wird die gemittelte Leistungsaufnahme des Kompressors
\begin{align*}
    N = \frac{1}{n} \sum_{m=1}^{n} P_m.
\end{align*}
benötigt.
Man erhält
\begin{align*}
    N_1  &=  \qty[]{7200}{\J\per\min} &
    N_2  &=  \qty[]{7360}{\J\per\min} &
    N_3  &=  \qty[]{7376}{\J\per\min} &
    N_4  &=  \qty[]{7240}{\J\per\min}
\end{align*}
Sodass sich schließlich  $\nu_\text{real}$ in Tabelle \ref{tab:gueteziffern} ergibt.
Man erkennt große Diskrepanzen zwischen idealen und realen Werten,
die rechnerisch als prozentuale Abweichungen der Mittelwerte $\Delta \nu$ in der selben Tabelle angegeben sind.

\begin{table}
    \caption[]{Ideale und reale Güteziffern mit ihrer Abweichung}
    \label{tab:gueteziffern}
    \sisetup{table-format = 2.2}
    \begin{tabular}{S S S @{${}\pm{}$} S S}
        \toprule
        {$t /\unit[]{\min}$} & {$\nu_\text{ideal}$} & \multicolumn{2}{c}{$\nu_\text{real}$} & {$\Delta \nu$} \\
        \midrule
        3  & 45.9 & 2.77 & 0.13 & 93.90 \\
        9  & 14.2 & 2.30 & 0.12 & 83.80 \\
        15 & 9.3  & 1.73 & 0.08 & 81.72 \\
        21 & 7.4  & 1.33 & 0.06 & 82.43 \\ 
        \bottomrule 
    \end{tabular}
    \centering
\end{table}
%Abweichung nu:  [0.93899782 0.83802817 0.8172043  0.82432432]

%nu real 0:  2.77+/-0.13
%nu real 1:  2.30+/-0.12
%nu real 2:  1.73+/-0.08
%nu real 3:  1.33+/-0.06




\subsection[]{Der Massendurchsatz}
Um den Massendurchsatz zu bestimmen, werden die zuvor in Abschnitt \ref{sec: dif_quot} Differentialquotienten des kalten Reservoirs benötigt.
Des Weiteren muss die Verdampfungswärme $L$ des Transportmediums, hier: $\ce{Cl2F2C}$, für Formel \textbf{REFERENZ!!!} ermittelt werden.


\subsubsection[]{Die Verdampfungswärme}
Zur Bestimmung der Verdampfungswärme $L$ werden die Wertepaare $\left(p_{\text{k}}, \frac{1}{T_{\text{k}}}\right)$ logarithmisch dargestellt
und anschließend eine lineare Ausgleichsrechnung zur Bestimmung von $L$ durchgeführt.
Dies baut auf der Formel 
\begin{align*}
    p = p_0 \exp{\left(-\frac{L}{RT}\right)}
\end{align*}
aus Versuch \cite[]{man:v203} auf.
$p_0$ ist dabei der Umgebungsdruck bei Zimmertemperatur und $R$ die Gaskonstante.
Sobald die Steigung $m$ der Ausgleichsgeraden ermittelt wurde, kann somit $L = -R \cdot m$ bestimmt werden.
Laut der Wetterstation \cite*[][]{wetterstation} betrug der Luftdruck am Tag des Experiments
$p_0 = \qty[]{1018.1}{\hecto\pascal} \simeq \qty[]{1.0}{\bar}$.
Mittels der Python \cite[]{python} Funktion \texttt{polyfit} aus dem Paket \texttt{numpy} \cite[]{numpy} wird die lineare Regression durchgeführt und
der Plot \ref{fig:ausgleichsgerade} erstellt.
Es ergibt sich die Steigung $m = \left(-2001.8141\pm 111.7750 \right)$ und ein Achsenabschnitt von $n = \left( 8.3639 \pm 0.3961 \right)$, 
wobei letzterer für die Rechnungen nicht relevant ist. 
Folglich gilt $L = $ %%%%%%%%%%%%%%%%%%%%%%%%%%%%%%%%%%%%%%%%%%%%%%%%%%%%%%%%%%%%%%%%%%%%%%%%%%%%%%%%%%
\begin{figure}
    \includegraphics[]{build/plot_verdampfungswaerme.pdf}
    \caption[]{lineare Ausgleichsgerade zur Bestimmung von $L$}
    \label{fig:ausgleichsgerade}
\end{figure}

%Steigung m= -2001.814130953005
%achsenabschn b=  8.363939543080686
%#Fehler:  [111.77504319   0.39608049]





%fragen: p0, standardabweichung?