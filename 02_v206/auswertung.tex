\section{Auswertung}
Test.....................
warum?


\subsection{Bestimmung der Güteziffer}
Um die Güteziffer $\nu$ zu bestimmen, muss gemäß \textbf{GLEICHUNG AUS THEORIE REFERENZIEREN SOBALD DIE DA IST} zunächst
die zeitliche Änderung des Temperaturverlaufs ermittelt werden.
Dies geschieht, indem die die Messdaten der Temperaturen beider Reservoire aus Tabelle \ref{tab:messdaten} geplottet werden.
Anschließend wird mit Hilfe einer nicht-linearen Ausgleichsrechnung eine Funktion $T \left(t\right)$ ermittelt, die den Verlauf 
für das jeweilige Reservoir beschreiben soll.
Diese Funktion wird nach der Zeit differenziert, sodass man $\nu$ für je vier unterschiedliche Temperaturen bestimmen kann.


\subsubsection{Plot der Temperaturverläufe}
Plottet man die Temperaturverläufe der beiden Reservoire in Abhängigkeit von der Zeit ergibt sich Abbildung \ref{fig:temperaturverlauf}.

\begin{figure}
    \includegraphics[]{build/plot_temp_verlauf.pdf}
    \caption[]{Zeitabhängige Temperaturverläufe beider Reservoire}
    \label{fig:temperaturverlauf}
\end{figure}


\subsubsection[]{Ermittlung der Ausgleichsfunktionen}
Um die Ausgleichskurve zu approximieren, sind folgende Funktionen möglich:
\begin{align}
    T \left(t\right) &= A t^2 + B t + C \label{eq:ausgleichsfunktion_1} \\
    T \left(t\right) &= \frac{A}{1 + B t^{\alpha}} \label{eq:ausgleichsfunktion_2} \\
    T \left(t\right) &= \frac{A t^{\alpha}}{1 + B t^{\alpha}} + C \label{eq:ausgleichsfunktion_3}
\end{align}
Hierbei sind $A$, $B$, $C$ und $\alpha$ die zu bestimmenden Parameter, wobei $1 \leq \alpha \leq 2$ gelten soll.
Mit Hilfe der Python \cite[]{python} Funktion \texttt{curve\_fit} aus dem Paket \texttt{scipy.optimize} 
\cite[]{scipy} lassen sich die nicht-linearen Ausgleichsrechnungen durchführen.
\begin{figure}
    \includegraphics[]{build/plot_ausgleich_1.pdf}
    \caption[]{Ausgleichskurve zu Funktion \eqref{eq:ausgleichsfunktion_1}}
    \label{fig:ausgleichsplot_1}
\end{figure}
\begin{figure}
    \includegraphics[]{build/plot_ausgleich_2.pdf}
    \caption[]{Ausgleichskurve zu Funktion \eqref{eq:ausgleichsfunktion_2}}
    \label{fig:ausgleichsplot_2}
\end{figure}
\begin{figure}
    \includegraphics[]{build/plot_ausgleich_3.pdf}
    \caption[]{Ausgleichskurve zu Funktion \eqref{eq:ausgleichsfunktion_3}}
    \label{fig:ausgleichsplot_3}
\end{figure}
Vergleicht man die Plots \ref{fig:ausgleichsplot_1},\ref{fig:ausgleichsplot_2} und \ref{fig:ausgleichsplot_3} mit einander,
erkennt man unmittelbar, dass die Approximation \eqref{eq:ausgleichsfunktion_2} insgesamt am ungenausten ist.
\eqref{eq:ausgleichsfunktion_1} und \eqref{eq:ausgleichsfunktion_3} erscheinen zunächst in etwa den gleichen Exaktheitsgrad zu haben.
Im Folgenden wird mit \eqref{eq:ausgleichsfunktion_3} weiter gerechnet, da diese besonders für kleine Zeiten näher an den eigentlichen Messdaten liegt
als \eqref{eq:ausgleichsfunktion_1}.
Die entsprechenden Parameter betragen dann für das kalte Reservoir
\begin{align*}
    A_{\text{k}} &= -0.91132029 & B_{\text{k}} &= 0.02042680 \\
    C_{\text{k}} &= 21.51604450 & \alpha_{\text{k}} &= 1.19462353 \\
    \intertext{und für das warme Reservoir}
    A_{\text{w}} &= 1.13913878 & B_{\text{w}} &= 0.02290029 \\
    C_{\text{w}} &= 20.89673747 & \alpha_{\text{w}} &= 1.22110022.
\end{align*}


\subsubsection[]{Bestimmung des Differentialquotienten}
Differenziert man \eqref{eq:ausgleichsfunktion_3} nach der Zeit, ergibt sich 

\begin{align*}
    \dv{T}{t} = \frac{\alpha A t^{\alpha -1}}{\left(1 + B t^{\alpha}\right)^2}.
\end{align*}

%Mit Hilfe der Wärmekapazität von Wasser $c_{\text{w}} = \qty{4190}{\joule\per\kg\per\kelvin}$ \cite{leifi}