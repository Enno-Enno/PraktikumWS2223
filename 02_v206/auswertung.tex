\section{Auswertung}

\subsection{Die Güteziffer}
Um die Güteziffer $\nu$ zu bestimmen, muss gemäß \ref*{eq:guetePraxis} zunächst
die zeitliche Änderung des Temperaturverlaufs ermittelt werden.
Dies geschieht, indem die die Messdaten der Temperaturen beider Reservoire aus Tabelle \ref{tab:messdaten} geplottet werden.
Anschließend wird mit Hilfe einer nicht-linearen Ausgleichsrechnung eine Funktion $T \left(t\right)$ ermittelt, die den Verlauf 
für das jeweilige Reservoir beschreiben soll.
Diese Funktion wird nach der Zeit differenziert, sodass man $\nu$ für je vier unterschiedliche Temperaturen bestimmen kann.
Für die vier Berechnungen der Güteziffer werden Tabelle \ref{tab:messdaten} jeweils die Temperaturen $T_{\text{k}}$ und $T_{\text{w}}$ nach 
$t_1 = \qty[]{3}{\s}$, $t_2 = \qty[]{9}{\s}$, $t_3 = \qty[]{15}{\s}$ und $t_4 = \qty[]{21}{\s}$ entnommen.


\subsubsection{Plot der Temperaturverläufe}
Plottet man die Temperaturverläufe der beiden Reservoire in Abhängigkeit von der Zeit ergibt sich Abbildung \ref{fig:temperaturverlauf}.

\begin{figure}
    \includegraphics[]{build/plot_temp_verlauf.pdf}
    \caption[]{Zeitabhängige Temperaturverläufe beider Reservoire}
    \label{fig:temperaturverlauf}
\end{figure}


\subsubsection[]{Ermittlung der Ausgleichsfunktionen}
Um die Ausgleichskurve zu approximieren, sind folgende Funktionen möglich:
\begin{align}
    T \left(t\right) &= A t^2 + B t + C \label{eq:ausgleichsfunktion_1} \\
    T \left(t\right) &= \frac{A}{1 + B t^{\alpha}} \label{eq:ausgleichsfunktion_2} \\
    T \left(t\right) &= \frac{A t^{\alpha}}{1 + B t^{\alpha}} + C \label{eq:ausgleichsfunktion_3}
\end{align}
Hierbei sind $A$, $B$, $C$ und $\alpha$ die zu bestimmenden Parameter, wobei $1 \leq \alpha \leq 2$ gelten soll.
Mit Hilfe der Python \cite[]{python} Funktion \texttt{curve\_fit} aus dem Paket \texttt{scipy.optimize} 
\cite[]{scipy} lassen sich die nicht-linearen Ausgleichsrechnungen durchführen.
\begin{figure}
    \includegraphics[]{build/plot_ausgleich_1.pdf}
    \caption[]{Ausgleichskurve zu Funktion \eqref{eq:ausgleichsfunktion_1}}
    \label{fig:ausgleichsplot_1}
\end{figure}
\begin{figure}
    \includegraphics[]{build/plot_ausgleich_2.pdf}
    \caption[]{Ausgleichskurve zu Funktion \eqref{eq:ausgleichsfunktion_2}}
    \label{fig:ausgleichsplot_2}
\end{figure}
\begin{figure}
    \includegraphics[]{build/plot_ausgleich_3.pdf}
    \caption[]{Ausgleichskurve zu Funktion \eqref{eq:ausgleichsfunktion_3}}
    \label{fig:ausgleichsplot_3}
\end{figure}
Vergleicht man die Plots \ref{fig:ausgleichsplot_1},\ref{fig:ausgleichsplot_2} und \ref{fig:ausgleichsplot_3} mit einander,
erkennt man unmittelbar, dass die Approximation \eqref{eq:ausgleichsfunktion_2} auch nach einstellen der Startparameter am ungenausten ist.
\eqref{eq:ausgleichsfunktion_1} und \eqref{eq:ausgleichsfunktion_3} erscheinen zunächst in etwa den gleichen Exaktheitsgrad zu haben.
Im Folgenden wird mit \eqref{eq:ausgleichsfunktion_3} weiter gerechnet, da diese besonders für kleine Zeiten näher an den eigentlichen Messdaten liegt
als \eqref{eq:ausgleichsfunktion_1}.
Die entsprechenden Parameter betragen dann für das kalte Reservoir
\begin{align*}
    A_{\text{k}} &= -0.91131991 & B_{\text{k}} &= 0.02042679 \\
    C_{\text{k}} &= 294.66604398 & \alpha_{\text{k}} &= 1.19462372 \\
    \intertext{und für das warme Reservoir}
    A_{\text{w}} &= 1.13913900 & B_{\text{w}} &= 0.02290029 \\
    C_{\text{w}} &= 294.04673712 & \alpha_{\text{w}} &= 1.22110015.
\end{align*}

%#Funktion 3 für kalt:
%#a = -0.91131991
%#b = 0.02042679
%#c = 294.66604398
%#p = 1.19462372
%#Funktion 3 für warm:
%#a = 1.13913900
%#b = 0.02290029
%#c = 294.04673712
%#p = 1.22110015


\subsubsection[]{Bestimmung der Differentialquotienten}
Differenziert man \eqref{eq:ausgleichsfunktion_3} nach der Zeit, erhält man mittels Quotientenregel
\begin{align*}
    \frac{\symup{d}T}{\symup{d}t} = \frac{\alpha A t^{\alpha -1}}{\left(1 + B t^{\alpha}\right)^2}.
\end{align*}
%\dv{T}{t} läuft nur mit paket physics und das mag weder qty noch biber
%
Setzt man für beide Reservoire jeweils die obigen Parameter und entsprechenden Zeiten ein, erhält man 
\begin{align*}
    \frac{\symup{d}}{\symup{d}t} T_\text{k} \left(t_1\right) &= -1.1647318102435604 &
    \frac{\symup{d}}{\symup{d}t} T_\text{w} \left(t_1\right) &=  1.4993025822691834  \\
    \frac{\symup{d}}{\symup{d}t} T_\text{k} \left(t_2\right) &= -1.0159749249040468 &
    \frac{\symup{d}}{\symup{d}t} T_\text{w} \left(t_2\right) &= 1.2686378707296608 \\
    \frac{\symup{d}}{\symup{d}t} T_\text{k} \left(t_3\right) &= -0.799224014090223 &
    \frac{\symup{d}}{\symup{d}t} T_\text{w} \left(t_3\right) &= 0.9584929746374173 \\
    \frac{\symup{d}}{\symup{d}t} T_\text{k} \left(t_4\right) &= -0.6243740742879325 &
    \frac{\symup{d}}{\symup{d}t} T_\text{w} \left(t_4\right) &= 0.7224879854278933 
\end{align*}
%

\subsubsection[]{Berechnung der Güteziffer}
Gemäß Formel \eqref{eq:gueteTheorie} ergeben sich folgende ideale Güteziffern $\nu_\text{ideal}$
\begin{align*}
    \nu_{\text{ideal}, 1} &= 45.9 & \nu_{\text{ideal}, 2} &=14.2 & \nu_{\text{ideal}, 3} &= 9.3 & \nu_{\text{ideal}, 4} &= 7.4
\end{align*}
%nu ideal: [45.85384615 14.1875      9.29228487  7.39883721]
Mit Hilfe der spezifischen Wärmekapazität von Wasser $c_{\text{w}} = \qty{4190}{\joule\per\kg\per\kelvin}$ \cite[]{leifi}, der gegebenen Wärmekapazität
der Kupferspirale $m_{\text{K}} c_{\text{K}} = \qty{750}{\joule\per\kelvin}$ und der zuvor bestimmten Differentialquotienten
kann bei bei einer Wassermenge von $\qty[]{3}{\liter}$ pro Behälter die reale Güteziffer $\nu_{\text{real}}$ gemäß \eqref{eq:guetePraxis} bestimmt werden.
Hier für wird die gemittelte Leistungsaufnahme des Kompressors
\begin{align*}
    N = \frac{1}{t_n} \sum_{m=1}^{t_n} P_m.
\end{align*}
nach der jeweiligen Zeit $t_n \ \left(n = 1, \dotsc, 4\right)$ in Minuten benötigt.
Man erhält
\begin{align*}
    N_1  &=  120 &
    N_2  &=  123 &
    N_3  &=  123 &
    N_4  &=  121
\end{align*}
Sodass sich schließlich ergibt
\begin{align*}
    \nu_{\text{real}, 1} &=  & \nu_{\text{real}, 2} &=  & \nu_{\text{real}, 3} &=  & \nu_{\text{real}, 4} &=  & 
\end{align*}
%
