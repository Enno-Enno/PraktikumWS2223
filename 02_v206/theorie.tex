\section{Theorie}

\subsection{Thermodynamik in der Wärmepumpe}
Eine Wärmepumpe ist ein Gerät, welches die Wärmeenergie von einem Ort an einen anderen transportiert.
Dazu macht sie sich den zweiten Hauptsatz der Thermodynamik zunutze, laut dem Wärmeenergie immer von einem 
wärmeren zu einem kälteren Reservoir fließt \cite[vgl][S. 321]{demtroeder}. 
Der Energie die einem der Reservoire hinzugefügt wird, heißt Wärmemenge $ Q = c m \Delta T $. % Drüber nachdenken, wie man das richtig ausdrückt
$ c $ ist hier die spezifische Wärmekapazität des Materials $\Delta T $ die Anderung der Temperatur, 
die bei unveränderlichem Druck diese 
In diesem Versuch geht es darum die Güteziffer $ \nu $
herauszufinden, die die Wärmepumpe beschreibt.
\begin{align}
    \nu = \frac{\Delta Q_{\text{transp}}}{A}
\end{align}
Der erste Hauptsatz der Thermodynamik bringt die bewegte Wärmemenge mit der inneren Energie d.h. der gemittelten Temperatur
der von aussen wirkenden Arbeit ins Verhältnis \cite[vgl][S. 318]{demtroeder}. 
Die Wärmepumpe kann als ganzes System betrachtet werden. % Ünnötig finde einen besseren Satz dafür
\begin{align}
                \Delta U &= \Delta W + \Delta Q \\
    \nonumber   \Delta U &= A + (Q_{2} - Q_{1}) \\
    \intertext{Da die Wärmepumpe die Wärmeenergie von einem Reservoir in das andere transportiert
    bliebt die innere Energie konstant $ \Delta U = 0 $ .} 
                Q_{1}    &= A + Q_{2} % DIe Q_i hängen selber auch mit veränderungen in der Temperatur zusammen...
\end{align}
Die Wärmepumpe muss also die Arbeit $\Delta W = A$ aufwenden um die Wärmeenergie aus Reservoir 2 in Reservoir 1 zu überführen.
\begin{align}
    \frac{Q_{1}}{}
\end{align}