\section{Theorie}

\subsection{Die Funktionsweise der Wärmepumpe} % Zitieren soweit notwendig
\label{sec:funktionsweise}
Die Wärmepumpe pumpt ein Gas in einem abgeschlossenen Kreislauf durch die beiden Reservoirs.
In den Reservoirs besteht der Kreislauf aus Kupferrohren, die zu einer Spirale aufgewickelt sind.
So kann wärme zwischen den Reservoirs und dem Kreislauf fließen.
Dabei wird durch einen Kompressor und ein Drosselventil ein höherer Druck in der Kupferspirale 1 erzeugt.
Das Gas kondensiert,  verliert an Volumen und gibt Wärme an die Umgebung ab.
Die Temperatur in Reservoir 1 steigt also. 
Das Flüssige Gas wird durch das Drosselventil in den zweiten Teil des Kreislaufes gelassen.
Durch eine technische Vorrichtung wird sichergestellt, dass nur das flüssige Gas den Kreislauf passiert.
Im zweiten Teil des Kreislaufs wird durch die Pumpe ein Unterdruck im Vergleich zum ersten Teil erzeugt.
Dabei verdampft das Gas wieder, es nimmt an Volumen zu und nimmt Wärme aus der Umgebung auf.
Das Reservoir 2 wird also kälter.

\subsection{Thermodynamik in der Wärmepumpe}
% Die Wärmepumpe macht sich in ihrer Funktionsweise den zweiten Hauptsatz der Thermodynamik zunutze laut dem
% wärmeren zu einem kälteren Reservoir fließt \cite[vgl][321]{demtroeder}.

Der Energie die einem der Reservoire durch Änderung der Temperatur hinzugefügt wird,
heißt Wärmemenge $ Q = c m \Delta T $. 
%% Es ist wichtig sich zu entscheiden, wie die Hinzugefügte Wärmemenge heißen soll: Q_1 oder \Delta Q_1
$ c $ ist hier die spezifische Wärmekapazität des Materials, die im Allgemeinen von der Temperatur abhängen kann.
Der erste Hauptsatz der Thermodynamik bringt die bewegte Wärmemenge mit der inneren Energie d.h. der gemittelten Temperatur
der von aussen wirkenden Arbeit ins Verhältnis \cite[vgl][318]{demtroeder}. 
Die Wärmepumpe kann als abgeschlossenes System betrachtet werden. 
\begin{align}
                \Delta U &= \Delta W + \Delta Q \\
    \nonumber   \Delta U &= A + (Q_{2} - Q_{1}) \\
\intertext{Da die Wärmepumpe die Wärmeenergie von einem Reservoir in das andere transportiert
    bliebt die innere Energie konstant $ \Delta U = 0 $ .} 
                Q_{1}    &= A + Q_{2} % DIe Q_i hängen selber auch mit veränderungen in der Temperatur zusammen...
\end{align}
%
Die Wärmepumpe muss also die Arbeit $\Delta W = A$ aufwenden um die Wärmeenergie aus Reservoir 2 in Reservoir 1 zu überführen.
An den einzelnen Reservoirs wird keine Arbeit in Form von einer Druckänderung gewirkt, da der Luftdruck konstant ist.
\begin{align}
    \Delta W = - p \Delta V = 0 
\end{align}
Wenn man jetzt idealisierend annimmt, das in dem Prozess keine Wärme verloren geht, kann man den zweiten Hauptsatz der Thermodynamik
verwenden um die Wärmemengen mit der abgegeben Temperatur in Verbindung zu setzen \cite[vgl.][1]{man:v206}. 
Es gilt:
\begin{align}
    \int \frac{1}{T} dQ = 0 % Warum auch immer das aus dem zweiten Hauptsatz folgt
\end{align}
Daraus folgt anscheinend:
\begin{align}
    \frac{Q_{1}}{T_{1}} - \frac{Q_{2}}{T_{2}} &= 0
    %\frac{Q_{1}}{T_{1}} - \frac{Q_{2}}{T_{2}} &= c m \left(\frac{T_{0}}{T_{1}} - 1 \right) -c m \left(\frac{T_{0}}{T_{1}} - 1 \right) \\
    % &= c m T_{0} \left(\frac{1}{T_{1}} - \frac{1}{T_{2}}\right) = 0
\end{align}
Wenn man diese beiden Formeln ineinander einsetzt erhält man für die Wärmemenge $Q_1$ folgende relation \cite{man:v206}: 
\begin{align}
    Q_1 &= A + \frac{T_2}{T_1}Q_1 \\
    Q_1 &= A \frac{T_1}{T_1 - T_2}
\end{align}

\subsection{Die Güteziffer}
Die Güteziffer $ \nu $ beschreibt das Verhältnis von Aufgewendeter Arbeit zu transportierter Wärme \cite[vgl.][1]{man:v206}
Sie wird folgendermaßen berechnet.
\begin{align}
    \nu_{\text{ideal}} &= \frac{Q_{1}}{A} \\
    \nu_{\text{ideal}} &= \frac{T_1}{T_1 - T_2}
    \label{eq:gueteTheorie}
    \intertext{Da für $ Q_1 $ eine ideale Wärmeübertragung angenommen wurde ist die reale Güteziffer kleiner}
    \nu_{\text{real}} &< \frac{Q_{1}}{A} 
\end{align}
% Der folgende Teil sollte vielleicht zur Auswertung wandern: 
Um die reale Güteziffer zu berechnen muss die gemessene Wärmemenge mit der tatsächlich geleisteten
Arbeit der Wärmepumpe in Verhältnis gesetzt werden.
\begin{align}
    \nonumber A &= N \Delta t \\
    \nonumber \frac{Q_1}{\Delta t} &= (m_1 c_w + m_k c_k) \frac{\Delta T}{\Delta t} \\
    \nu_{\text{real}} &= \frac{Q_{1}}{A} 
    \label{eq:guetePraxis}
\end{align}



\subsection{Der Massendurchsatz}
Um die Masse an Gas herauszufinden, die pro Zeiteinheit 
