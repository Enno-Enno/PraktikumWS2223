\subsection[]{Die mechanische Leistung}
% Man errechne die mechanische Leistung des Kompressors, die dieser abgibt, wenn 
% er zwischen den Drücken pa und pb arbeitet, für die 4 Temperaturen aus 5c. Daten 
% für Cl2F2C: ρ0= 5,51 g/l bei T = 0°C und p = 1 Bar, κ = 1,14 
Um die mechanische Leistung $N_\text{mech}$ zu ermitteln, wird Gleichung \eqref{eq:mech_leistung} verwendet.
Hierfür sind in \cite[]{man:v206} folgende Größen angegeben: $\rho_\text{0} = \qty[]{5510}{\gram\per\cubic\meter}$ 
und $\kappa = \num[]{1.14}$ gegeben.
Nach \eqref{eq:rho_k} ergeben sich somit die Dichten $\rho_\text{k}$ in Tabelle \ref{tab:mech_leistung}.
Für die vier verschiedenen Zeitpunkte folgen des Weiteren die gesuchten Werte für $N_\text{mech}$ in der selben Tabelle.

\begin{table}
    \caption[]{Mechanische Leistungen zu den vier betrachteten Zeiten}
    \label{tab:mech_leistung}
    \sisetup{table-format = 1.1}
    \begin{tabular}{S[table-format = 2.0] S[table-format=5.1] S[table-format=2.1] @{${}\pm{}$} S}
        \toprule
        {$t /\unit[]{\min}$} & {$\rho_\text{k} / \unit[]{\gram\per\cubic\meter}$} & \multicolumn{2}{c}{$N_\text{mech} / (\unit[]{\W})$} \\
        \midrule
        3  & 22713.9 & -2.0 & 0.2 \\
        9  & 21134.7 & -3.0 & 0.4 \\
        15 & 17773.1 & -3.7 & 0.4 \\
        21 & 15862.9 & -3.8 & 0.4 \\ 
        \bottomrule 
    \end{tabular}
    \centering
\end{table}

%rho:  [24543.13300492611+/-0 20466.517083120856+/-0 22605.388632872502+/-0
% 22713.937918024356+/-0 23342.25142167844+/-0 23439.191036511504+/-0
% 22997.911442958848+/-0 21504.553580763193+/-0 21060.787126115094+/-0
% 21134.72353870458+/-0 20141.62599049128+/-0 20212.810390528357+/-0
% 19210.081191278143+/-0 18736.253779121467+/-0 18259.383050847457+/-0
% 17773.077294685987+/-0 17824.103534900412+/-0 17327.507825148405+/-0
% 16834.476456792352+/-0 16332.680412371132+/-0 16374.141432456934+/-0
% 15862.852444121389+/-0 15349.328719723184+/-0 15382.946523088152+/-0
% 15411.07405375754+/-0 15444.96316657504+/-0]

%mechanische leistung nach 3s:  -2.0+/-0.2
%mechanische leistung nach 9s:  -3.0+/-0.4
%mechanische leistung nach 15s: -3.7+/-0.4
%mechanische leistung nach 21s: -3.8+/-0.4