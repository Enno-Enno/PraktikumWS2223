\subsection{Die mechanische Leistung}
\label{sec:mech_Leistung}
Um die mechanische Leistung $N_\text{mech}$ zu ermitteln, wird Gleichung \eqref{eq:mech_leistung} verwendet.
Hierfür sind in Versuchsanleitung \cite[]{man:v206} folgende Größen angegeben: $\rho_\text{0} = \qty[]{5510}{\gram\per\cubic\meter}$ 
und $\kappa = \num[]{1.14}$ gegeben.
Nach \eqref{eq:rho_k} ergeben sich somit die Dichten $\rho_\text{k}$ in Tabelle \ref{tab:mech_leistung}.
Für die vier verschiedenen Zeitpunkte folgen des Weiteren die gesuchten Werte für $N_\text{mech}$ in der selben Tabelle.
Diese werden mit der in Abschnitt \ref{sec:berechnung_guete} berechneten Leistungsaufnahme $N$ des Kompressors verglichen 
und die prozentuale Abweichung dargestellt.

\begin{table}
    \caption[]{Druck, elektrische und mechanische Leistung, Leistungsabweichung.}
    \label{tab:mech_leistung}
    \sisetup{table-format = 1.1} %2.0 5.1 3.2 3.1 1.1 2.1 1.1
    \begin{tabular}{S[table-format = 2.0] S[table-format=5.1] S[table-format=3.2] S[table-format=3.1] @{${}\pm{}$} S S[table-format=2.1] @{${}\pm{}$} S}
        \toprule
        {$t /\unit[]{\min}$} & {$\rho_\text{k} / \frac{\unit{\gram}}{\unit{\cubic\meter}}$} & {$N / \unit[]{\watt}$} &
        \multicolumn{2}{c}{$N_\text{mech} / \unit[]{\W}$} & \multicolumn{2}{c}{$\lvert\frac{N_\text{mech}}{N}\rvert$ in \%}\\
        \midrule
        3  & 22713.9 & 120.00 & -13.3 & 0.8 & 11.1 & 0.6 \\
        9  & 21134.7 & 122.67 & -19.9 & 1.2 & 16.2 & 1.0 \\
        15 & 17773.1 & 122.93 & -25.3 & 1.7 & 20.6 & 1.4 \\
        21 & 15862.9 & 120.67 & -25.5 & 2.0 & 21.1 & 1.7 \\ 
        \bottomrule 
    \end{tabular}
    \centering
\end{table}
%# leisten_notaufnahme_0 :  120.00
%# leisten_notaufnahme_1 :  122.66666666666667
%# leisten_notaufnahme_2 :  122.93333333333334
%# leisten_notaufnahme_3 :  120.66666666666667

%#mechanische leistung nach  3s: -13.3+/-0.8
%#mechanische leistung nach  9s: -19.9+/-1.2
%#mechanische leistung nach 15s: -25.3+/-1.7
%#mechanische leistung nach 21s: -25.5+/-2.0

%#abweichung_lesitung_0 in % :  11.1+/-0.6
%#abweichung_lesitung_1 in % :  16.2+/-1.0
%#abweichung_lesitung_2 in % :  20.6+/-1.4
%#abweichung_lesitung_3 in % :  21.1+/-1.7