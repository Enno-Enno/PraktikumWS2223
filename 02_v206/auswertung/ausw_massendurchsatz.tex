\subsection[]{Der Massendurchsatz}
Um den Massendurchsatz zu bestimmen, werden die zuvor in Abschnitt \ref{sec: dif_quot} Differentialquotienten des kalten Reservoirs benötigt.
Des Weiteren muss die Verdampfungswärme $L$ des Transportmediums, hier: $\ce{Cl2F2C}$, für Formel \eqref{eq:massedurchsatz} ermittelt werden.


\subsubsection[]{Die Verdampfungswärme}
Zur Bestimmung der Verdampfungswärme $L$ werden die Wertepaare $\left(p_{\text{k}}, \frac{1}{T_{\text{k}}}\right)$ logarithmisch dargestellt
und anschließend eine lineare Ausgleichsrechnung zur Bestimmung von $L$ durchgeführt.
Dies baut auf der Formel 
\begin{align*}
    p = p_0 \exp{\left(-\frac{L}{RT}\right)}
\end{align*}
aus Versuch \cite[]{man:v203} auf.
$p_0$ ist dabei der Umgebungsdruck bei Zimmertemperatur und $R$ die Gaskonstante.
Sobald die Steigung $m$ der Ausgleichsgeraden ermittelt wurde, kann somit $L = -R \cdot m$ bestimmt werden.
Laut der Wetterstation \cite*[][]{wetterstation} betrug der Luftdruck am Tag des Experiments
$p_0 = \qty[]{1018.1}{\hecto\pascal} \simeq \qty[]{1.0}{\bar}$.
Mittels der Python \cite[]{python} Funktion \texttt{polyfit} aus dem Paket \texttt{numpy} \cite[]{numpy} wird die lineare Regression durchgeführt und
der Plot \ref{fig:ausgleichsgerade} erstellt.
Es ergibt sich die Steigung $m = \left(-2001.8141\pm 111.7750 \right)$ und ein Achsenabschnitt von $n = \left( 8.3639 \pm 0.3961 \right)$, 
wobei letzterer für die Rechnungen nicht relevant ist. 
Folglich gilt $L = $ %%%%%%%%%%%%%%%%%%%%%%%%%%%%%%%%%%%%%%%%%%%%%%%%%%%%%%%%%%%%%%%%%%%%%%%%%%%%%%%%%%
\begin{figure}
    \includegraphics[]{build/plot_verdampfungswaerme.pdf}
    \caption[]{lineare Ausgleichsgerade zur Bestimmung von $L$}
    \label{fig:ausgleichsgerade}
\end{figure}

%Steigung m= -2001.814130953005
%achsenabschn b=  8.363939543080686
%#Fehler:  [111.77504319   0.39608049]

