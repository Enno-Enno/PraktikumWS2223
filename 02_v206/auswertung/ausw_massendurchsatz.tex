\subsection[]{Der Massendurchsatz}
Um den Massendurchsatz zu bestimmen, werden die zuvor in Abschnitt \ref{sec: dif_quot} Differentialquotienten des kalten Reservoirs benötigt.
Des Weiteren muss die Verdampfungswärme $L$ des Transportmediums, hier: $\ce{Cl2F2C}$, für Formel \eqref{eq:massedurchsatz} ermittelt werden.


\subsubsection[]{Bestimmung der Verdampfungswärme}
Zur Bestimmung der Verdampfungswärme $L$ werden die Wertepaare $\left(p_{\text{k}}, \frac{1}{T_{\text{k}}}\right)$ logarithmisch dargestellt (vgl. \eqref{eq:Dampfdruckkurve})
und anschließend eine lineare Ausgleichsrechnung zur Bestimmung von $L$ durchgeführt.
Dies baut auf der Formel 
\begin{align*}
    p = p_0 \exp{\left(-\frac{L}{RT}\right)}
\end{align*}
aus Versuch \cite[]{man:v203} auf.
$p_0$ ist dabei der Umgebungsdruck bei Zimmertemperatur und $R$ die universelle Gaskonstante.
Sobald die Steigung $s$ der Ausgleichsgeraden ermittelt wurde, kann somit $L = -R \cdot s$ bestimmt werden.
Laut der Wetterstation \cite*[][]{wetterstation} betrug der Luftdruck am Tag des Experiments
$p_0 = \qty[]{1018.1}{\hecto\pascal} \simeq \qty[]{1.0}{\bar}$.
Mittels der Python \cite[]{python} Funktion \texttt{polyfit} aus dem Paket \texttt{numpy} \cite[]{numpy} wird die lineare Regression durchgeführt und
der Plot \ref{fig:ausgleichsgerade} erstellt.
Es ergibt sich die Steigung $s = \left(\num[]{-2001.8141}\pm \num[]{111.7750} \right)$ und ein Achsenabschnitt von 
$b = \left( \num[]{8.3639} \pm \num[]{0.3961} \right)$, 
wobei letzterer für die Rechnungen nicht relevant ist. 
Folglich gilt $L = \left(1.66 \pm 0.09 \right) 10^{4} \, \unit[]{\joule\per\mol}$. 
$\qty[]{1}{\mol}$ Wasser entspricht nach \cite[]{chemie_schule} einer Masse von $m = \qty[]{18.016}{\gram}$, sodass gilt 
$L = \left(920 \pm 50 \right) \, \unit[]{\joule\per\gram}$.
%L:  (1.66+/-0.09)e+04
%T in K, R in J/(mol*K) -> L in J/mol
%L in J/g:  (9.2+/-0.5)e+02
\begin{figure}
    \includegraphics[]{build/plot_verdampfungswaerme.pdf}
    \caption[]{lineare Ausgleichsgerade zur Bestimmung von $L$}
    \label{fig:ausgleichsgerade}
\end{figure}

%Steigung m= -2001.814130953005
%achsenabschn b=  8.363939543080686
%#Fehler:  [111.77504319   0.39608049]

\subsubsection[]{Berechnung des Massendurchsatzes}
Nach den Formeln \eqref{eq:massedurchsatz} und \eqref{eq:aenderung_q2} folgt somit für den Massendurchsatz
$\frac{\diff m}{\diff t} = \frac{1}{L} \frac{\diff Q_2}{\diff t}$, die Werte sind in Tabelle \ref{tab:Massendurchsatz} einzusehen.

\begin{table}
    \caption[]{Massendurchsatz zu den vier betrachteten Zeiten}
    \label{tab:Massendurchsatz}
    \sisetup{table-format = 3.1}
    \begin{tabular}{S[table-format = 2.0] S @{${}\pm{}$} S[table-format =1.1]}
        \toprule
        {$t /\unit[]{\min}$} & \multicolumn{2}{c}{$\frac{\diff m}{\diff t} / (\unit[]{\gram\per\min})$} \\
        \midrule
        3  & -16.8 & 2.0\\
        9  & -14.7 & 1.9\\
        15 & -11.5 & 1.4\\
        21 & -9.0  & 1.0\\ 
        \bottomrule 
    \end{tabular}
    \centering
\end{table}

%massendurchsatz 0:  -16.8+/-2.0
%massendurchsatz 1:  -14.7+/-1.9
%massendurchsatz 2:  -11.5+/-1.4
%massendurchsatz 3:  -9.0 +/-1.0