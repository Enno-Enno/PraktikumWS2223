\subsection{Der Massendurchsatz}
Um den Massendurchsatz zu bestimmen, werden die zuvor in Abschnitt \ref{sec: dif_quot}
berechneten Differentialquotienten des kalten Reservoirs benötigt.
Des Weiteren muss die Verdampfungswärme $L$ des Transportmediums, hier $\ce{Cl2F2C}$, für Formel \eqref{eq:massedurchsatz} ermittelt werden.


\subsubsection{Bestimmung der Verdampfungswärme}
Zur Bestimmung der Verdampfungswärme $L$ werden die Wertepaare $\left(\frac{1}{T_{\text{k}}}, \ln{\left(\frac{p_{\text{k}}}{p_0}\right)}\right)$ logarithmisch dargestellt (vgl. \eqref{eq:Dampfdruckkurve})
und anschließend eine lineare Ausgleichsrechnung zur Bestimmung von $L$ durchgeführt.
Dies baut auf der Formel 
\begin{align*}
    p = p_0 \exp{\left(-\frac{L}{RT}\right)}
\end{align*}
aus Versuch \cite[]{man:v203} auf.
Der Umgebungsdruck bei Zimmertemperatur ist dabei $p_0$ und $R$ die universelle Gaskonstante.
Sobald die Steigung $s$ der Ausgleichsgeraden ermittelt wurde, kann somit $L = -R \cdot s$ bestimmt werden.
Laut der Wetterstation \cite*[][]{wetterstation} betrug der Luftdruck am Tag des Experiments
$p_0 = \qty[]{1018.1}{\hecto\pascal} \simeq \qty[]{1.0}{\bar}$.
Mittels der Python \cite[]{python} Funktion \texttt{polyfit} aus dem Paket \texttt{numpy} \cite[]{numpy} wird die lineare Regression durchgeführt und
der Plot \ref{fig:ausgleichsgerade} erstellt.
Es ergibt sich die Steigung $s = \left(\num[]{-2001.81}\pm \num[]{111.78} \right)$ und ein Achsenabschnitt von 
$b = \left( \num[]{8.36} \pm \num[]{0.40} \right)$, 
wobei letzterer für die Rechnungen nicht relevant ist. 
Folglich gilt $L = \left(\num[]{1.66 \pm 0.09} \right) \cdot 10^{4} \, \unit[]{\joule\per\mol}$. 
%$\qty[]{1}{\mol}$ Wasser entspricht nach \cite[]{chemie_schule} einer Masse von $m = \qty[]{18.016}{\gram}$, sodass gilt 
%$L = \left(920 \pm 50 \right) \, \unit[]{\joule\per\gram}$.
Die molare Masse des Transportmediums beträgt nach \cite[]{gestis} \qty[]{120.91}{\gram\per\mol}, sodass gilt
$L = \left(\num[]{138 \pm 8} \right) \, \unit[]{\joule\per\gram}$.
%L in J/mol:  (1.66+/-0.09)e+04
%L in J/g:  138+/-8

\begin{figure}[H]
    \includegraphics[]{build/plot_verdampfungswaerme.pdf}
    \caption[]{Lineare Ausgleichsgerade zur Bestimmung von $L$.}
    \label{fig:ausgleichsgerade}
\end{figure}

%Steigung m= -2001.814130953005
%achsenabschn b=  8.363939543080686
%#Fehler:  [111.77504319   0.39608049]

\subsubsection{Berechnung des Massendurchsatzes}
Nach den Formeln \eqref{eq:massedurchsatz} und \eqref{eq:aenderung_q2} folgt somit für den Massendurchsatz
$\frac{\diff m}{\diff t} = \frac{1}{L} \frac{\diff Q_2}{\diff t}$, die Werte sind in Tabelle \ref{tab:Massendurchsatz} einzusehen.

\begin{table}
    \caption[]{Massendurchsatz zu den vier betrachteten Zeiten.}
    \label{tab:Massendurchsatz}
    \sisetup{table-format = 3.0}
    \begin{tabular}{S[table-format = 2.0] S @{${}\pm{}$} S[table-format =1.0]}
        \toprule
        {$t /\unit[]{\min}$} & \multicolumn{2}{c}{$\frac{\diff m}{\diff t} / \frac{\unit{\gram}}{\unit{\min}}$} \\
        \midrule
        3  & -113 & 7 \\
        9  & - 96 & 6 \\
        15 & - 79 & 5 \\
        21 & - 61 & 5 \\ 
        \bottomrule 
    \end{tabular}
    \centering
\end{table}

% Massenfurchtsatz in g/min 0:  -113+/-7
% Massenfurchtsatz in g/min 1:  - 96+/-6
% Massenfurchtsatz in g/min 2:  - 79+/-5
% Massenfurchtsatz in g/min 3:  - 61+/-5
