\subsection{Die Güteziffer}
Um die Güteziffer $\nu$ zu bestimmen, muss gemäß \eqref{eq:guetePraxis} zunächst
die zeitliche Änderung des Temperaturverlaufs ermittelt werden.
Dies geschieht, indem die die Messdaten der Temperaturen beider Reservoire aus Tabelle \ref{tab:messdaten} geplottet werden.
Anschließend wird mit Hilfe einer nicht-linearen Ausgleichsrechnung eine Funktion $T \left(t\right)$ ermittelt, die den Verlauf 
für das jeweilige Reservoir beschreiben soll.
Diese Funktion wird nach der Zeit differenziert, sodass $\nu$ für je vier unterschiedliche Temperaturen bestimmt werden kann.
Für die vier Berechnungen der Güteziffer werden Tabelle \ref{tab:messdaten} jeweils die Temperaturen $T_{\text{k}}$ und $T_{\text{w}}$ nach 
$t_1 = \qty[]{3}{\s}$, $t_2 = \qty[]{9}{\s}$, $t_3 = \qty[]{15}{\s}$ und $t_4 = \qty[]{21}{\s}$ entnommen.


\subsubsection{Plot der Temperaturverläufe}
Werden die Temperaturverläufe der beiden Reservoire in Abhängigkeit von der Zeit geplottet, ergibt sich Abbildung \ref{fig:temperaturverlauf}.

\begin{figure}[H]
    \includegraphics[]{build/plot_temp_verlauf.pdf}
    \caption[]{Zeitabhängige Temperaturverläufe beider Reservoire.}
    \label{fig:temperaturverlauf}
\end{figure}


\subsubsection{Ermittlung der Ausgleichsfunktionen}
Um die Ausgleichskurve zu approximieren, sind folgende Funktionen möglich:
\begin{align}
    T \left(t\right) &= A t^2 + B t + C \label{eq:ausgleichsfunktion_1}, \\
    T \left(t\right) &= \frac{A}{1 + B t^{\alpha}} \label{eq:ausgleichsfunktion_2}, \\
    T \left(t\right) &= \frac{A t^{\alpha}}{1 + B t^{\alpha}} + C \label{eq:ausgleichsfunktion_3}.
\end{align}
Hierbei sind $A$, $B$, $C$ und $\alpha$ die zu bestimmenden Parameter, wobei $1 \leq \alpha \leq 2$ gelten soll.
Mit Hilfe der Python \cite[]{python} Funktion \texttt{curve\_fit} aus dem Paket \texttt{scipy.optimize} 
\cite[]{scipy} lassen sich die nicht-linearen Ausgleichsrechnungen durchführen.
\begin{figure}[H]
    \includegraphics[]{build/plot_ausgleich_1.pdf}
    \caption[]{Ausgleichskurve zu Funktion \eqref{eq:ausgleichsfunktion_1}.}
    \label{fig:ausgleichsplot_1}
\end{figure}
\begin{figure}[H]
    \includegraphics[]{build/plot_ausgleich_2.pdf}
    \caption[]{Ausgleichskurve zu Funktion \eqref{eq:ausgleichsfunktion_2}.}
    \label{fig:ausgleichsplot_2}
\end{figure}
\begin{figure}[H]
    \includegraphics[]{build/plot_ausgleich_3.pdf}
    \caption[]{Ausgleichskurve zu Funktion \eqref{eq:ausgleichsfunktion_3}.}
    \label{fig:ausgleichsplot_3}
\end{figure}
\noindent
Werden die Plots \ref{fig:ausgleichsplot_1}, \ref{fig:ausgleichsplot_2} und \ref{fig:ausgleichsplot_3} mit einander verglichen,
ist unmittelbar zu erkennen, dass die Approximation \eqref{eq:ausgleichsfunktion_2} auch nach einstellen der Startparameter am ungenausten ist.
\eqref{eq:ausgleichsfunktion_1} und \eqref{eq:ausgleichsfunktion_3} erscheinen zunächst in etwa den gleichen Exaktheitsgrad zu haben.
Im Folgenden wird mit \eqref{eq:ausgleichsfunktion_1} weiter gerechnet.
Die entsprechenden Parameter betragen dann
\begin{align*}
    A_{\text{k}} &= \left(\num[]{  0.0150} \pm \num[]{0.0007}\right) \unit[]{\kelvin\per\min\squared}, & 
    A_{\text{w}} &= \left(\num[]{ -0.0215} \pm \num[]{0.0006}\right) \unit[]{\kelvin\per\min\squared}, \\
    B_{\text{k}} &= \left(\num[]{ -1.2606} \pm \num[]{0.0191}\right) \unit[]{\kelvin\per\min}, &  
    B_{\text{w}} &= \left(\num[]{  1.6268} \pm \num[]{0.0161}\right) \unit[]{\kelvin\per\min}, \\             
    C_{\text{k}} &= \left(\num[]{295.0128} \pm \num[]{0.1030}\right) \unit[]{\kelvin}, &                  
    C_{\text{w}} &= \left(\num[]{293.5784} \pm \num[]{0.0867}\right) \unit[]{\kelvin},                  
\end{align*}
wobei die Fehler der Kovarianzmatrix entnommen wurden.

% a_k = ufloat(  0.01497863, 0.00073754)
% b_k = ufloat( -1.26058547, 0.01908795)
% c_k = ufloat(295.01282051, 0.1030449 )
% 
% a_w = ufloat( -0.02154457 , 0.00062077)
% b_w = ufloat(  1.62675092 , 0.01606575)
% c_w = ufloat(293.57844933 , 0.08672976)


\subsubsection{Bestimmung der Differentialquotienten}
\label{sec: dif_quot}
Wird \eqref{eq:ausgleichsfunktion_1} nach der Zeit differenziert, ergibt sich mittels Quotientenregel
%\begin{align*}
%    \frac{\symup{d}T}{\symup{d}t} = \frac{\alpha A t^{\alpha -1}}{\left(1 + B t^{\alpha}\right)^2}.
%\end{align*}
\begin{align*}
    \frac{\symup{d}T}{\symup{d}t} = 2 A t + B.
\end{align*}
%
Werden für beide Reservoire jeweils die obigen Parameter und entsprechenden Zeiten eingesetzt, ergibt sich 
\begin{align*}
    \frac{\symup{d}}{\symup{d}t} T_\text{k} \left(t_1\right) &= \left(\num[]{-1.171} \pm \num[]{0.020} \right)\, \unit[]{\kelvin\per\min}  &
    \frac{\symup{d}}{\symup{d}t} T_\text{w} \left(t_1\right) &= \left(\num[]{ 1.497} \pm \num[]{0.016} \right)\, \unit[]{\kelvin\per\min}   \\
    \frac{\symup{d}}{\symup{d}t} T_\text{k} \left(t_2\right) &= \left(\num[]{-0.991} \pm \num[]{0.023} \right)\, \unit[]{\kelvin\per\min}  &
    \frac{\symup{d}}{\symup{d}t} T_\text{w} \left(t_2\right) &= \left(\num[]{ 1.239} \pm \num[]{0.020} \right)\, \unit[]{\kelvin\per\min} \\
    \frac{\symup{d}}{\symup{d}t} T_\text{k} \left(t_3\right) &= \left(\num[]{-0.811} \pm \num[]{0.029} \right)\, \unit[]{\kelvin\per\min} &
    \frac{\symup{d}}{\symup{d}t} T_\text{w} \left(t_3\right) &= \left(\num[]{ 0.980} \pm \num[]{0.025} \right)\, \unit[]{\kelvin\per\min} \\
    \frac{\symup{d}}{\symup{d}t} T_\text{k} \left(t_4\right) &= \left(\num[]{-0.630} \pm \num[]{0.040} \right)\, \unit[]{\kelvin\per\min}  &
    \frac{\symup{d}}{\symup{d}t} T_\text{w} \left(t_4\right) &= \left(\num[]{ 0.722} \pm \num[]{0.031} \right)\, \unit[]{\kelvin\per\min}.
\end{align*}
%
% #diff_quot k0:  -1.171 +/-0.020
% #diff_quot w0:   1.497 +/-0.016
% #diff_quot k1:  -0.991 +/-0.023
% #diff_quot w1:   1.239 +/-0.020
% #diff_quot k2:  -0.811 +/-0.029
% #diff_quot w2:   0.980 +/-0.025
% #diff_quot k3:  -0.630 +/-0.040
% #diff_quot w3:   0.722 +/-0.031

\subsubsection{Berechnung der Güteziffer}
Gemäß Formel \eqref{eq:gueteTheorie} ergeben sich folgende ideale Güteziffern $\nu_\text{ideal}$ in Tabelle \ref{tab:gueteziffern}.
%
Mit Hilfe der spezifischen Wärmekapazität von Wasser $c_{\text{H2O}} = \qty{4190}{\joule\per\kg\per\kelvin}$ \cite[]{leifi}, der gegebenen Wärmekapazität
der Kupferspirale $m_{\text{C}} c_{\text{C}} = \qty{750}{\joule\per\kelvin}$ und der zuvor bestimmten Differentialquotienten
kann bei einer Wassermenge von $\qty[]{3}{\liter}$ pro Behälter (entspricht $m_\text{H2O}=\qty[]{3}{kg}$) die reale Güteziffer 
$\nu_{\text{real}}$ gemäß \eqref{eq:guetePraxis} bestimmt werden.
Hierfür wird die gemittelte Leistungsaufnahme des Kompressors
\begin{align*}
    N = \frac{1}{n} \sum_{m=1}^{n} P_m
\end{align*}
benötigt.
Somit ergibt sich
\begin{align*}
    N_1  &=  \qty[]{7200}{\J\per\min} &
    N_2  &=  \qty[]{7360}{\J\per\min} &
    N_3  &=  \qty[]{7376}{\J\per\min} &
    N_4  &=  \qty[]{7240}{\J\per\min}
\end{align*}
sodass sich schließlich  $\nu_\text{real}$ in Tabelle \ref{tab:gueteziffern} ergibt.
%Große Diskrepanzen zwischen idealen und realen Werten sind zu erkennen,
%die rechnerisch als Quotient der beiden Werte in der Tabelle angegeben sind.

\begin{table}
    \caption[]{Ideale und reale Güteziffern mit ihrer Abweichung.}
    \label{tab:gueteziffern}
    \sisetup{table-format = 1.2}
    \begin{tabular}{S[table-format = 2.0] S[table-format = 2.2] S @{${}\pm{}$} S S[table-format = 2.1] @{${}\pm{}$} S[table-format = 2.1]}
        \toprule
        {$t /\unit[]{\min}$} & {$\nu_\text{ideal}$} & \multicolumn{2}{c}{$\nu_\text{real}$} & \multicolumn{2}{c}{$\frac{\nu_\text{real}}{\nu_\text{ideal}} \text{ in \%}$} \\
        \midrule
        3  & 45.85 & 2.77 & 0.03 &  6.0 & 0.1 \\
        9  & 14.19 & 2.24 & 0.04 & 15.8 & 0.3 \\
        15 &  9.29 & 1.77 & 0.04 & 19.1 & 0.5 \\
        21 &  7.40 & 1.33 & 0.06 & 18.0 & 0.8 \\ 
        \bottomrule 
    \end{tabular}
    \centering
\end{table}


%#Abweichungen Nu:
%#Abw 0:   6.0 +/-0.1
%#Abw 1:  15.8 +/-0.3
%#Abw 2:  19.1 +/-0.5
%#Abw 3:  18.0 +/-0.8
