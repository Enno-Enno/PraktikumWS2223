\section{Durchführung}

Eine Wärmepumpe wie in \ref{sec:funktionsweise} beschrieben wird aufgestellt.
Die Beiden Reservoirs werden mit jeweils 3 Litern Wasser befüllt.
Der Druck in beiden Teilen des Gaskreislaufs wird mithilfe von Barometern gemessen.
Die Temperatur in beiden Reservoirs wird mit digitalen Thermometern gemessen.
Die Stromversorgung des Kompressors der Wärmepumpe läuft über ein Wattmeter um die elektrische Leistung herauszufinden.
Die Werte bevor die Wärmepumpe angeschaltet wird werden in der Tabelle bei Minute 0 eingetragen.
Schließlich wird mit dem Handy die Zeit gestoppt.
Jede volle Minute werden die beiden Temperaturen, die beiden Gasdrücke und die elektrische Leistung abgelesen und notiert.
Hierbei werden die Geräte immer in der gleichen Reihenfolge abgelesen, 
damit die Zeitabstände zwischen den Messungen konsistent bleiben.