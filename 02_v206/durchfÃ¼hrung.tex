\section{Durchführung}

Eine Wärmepumpe wie in \ref{sec:funktionsweise} beschrieben wird aufgestellt.
Die beiden Reservoirs werden mit jeweils 3 Litern Wasser befüllt.
Der Druck in beiden Teilen des Gaskreislaufs wird mithilfe von Barometern gemessen.
Im Folgenden steht $p_\text{k}$ für den Druck im kalten Reservoir und $p_\text{w}$ für den Druck im warmen Reservoir.
Dabei werden für $p_\text{k}$ und $p_\text{w}$ zu den eigentlichen Messdaten jeweils $\qty[]{1}{bar}$ hinzu addiert,
was in der Anleitung \cite[]{man:v206} vorgegeben wird und mit dem Versuchsaufbau zu erklären ist. 
Die Temperatur in beiden Reservoirs wird mit digitalen Thermometern gemessen,
wobei $T_\text{k}$ die Temperatur im kalten und $T_\text{w}$ die Temperatur im warmen Reservoir beschreibt.
Die Stromversorgung des Kompressors der Wärmepumpe läuft über ein Wattmeter um die elektrische Leistung $P$ herauszufinden.
Die Werte bevor die Wärmepumpe angeschaltet wird werden in der Tabelle bei Minute 0 eingetragen.
Schließlich wird mit dem Handy die Zeit $t$ gestoppt.
Jede volle Minute werden die beiden Temperaturen, die beiden Gasdrücke und die elektrische Leistung abgelesen und notiert.
Hierbei werden die Geräte immer in der gleichen Reihenfolge abgelesen, 
damit die Zeitabstände zwischen den Messungen konsistent bleiben.