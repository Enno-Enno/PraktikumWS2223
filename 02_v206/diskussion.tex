\section{Diskussion}

Anhand der Plots \ref{fig:ausgleichsplot_1}, \ref{fig:ausgleichsplot_2} und \ref{fig:ausgleichsplot_3}
kann \eqref{eq:ausgleichsfunktion_2} aufgrund der relativ Abweichungen ausgeschlossen werden.
Zwischen \eqref{eq:ausgleichsfunktion_1} und \eqref{eq:ausgleichsfunktion_3} bestehen allerdings nur kleine Unterschiede
im betrachteten Intervall,
wodurch es ebenfalls möglich ist, dass \eqref{eq:ausgleichsfunktion_1} die bessere Näherung ist. 

Die realen Güteziffern sind in etwa um den Faktor 10 kleiner als die idealen (siehe \ref{tab:gueteziffern}).
Das hängt mit verschiedenen Schritten zusammen, an denen Energie verloren geht.
Zunächst wird die elektrische Energie, die von dem Kompressor verwendet wird
nicht verlustfrei in mechanische Energie umgewandelt.
Die Rohre und Behälter der Wärmepumpe sind zwar isoliert, 
Wärmeenergie geht trotzdem nach außen verloren.
In \ref{sec:mech_Leistung} wurde die mechanische Leistung anhand der gemessenen Wirkung im Reservoir k   ermittelt.
Die dabei ermittelte Mechanische Leistung war um einen Faktor 100 kleiner als die elektrische Leistung.
Auch hier gibt es Wirkungsverluste in allen Zwischenschritten, die eine Erklärung sein können.

Bei der Mechanischen Leistung in \ref{sec:mech_Leistung} ist ein negativer Wert herausgekommen. 
Das hängt mit der Perspektive zusammen aus dem man den Massendurchsatz bzw. die Leistung betrachtet.
Wenn man sie von dem k-Teil der Wärmepumpe betrachtet dann wird diesem Gas, Druck und damit Energie entzogen,
eine negative Leistung ist sinnvoll.
Für den Vergleich mit der elektrischen Leistung der Pumpe ist allerdings der Betrag der Leistung sinnvoller.