\section{Diskussion}
Anhand der Plots \ref{fig:ausgleichsplot_1}, \ref{fig:ausgleichsplot_2} und \ref{fig:ausgleichsplot_3}
kann \eqref{eq:ausgleichsfunktion_2} ausgeschlossen werden. 
Zwischen \eqref{eq:ausgleichsfunktion_1} und \eqref{eq:ausgleichsfunktion_3} bestehen im betrachteten Intervall nur kleine Abweichungen,
jedoch ist \eqref{eq:ausgleichsfunktion_3} etwas exakter.
%Die realen Güteziffern sind in etwa um den Faktor 10 kleiner als die idealen (siehe \ref{tab:gueteziffern}).
Tabelle \ref{tab:gueteziffern} ist zu entnehmen, dass reale und ideale Güteziffern prozentuale Abweichungen zwischen etwa
6 und 20 \% haben.
Es liegen also große Diskrepanzen vor.
Das hängt mit verschiedenen Schritten zusammen, an denen Energie verloren geht.
Zunächst wird die elektrische Energie, die von dem Kompressor verwendet wird
nicht verlustfrei in mechanische Energie umgewandelt.
Die Rohre und Behälter der Wärmepumpe sind zwar isoliert, 
Wärmeenergie geht trotzdem nach außen verloren.
In \ref{sec:mech_Leistung} wurde die mechanische Leistung anhand der gemessenen Wirkung im Reservoir k   ermittelt.
Die dabei ermittelte Mechanische Leistung war um einen Faktor 100 kleiner als die elektrische Leistung.
Auch hier gibt es Wirkungsverluste in allen Zwischenschritten, die eine Erklärung sein können.

Außerdem können die statistischen Messfehler noch größer sein als bisher ermittelt.
An den Geräten gab es oft keine Angaben zum typischen Messfehler,
weshalb diese nicht in die Abschätzung des Messfehlers eingehen konnten.
Eines der Manometer hatte eine um $\qty{1}{\bar}$ verschobene Messanzeige.
Diese Verschiebung wurde in den Rechnungen berücksichtigt.
Ob es weitere systematische Messfehler gibt kann ohne weiteres nicht gesagt werden.

Bei der Mechanischen Leistung in \ref{sec:mech_Leistung} ist ein negativer Wert herausgekommen. 
Das hängt mit der Perspektive zusammen aus der der Massendurchsatz bzw. die Leistung betrachtet wird.
Wird sie von dem k-Teil der Wärmepumpe betrachtet, dann wird diesem Gas, Druck und damit Energie entzogen,
eine negative Leistung ist sinnvoll.
Für den Vergleich mit der elektrischen Leistung der Pumpe ist allerdings der Betrag der Leistung sinnvoller.