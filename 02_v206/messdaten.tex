\section{Messdaten}
Die gemessenen Daten sind in Tabelle \ref{tab:messdaten} einzusehen. 
$t$ steht dabei für die Zeit, $T_\text{k}$ und $p_\text{k}$ geben jeweils die Temperatur und den Druck im kalten Reservoir an.
Analog geben $T_\text{k}$ und $p_\text{k}$ Temperatur und Druck des warmen Reservoirs an.
Dabei wurden für $p_\text{k}$ und $p_\text{w}$ jeweils zu den eigentlichen Messdaten $\qty[]{1}{bar}$ hinzu addiert,
was in der Anleitung \cite[]{man:v206} vorgegeben wurde und mit Versuchsaufbau zu erklären ist. 
$P$ gibt die elektrische Leistung an.

\begin{table}
    \centering
    \sisetup{table-format=2.1}
    \begin{tabular}[]{S[table-format=2.0] S S[table-format=1.1] S S S[table-format=3.0]}
        \toprule
        {$t / \unit[]{\minute}$} & {$T_\text{k} / \unit[]{\degreeCelsius}$} & {$p_\text{k} / \unit[]{\bar}$} & {$T_\text{w} / \unit[]{\degreeCelsius}$} & {$p_\text{w} / \unit[]{\bar}$} & {$P / \unit[]{\watt}$} \\
        \midrule
        0 & 21.2  &  4.8 & 20.9  &   4.2  &  0   \\  
        1 & 21.0  &  4.0 & 21.9  &   6.0  &  115 \\
        2 & 19.8  &  4.4 & 23.6  &   6.4  &  120 \\
        3 & 18.4  &  4.4 & 24.9  &   6.6  &  125 \\
        4 & 17.0  &  4.5 & 26.4  &   7.0  &  125 \\
        5 & 15.8  &  4.5 & 27.8  &   7.1  &  125 \\
        6 & 14.8  &  4.4 & 29.4  &   7.5  &  125 \\
        7 & 13.8  &  4.1 & 30.7  &   7.9  &  123 \\
        8 & 12.7  &  4.0 & 32.0  &   8.0  &  123 \\
        9 & 11.7  &  4.0 & 33.3  &   8.2  &  123 \\
       10 & 10.8  &  3.8 & 34.6  &   8.5  &  125 \\
       11 &  9.8  &  3.8 & 35.8  &   8.9  &  125 \\
       12 &  8.9  &  3.6 & 37.0  &   9.1  &  125 \\
       13 &  8.0  &  3.5 & 38.2  &   9.3  &  125 \\
       14 &  7.1  &  3.4 & 39.2  &   9.6  &  125 \\
       15 &  6.3  &  3.3 & 40.0  &  10.0  &  115 \\
       16 &  5.5  &  3.3 & 40.9  &  10.0  &  115 \\
       17 &  4.8  &  3.2 & 41.8  &  10.3  &  115 \\
       18 &  4.0  &  3.1 & 42.7  &  10.5  &  115 \\
       19 &  3.3  &  3.0 & 43.5  &  10.8  &  115 \\
       20 &  2.6  &  3.0 & 44.2  &  11.0  &  115 \\
       21 &  2.0  &  2.9 & 45.0  &  12.1  &  115 \\
       22 &  1.4  &  2.8 & 45.8  &  11.4  &  113 \\
       23 &  0.8  &  2.8 & 46.4  &  11.5  &  115 \\
       24 &  0.3  &  2.8 & 47.1  &  11.8  &  113 \\
       25 & -0.3  &  2.8 & 47.7  &  12.0  &  112 \\
    \bottomrule
    \end{tabular}
    \caption[]{Die erfassten Messdaten des Versuchs}
    \label{tab:messdaten}
\end{table}