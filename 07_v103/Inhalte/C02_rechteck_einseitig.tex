\subsection{Rechteckiger Querschnitt, einseitige Einspannung}

\begin{table}[H]
    \centering
    \caption{Messdaten des Stabes mit rechteckigem Querschnitt bei einseitiger Einspannung.}
    \label{tab:messdaten_rechteck_einseitig}
    \sisetup{table-format = 3.0}
    \begin{tabular}[]{S[table-format = 2.0] S S S}
        \toprule
        {$x / \unit{\centi\meter}$} & {$D_0 / (\qty{0.01}{\milli\meter})$} & {$D_\text{m} / (\qty{0.01}{\milli\meter})$} & {$D / (\qty{0.01}{\milli\meter})$} \\
        \midrule
        49 & 189 &  87 & 102 \\
        47 & 207 & 103 & 104 \\
        45 & 212 & 131 &  81 \\
        43 & 235 & 149 &  86 \\
        41 & 251 & 170 &  81 \\
        39 & 268 & 190 &  78 \\
        37 & 282 & 210 &  72 \\
        35 & 295 & 233 &  62 \\
        33 & 306 & 251 &  55 \\
        31 & 317 & 265 &  52 \\
        29 & 329 & 283 &  46 \\
        27 & 341 & 300 &  41 \\
        25 & 354 & 317 &  37 \\
        23 & 369 & 338 &  31 \\
        21 & 381 & 355 &  26 \\
        19 & 391 & 368 &  23 \\
        17 & 404 & 386 &  18 \\
        15 & 414 & 400 &  14 \\
        13 & 425 & 413 &  12 \\
        11 & 434 & 426 &   8 \\
         9 & 443 & 437 &   6 \\
         7 & 449 & 445 &   4 \\
         5 & 451 & 450 &   1 \\
         3 & 457 & 456 &   1 \\
        \bottomrule
    \end{tabular}
\end{table}
