\subsection{Rechteckiger Querschnitt, einseitige Einspannung}
\label{sec:rechteck_einseitig}
Das Probegewicht hängt bei einer Länge von $L = \qty{51.5}{\cm}$.
Die Gesamtmasse des angehängten Gewichts bei einseitiger Einspannung ist die Summe aus kleinerer Masse und Halterung in Tabelle
\ref{tab:massen}.
Somit ergibt sich $m_\text{eins} = (\num{249.70} \pm \num{0.14}) \, \unit{\gram}$.
Die Durchbiegung $D$ in Abhängigkeit der Position $x$ ist in Tabelle \ref{tab:messdaten_rechteck_einseitig} einzusehen.
Dabei ist $D_0$ die Auslenkung ohne angehängte Masse und $D_\text{m}$ die Auslenkung mit Belasungsgewicht.
$D$ errechnet sich somit aus der Differenz beider Größen.

\begin{table}[H]
    \centering
    \caption{Messdaten des Stabes mit rechteckigem Querschnitt bei einseitiger Einspannung.}
    \label{tab:messdaten_rechteck_einseitig}
    \sisetup{table-format = 3.0}
    \begin{tabular}[]{S[table-format = 2.0] S S S}
        \toprule
        {$x / \unit{\centi\meter}$} & {$D_0 / (\qty{0.01}{\milli\meter})$} & {$D_\text{m} / (\qty{0.01}{\milli\meter})$} & {$D / (\qty{0.01}{\milli\meter})$} \\
        \midrule
        49 & 189 &  87 & 102 \\
        47 & 207 & 103 & 104 \\
        45 & 212 & 131 &  81 \\
        43 & 235 & 149 &  86 \\
        41 & 251 & 170 &  81 \\
        39 & 268 & 190 &  78 \\
        37 & 282 & 210 &  72 \\
        35 & 295 & 233 &  62 \\
        33 & 306 & 251 &  55 \\
        31 & 317 & 265 &  52 \\
        29 & 329 & 283 &  46 \\
        27 & 341 & 300 &  41 \\
        25 & 354 & 317 &  37 \\
        23 & 369 & 338 &  31 \\
        21 & 381 & 355 &  26 \\
        19 & 391 & 368 &  23 \\
        17 & 404 & 386 &  18 \\
        15 & 414 & 400 &  14 \\
        13 & 425 & 413 &  12 \\
        11 & 434 & 426 &   8 \\
         9 & 443 & 437 &   6 \\
         7 & 449 & 445 &   4 \\
         5 & 451 & 450 &   1 \\
         3 & 457 & 456 &   1 \\
        \bottomrule
    \end{tabular}
\end{table}

\noindent
Zur Bestimmung des Elastizitätsmoduls $E$ wird gemäß Gleichung \textbf{REF!!!} die Durchbiegung $D$ in Abhängigkeit von 
\begin{align*}
    L x^2 - \frac{x^3}{3}
\end{align*}
geplottet.
Außerdem wird eine linearen Ausgleichsrechnung mit Hilfe der Python \cite[]{python} Funktion \texttt{curve\_fit} aus dem Paket
\texttt{scipy.optimize} \cite[]{scipy} durchgeführt, mit dem die Steigung $s$ der Funktion 
\begin{align}
    D = s \left(L x^2 - \frac{x^3}{3}\right) + b
    \label{eq:ausgleichsfunktion_einseitig}
\end{align}
bestimmt wird.
Hieraus werden die Werte 
\begin{align*}
    s &= (\num{0.0126} \pm \num{0.0002}) \, \frac{1}{\unit{\meter^2}}, & b &= (\num{0.000019} \pm \num{0.000011}) \, \unit{\meter}
\end{align*}
ermittelt.
% parameter:
% s = 0.01225490 1/m^2
% b = 0.00001901 m
% Fehler:  [2.49712761e-04 1.07332779e-05]
Der Plot der Messdaten und der Ausgleichsfunktion ist in Abbildung \ref{fig:plot_rechteck_einseitig} zu sehen.
Gemäß Gleichung \textbf{REF!!!} errechnet sich das Flächenträgheitsmoment des Stabes zu 
$I = (\num{8.33} \pm \num{0.08}) \cdot 10^{-6} \, \unit{\meter^4}$.
%(8.33+/-0.08)e-06 m^4
Mit der Gewichtskraft $F = m_\text{eins} \, g$ folgt somit für den Elastizitätsmodul gemäß Gleichung \textbf{REF!!!}
\begin{align}
    E_\text{Recht} = \frac{F}{2 I s} = (\num{119.9} \pm \num{2.4}) \cdot 10^9 \, \unit{\newton\per\meter^2}.
\end{align}
%elastizitaet:  (1.199+/-0.024)e+11 N/m^2 N/m^2

\begin{figure}[H]
    \centering
    \includegraphics[]{build/C02_rechteck_einseitig.pdf}
    \caption{Ausgleichsgerade \eqref{eq:ausgleichsfunktion_einseitig} zu den Messdaten aus Tabelle \ref{tab:messdaten_rechteck_einseitig}.}
    \label{fig:plot_rechteck_einseitig}
\end{figure}