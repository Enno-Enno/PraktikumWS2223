%bei beidseitiger Auflegung: Fehler dadurch dass Masse (30cm) nicht genau bei Hälfte der Stäbe(56cm)

\section{Diskussion}
Die Ergebnisse für die Elastizitätsmodule in den  verschiedenen Konfigurationen
unterscheiden sich deutlich voneinander. Die berechneten Messunsicherheiten sind
um ein mehrfaches kleiner als die tatsächlichen Unterschiede.
Der einfachste Grund dafür ist, dass die Stangen bereits plastisch verformt sind,
was deren Biegeeigenschaften verändert.
In den Messungen konnten große Unterschiede über den verlauf der Stange festgestellt werden,
die für diese Einschätzung sprechen.
Die Messungen konnten dazu in den linearen Fits unterschiedlich gut abgebildet werden.
Bei den Messreihen für die einseitige Einspannung ging dies deutlich besser als in
den Messreihen für die Beidseitige Einspannung.
Einer der Gründe kann es sein dass bei der Beidseitigen Variante die Masse nicht
genau in der Mitte plaziert wurde, wie es bei der Formel \eqref{eq:D_x_beidseitig_links} angenommen war.
Die einzelnen Messungen sind auch von starken Schwankungen betroffen.
Bei der Ausführung konnte ein leichtes anstoßen an den Tisch die einstellung der Uhr verändern.

Das tatsächliche Elastizitätsmodul des Materials liegt also vermutlich irgendwo 
zwischen \qty{120 e9}{\newton\per\square\meter} und \qty{260 e9}{\newton\per\square\meter}.
Da die kleineren Werte für E aus besseren Messungen stammen ist es vermutlich näher an dem kleineren Wert.