\subsection{Kreisförmiger Querschnitt, beidseitige Auflegung}

\begin{table}[H]
    \centering
    \caption{Messdaten des Stabes mit runder Querschnitt bei beidseitiger Auflegung.}
    \label{tab:messdaten_kreis_beidseitig}
    \sisetup{table-format = 3.0}
    \begin{tabular}[]{S[table-format = 2.0] S S S[table-format = 2.0]}
        \toprule
        {$x / \unit{\centi\meter}$} & {$D_0 / (\qty{0.01}{\milli\meter})$} & {$D_\text{m} / (\qty{0.01}{\milli\meter})$} & {$D / (\qty{0.01}{\milli\meter})$} \\
        \midrule
         3 & 86 & 86 &  0 \\
         5 & 87 & 86 &  1 \\
         7 & 89 & 86 &  3 \\
         9 & 88 & 85 &  3 \\
        11 & 88 & 83 &  5 \\
        13 & 86 & 79 &  7 \\
        15 & 84 & 76 &  8 \\
        17 & 81 & 72 &  9 \\
        19 & 79 & 68 & 11 \\
        21 & 81 & 69 & 12 \\
        22 & 79 & 68 & 11 \\
        23 & 80 & 67 & 13 \\
        24 & 79 & 66 & 13 \\
        25 & 79 & 64 & 15 \\
        26 & 79 & 63 & 16 \\
        27 & 79 & 63 & 16 \\
        28 & 79 & 63 & 16 \\
        29 & 78 & 62 & 16 \\
        31 & 78 & 61 & 17 \\
        32 & 78 & 61 & 17 \\
        33 & 78 & 62 & 16 \\
        34 & 77 & 62 & 15 \\
        35 & 79 & 64 & 15 \\
        36 & 80 & 64 & 16 \\
        37 & 79 & 63 & 16 \\
        38 & 79 & 63 & 16 \\
        39 & 77 & 64 & 13 \\
        40 & 77 & 63 & 14 \\
        41 & 78 & 62 & 16 \\
        42 & 77 & 62 & 15 \\
        44 & 75 & 63 & 12 \\
        46 & 71 & 63 &  8 \\
        48 & 70 & 63 &  7 \\
        49 & 71 & 63 &  8 \\
        \bottomrule
    \end{tabular}
\end{table}
