\subsection{Kreisförmiger Querschnitt, beidseitige Auflegung}
Es werden das gleiche Belastungsgewicht und die gleichen Abstände wie in Abschnitt \ref{sec:rechteck_beidseitig} verwendet.
Die Messdaten sind in Tabelle \ref{tab:messdaten_kreis_beidseitig} zu sehen.
Die Auswertung erfolgt somit komplett analog.
%
\begin{table}[H]
    \centering
    \caption{Messdaten des Stabes mit runder Querschnitt bei beidseitiger Auflegung.}
    \label{tab:messdaten_kreis_beidseitig}
    \sisetup{table-format = 3.0}
    \begin{tabular}[]{S S[table-format = 2.0] S S S[table-format = 2.0]}
        \toprule
        {} & {$x / \unit{\centi\meter}$} & {$D_0 / (\qty{0.01}{\milli\meter})$} & {$D_\text{m} / (\qty{0.01}{\milli\meter})$} & {$D / (\qty{0.01}{\milli\meter})$} \\
        \midrule
        {rechts} &  3 & 86 & 86 &  0 \\
                 &  5 & 87 & 86 &  1 \\
                 &  7 & 89 & 86 &  3 \\
                 &  9 & 88 & 85 &  3 \\
                 & 11 & 88 & 83 &  5 \\
                 & 13 & 86 & 79 &  7 \\
                 & 15 & 84 & 76 &  8 \\
                 & 17 & 81 & 72 &  9 \\
                 & 19 & 79 & 68 & 11 \\
                 & 21 & 81 & 69 & 12 \\
                 & 22 & 79 & 68 & 11 \\
                 & 23 & 80 & 67 & 13 \\
                 & 24 & 79 & 66 & 13 \\
                 & 25 & 79 & 64 & 15 \\
                 & 26 & 79 & 63 & 16 \\
                 & 27 & 79 & 63 & 16 \\
                 & 28 & 79 & 63 & 16 \\
                 & 29 & 78 & 62 & 16 \\
        \midrule
         {links} & 31 & 78 & 61 & 17 \\
                 & 32 & 78 & 61 & 17 \\
                 & 33 & 78 & 62 & 16 \\
                 & 34 & 77 & 62 & 15 \\
                 & 35 & 79 & 64 & 15 \\
                 & 36 & 80 & 64 & 16 \\
                 & 37 & 79 & 63 & 16 \\
                 & 38 & 79 & 63 & 16 \\
                 & 39 & 77 & 64 & 13 \\
                 & 40 & 77 & 63 & 14 \\
                 & 41 & 78 & 62 & 16 \\
                 & 42 & 77 & 62 & 15 \\
                 & 44 & 75 & 63 & 12 \\
                 & 46 & 71 & 63 &  8 \\
                 & 48 & 70 & 63 &  7 \\
                 & 49 & 71 & 63 &  8 \\
        \bottomrule
    \end{tabular}
\end{table}

\subsubsection{Rechte Seite}
Für die lineare Ausgleichsrechnung ergeben sich die Werte
\begin{align*}
    s_\text{r} &= (\num{0.00109} \pm \num{0.00007}) \, \frac{1}{\unit{\meter^2}}, & 
    b_\text{r} &= (\num{-0.000046} \pm \num{0.000009}) \, \unit{\meter}.
\end{align*}
% parameter rechts:
% s = 0.00108506 1/m^2
% b = -0.00004673 m
% Fehler_rechts:  [6.64784236e-05 9.37093844e-06]
Der Plot der Ausgleichsgeraden \eqref{eq:ausgleichsfunktion_beidseitig_rechts} für den zylinderförmigen Stab
ist in Abbildung \ref{fig:plot_kreis_beidseitig_rechts} zu sehen.
%
\begin{figure}[H]
    \centering
    \includegraphics[]{build/C02_kreis_beidseitig_rechts.pdf}
    \caption{Ausgleichsgerade \eqref{eq:ausgleichsfunktion_beidseitig_rechts} zu den Messdaten aus Tabelle \ref{tab:messdaten_kreis_beidseitig} (rechts).}
    \label{fig:plot_kreis_beidseitig_rechts}
\end{figure}

\noindent
Der Elastizitätsmodul errechnet sich somit gemäß Gleichung \eqref{eq:D_x_rechts} zu
\begin{align}
    E_\text{Kreis,r} = (\num{212} \pm \num{4}) \cdot 10^9 \, \unit{\newton\per\meter^2}.
\end{align}
% elastizitaet_rechts:  (2.12+/-0.04)e+11
%was der Abweichung
%\begin{align*}
%    \frac{E_\text{Kreis,r} - E_\text{Kreis}}{E_\text{Kreis}} = (\num{61} \pm \num{4}) \, \%
%\end{align*}
% abweichung_rechts: 0.61+/-0.04
%entspricht.




\subsubsection{Linke Seite}
Analog zu Abschnitt \ref{sec:rechteck_beidseitig_links} ergeben sich die Werte
\begin{align*}
    s_\text{l} &= (\num{0.00088} \pm \num{0.00010}) \, \frac{1}{\unit{\meter^2}}, & 
    b_\text{l} &= (\num{0.000020} \pm \num{0.000014}) \, \unit{\meter}
\end{align*}
% parameter links:
% s = 0.00087826 1/m^2
% b = 0.00002034 m
% Fehler_links:  [1.04203935e-04 1.44064544e-05]
für die Ausgleichsgerade.
In Abbildung \ref{fig:plot_kreis_beidseitig_links} ist der Plot für die linke Seite des zylinderförmigen Stabes zu sehen.
%
\begin{figure}[H]
    \centering
    \includegraphics[]{build/C02_kreis_beidseitig_links.pdf}
    \caption{Ausgleichsgerade \eqref{eq:ausgleichsfunktion_beidseitig_links} zu den Messdaten aus Tabelle \ref{tab:messdaten_kreis_beidseitig} (links).}
    \label{fig:plot_kreis_beidseitig_links}
\end{figure}

\noindent
Der Elastizitätsmodul beträgt somit gemäß Gleichung \eqref{eq:D_x_beidseitig_links}
\begin{align}
    E_\text{Kreis,l} = (\num{262} \pm \num{5}) \cdot 10^9 \, \unit{\newton\per\meter^2}.
\end{align}
% elastizitaet_links:  (2.62+/-0.05)e+11 N/m^2
%Dadurch ergeben sich die Abweichungen
%\begin{align*}
%    \frac{E_\text{Kreis,l} - E_\text{Kreis}}{E_\text{Kreis}} &= (\num{99} \pm \num{6}) \, \%, &
%    \frac{E_\text{Kreis,l} - E_\text{Kreis,r}}{E_\text{Kreis,r}} &= (\num{23.5} \pm \num{0.0}) \, \%.
%\end{align*}
% abweichung_links:  0.99+/-0.06
% abweichung_rechts_links: -0.235470533236481899264+/-0.000000000000000000031