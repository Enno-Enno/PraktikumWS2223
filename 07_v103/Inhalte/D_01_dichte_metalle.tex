\subsection{Materialien}
Vergleicht man die Dichten der Stäbe aus den Tabellen \ref{tab:kreis} und \ref{tab:rechteck} mit dem Literaturwert \cite{geschke} 
der Dichte von Kupfer $\rho_\text{Cu} = \qty{8.92}{\gram / \cm ^3}$, so ergeben sich die relativen Abweichungen
\begin{align*}
    \frac{|\rho_\text{Kreis} - \rho_\text{Cu}|}{\rho_\text{Cu}} &= (\num{1.1} \pm \num{1.0}) \, \%, &
    \frac{|\rho_\text{Recht} - \rho_\text{Cu}|}{\rho_\text{Cu}} &= (\num{0.4} \pm \num{1.0}) \, \%. &
\end{align*}
%Abweichung Dichte Kreis:  -0.011+/-0.010
%Abweichung Dichte Rechteck:  -0.004+/-0.010
Daher lässt sich vermuten, dass die beiden Stäbe jeweils aus Kuper bestehen.

\noindent
Kupfer hat nach \cite{uni_kiel} einen Elastizitätsmodul von $E_\text{Cu} = {124} \cdot 10^9 \unit{\newton / \meter^2}$.
Bei einseitiger Auflegung ergeben sich für den Stab mit dem kreisförimigen Querschnitt dadurch die prozentuale Abweichung
\begin{align*}
    \frac{|E_\text{Kreis} - E_\text{Cu}|}{E_\text{Cu}} &= (\num{6.2} \pm \num{2.1}) \, \%,
    \intertext{und für den Stab mit dem rechteckigem Querschnitt}
    \frac{|E_\text{Recht} - E_\text{Cu}|}{E_\text{Cu}} &= (\num{3.3} \pm \num{1.9}) \, \%.
\end{align*}
Bei beidseitiger Auflegung ergeben sich für den Stab mit kreisförimigen Querschnitt links bzw. rechts
\begin{align*}
    \frac{|E_\text{Kreis,l} - E_\text{Cu}|}{E_\text{Cu}} &= (\num{111} \pm \num{4}) \, \%, &
    \frac{|E_\text{Kreis,r} - E_\text{Cu}|}{E_\text{Cu}} &= (\num{71.0} \pm \num{3.2}) \, \%,
    \intertext{und für den Stab mit dem rechteckigem Querschnitt}
    \frac{|E_\text{Recht,l} - E_\text{Cu}|}{E_\text{Cu}} &= (\num{1.7} \pm \num{2.0}) \, \%, &
    \frac{|E_\text{Recht,r} - E_\text{Cu}|}{E_\text{Cu}} &= (\num{34.7} \pm \num{2.7}) \, \%. %0.347+/-0.027
\end{align*}