\subsection{Materialien}
Die Dichten der Stäbe aus den Tabellen \ref{tab:kreis} und \ref{tab:rechteck} überschneiden sich in ihrer 
Messunsicherheit mit dem Literaturwert \cite{geschke} 
der Dichte von Kupfer $\rho_\text{Cu} = \qty{8.92}{\gram / \cm ^3}$.% so ergeben sich die relativen Abweichungen
% \begin{align*}
%     \frac{|\rho_\text{Kreis} - \rho_\text{Cu}|}{\rho_\text{Cu}} &= (\num{1.1} \pm \num{1.0}) \, \%, &
%     \frac{|\rho_\text{Recht} - \rho_\text{Cu}|}{\rho_\text{Cu}} &= (\num{0.4} \pm \num{1.0}) \, \%. &
% \end{align*}
%Abweichung Dichte Kreis:  -0.011+/-0.010
%Abweichung Dichte Rechteck:  -0.004+/-0.010
Daher lässt sich vermuten, dass die beiden Stäbe jeweils aus Kupfer bestehen.

\noindent
Kupfer hat nach \cite{uni_kiel} ein Elastizitätsmodul von $E_\text{Cu} = {124} \cdot 10^9 \unit{\newton / \meter^2}$.
In den Experimenten wurde das Elastizitätsmodul der Stangen bestimmt.
Bei einseitiger Auflage ergibt der Vergleich der theoretischen Werte mit dem Literaturwert eine Prozentuale Abweichung.
\begin{align*}
    E_\text{Kreis} &= (\num{131.7} \pm \num{2.6}) \cdot 10^9 \, \unit{\newton\per\meter^2}&
    \frac{|E_\text{Kreis} - E_\text{Cu}|}{E_\text{Cu}} &= (\num{6.2} \pm \num{2.1}) \, \%,
    \intertext{und für den Stab mit dem rechteckigem Querschnitt}
    E_\text{Recht} &= (\num{119.9} \pm \num{2.4}) \cdot 10^9 \, \unit{\newton\per\meter^2}&
    \frac{|E_\text{Recht} - E_\text{Cu}|}{E_\text{Cu}} &= (\num{3.3} \pm \num{1.9}) \, \%.
\end{align*}
Bei beidseitiger Auflage ergibt dieser Vergleich für einen Kreisförmigen Querschnitt links bzw. rechts
\begin{align*}
    E_\text{Kreis,l} &= (\num{262} \pm \num{5}) \cdot 10^9 \, \unit{\newton\per\meter^2}&
    \frac{|E_\text{Kreis,l} - E_\text{Cu}|}{E_\text{Cu}} &= (\num{111} \pm \num{4}) \, \%, \\
    E_\text{Kreis,r} &= (\num{212} \pm \num{4}) \cdot 10^9 \, \unit{\newton\per\meter^2}&
    \frac{|E_\text{Kreis,r} - E_\text{Cu}|}{E_\text{Cu}} &= (\num{71.0} \pm \num{3.2}) \, \%,\\
    \intertext{und für den Stab mit dem rechteckigem Querschnitt}
    E_\text{Recht,l} &= \frac{F}{48 I s_\text{l}} = (\num{126.1} \pm \num{2.5}) \cdot 10^9 \, \unit{\newton\per\meter^2}&
    \frac{|E_\text{Recht,l} - E_\text{Cu}|}{E_\text{Cu}} &= (\num{1.7} \pm \num{2.0}) \, \%, \\
    E_\text{Recht,r} &= \frac{F}{48 I s_\text{r}} = (\num{167.0} \pm \num{3.3}) \cdot 10^9 \, \unit{\newton\per\meter^2}&
    \frac{|E_\text{Recht,r} - E_\text{Cu}|}{E_\text{Cu}} &= (\num{34.7} \pm \num{2.7}) \, \%. %0.347+/-0.027
\end{align*}