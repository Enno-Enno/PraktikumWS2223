\subsection{Kenngrößen der Probekörper und Belastungsgewichte}

Die Kenngrößen des Stabes mit kreisförmiger Querschnittsfläche sind in Tabelle \ref{tab:kreis} einzusehen,
die Kenngrößen des Stabes mit der rechteckigen Querschnittsfläche in Tabelle \ref{tab:rechteck}.
Hierbei ist $m$ die Masse, $l$ die Länge, $d$ der Durchmesser, $b$ die Breite und $t$ die Tiefe der jeweiligen Probekörper.
Die angegebenen Fehler in den Tabellen sind dabei den Messgeräten entnommen.

\begin{table}[H]
    \centering
    \caption{Kenngrößen des zylinderförmigen Stabes.}
    \label{tab:kreis}
    \sisetup{table-format = 1.3}
    \begin{tabular}[]{S @{${}\pm{}$} S S[table-format = 2.1] S[table-format = 3.1] @{${}\pm{}$} S[table-format = 1.1]}
        \toprule
        \multicolumn{2}{c}{$d / \unit[]{\cm}$} & {$l_\text{Kreis} / \unit[]{\cm}$} & \multicolumn{2}{c}{$m_\text{Kreis} / \unit[]{\gram}$} \\
        \midrule
        1.000 & 0.005 & 59.2 & 410.0 & 0.1 \\
        \bottomrule
    \end{tabular}
\end{table}

\begin{table}[H]
    \centering
    \caption{Kenngrößen des Stabes mit rechteckigem Querschnitt.}
    \label{tab:rechteck}
    \sisetup{table-format = 1.3}
    \begin{tabular}[]{S @{${}\pm{}$} S S @{${}\pm{}$} S S[table-format=2.1] S[table-format=3.1] @{${}\pm{}$} S[table-format=1.1]}
        \toprule
        \multicolumn{2}{c}{$b / \unit[]{\cm}$} & \multicolumn{2}{c}{$t / \unit[]{\cm}$} & {$l_\text{Recht} / \unit[]{\cm}$} & \multicolumn{2}{c}{$m_\text{Recht} / \unit[]{\gram}$} \\
        \midrule
        1.000 & 0.005 & 1.000 & 0.005 & 60.3 & 535.9 & 0.1 \\
        \bottomrule
    \end{tabular}
\end{table}

\noindent
Die Massen der verwendeten Belastungsgewichte sowie der Halterung sind in Tabelle \ref{tab:massen} einzusehen.

\begin{table}[H]
    \centering
    \caption[]{Bestimmte Massen der Gewichte und der Halterung.}
    \label{tab:massen}
    \begin{tabular}[pos]{S S[table-format = 3.1] @{${}\pm{}$} S[table-format = 1.1]}
        \toprule
        {Körper} & \multicolumn{2}{c}{$m / \unit[]{\gram}$}  \\
        \midrule
        {Halterung}     &  50.4 & 0.1 \\
        {schwere Masse} & 501.5 & 0.1 \\
        {leichte Masse} & 199.3 & 0.1 \\
        \bottomrule
    \end{tabular}
\end{table}