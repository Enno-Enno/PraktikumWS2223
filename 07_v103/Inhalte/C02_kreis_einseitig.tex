\subsection{Kreisförmiger Querschnitt, einseitige Einspannung}
Bei diesem Versuchsteil hängt die Masse $m_\text{eins}$ bei $L = \qty{50.5}{\cm}$.
Die Messdaten sind in Tabelle \ref{tab:messdaten_kreis_einseitig} einzusehen.

\begin{table}[H]
    \centering
    \caption{Messdaten des Stabes mit rundem Querschnitt bei einseitiger Einspannung.}
    \label{tab:messdaten_kreis_einseitig}
    \sisetup{table-format = 3.0}
    \begin{tabular}[]{S[table-format = 2.0] S S S}
        \toprule
        {$x / \unit{\centi\meter}$} & {$D_0 / (\qty{0.01}{\milli\meter})$} & {$D_\text{m} / (\qty{0.01}{\milli\meter})$} & {$D / (\qty{0.01}{\milli\meter})$} \\
        \midrule
        49 & 193 &  32 & 161 \\
        47 & 192 &  40 & 152 \\
        45 & 191 &  54 & 137 \\
        43 & 195 &  72 & 123 \\
        41 & 196 &  77 & 119 \\
        39 & 197 &  89 & 108 \\
        37 & 200 & 100 & 100 \\
        35 & 201 & 108 &  93 \\
        33 & 199 & 114 &  85 \\
        31 & 198 & 120 &  78 \\
        29 & 197 & 128 &  69 \\
        27 & 196 & 135 &  61 \\
        25 & 195 & 141 &  54 \\
        23 & 194 & 147 &  47 \\
        21 & 195 & 154 &  41 \\
        19 & 191 & 158 &  33 \\
        17 & 192 & 164 &  28 \\
        15 & 192 & 170 &  22 \\
        13 & 192 & 175 &  17 \\
        11 & 193 & 180 &  13 \\
         9 & 193 & 184 &   9 \\
         7 & 192 & 186 &   6 \\
         5 & 190 & 186 &   4 \\
         3 & 188 & 186 &   2 \\
        \bottomrule
    \end{tabular}
\end{table}

\noindent
Analog zu Abschnitt \ref{sec:rechteck_einseitig} ergibt sich die Ausgleichsgerade in Abbildung \ref{fig:plot_kreis_einseitig}
mit einer Steigung von 
\begin{align*}
    s = (\num{0.0194} \pm \num{0.0001}) \, \unit{\per\meter^2}.
\end{align*}
Daraus folgt der Elastizitätsmodul
\begin{align}
    E_\text{kreis,eins} = (\num{8.0224} \pm \num{0.1605}) \cdot 10^9 \, \unit{\newton\per\meter^2}.
\end{align}
% [8,022,411,628.+/-160,512,553.] N/m^2

\begin{figure}[H]
    \centering
    \includegraphics[]{build/C02_kreis_einseitig.pdf}
    \caption{Ausgleichsgerade \eqref{eq:ausgleichsfunktion_einseitig} zu den Messdaten aus Tabelle \ref{tab:messdaten_kreis_einseitig}.}
    \label{fig:plot_kreis_einseitig}
\end{figure}