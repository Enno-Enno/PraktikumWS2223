\subsection{Rechteckiger Querschnitt, beidseitige Auflegung}

\begin{table}[H]
    \centering
    \caption{Messdaten des Stabes mit rechteckigem Querschnitt bei beidseitiger Auflegung.}
    \label{tab:messdaten_rechteck_beidseitig}
    \sisetup{table-format = 3.0}
    \begin{tabular}[]{S[table-format = 2.0] S S S[table-format = 2.0]}
        \toprule
        {$x / \unit{\centi\meter}$} & {$D_0 / (\qty{0.01}{\milli\meter})$} & {$D_\text{m} / (\qty{0.01}{\milli\meter})$} & {$D / (\qty{0.01}{\milli\meter})$} \\
        \midrule
         3 &  79 &  73 &  6 \\
         5 &  82 &  73 &  9 \\
         7 &  83 &  75 &  8 \\
         9 &  86 &  78 &  8 \\
        11 &  91 &  82 &  9 \\
        13 &  96 &  84 & 12 \\
        15 & 105 &  89 & 16 \\
        17 & 106 &  93 & 13 \\
        19 & 115 &  99 & 16 \\
        21 & 126 & 110 & 16 \\
        22 & 129 & 112 & 17 \\
        23 & 133 & 118 & 15 \\
        24 & 138 & 121 & 17 \\
        25 & 142 & 125 & 17 \\
        26 & 148 & 129 & 19 \\
        27 & 150 & 133 & 17 \\
        28 & 153 & 135 & 18 \\
        29 & 159 & 141 & 18 \\
        31 & 170 & 152 & 18 \\
        32 & 172 & 160 & 12 \\
        33 & 181 & 165 & 16 \\
        34 & 189 & 172 & 17 \\
        35 & 197 & 179 & 18 \\
        36 & 203 & 187 & 16 \\
        37 & 210 & 194 & 16 \\
        38 & 214 & 199 & 15 \\
        39 & 219 & 205 & 14 \\
        40 & 225 & 209 & 16 \\
        41 & 231 & 216 & 15 \\
        42 & 234 & 222 & 12 \\
        44 & 244 & 232 & 12 \\
        46 & 241 & 234 &  7 \\
        48 & 247 & 241 &  6 \\
        49 & 251 & 246 &  5 \\
        \bottomrule
    \end{tabular}
\end{table}
