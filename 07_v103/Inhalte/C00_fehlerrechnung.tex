\subsection{Gaußsche Fehlerfortpflanzung}

Wenn zu Messdaten die Standardabweichung bekannt ist, und mit diesen Messdaten weiter gerechnet werden soll,
wird die Gaußsche Fehlerfortpflanzung verwendet. 
Angenommen, es gibt $k$ Messwerte $x_i [i \in \mathbb{N}, i \leq k]$ mit den Standardabweichungen $\Delta x_i$
und eine abgeleitete Größe $f(x_i)$.
Dann ist der Fehler von $f$
\begin{align}
    \Delta f(x_i) = \sqrt{
    \left(\frac{\partial f}{\partial x_1} \Delta x_1\right)^2%
     + \left(\frac{\partial f}{\partial x_2} \Delta x_2\right)^2%
     + \dots%
     + \left(\frac{\partial f}{\partial x_k} \Delta x_k\right)^2%
    }.
    \label{eq:gauspflanz}
\end{align} 
Im Ergebnis ergibt sich der Mittelwert von $f$ mit der errechneten Abweichung $\overline{f} \pm \Delta f $.
Um Rechenfehler zu vermeiden, wird das Python \cite[]{python} Paket \texttt{uncertainties} \cite[][]{uncertainties} verwendet.
Hier wird die Fehlerfortpflanzung automatisch verrechnet, wenn die Variablen als \texttt{ufloat} definiert werden.