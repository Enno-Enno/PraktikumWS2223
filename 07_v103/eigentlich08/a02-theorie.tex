\input{header.tex}
\begin{document}

\section[Theorie]{{Theorie\footnote{Unter Verwendung der Quelle \cite{man:v601}.}}}

Der Franck-Hertz-Versuch ist ein Elektronenstoßexperiment. 
In diesem Zusammenhang bedeutet das, dass Hg-Atome mit Elektronen möglichst monoenergetischer Energie beschossen werden.
Es treten sowohl elastische als auch inelastische Stöße auf.
%Die kinetische Energie $E_\text{kin}$, die die Elektronen bei den inelastischen Stößen abgeben, dient dabei als Informationsquelle.
Bei einem inelastischen Stoß wird ein Hg-Atom aus seinem Grundzustand mit der Energie $E_0$ in den ersten angeregten Zustand mit der Energie $E_1$ versetzt.
Somit entspricht die Energiedifferenz 
\begin{align}
    \Delta E = E_{\text{kin,vor}} - E_{\text{kin,nach}} = \frac{m_0}{2} \left(v^2_\text{vor} -v^2_\text{nach}\right) = E_1 - E_0,
    \label{eq:energiedifferenz}
\end{align}
wobei $m_0$ die Ruhemasse und $E_\text{kin}$ die kinetische Energie des Elektrons ist.
Dabei sei angemerkt, dass theoretisch auch höhere Energieniveaus möglich sind, diese allerdings mit der hier verwendeten Apparatur nicht erreicht werden.
Ferner ist es nicht möglich, die Energie der emittierten Photonen beim Wechsel zurück in den Grundzustand zu messen.
Die Energiemessung der Elektronen erfolgt mittels der Gegenfeldmethode.



\subsection{Aufbau und Ablauf des Versuchs}



\printbibliography

\end{document}