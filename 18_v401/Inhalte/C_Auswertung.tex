\section{Auswertung}


\subsection{Mittelwerte und Fehler}
Das arithmetische Mittel $\overline{c}$ und die Standardabweichung $\Delta c$ einer Messreihe mit $N$ Werten $c_k$ errechnet sich gemäß der Formeln
\begin{align}
    \overline{c} &= \frac{1}{N} \sum_{k=1}^{N} c_k, & \Delta c = \sqrt{ \sum_{k=1}^{N} \left(\overline{c} - c_k \right)^2 }.
    \label{eq:mittelstand}
\end{align}

\subsection{Gaußsche Fehlerfortpflanzung}
Wenn zu Messdaten die Standardabweichung bekannt ist, und mit diesen Messdaten weiter gerechnet werden soll,
wird die Gaußsche Fehlerfortpflanzung verwendet. 
Angenommen, es gibt $k$ Messwerte $x_i [i \in \mathbb{N}, i \leq k]$ mit den Standardabweichungen $\Delta x_i$
und eine abgeleitete Größe $f(x_i)$.
Dann ist der Fehler von $f$
\begin{align}
    \Delta f(x_i) = \sqrt{
    \left(\frac{\partial f}{\partial x_1} \Delta x_1\right)^2%
     + \left(\frac{\partial f}{\partial x_2} \Delta x_2\right)^2%
     + \dots%
     + \left(\frac{\partial f}{\partial x_k} \Delta x_k\right)^2%
    }.
    \label{eq:gauss}
\end{align} 
Im Ergebnis ergibt sich der Mittelwert von $f$ mit der errechneten Abweichung $\overline{f} \pm \Delta f $.
Um Rechenfehler zu vermeiden, wird das Python \cite[]{python} Paket \texttt{uncertainties} \cite[][]{uncertainties} verwendet.
Hier wird die Fehlerfortpflanzung automatisch verrechnet, wenn die Variablen als \texttt{ufloat} definiert werden.

\subsection{Wellenlänge der Laserdiode}
Die gemessene Impulsanzahl $z$ und die daraus resultierende Wellenlänge $\lambda$ ist in Tabelle \ref{tab:lambda} zu sehen.
Die Wellenlänge ergibt sich dabei aus Gleichung \eqref{eq:d_mod}, wobei die Distanz $\Delta d$ von der vom Mikrometerschraube
abgelesen und mit der Hebeluntersetzung $\frac{1}{\num{5.017}}$ gewichtet wurde, um die tatsächliche Verschiebung zu erhalten.
Da hier der an der Mikrometerschraube jeweils abgelesene Wert nicht genauer notiert wurde, wird für jeden Wert eine Ungenauigkeit
von \qty{0.02}{\milli\meter} anstelle des angegebenen Messfehlers \qty{0.01}{\milli\meter} angenommen.
Somit gilt jeweils $\Delta d = \qty{5.00 +- 0.02}{\milli\meter}$.
Die Abweichungen der einzelnen Wellenlängen ergeben sich aus der Gaußschen Fehlerfortpflanzung \eqref{eq:gauss}.

\begin{table}[H]
    \centering
    \caption{Die Wellenlänge $\lambda$ in Abhängigkiet der gemessenen Impulsanzahl $z$.}
    \label{tab:lambda}
    \begin{tabular}{
        S[table-format = 4.0] % z
        S[table-format = 3.2] % nom(lambda)
        @{${}\pm{}$}
        S[table-format = 1.2] %std(lambda)
        }
        \toprule
        {$z$} & \multicolumn{2}{c}{$\lambda / \unit{\nano\meter}$} \\
        \midrule
        2992 & 666.18 & 2.66 \\
        2997 & 665.07 & 2.66 \\
        3026 & 658.70 & 2.63 \\
        2999 & 664.63 & 2.66 \\
        3037 & 656.31 & 2.63 \\
        2955 & 674.53 & 2.70 \\
        2971 & 670.89 & 2.68 \\
        3010 & 662.20 & 2.65 \\
        2953 & 674.98 & 2.70 \\
        2955 & 674.53 & 2.70 \\
        \bottomrule
    \end{tabular}
\end{table}

\noindent
Durch eine Mittelung (vgl. Gleichung \eqref{eq:mittelstand}) ergibt sich schließlich die experimentell bestimmte Wellenlänge der Diode
zu $\overline{\lambda} = \qty{666.80 +- 2.67}{\nano\meter}$.


\subsection{Brechungsindex von Luft}