\section{Diskussion}

Der gegebene Literatuwert für die Wellenlänge der Laserdiode beträgt $\lambda_\text{lit} = \qty{682 \pm 2}{\nm}$.
Der gemessene Wert von $\overline{\lambda} = \qty{666.80 \pm 2.67}{\nano\meter}$ weicht um 
$\frac{\left|\lambda_\text{lit} - \overline{\lambda}\right|}{\lambda_\text{lit}} = \qty{2.2}{\percent}$ vom Literaturwert ab.
Der Fehlerbereich der Messung enthält nicht den Literaturwert oder den Fehlerbereich davon.
Ein systematischer Fehler, der aufgetaucht sein könnte ist, dass durch berühren des Tisches oder andere Störungen zu viele Impulse gezählt wurden.
Dadurch könnte das Ergebnis für die Wellenlänge nach unten hin verfälscht worden sein.

Der Literaturwert für den Brechungsindex von Luft beträgt laut \cite{chemie_n} $n_\text{lit} = \num{1.000292}$.
Die Messung ergab einen Brechungsindex von $\overline{n} = \num{1.0002891 \pm 0.0000050}$.
Der Fehlerbereich dieser Messung überlappt mit dem Theoriewert. 
Eine Erklärung für die Bessere Messung als bei der Wellenlänge kann sein, dass sich durch die insgesamt kürzere Messzeit 
weniger Gelegenheiten ergeben haben das Messergebnis zu verfälschen.