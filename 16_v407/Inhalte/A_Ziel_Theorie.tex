\section{Ziel}
In diesem Versuch wird das Verhalten von parallel und senkrecht polarisiertem Licht bei Reflexion an Grenzflächen untersucht,
d.h. es werden die Fresnelschen Gleichungen für die Reflexion überprüft.
Außerdem wird der Brewsterwinkel sowie der Brechungsindex von Silizium ermittelt.


\section[Theorie]{Theorie\footnote[1]{Unter Verwendung von \cite{man:v407}.}}

\subsection{Polarisation}
Elektromagnetische Wellen schwingen stets senkrecht zu ihrer Ausbreitungsrichtung.
Je nach der Form mit der sich das elektrische bzw. magnetische Feld dabei bewegt
wird beispielsweise von linear polarisiertem Licht (die Felder schwingen auf einer Geraden senkrecht zur Ausbreitung)
oder zirkular polarisiertem Licht (die Felder schwingen auf einem Kreis senkrecht zur Ausbreitung) gesprochen.
In diesem Versuch ist das linear polarisierte Licht von Interesse, das beispielsweies mittels Polarisationsfolien aus unpolarisiertem Licht erzeugt werden kann.

\subsection{Die Fresnelschen Gleichungen}
% Trifft Licht auf eine Grenzfläche zwischen zwei Brechungsindizes, wird es zum Teil transmittiert und zum Teil reflektiert.
% Dabei ist unter anderem der Einfallswinkel ausschlaggebend.
Theoretische Grundlage dieses Versuchs sind die Maxwellschen Gleichungen
\begin{align}
    \nabla \times \symbf{H} &= \symbf{j} + \epsilon \epsilon_0 \dot{\symbf{E}}
     & \nabla \times \symbf{E} &= - \mu \mu_0 \dot{\symbf{H}}
\end{align}
mit der Dielektrizitätskonstante $\epsilon$ und der elekrischen Feldkonstante $\epsilon_0$ sowie der magnetischen Permeabilität $\mu$ und der Induktionskonstante $\mu_0$.
Dabei ist $\symbf{E}$ die elektrische Feldstärke und $\symbf{H}$ die magnetische Erregung.
Im Folgenden werden nur nicht-ferromagnetische und nicht elektrisch leitende Materialien ($\mu = 1, \symbf{j} = 0$) betrachtet.
Aus den Maxwellschen Gleichungen lassen sich bestimmte Bedingungen für Licht als elektromagnetische Welle herleiten:

\noindent
Zum einen kann hergeleitet werden, dass Strahlung ein Energietransport ist, der durch den Poynting-Vektor
\begin{align}
    \symbf{S} = \symbf{E} \times \symbf{H},
    \label{eq:poynting}
\end{align}
beschrieben wird und für den kein Medium notwendig ist.
Die Strahlungsleistung pro Fläche beträgt
\begin{align}
    \left|\symbf{S}\right| = v \epsilon \epsilon_0 \symbf{E}^2
    \label{eq:leistung}
\end{align}
mit der Ausbreitungsgeschwindigkeit $v$ der Welle.

\noindent
Zum anderen kann unter Verwendung des Snelliusschen Brechungsgesetzes
\begin{align}
    \frac{n_2}{n_1} = \frac{\sin(\alpha)}{\sin(\beta)}
    \label{eq:snellius}
\end{align}
das Verhalten von elektromagnetischen Wellen an Grenzflächen zweier Materialien mit unterschiedlichen Brechungsindizes untersucht werden.
Der Brechungsindex $n$ beschreibt dabei das Geschwindingkeitsverhältnis der Welle im Vakuum im Vergleich zum betrachteten Material, also $n = c/v$.
Aus den Maxwellschen Gleichungen lässt sich ferner die Relation $n^2 = \epsilon$ herleiten.
Mit Hilfe dieser Relationen können schließlich die Fresnelschen Gleichungen hergeleitet werden.

%Brewsterwinkel, Gleichung 18 und 21
