\section{Auswertung}
\subsection{Gaußsche Fehlerfortpflanzung}

Wenn zu Messdaten die Standardabweichung bekannt ist, und mit diesen Messdaten weiter gerechnet werden soll,
wird die Gaußsche Fehlerfortpflanzung verwendet. 
Angenommen, es gibt $k$ Messwerte $x_i [i \in \mathbb{N}, i \leq k]$ mit den Standardabweichungen $\Delta x_i$
und eine abgeleitete Größe $f(x_i)$.
Dann ist der Fehler von $f$
\begin{align}
    \Delta f(x_i) = \sqrt{
    \left(\frac{\partial f}{\partial x_1} \Delta x_1\right)^2%
     + \left(\frac{\partial f}{\partial x_2} \Delta x_2\right)^2%
     + \dots%
     + \left(\frac{\partial f}{\partial x_k} \Delta x_k\right)^2%
    }.
    \label{eq:gauspflanz}
\end{align} 
Im Ergebnis ergibt sich der Mittelwert von $f$ mit der errechneten Abweichung $\overline{f} \pm \Delta f $.
Um Rechenfehler zu vermeiden, wird das Python \cite[]{python} Paket \texttt{uncertainties} \cite[][]{uncertainties} verwendet.
Hier wird die Fehlerfortpflanzung automatisch verrechnet, wenn die Variablen als \texttt{ufloat} definiert werden.
\subsection{Brechungsindex aus Fresnel-Gleichungen}
\begin{table}
    \centering
    \begin{tabular}{S S S S}
        \toprule
        {$\alpha / \unit{\degree}$} & {$I_0/ \unit{\nano\ampere}$} & {$I_\perp/ \unit{\nano\ampere}$} & {$I_\parallel/ \unit{\nano\ampere}$}\\
        \midrule
            5.0   & 5.00 \pm 0.10 & 8.00 \pm 0.10  & 7.20 \pm 0.10  \\
            10.0  & 2.60 \pm 0.10 & 9.20 \pm 0.10  & 5.70 \pm 0.10  \\
            15.0  & 1.00 \pm 0.10 & 10.0 \pm 0.5   & 6.00 \pm 0.10  \\
            20.0  & 1.80 \pm 0.10 & 11.0 \pm 0.5   & 4.80 \pm 0.10  \\
            25.0  & 0.80 \pm 0.10 & 12.0 \pm 0.5   & 5.60 \pm 0.10  \\
            30.0  & 1.60 \pm 0.10 & 12.0 \pm 0.5   & 5.20 \pm 0.10  \\
            35.0  & 0.80 \pm 0.10 & 20.0 \pm 0.5   & 7.40 \pm 0.10  \\
            40.0  & 1.10 \pm 0.10 & 28.0 \pm 0.5   & 8.00 \pm 0.10  \\
            45.0  & 0.80 \pm 0.10 & 36.0 \pm 0.5   & 7.80 \pm 0.10  \\
            50.0  & 4.00 \pm 0.10 & 38.0 \pm 0.5   & 7.40 \pm 0.10  \\
            55.0  & 1.90 \pm 0.10 & 48.0 \pm 0.5   & 8.00 \pm 0.10  \\
            60.0  & 1.80 \pm 0.10 & 62.0 \pm 0.5   & 7.00 \pm 0.10  \\
            65.0  & 1.80 \pm 0.10 & 54.0 \pm 0.5   & 4.70 \pm 0.10  \\
            70.0  & 1.90 \pm 0.10 & 92.0 \pm 0.5   & 3.40 \pm 0.10  \\
            75.0  & 1.40 \pm 0.10 & 82.0 \pm 0.5   & 1.60 \pm 0.10  \\
            76.0  & 1.60 \pm 0.10 & 100  \pm 5     & 1.70 \pm 0.10  \\
            77.0  & 1.40 \pm 0.10 & 100  \pm 5     & 1.80 \pm 0.10  \\
            78.0  & 1.30 \pm 0.10 & 110  \pm 5     & 2.20 \pm 0.10  \\
            79.0  & 1.10 \pm 0.10 & 110  \pm 5     & 2.40 \pm 0.10  \\
            80.0  & 1.30 \pm 0.10 & 100  \pm 5     & 4.00 \pm 0.10  \\
            85.0  & 1.00 \pm 0.10 & 120  \pm 5     & 18.0 \pm 0.5   \\
            88.0  & 3.70 \pm 0.10 & 150  \pm 5     & 46.0 \pm 0.5   \\
        \bottomrule
    \end{tabular}
    \caption{Gemessene Intensitäten für die senkrechte und die parallele Polarisation, sowie die Dunkelintensität.}
    \label{tab:messungen}
\end{table}
\begin{table}
    \centering
    \begin{tabular}{S S S}
        \toprule
        {$\alpha / \unit{\degree}$} & {$I_\perp/ \unit{\nano\ampere}$} & {$I_\parallel/ \unit{\nano\ampere}$}\\
        \midrule
        5.0   & 3.00 \pm 0.14   & 2.20 \pm 0.14 \\ 
        10.0  & 6.60 \pm 0.14   & 3.10 \pm 0.14 \\  
        15.0  & 9.0  \pm 0.5    & 5.00 \pm 0.14 \\
        20.0  & 9.2  \pm 0.5    & 3.00 \pm 0.14 \\
        25.0  & 11.2 \pm 0.5    & 4.80 \pm 0.14 \\ 
        30.0  & 10.4 \pm 0.5    & 3.60 \pm 0.14 \\ 
        35.0  & 19.2 \pm 0.5    & 6.60 \pm 0.14 \\ 
        40.0  & 26.9 \pm 0.5    & 6.90 \pm 0.14 \\ 
        45.0  & 35.2 \pm 0.5    & 7.00 \pm 0.14 \\ 
        50.0  & 34.0 \pm 0.5    & 3.40 \pm 0.14 \\ 
        55.0  & 46.1 \pm 0.5    & 6.10 \pm 0.14 \\ 
        60.0  & 60.2 \pm 0.5    & 5.20 \pm 0.14 \\ 
        65.0  & 52.2 \pm 0.5    & 2.90 \pm 0.14 \\ 
        70.0  & 90.1 \pm 0.5    & 1.50 \pm 0.14 \\ 
        75.0  & 80.6 \pm 0.5    & 0.20 \pm 0.14 \\ 
        76.0  & 98   \pm 5      & 0.10 \pm 0.14 \\ 
        77.0  & 99   \pm 5      & 0.40 \pm 0.14 \\ 
        78.0  & 109  \pm 5      & 0.90 \pm 0.14 \\ 
        79.0  & 109  \pm 5      & 1.30 \pm 0.14 \\ 
        80.0  & 99   \pm 5      & 2.70 \pm 0.14 \\ 
        85.0  & 119  \pm 5      & 17.0 \pm 0.5  \\
        88.0  & 146  \pm 5      & 42.3 \pm 0.5  \\
        \bottomrule
    \end{tabular}
    \caption{Korrigierte Intensitäten nach Abzug der Dunkelintensität}
    \label{tab:korr_messungen}
\end{table}
\begin{table}
    \centering
    \begin{tabular}{S S S S}
        \toprule
        {$\alpha / \unit{\degree}$} & {$n_\perp$} & {$n_\parallel$}\\
        \midrule
        5.0    & 1.000337 \pm 0.000010  & 4.043 \pm 0.015  \\
        10.0   & 1.00207 \pm 0.00004    & 1.284 \pm 0.006  \\
        15.0   & 1.00547 \pm 0.00020    & 1.402 \pm 0.007  \\
        20.0   & 1.00965 \pm 0.00034    & 2.746 \pm 0.011  \\
        25.0   & 1.0165 \pm 0.0005      & 1.204 \pm 0.004  \\
        30.0   & 1.0220 \pm 0.0007      & 7.670 \pm 0.027  \\
        35.0   & 1.0414 \pm 0.0010      & 1.259 \pm 0.005  \\
        40.0   & 1.0635 \pm 0.0013      & 1.610 \pm 0.008  \\
        45.0   & 1.0904 \pm 0.0018      & 2.110 \pm 0.010  \\
        50.0   & 1.1031 \pm 0.0020      & 1.205 \pm 0.004  \\
        55.0   & 1.1421 \pm 0.0028      & 54.56 \pm 0.20   \\
        60.0   & 1.188 \pm 0.004        & 1.223 \pm 0.005  \\
        65.0   & 1.186 \pm 0.004        & 1.901 \pm 0.008  \\
        70.0   & 1.287 \pm 0.006        & 1.627 \pm 0.008  \\
        75.0   & 1.279 \pm 0.005        & 1.156 \pm 0.015  \\
        76.0   & 1.323 \pm 0.013        & 1.196 \pm 0.034  \\
        77.0   & 1.326 \pm 0.013        & 36.90 \pm 0.24   \\
        78.0   & 1.352 \pm 0.014        & 1.228 \pm 0.006  \\
        79.0   & 1.354 \pm 0.014        & 1.213 \pm 0.005  \\
        80.0   & 1.332 \pm 0.014        & 10.69 \pm 0.04   \\
        85.0   & 1.387 \pm 0.014        & 1.246 \pm 0.005  \\
        88.0   & 1.454 \pm 0.015        & 1.272 \pm 0.006  \\
        \bottomrule
    \end{tabular}
    \caption{Berechnete Brechungsindizes bei paralleler und senkrechter Polarisation}
    \label{tab:brechungsindizes}
\end{table}
In Tabelle \ref{tab:messungen} sind die gemessenen Intensitäten für die parallel 
und senkrecht polarisierten Reflexionen abzulesen.
Um Störeffekte zu berücksichtigen wird zusätzlich zu jedem Winkel eine Dunkelintensität aufgenommen.
Alle gemessen Werte haben eine Messunsicherheit, die von der gemessenen Intensität abhängt.
Für Intensitäten $\geq \qty{1000}{\nano\ampere}$ wird diese auf \qty{50}{\nano\ampere} abgeschätzt.
Bei kleineren Intensitäten $\geq \qty{100}{\nano\ampere}$ liegt Messunsicherheit bei \qty{5}{\nano\ampere} und 

In Tabelle \ref{tab:korr_messungen} werden die mit der Dunkelintensität verrechneten Werte ($I_\text{gemessen} - I_\text{dunkel}$)
für die Intensität aufgeführt.
Nach den umgestellten Fresnel Formeln \eqref{eq:n_senkrecht},\eqref{eq:n_parallel} werden die Brechungsindizes berechnet.
Dazu wird zur Berechnung des elektrischen Feldes die wurzeln aus den Intensitäten eingesetzt.
Das Einfallende elektrische Feld wird durch eine Messung ohne eingebauten Spiegel ermittelt.
\begin{align}
    I_{\text{e},\perp} &= \qty{1800 \pm 50}{\nano\ampere} & I_{\text{e},\parallel} &= \qty{1000 \pm 50}{\nano\ampere}
\end{align} 
Für die jeweiligen E-Felder die Quadratwurzeln der gemessenen Intensitäten eingesetzt.
Die Ergebnisse in Tabelle \ref{tab:brechungsindizes} enthalten besonders bei der 
parallelen Polarisation statistische Ausreißer. 
Um diesen zu entgehen werden bei der Mittelung über die Brechungsindizes alle Werte für $n \geq 8$ herausgefiltert.
Aus den Messreihen ergeben sich die mittleren Brechungsindizes von 
\begin{align}
    \overline{n_\perp} &= \num{1.180 \pm 0.004} & \overline{n_\parallel} &= \num{1.962 \pm 0.008}
\end{align}
Die Ergebnisse für den Brechungsindex weichen stark voneinander ab.

\subsection{Brechungsindex aus dem Brewsterwinkel}
\begin{figure}
    \centering
    \includegraphics[width=0.75\textwidth]{build/02_plot.pdf}
    \caption{Reflektierte Intensitäten auf die einfallende Intensität normiert.}
    \label{fig:I_r_einfach}
\end{figure}
In Abbildung \ref{fig:I_r_einfach} wird die reflektierte Intensität gegen den Reflexionswinkel $\alpha$ aufgetragen.
In dieser Abbildung lässt sich am Minimum des parallel polarisierten Lichtes der Brewsterwinkel ablesen.
Es ergeben sich nach Formel \eqref{eq:brewster}
\begin{align}
    \alpha_\text{b} &= qty{76 \pm 0.5} & n_\text{b} &= \num{4.01 \pm 0.15}
\end{align}
Es gibt also drei verschiedene Werte für den Brechungsindex $n$.
In Abbildung \ref{fig:I_r_kompliziert} werden die erwarteten Reflexionsintensitäten basierend auf den Formeln \eqref{eq:senkrecht} und 
\eqref{eq:parallel} geplottet.
\begin{figure}
    \centering
    \includegraphics[width=0.75\textwidth]{build/03_plot.pdf}
    \caption{Messwerte mit den erwarteten Funktionen für die verschiedenen ermittelten Brechungsindizes $n$.}
    \label{fig:I_r_kompliziert}
\end{figure}