\section{Auswertung}

\subsection{Mittelwerte und Fehler}
Das arithmetische Mittel $\overline{c}$ und die Standardabweichung $\Delta c$ einer Messreihe mit $N$ Werten $c_k$ errechnet sich gemäß der Formeln
\begin{align}
    \overline{c} &= \frac{1}{N} \sum_{k=1}^{N} c_k, & \Delta c = \sqrt{ \sum_{k=1}^{N} \left(\overline{c} - c_k \right)^2 }.
    \label{eq:mittelstand}
\end{align}

\subsection{Gaußsche Fehlerfortpflanzung}
Wenn zu Messdaten die Standardabweichung bekannt ist, und mit diesen Messdaten weiter gerechnet werden soll,
wird die Gaußsche Fehlerfortpflanzung verwendet. 
Angenommen, es gibt $k$ Messwerte $x_i [i \in \mathbb{N}, i \leq k]$ mit den Standardabweichungen $\Delta x_i$
und eine abgeleitete Größe $f(x_i)$.
Dann ist der Fehler von $f$
\begin{align}
    \Delta f(x_i) = \sqrt{
    \left(\frac{\partial f}{\partial x_1} \Delta x_1\right)^2%
     + \left(\frac{\partial f}{\partial x_2} \Delta x_2\right)^2%
     + \dots%
     + \left(\frac{\partial f}{\partial x_k} \Delta x_k\right)^2%
    }.
    \label{eq:gauss}
\end{align} 
Im Ergebnis ergibt sich der Mittelwert von $f$ mit der errechneten Abweichung $\overline{f} \pm \Delta f $.
Um Rechenfehler zu vermeiden, wird das Python \cite[]{python} Paket \texttt{uncertainties} \cite[][]{uncertainties} verwendet.
Hier wird die Fehlerfortpflanzung automatisch verrechnet, wenn die Variablen als \texttt{ufloat} definiert werden.
\subsection{Reflexionskoeffizient aus Fresnel-Gleichungen}
In Tabelle \textbf{Tabelle Einfügen} sind die gemessenen Intensitäten für die parallel 
und senkrecht polarisierten Reflexionen abzulesen.
Um Störeffekte zu berücksichtigen wird zusätzlich zu jedem Winkel eine Dunkelintensität aufgenommen.
In Tabelle \textbf{Tabelle einfügen} werden die mit der Dunkelintensität verrechneten Werte ($I_\text{gemessen} - I_\text{dunkel}$)
für die Intensität aufgeführt.
Nach den umgestellten Fresnel Formeln \textbf{Referenzen} werden die Brechungsindizes berechnet.
Um die Richtigen werte herauszubekommen werden für die E-Felder die Quadratwurzeln der gemessenen Intensitäten eingesetzt
Die Ergebnisse in Tabelle \textbf{Tabelle Brechungsindizes erstellen} enthalten besonders bei der 
parallelen Polarisation statistische Ausreißer. 
Um diesen zu entgehen werden bei der Mittelung über die Brechungsindizes alle Werte für $n \geq 10$ herausgefiltert.
Aus den Messreihen ergeben sich die mittleren Brechungsindizes von 
\begin{align}
    \overline{n_\perp} &= \num{1.180 \pm 0.004} & \overline{n_\parallel} = \num{4.299 \pm 0.018}
\end{align}