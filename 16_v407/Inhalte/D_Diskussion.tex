\section{Diskussion}


% Die abweichenden Brechungsindizes bei den beiden Polarisationen sind ein Zeichen, dass wir etwas wichtiges bei der Berechnung 
% der Brechungsindizes übersehen haben. (wichtiger systematischer Fehler)
% Es gibt statistische Ausreißer für die Brechungsindizes im Bereich des Brewsterwinkels (bei kleinen Intensitäten)
% aber auch bei größeren Intensitäten. Das ist eher seltsam ¯\_(ツ)_/¯
% Mögliche Erklärung: Der Lichtsensor wurde nicht richtig getroffen.
% Andere Möglichkeit: vielleicht war die Oberfläche des Silziums irgendwie verdreckt.
% Das minimum der Funktion passt nur bei n_b mit dem Minimum der gemessenen werte zusammen. Dahinter steckt aber keine notwendigerweise
% Interessante Aussage, weil wir ja auch die Funktion mit genau diesem Minimum generiert haben.