\section{Diskussion}
Die berechneten Brechungsindizes von $\overline{n_\perp} = \num{1.180 \pm 0.004} $
und $\overline{n_\parallel} = \num{1.962 \pm 0.008}$ haben eine Abweichung von 
$\left|\overline{n_\perp} - \overline{n_\parallel}\right|/\overline{n_\parallel} = \left(\num{39.86 +- 0.32}\right) \, \%$.
Im Vergleich zum Literaturwert \cite{si_n} $n_\text{Lit} = \num{3.353}$ ergeben sich Abweichungen von
$\left|\overline{n_\perp} - n_\text{Lit}\right|/n_\text{Lit} = \left(\num{64.81 +- 0.12}\right) \, \%$
und $\left|\overline{n_\parallel} - n_\text{Lit}\right|/n_\text{Lit} = \left(\num{41.49+-0.24}\right) \, \%$.
Dies lässt auf erhebliche systematische Fehler schließen.

\noindent
Der Brewsterwinkel konnte zu $\alpha_\text{b} = \qty{76 \pm 0.5}{\degree}$ bestimmt werden.
Über Gleichung \eqref{eq:brewster} kann hierdurch ein Vergleich zum Literaturwert des Brechungsindexes erzielt werden.
Die Abweichung beträgt in diesem Fall
$\left|\overline{n_\text{b}} - n_\text{Lit}\right|/n_\text{Lit} = \left(\num{20+-4}\right) \, \%$
bei $n_\text{b} = \num{4.01 \pm 0.15}$.
Auch hier kann aufgrund der Abweichung auf systematische Fehler geschlossen werden.

\noindent
Ein systematischer Fehler durch Umgebungseinflüsse wurde dadurch vorgebeugt, dass zu jedem eingestellten Winkel die Dunkelintensität gemessen wurde.
Die Fehlerquelle muss also woanders liegen.
Beispielsweise könnte es sein, dass der Lichtsensor durch eine geneigte Polarisationsfolie beim senkrecht polarisierten Licht anders getroffen wird als beim parallel
polarisierten Licht. 
Dies erklärt zwar die Abweichung zwischen den experimentell bestimmten Werten, aber nicht die jeweilige Abweichung zum Literaturwert.
Geringe Abweichungen zum Literaturwert könnten ihre Ursache im ungenauen Einstellen des Winkels haben, diese sind aber keine Erklärung für die oben angegebenen Fehler.

% Die abweichenden Brechungsindizes bei den beiden Polarisationen sind ein Zeichen, dass wir etwas wichtiges bei der Berechnung 
% der Brechungsindizes übersehen haben. (wichtiger systematischer Fehler)

% Es gibt statistische Ausreißer für die Brechungsindizes im Bereich des Brewsterwinkels (bei kleinen Intensitäten)
% aber auch bei größeren Intensitäten. Das ist eher seltsam ¯\_(ツ)_/¯
% Mögliche Erklärung: Der Lichtsensor wurde nicht richtig getroffen.
% Andere Möglichkeit: vielleicht war die Oberfläche des Silziums irgendwie verdreckt.
% Das minimum der Funktion passt nur bei n_b mit dem Minimum der gemessenen werte zusammen. Dahinter steckt aber keine notwendigerweise
% Interessante Aussage, weil wir ja auch die Funktion mit genau diesem Minimum generiert haben.