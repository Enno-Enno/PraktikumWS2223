% Diskrepanz Widerstand durch innenwiderstand am Generator 
% (bin mir hier nicht sicher ob ich mich verrechnet hab wegen hoher abweichung aber T_eff sollte passen 
% wenn man das mit dem Graphen vergleicht (hat ca Wert A_0/e). Und wenn man aus T_eff das R_eff berechnet ist das der gleiche Wert also 
% sollte der Wert eigentlich stimmen)

% für nu_+ - nu_- voraussetzung R^2/L^2 bei hoher abweichung überprüfen

\section{Diskussion}

\subsection{Der gedämpfte Oszillator}
Die Amplitude der einhüllenden Funktion konnte mithilfe der gemessenen Werte als e-Funktion verifiziert werden.
Der effektive Widerstand, der aus dem bekannten $R_1$ und dem Innenwiderstand des Frequenzgenerators bestehen sollte weicht
um ein Vielfaches von $R_1$ ab. 
Das liegt vermutlich an einer falschen Einstellung im Finetuning des Oszilloskops, welche dazu geführt haben kann, dass die Werte für die
Zeit und/ oder die Werte für die Amplitude um einen konstanten Faktor verzerrt sind.

Der Dämpfungswiderstand des aperiodischen Grenzfalls weicht von dem erwarteten Wert ab.
Das kann an einem Ablesefehler liegen, wenn die charakteristische Form des aperiodischen Grenzfalls zu früh erkannt wurde.
Ansonsten kann der Innenwiderstand des Frequenzgenerators eine mögliche Quelle der Verzerrung des tatsächlich wirkenden Widerstands sein.

\subsection{Resonanzfrequenzen und Peak}
Die Güte $q$ des Schwingkreises ist bei $\qty{2}{\percent}$  des erwarteten Wertes.
Auch hier können Innenwiederstände und Fehleinstellungen eine Rolle spielen.
Außerdem ist es möglich, dass ein größerer Peak zwishen zwei Messwerten verloren gegangen ist.

Die Resonanzfrequenzen liegen bei etwa $\qty{240}{\percent}$ der erwarteten Werte. 
Die Peaks weichen auch in der breite um den gleichen Faktor ab.
Diese Abweichung kann also gut mit einer Fehleinstellung der Zeiteinheit des Oszilloskops begründet werden.