\subsection{Amplitude der Einhüllenden einer gedämpften Schwingung}
Die gemessenen Werte für die Amplitude $A$ in Abhängigkeit der Zeit $t$ der gedämpften Schwingung 
sind in Tabelle \ref{tab:amplitude} einzusehen.
%
\begin{table}[H]
    \centering
    \caption{Spannungsamplitude $A$ in Abhängigkeit der Zeit $t$.}
    \label{tab:amplitude}
    \begin{tabular}[pos]{S[table-format = 3.0] S[table-format= 1.1]}
        \toprule
        {$t/ \mu \unit[]{\second}$} & {$A / \unit{\volt}$} \\
        \midrule
         0 & 2.8 \\
        14 & 2.4 \\
        26 & 2.0 \\
        38 & 1.7 \\
        52 & 1.5 \\
        64 & 1.2 \\
        78 & 1.1 \\
        92 & 1.0 \\
       104 & 0.9 \\
       118 & 0.8 \\
       130 & 0.7 \\
       144 & 0.6 \\
       156 & 0.4 \\
       170 & 0.4 \\
        \bottomrule
    \end{tabular}
\end{table}
\noindent
Anhand von Gleichung \eqref{eq:Schwingfall} ergibt sich die Amplitude
\begin{align}
    A = A_0 \exp\left(- 2 \pi \mu t\right).
\end{align}
Für $A$ wird mit Hilfe der Python \cite[]{python} Funktion \texttt{curve\_fit} aus dem Paket
\texttt{scipy.optimize} \cite[]{scipy} eine nicht-lineare Ausgleichsrechnung durchgeführt.
Es ergibt sich Plot \ref{fig:plot_amplitude} sowie die Parameter 
\begin{align}
    A_0 &= (\num[]{2.7462} \pm \num[]{0.04881}) \, \unit{\volt}, & \mu &= (\num[]{1808.5104} \pm \num[]{55.5975}) \, \unit{\per\second}.
\end{align}
% #effektiver Widerstand R =  230+/-7
% #effektive Abklingdauer T =  (8.80+/-0.27)e-05
Hieraus folgen gemäß der Gleichung \eqref{eq:Abklingdauer} für die effektive Abklingdauer $T_\text{eff}$ und den effektiven Widerstand 
\begin{align*}
    R_\text{eff} = 4 \pi \mu L
\end{align*}
die Werte 
\begin{align}
    R_\text{eff} &= (\num[]{230} \pm \num[]{7}) \, \unit{\ohm}, & 
    T_\text{eff} &= (\num[]{88} \pm \num[]{2.7}) \, \mu \unit{\second}.
\end{align}
Im Vergleich zu dem angeschlossenen Widerstand $R_1$ ergibt sich eine
Abweichung von $\frac{R_\text{eff}}{R_1} = (\num[]{4.78} \pm \num[]{0.15})$.
% effektiver Widerstand R =  230+/-7 ohm
% effektive Abklingdauer T =  (8.80+/-0.27)e-05 s
% Abweichung R =  4.78+/-0.15
\begin{figure}[H]
    \centering
    \includegraphics{build/C01_a.pdf}
    \caption{Ausgleichsplot für die Amplitude der gedämpften Schwingung.}
    \label{fig:plot_amplitude}
\end{figure}
