\subsection{Angetriebener Harmonischer Oszillator}
Nun wird der Gedämpfte Harmonische Oszillator mit einer Sinusschwingung angetrieben.
Die Homogene Differenzialgleichung \ref{eq:Hom_DGL} wird um eine Treibende Sinusschwingung ergänzt
\begin{align}
    \frac{\diff^2 U_C}{\diff t^2}+ \frac{R}{L} \frac{\diff U_C}{\diff t} + \frac{1}{LC} U_C = U_0 e^{i \omega t} .
    \label{eq:inhom_DGL}
\end{align}
Mit dem Ansatz $U_C(\omega, t) = U(\omega) e^{i \omega t}$ entsteht die Gleichung
\begin{align*}
    -LC \omega^2 U + i \omega RC U + U = U_0
\end{align*}
Für $U(\omega)$ ergibt sich der Betrag
\begin{align}
    \mid U \mid = U_0 \sqrt{\frac{\left(1-LC \omega^2 \right)^2+ \omega^2 R^2 C^2}{\left(\left(1-LC \omega^2 \right)^2 + \omega^2 R^2 C^2\right)^2 }}
    \label{eq:Betrag_U}
\end{align}
und die Phase 
\begin{align}
    \phi(\omega) = \arctan\left(\frac{-\omega RC}{1-LC \omega^2}\right)
    \label{eq:Phi_omega}
\end{align}
Aus Gleichung \ref{eq:Betrag_U} ergibt sich auch die Lösungsfunktion $U_C$
\begin{align}
    U_C(\omega) = \frac{U_0}{\sqrt{\left(1-LC \omega^2 \right)^2+ \omega^2 R^2 C^2}}
    \label{eq:U_c_omega}
\end{align}
Für $\omega \rightarrow \infty$ geht diese Funktion gegen 0 während sie für $\omega \rightarrow 0$ gegen $U_0$ geht.
Bei der Resonanzfrequenz $\omega_\text{res}$ hat die Kurve von $U_C$ ein Maximum, das größer als die Eingangsfrequenz sein kann. 
\begin{align}
    \omega_\text{res} = \sqrt{\frac{1}{LC}- \frac{R^2}{2L^2}}
    \label{eq:Resonanzfrequenz}
\end{align}
Besonders interessant ist der Fall Schwacher Dämpfung 
\begin{align*}
    \frac{1}{LC} >> \frac{R^2}{2L^2}
\end{align*}
Für diesen Fall nähert sich die Resonanzfrequenz der Schwingung an die Grundfrequenz $\omega_0$ an (vgl. Gleichung \eqref{eq:Schwingungsdauer_0}).
Die Maximale Amplitude erreicht so einen Wert von
\begin{align}
    U_\text{C, max} = \frac{1}{\omega_0 RC}U_0 = \frac{1}{R}\sqrt{\frac{L}{C}}U_0
    \label{eq:UC_max}
\end{align}
Der Vorfaktor $1/(\omega_0 RC)$ wird auch als Resonanzüberhöhung oder Güte $q$ bezeichnet.
Die Breite des Peaks wird mit $\omega_+$ und $\omega_-$ beschrieben.
Das sind die Frequenzen, bei denen die Amplitude von $U_C$ auf $U_\text{C, max}/\sqrt{2}$ absinkt.
$\omega_-$ und $\omega_+$ werden nun durch die Gleichung
\begin{align*}
    \frac{U_0}{\sqrt{2}} \frac{1}{\omega_0 RC} = \frac{U_0}{C\sqrt{\omega_\pm^2 R^2 + \left(\omega_\pm^2 L - 1/C\right)^2}}
\end{align*}
Da $R^2/ L^2 << \omega_0^2$ kann in der Näherung für die Breite der Resonanzkurve geschlossen werden, dass
\begin{align}
    \omega_+ - \omega_- \simeq \frac{R}{L}
    \label{eq:omega_Breite_pm}
\end{align}
Diese Rechnungen halten auch noch ihre Gültigkeit, 
wenn sie nicht mit der Winkelgeschwindigkeit $\omega$ sondern mit der Frequenz $\nu$ berechnet werden.
Im Fall der starken Dämpfung gibt es keinen Peak bei der Resonanzfrequenz, 
die Intensität von $U_C$ fällt monoton mit ansteigender Frequenz ab.

Die Phasenverschiebung wird mit Gleichung \eqref{eq:Phi_omega} beschrieben.
An der Stelle $\omega_0^2 = \frac{1}{LC}$ ist $\phi = -\pi/2$
Die Werte $\omega_1$ und $\omega_2$ sind die werte bei denen die Phasenverschiebung bei $\pi/4$ bzw $\frac{3}{4}\pi$ liegt.
Sie lassen sich folgendermaßen berechnen:
\begin{align}
    \omega_{1,2} = \pm \frac{R}{2L} + \sqrt{\frac{R^2}{4L^2} + \frac{1}{LC}}
    \label{eq:omega12}
\end{align}
Daraus erhält man für den fall mit niedriger Dämpfung
\begin{align}
    \omega_1 -\omega_2 = \frac{R}{L}
    \label{eq:omega_Breite_12}
\end{align}
Damit sind beide Definitonen der Breite des Peaks gleich breit.