\subsubsection{Phasenverschiebung}
Eine halblogarithmische Darstellung der frequenzabhägigen Phasenverschiebung ist in Abbildung  \ref{fig:plot_phase} zu sehen.
\begin{figure}[H]
    \centering
    \includegraphics{build/C03_c_d_phase.pdf}
    \caption{Halblogarithmische Darstellung der Phasenverschiebung in Abhängigkeit der Frequenz.}
    \label{fig:plot_phase}
\end{figure}

\noindent
Des Weiteren ist der Bereich um $\phi = \frac{\pi}{2}$ in Abbildung \ref{fig:plot_phase_linear} auch linear dargestellt.
Hieraus lassen sich die Werte 
\begin{align}
    \nonumber \nu_\text{res} &= \qty{87}{\kilo\hertz}, \\
    \nu_1 &= \qty{76}{\kilo\hertz}, & 
    \nu_2 &= \qty{95}{\kilo\hertz}
    \intertext{entnehmen.
    Die jeweiligen aus den Gleichungen \eqref{eq:Resonanzfrequenz} und \eqref{eq:omega12} berechneten theoretischen Werte betragen}
    \nonumber \nu_\text{res, theo} &= (\num{34.13} \pm \num{0.06}) \, \unit{\kilo\hertz}, \\
    \nu_{1,\text{theo}} &=  (\num{30.82} \pm \num{0.05}) \, \unit{\kilo\hertz}, &
    \nu_{2,\text{theo}} &=  (\num{38.84} \pm \num{0.07}) \, \unit{\kilo\hertz}.
    \intertext{Dies entspricht den Abweichungen}
    \nonumber \frac{\nu_\text{res}}{\nu_\text{res,theo}} &= (\num{254.9} \pm \num{0.4}) \, \%, \\
    \frac{\nu_1}{\nu_\text{1,theo}} &= (\num{246.6} \pm \num{0.4}) \, \%, &
    \frac{\nu_2}{\nu_\text{2,theo}} &= (\num{244.6} \pm \num{0.4}) \, \%.
\end{align}


%
%entnehmen.
%Die jeweiligen aus den Gleichungen \textbf{REF!!!} berechneten theoretischen Werte betragen 
%\begin{align}
%    \nu_\text{res, theo} &= (\num{34.13} \pm \num{0.06}) \, \unit{\kilo\hertz}, &
%    \nu_{1,\text{theo}} &= , &
%    \nu_{2,\text{theo}} &= .
%\end{align}
%Dies entspricht den Abweichungen 
%\begin{align}
%    \frac{\nu_\text{res}}{\nu_\text{res,theo}} &= (\num{4.60} \pm \num{0.01}) \, \%, &
%    \frac{\nu_1}{\nu_\text{1,theo}} &= \, \%, &
%    \frac{\nu_\text{2}}{\nu_\text{2,theo}} &= \, \%.
%\end{align}
%

\begin{figure}[H]
    \centering
    \includegraphics{build/C03_c_d_phase_linear.pdf}
    \caption{Lineare Darstellung der Phasenverschiebung in Abhängigkeit der Frequenz.}
    \label{fig:plot_phase_linear}
\end{figure}