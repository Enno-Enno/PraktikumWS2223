\subsection{Dämpfungswiderstand beim aperiodischen Grenzfall}
Für den aperiodischen Grenzfall wird der Dämpfungswiderstand $R_\text{ap} = \qty{3.54}{\kilo\ohm}$ gemessen.
Gemäß Gleichung \eqref{eq:ap_bed} beträgt der erwartete Wert 
\begin{align}
    R_\text{ap,theo} = 2 \cdot \sqrt{\left(\frac{L}{C}\right)} =  (\num{4.396} \pm \num{0.007}) \, \unit{\kilo\ohm}.
\end{align}
Dies entspricht einer Abweichung von
\begin{align*}
    \frac{|R_\text{ap} - R_\text{ap,theo}|}{R_\text{ap,theo}} &= (\num[]{19.47} \pm \num[]{0.13}) \, \%.
\end{align*}
%abweichung =  -0.1947+/-0.0013