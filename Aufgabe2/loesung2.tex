\documentclass{scrartcl}

\usepackage{scrhack}

\usepackage[aux]{rerunfilecheck}

\usepackage[ngerman]{babel}

\usepackage[
locale=DE,
seperate-uncertainties=true,
per-mode-symbol-or-fraction]{}

\usepackage{fontspec}

\usepackage{amsmath}

\usepackage{amssymb}

\usepackage{mathtools}

\begin{document}

\section*{Nr. 1}

\subsection*{1.1 Der Mittelwert}
Der Mittelwert $\bar{x}$ bezeichnet den statistischen Durchschnittswert. 
Dabei summiert man alle Werte $x_k$ auf und teilt anschließend durch die Anzahl aller Werte $N$.

\begin{equation}
    \bar{x} = \frac{1}{N} \sum_{k=1}^{N} x_k
\end{equation}

\subsection*{1.2 Die Standardabweichung}
Die Standardabweichung $s$ ist ein Maß für die Abweichung bzw. Streuung der Messwerte vom Mittelwert.
Dabei bildet man die Summe der Abstandsquadrate und teilt durch $N-1$.

\begin{equation}
    s = \sqrt{\frac{1}{N-1} \sum_{k=1}^{N} \left(x_k - \bar{x}\right)^2 }
\end{equation}

\subsection*{1.3 Streuung der Messwerte vs Fehler des Mittelwerts}
Der Fehler des Mittelwerts $s_{\bar{x}}$ ist ein Maß für die Abweichung des Mittelwerts $\bar{x}$ vom stochastischen Erwartungswert $\mu$.
Dahingegen gibt die Streuung der Messewerte die Standardabweichung $s$ an,
d.h. sie beschreibt die Streuung der Messwerte um den Mittelwert $\bar{x}$.

\begin{equation}
    s_{\bar{x}}=\frac{s}{\sqrt{N}}
\end{equation}

\section*{Nr. 2}
Die Standardabweichung beträgt $\sigma_u = 10\,\mathrm{m/s}$.
Mit der Formel für die Standardabweichung des Mittelwerts ergibt sich:



\end{document}