\section{Auswertung}
\subsection{}

\begin{table}[H]
    \sisetup{table-format=3.0}
    \OverfullCenter{
    \begin{tabular}{S S SS SS SS SS SS}
        \toprule
        & & \multicolumn{2}{c}{\qty{1.5}{\liter\per\minute}} &
        \multicolumn{2}{c}{\qty{2.0}{\liter\per\minute}} &
        \multicolumn{2}{c}{\qty{3.0}{\liter\per\minute}} &
        \multicolumn{2}{c}{\qty{4.0}{\liter\per\minute}} &
        \multicolumn{2}{c}{\qty{5.0}{\liter\per\minute}} \\
        {$\theta/ \unit{\degree}   $} & {$\alpha/ \unit{\degree} $} & {$\Delta f_{\text{max}}  / \unit{\hertz}  $} & {$\Delta f_{\text{mean}}  / \unit{\hertz}  $} &
        {$\Delta f_{\text{max}}  / \unit{\hertz}  $}  &  {$\Delta f_{\text{mean}}  / \unit{\hertz}  $} & 
        {$\Delta f_{\text{max}}  / \unit{\hertz}  $}  &  {$\Delta f_{\text{mean}}  / \unit{\hertz}  $} & 
        {$\Delta f_{\text{max}}  / \unit{\hertz}  $}  &  {$\Delta f_{\text{mean}}  / \unit{\hertz}  $} & 
        {$\Delta f_{\text{max}}  / \unit{\hertz}  $}  &  {$\Delta f_{\text{mean}}  / \unit{\hertz}  $} \\
        \midrule
        15& 80 & -104 & -67 & -140 & -85 & -255 & -134 & -400 & -220 & -507 & -269\\
        30& 71 & 163 & 98 & 232 & 134 & 430 & 256 & 723 & 403 & 1048 & 562\\
        60& 55 & -280 & -171 & -410 & -232 & -732 & -439 & -1234 & -708 & -1740 & -1050\\   
        \bottomrule
    \end{tabular}}
    \caption{Die gemessenen Frequenzverschiebungen bei einer Schalllaufzeit von \qty{15.5}{\micro\second}}
\end{table}

\begin{table}[H]
    \centering
    \begin{tabular}{S|SSS}
        \toprule
        & {$v / \unit{\cm\per\second}$}& {$v / \unit{\cm\per\second}$}& {$v / \unit{\cm\per\second}$}\\
        \midrule
        {$\theta / \unit{\degree}$}   & 15 & 30 & 60 \\
        \midrule
        {W/\unit{\liter\per\minute}} &&&\\
         1.5                         & -17.4 & 13.5 & -13.4 \\
         2.0                         & -22.0 & 18.5 & -18.2 \\
         3.0                         & -34.7 & 35.4 & -34.4 \\
         4.0                         & -57.0 & 55.7 & -55.5 \\
         5.0                         & -69.7 & 77.7 & -82.4 \\
         \bottomrule
    \end{tabular}
    \caption{Die aus den Frequenzverschiebungen berechneten Mittleren Geschwindigkeiten unter Verwendung von $f_\text{mean}$.}
    \label{tab:geschwindigkeiten}
\end{table}
\begin{figure}
    \centering
    \includegraphics[width=0.75\textwidth]{build/01_plot.pdf}
    \caption{Die Frequenzverschiebungen $f_\text{max}$ und $f_\text{mean}$ zu den Verschiedenen Winkeln mit linearen Ausgleichsrechnungen.}
    \label{fig:frequenzen}
\end{figure}

In Abbildung \ref{fig:frequenzen} werden die Frequenzverschiebungen nach der eingestellten Strömungsgeschwindigkeit eingestellt.
Es kann ein linearer Zusammenhang zwischen Pumpleistung und und auf $\cos(\alpha)$ normierter Frequenzverschiebung gefunden werden.
Mit einer linearen Ausgleichsrechnung  der Form $a W +b$ ergibt sich für die gemessenen Werte folgende Parameter
\begin{align}
    a_\text{mean} &=  \qty{399  \pm 17}{\liter\per\minute\per\second} &
    b_\text{mean} &=  \qty{-340 \pm 85}{1\per\second}                 \\
    a_\text{max}  &=  \qty{733  \pm 27}{\liter\per\minute\per\second} &
    b_\text{max}  &=  \qty{-690 \pm 90}{1\per\second}                 
\end{align}
In Tabelle \ref{tab:geschwindigkeiten} werden die Geschwindigkeiten aus der mittleren Frequenzverschiebung mit
\begin{align}
    v = \frac{\Delta f  c_L }{2 f_0  \cos(\alpha)}
\end{align}
berechnet.

\subsection{Untersuchung des Strömungsprofils}
\begin{table}[H]
    \centering
    \sisetup{table-format=3.0}
    \begin{tabular}{SSSSSS}
        \toprule
        & & \multicolumn{2}{c}{$W = \qty{3}{\liter\per\minute}$} & \multicolumn{2}{c}{$W = \qty{3}{\liter\per\minute}$} \\
        {$T /\unit{\micro\second}$} & { $D / \unit{\mm}$} & {$\text{Signal}_3$}  & {$\text{speed}_3$} & {$\text{signal}_5$} & {$\text{speed}_5$} \\
        \midrule
        12.5 & 31.03 & 20   & 0  & 20  & 0   \\
        13.0 & 31.78 & 34   & 33 & 29  & 70  \\
        13.5 & 32.53 & 40   & 37 & 55  & 90  \\
        14.0 & 33.28 & 50   & 42 & 57  & 100 \\
        14.5 & 34.03 & 52   & 45 & 70  & 110 \\
        15.0 & 34.78 & 60   & 48 & 70  & 115 \\
        15.5 & 35.53 & 56   & 48 & 77  & 108 \\
        16.0 & 36.28 & 61   & 42 & 80  & 100 \\
        16.5 & 37.03 & 61   & 37 & 85  & 82  \\
        17.0 & 37.78 & 56   & 34 & 70  & 80  \\
        17.5 & 38.53 & 55   & 34 & 50  & 80  \\
        18.0 & 39.28 & 105  & 39 & 135 & 90  \\
        18.5 & 40.03 & 80   & 35 & 300 & 98  \\
        19.0 & 40.78 & 100  & 40 & 300 & 98  \\
        19.5 & 41.53 & 60   & 38 & 230 & 90  \\
        \bottomrule
    \end{tabular}
    \caption{Geschwindigkeitsprofil der Flüssigkeit in Abhängigkeit von $D$ }
    \label{tab:geschwindigkeitsprofil}
\end{table}
\begin{figure}[H]
    \centering
    \includegraphics{build/02_plot.pdf}
    \caption{Graphische Darstellung des Strömungsprofils}
    \label{fig:geschwindigkeitsprofil}
\end{figure}

In Tabelle \ref{tab:geschwindigkeitsprofil} werden die Geschwindigkeiten und die Signalstärken in Abhängigkeit von der
Strecke $D$ in dem Material aufgelistet.
Die Strecke $D$ wird aus den Schallgeschwindigkeiten in den Materialien berechnet.
Es entsprechen $\qty{4}{\micro\s} \simeq \qty{10}{\mm}$ im Acrylglas und $\qty{4}{\micro\s} \simeq \qty{6}{\mm}$ in der Flüssigkeit.
Der Kristall hat eine Tiefe von \qty{30.7}{\mm} bei einem Winkel von $\theta = \qty{15}{\degree}$ und $\alpha = \qty{80}{\degree}$
hat das Rohr eine diagonale Tiefe von
\begin{align}
    H = \frac{\qty{10}{\mm}}{\sin(\qty{80}{\degree})} = \qty{10.15}{mm}
\end{align}

Es sind also nur Werte von $D \leq \qty{40.85}{\mm}$ für das Strömungsprofil relevant.