\section{Auswertung}
\subsection{Geschwindigkeit der Flüssigkeit und das Dopplersignal}
\begin{figure}
    \centering
    \includegraphics[width=0.75\textwidth]{build/01_plot.pdf}
    \caption{Die Frequenzverschiebungen $f_\text{max}$ und $f_\text{mean}$ zu den Verschiedenen Winkeln mit linearen Ausgleichsrechnungen.}
    \label{fig:frequenzen}
\end{figure}

\begin{table}[H]
    \sisetup{table-format=3.0}
    \OverfullCenter{
    \begin{tabular}{S SSS SSS}
        \toprule
        {$\theta/ \unit{\degree} $} & 15 & 30   & 60     & 15 & 30 & 60 \\
        {$\alpha/ \unit{\degree} $} & 80 & 71   & 55     & 80 & 71 & 55 \\
        \midrule
         {$W/ (\unit{\liter\per\minute})$}  & \multicolumn{3}{c}{{$\Delta f_{\text{mean}}  / \unit{\hertz}  $}} & \multicolumn{3}{c}{{$\Delta f_{\text{max}}  / \unit{\hertz}  $}} \\
           \midrule
             1.5                  &  -67 &   98 &  -171               & -104 &  163 &  -280 \\
             2.0                  &  -85 &  134 &  -232               & -140 &  232 &  -410 \\
             3.0                  & -134 &  256 &  -439               & -255 &  430 &  -732 \\
             4.0                  & -220 &  403 &  -708               & -400 &  723 & -1234 \\
             5.0                  & -269 &  562 & -1050               & -507 & 1048 & -1740 \\
        \bottomrule
    \end{tabular}}
    \caption{Die gemessenen Frequenzverschiebungen bei einer Schalllaufzeit von \qty{15.5}{\micro\second}}
\end{table}

\begin{table}[H]
    \centering
    \begin{tabular}{S|SSS}
        \toprule
        {$\theta / \unit{\degree}$}   & 15 & 30 & 60 \\
        \midrule
        {W/\unit{\liter\per\minute}} & {$v / (\unit{\cm\per\second})$}& {$v / (\unit{\cm\per\second})$}& {$v / (\unit{\cm\per\second})$}\\
        \midrule
         1.5                         & -17.4 & 13.5 & -13.4 \\
         2.0                         & -22.0 & 18.5 & -18.2 \\
         3.0                         & -34.7 & 35.4 & -34.4 \\
         4.0                         & -57.0 & 55.7 & -55.5 \\
         5.0                         & -69.7 & 77.7 & -82.4 \\
         \bottomrule
    \end{tabular}
    \caption{Die aus den Frequenzverschiebungen berechneten Mittleren Geschwindigkeiten unter Verwendung von $f_\text{mean}$.}
    \label{tab:geschwindigkeiten}
\end{table}

\noindent
In Abbildung \ref{fig:frequenzen} werden die Frequenzverschiebungen nach der eingestellten Strömungsgeschwindigkeit eingestellt.
Es kann ein linearer Zusammenhang zwischen Pumpleistung und und auf $\cos(\alpha)$ normierter Frequenzverschiebung gefunden werden.
Mit einer linearen Ausgleichsrechnung  der Form $a W +b$ ergibt sich für die gemessenen Werte folgende Parameter
\begin{align}
    a_\text{mean} &=  \qty{399  \pm 17}{\liter\per\minute\per\second} &
    b_\text{mean} &=  \qty{-340 \pm 85}{1\per\second}                 \\
    a_\text{max}  &=  \qty{733  \pm 27}{\liter\per\minute\per\second} &
    b_\text{max}  &=  \qty{-690 \pm 90}{1\per\second}                 
\end{align}
In Tabelle \ref{tab:geschwindigkeiten} werden die Geschwindigkeiten aus der mittleren Frequenzverschiebung mit
\begin{align}
    v = \frac{\Delta f  c_L }{2 f_0  \cos(\alpha)}
\end{align}
berechnet.

\subsection{Untersuchung des Strömungsprofils}
\begin{table}
    \centering
    \sisetup{table-format=3.0}
    \begin{tabular}{}
        {}
        12.5 & 20 & 0 & 20 & 0
        13.0 & 34 & 33 & 29 & 70
        13.5 & 40 & 37 & 55 & 90
        14.0 & 50 & 42 & 57 & 100
        14.5 & 52 & 45 & 70 & 110
        15.0 & 60 & 48 & 70 & 115
        15.5 & 56 & 48 & 77 & 108
        16.0 & 61 & 42 & 80 & 100
        16.5 & 61 & 37 & 85 & 82
        17.0 & 56 & 34 & 70 & 80
        17.5 & 55 & 34 & 50 & 80
        18.0 & 105 & 39 & 135 & 90
        18.5 & 80 & 35 & 300 & 98
        19.0 & 100 & 40 & 300 & 98
        19.5 & 60 & 38 & 230 & 90
    \end{tabular}
\end{table}
\begin{figure}[H]
    \centering
    \includegraphics{build/02_plot.pdf}
    \caption{Graphische Darstellung des Strömungsprofils}
    \label{fig:geschwindigkeitsprofil}
\end{figure}

In Tabelle \ref{tab:geschwindigkeitsprofil} werden die Geschwindigkeiten und die Signalstärken in Abhängigkeit von der
Strecke $D$ in dem Material aufgelistet.
Die Strecke $D$ wird aus den Schallgeschwindigkeiten in den Materialien berechnet.
Es entsprechen $\qty{4}{\micro\s} \simeq \qty{10}{\mm}$ im Acrylglas und $\qty{4}{\micro\s} \simeq \qty{6}{\mm}$ in der Flüssigkeit.
Der Kristall hat eine Tiefe von \qty{30.7}{\mm} bei einem Winkel von $\theta = \qty{15}{\degree}$ und $\alpha = \qty{80}{\degree}$
hat das Rohr eine diagonale Tiefe von
\begin{align}
    H = \frac{\qty{10}{\mm}}{\sin(\qty{80}{\degree})} = \qty{10.15}{mm}
\end{align}

Es sind also nur Werte von $D \leq \qty{40.85}{\mm}$ für das Strömungsprofil relevant.