\section{Diskussion}

\subsection{Flussgeschwindigkeit}
Es wurde gezeigt, dass ein linearer Zusammenhang zwischen Pumpleistung und Frequenzverschiebung existiert.
In Tabelle \ref{tab:geschwindigkeiten} ist darüber hinaus erkennbar, dass der Betrag der Strömungsgeschwindigkeit $v$ 
für jeden Winkel $\theta$ mit der Fließgeschwindigkeit $W$ steigt.
Die entgegen gesetzten Vorzeichen lassen sich dabei durch die \enquote{Blickrichtung} begründen, von der aus beim Prisma gemessen wird,
vgl. Abbildung \ref{fig:prisma}.


\subsection{Strömungsprofil}
In Abbildung \ref{fig:geschwindigkeitsprofil} ist erkennbar, dass für eine größere $W$ höhere Momentangeschwindigkeiten vorhanden sind.
Für die Tiefe \qty{31.03}{\milli\meter} sind noch keine Geschwindigkeiten vorhanden.
Es lässt sich darauf schließen, dass durch die Ausdehnung des Kontaktgels tatsächlich nicht in der Flüssigkeit, sondern in der Rohrhülle gemessen wurde.
Vernachlässigt man diesen Wert, ergeben sich durchschnittliche Werte von $v_3 = \qty{37+-11}{\centi\meter\per\second}$
und $v_5 = \qty{94+-12}{\centi\meter\per\second}$.
Aufgrund der nicht allzu großen Standardabweichungen dieser Werte kann von laminaren Strömungen ausgegangen werden.
Ferner ist erkennbar, dass für beide Fließgeschwindigkeiten die Maxima in einer Tiefe von ca. \qty{34.78}{\milli\meter} liegen, 
was nahe am Mittelpunkt des Rohres liegt.
Bei der höheren Strömungsgeschwindigkeit $W$ ist dies deutlicher erkennbar als bei einer Leistung von $45 \%$.

\noindent
Wird die Streuintensität betrachtet, so ist erkennbar, dass je tiefer gemessen wird der Dopplereffekt umso größer ist.
Dies wird insbesondere bei der höheren Pumpleistung deutlich.

\subsection{Fehlerabschätzung}
Da die Messgeräte und Instrumente dieses Versuchs keine Fehlerangaben haben und das Ablesen ausschließlich anhand von digitalen Werten
erfolgt,
ist es nicht möglich eine Fehlerabschätzung über die statistischen Abweichungen, die beispielsweise aus der Ausgleichsrechnung folgen,
hinaus vorzunehmen.
Ferner gibt es keine Literaturwerte für das hier verwendete Gemisch bzw. die ermittelten Werte, sodass eine Beurteilung der
Messergebnisse nicht möglich ist.