
    \begin{table}[H]
        \sisetup{table-format=3.0}
        \begin{tabular}{S S SS SS SS SS SS}
            \toprule
            & & \multicolumn{2}{c}{\qty{1.5}{\liter\per\minute}} &
            \multicolumn{2}{c}{\qty{2.0}{\liter\per\minute}} &
            \multicolumn{2}{c}{\qty{3.0}{\liter\per\minute}} &
            \multicolumn{2}{c}{\qty{4.0}{\liter\per\minute}} &
            \multicolumn{2}{c}{\qty{5.0}{\liter\per\minute}} \\
            {$\theta/ \unit{\degree}$} & {$\alpha/ \unit{\degree}$} & {$\Delta f_{\text{max}} $} & {$\Delta f_{\text{mean}} $} &
            {$\Delta f_{\text{max}} $}  &  {$\Delta f_{\text{mean}} $} & 
            {$\Delta f_{\text{max}} $}  &  {$\Delta f_{\text{mean}} $} & 
            {$\Delta f_{\text{max}} $}  &  {$\Delta f_{\text{mean}} $} & 
            {$\Delta f_{\text{max}} $}  &  {$\Delta f_{\text{mean}} $} \\
            \midrule
            15& 80 & -104 & -67 & -140 & -85 & -255 & -134 & -400 & -220 & -507 & -269\\
            30& 71 & 163 & 98 & 232 & 134 & 430 & 256 & 723 & 403 & 1048 & 562\\
            60& 55 & -280 & -171 & -410 & -232 & -732 & -439 & -1234 & -708 & -1740 & -1050\\   
            \bottomrule
        \end{tabular}
        \caption{Die gemessenen Frequenzverschiebungen bei einer Schalllaufzeit von \qty{15.5}{\micro\seconds}}
    \end{table}