\section{Durchführung}
In diesem Versuch wird ein Halbleiter-Sperrschichtzähler in einer Röhre mit einem festen Abstand von zunächst
$x_0 = \qty{6}{\cm}$ zu einem $\alpha$-Strahler eingestellt.
Dieser Zähler kann jedem Impuls einen Wert für die Energie zuordnen.
Der hier verwendete $\alpha$-Strahler ist ein Am-Präparat mit einer Halbwertszeit von $T_{1/2} = \num{458}$\,a.
\begin{align*}
    \ce{^{241}_{95}Am} \rightarrow \ce{^{237}_{93}Np}+ \ce{^{4}_{2}He^{++}}
\end{align*}
Mit dem Computerprogramm \enquote{Multi Channel Analyzer} werden die gemessenen Impulse 
nach ihrer Energie in einem Histogramm aufgestellt.
Die Röhre wird evakuiert das Histogramm wird für \qty{120}{\s} aufgenommen. 
Die Position der häufigsten Impulsenergie und die gesamte Zählrate wird notiert.
Das Maximum dieses Histogramms sollte bei etwa \qty{4}{\mega\electronvolt} liegen.
Aus diesem Wert wird auf die Energien der anderen Channels geschlossen.
Der Luftdruck in der Röhre wird nun in \qty{50}{\milli\bar} Schritten erhöht.
bei jedem Druck wird dieses Histogram von neuem aufgenommen.
Auf diese Weise lässt sich die mittlere Energie des $\alpha$-Teilchens in Abhängigkeit des Drucks p ermitteln.
Die Messungen werden anschließend noch einmal für die Entfernung $x_0 = \qty{7}{\cm}$ wiederholt.

\subsection{Statistik des Radioaktiven Zerfalls}
Bei evakuiertem Glaszylinder wird die Messrate über \qty{10}{\s} 100 Mal gemessen. 
Aus diesen Messungen wird ein Histogramm generiert um herauszufinden in welcher Form die Zerfälle Wahrscheinlichkeitsverteilt sind.