\section{Diskussion}
In den Messreihen konnten Reichweiten von \qty{2.03 \pm 0.06}{\cm}(bei $x_0 = \qty{6}{\cm}$)  und 
\qty{1.54 \pm 0.06}{\cm} (bei $x_0 = \qty{7}{\cm}$) ermittelt werden.
Die Fehlerbereiche dieser Messungen überlappen nicht.
Das kann mit verschiedenen systematischen Messfehlern zusammenhängen.
Die Skala für die Energie kann verschoben gewesen sein, sodass die Histogrammeinträge
konsistent nicht den richtigen Energien zugeordnet wurden.
Ansonsten kann es noch zu weiteren systematischen Verschiebungen gekommen sein, die hier nicht beachtet wurden.

In Abbildung \ref{fig:stat} wird die gemessene Verteilung mit zwei Wahrscheinlichkeitsverteilungen verglichen.
Die Gaußverteilung ist etwas ähnlicher zu den Messwerten als die Poissonverteilung.
Es ist allerdings aufgrund von starken Schwankungen nicht eindeutig zu erkennen, welche Verteilung besser geeignet ist um
die Messwerte zu modellieren.