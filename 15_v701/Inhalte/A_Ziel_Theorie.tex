\section{Ziel}
In diesem Versuch soll die Reichweite von $\alpha$-strahlung in der Luft ermittelt werden.

\section[Theorie]{Theorie\footnote[1]{Unter Verwendung von \cite{man:v701}.}}
Die Energie von $\alpha$-Strahlung wird durch die Reichweite dieser Strahlung in der Luft gemessen.
Der Energieverlust der $\alpha$-Teilchen wird durch die Bethe-Bloch-Gleichung beschrieben
\begin{align}
    -\frac{\diff E_\alpha}{\diff x} = \frac{z^2 e^4}{4 \pi \epsilon_0^2 m_e} \frac{n Z}{v^2} \ln\left(\frac{2 m_e v^2}{I}\right).
    %% Hier mit \epsilon_0^2 nicht ganz gleich dem was in der Anleitung stand aber Laurins Tipp.
\end{align}%
                                                        %%%  ... Falsch!
Hierbei ist $z$ die Ladungszahl und $e$ die Elementarladung. %und $Z$ die Ordnungszahl des $\alpha$ Teilchens.
$v$ ist die Geschwindigkeit der Teilchen, $n$ die Teilchendichte und $I$ die mittlere Anregungsenergie der Luftmoleküle.

Die mittlere Reichweite des $\alpha$-Teilchens wird durch 
\begin{align}
    R = \int_0^{E_\alpha} \frac{\diff E_\alpha}{-\diff E_\alpha / \diff}
\end{align}
berechnet.
Die Bethe-Bloch-Gleichung ist für niedrige Energien $E \leq \qty{2.5}{\mega\electronvolt}$ nicht mehr aussagekräftig.
Für diesen Bereich kann auch die Beziehung 
\begin{align}
    $R_m = 3.1 \cdot E_\alpha^{3/2}$ 
    \label{eq:reichweite}
\end{align}
verwendet werden.
Weil die mittlere Reichweite der $\alpha$-Teilchen proportional zu dem Luftdruck ist kann man die Reichweite auch herausfinden,
indem man den Druck variiert.
Bei einem Festen Abstand $x_0$ zwischen Detektor $\alpha$-Strahler kann die effektive Länge $x$ wie folgt errechnet werden
\begin{align}
    x = x_0 \frac{p}{p_0}.
    \label{eq:effektive_laenge}
\end{align}
wobei für $p_0 = \qty{1013}{\milli\bar}$ eingesetzt werden muss.



\section{Vorbereitung}
Ein Halbleiter-Sperrschichtzähler ist eine Form des Strahlungsdetektors, der sich die Eigenschaften eines Halbleiters zunutze macht.
Wenn dieser Halbleiter mit einem Ion getroffen wird, wird er leitend. 
Mit der passenden Analyse kann die Energie der Ionen herausgefunden werden \cite[vgl.]{man:v701}.
