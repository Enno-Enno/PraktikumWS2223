\section{Auswertung}


\subsection{Mittelwerte und Fehler}
Das arithmetische Mittel $\overline{c}$ und die Standardabweichung $\Delta c$ einer Messreihe mit $N$ Werten $c_k$ errechnet sich gemäß der Formeln
\begin{align}
    \overline{c} &= \frac{1}{N} \sum_{k=1}^{N} c_k, & \Delta c = \sqrt{ \sum_{k=1}^{N} \left(\overline{c} - c_k \right)^2 }.
    \label{eq:mittelstand}
\end{align}

\subsection{Gaußsche Fehlerfortpflanzung}
Wenn zu Messdaten die Standardabweichung bekannt ist, und mit diesen Messdaten weiter gerechnet werden soll,
wird die Gaußsche Fehlerfortpflanzung verwendet. 
Angenommen, es gibt $k$ Messwerte $x_i [i \in \mathbb{N}, i \leq k]$ mit den Standardabweichungen $\Delta x_i$
und eine abgeleitete Größe $f(x_i)$.
Dann ist der Fehler von $f$
\begin{align}
    \Delta f(x_i) = \sqrt{
    \left(\frac{\partial f}{\partial x_1} \Delta x_1\right)^2%
     + \left(\frac{\partial f}{\partial x_2} \Delta x_2\right)^2%
     + \dots%
     + \left(\frac{\partial f}{\partial x_k} \Delta x_k\right)^2%
    }.
    \label{eq:gauss}
\end{align} 
Im Ergebnis ergibt sich der Mittelwert von $f$ mit der errechneten Abweichung $\overline{f} \pm \Delta f $.
Um Rechenfehler zu vermeiden, wird das Python \cite[]{python} Paket \texttt{uncertainties} \cite[][]{uncertainties} verwendet.
Hier wird die Fehlerfortpflanzung automatisch verrechnet, wenn die Variablen als \texttt{ufloat} definiert werden.

\subsection{Reichweite der Alpha-Strahlung}
In den beiden Messreihen errechnet sich die effektive Länge $x$ gemäß Gleichung \textbf{REF!!! EFFEKTIVE LÄNGE}.
Da für den Druck jeweils ein Ablesefehler von \qty[]{5}{\milli\bar} angenommen wird, ergibt sich somit ein Fehler von 
\qty[]{3}{\mm} auf die Länge.
Für die Zählrate gibt es jeweils den Fehler $\Delta N = \sqrt{N}$.
Um von der Position \enquote{Channel} des Energiemaximums auf die entsprechende Energie zu kommen, wird der Dreisatz angewandt.
Dabei entspricht die Energie bei \qty{0}{\milli\bar} etwa \qty[]{4}{\mega\electronvolt}. 


\subsubsection[]{Erste Messreihe: 6 cm}
Die gemessenen und resultierenden Messgrößen bei einem Abstand von \qty[]{6}{\cm} sind in Tabelle \ref{tab:6cm} zu sehen.
Da die am Messgerät eingestellte Schwelle zwischen den Channels 696 und 697 liegt, werden die entsprechenden Werte
in der Auswertung nicht weiter berücksichtigt, um keine Ergebnisse zu verfälschen.
In der Tabelle sind die vernachlässigten Werte eingeklammert.

\begin{table}[H]
    \centering
    \caption{Druck $p$, effektive Länge $x$, Channel $C$, Energie $E$ sowie Zählrate $N$ bei einem Abstand von \qty[]{6}{\cm}.}
    \label{tab:6cm}
    \begin{tabular}{
        S[table-format=3.0] %druck
        S[table-format = 1.2] % x = eff laenge
        S[table-format=3.0] @{${}\pm{}$} S[table-format = 2.0] %N +- Fehler
        S[table-format=3.0] %channel
        S[table-format=1.2] % energie
    }
    \toprule
    {$p / \unit[]{\milli\bar}$} & {$x / \unit[]{\cm}$}
    & \multicolumn{2}{c}{$N / (1/\unit[]{\second})$} 
    & {$C$} & {$E / \unit[]{\mega\electronvolt}$} \\
    \midrule
     0  & 0.00 & 164 & 13 & 768 & 4.00 \\ 
     50 & 0.30 & 151 & 12 & 830 & 4.32 \\
    100 & 0.59 & 153 & 12 & 824 & 4.29 \\
    150 & 0.89 & 159 & 13 & 783 & 4.08 \\ 
    200 & 1.18 & 136 & 12 & 775 & 4.04 \\
    260 & 1.54 & 132 & 11 & 754 & 3.93 \\
    300 & 1.78 & 128 & 11 & 719 & 3.74 \\
    350 & 2.07 &  81 &  9 & 699 & 3.64 \\
    400 & 2.37 &  30 &  5 & {$(697)$} & {$(\num{3.63})$} \\
    450 & 2.67 &  14 &  4 & {$(697)$} & {$(\num{3.63})$} \\ 
    500 & 2.96 &   7 &  3 & {$(696)$} & {$(\num{3.62})$} \\
    560 & 3.32 &   6 &  2 & {$(696)$} & {$(\num{3.62})$} \\ 
    600 & 3.55 &   1 &  1 & {$(696)$} & {$(\num{3.62})$} \\
    650 & 3.85 &   2 &  2 & {$(696)$} & {$(\num{3.62})$} \\
    \bottomrule     
    \end{tabular}
\end{table}

\noindent
Wird $N$ gegen $x$ geplottet, ergibt sich Abbildung \ref{fig:rate_6cm}.
Mit Hilfe der Funktion \texttt{ODR} aus dem Python Paket \texttt{scipy} \cite[]{scipy}, 
die auf der Methode der kleinsten Quadrate beruht, 
wird eine Ausgleichsrechnung gemäß
\begin{align}
    N = a \cdot x + b
    \label{eq:lin}
\end{align}
durchgeführt.
Es ergeben sich die Parameter $a = \qty{-42.15 +- 6.69}{\per\cm\per\second}$ und $b = \qty{150.59 +- 22.35}{\per\second}$.
Hieraus lässt sich die mittlere Reichweite
\begin{align}
    R_\text{m} = \frac{N_\text{max} / 2 - b}{a} = \qty{1.63 +- 0.15}{\cm}
\end{align}
mit der maximalen Zählrate aus Tabelle \ref{tab:6cm} ermitteln.
Dies entspricht gemäß Gleichung \textbf{REF!!! ENERGIE ZUR REICHWEITE} der Energie $E_\alpha = \qty{3.02 +- 0.19}{\mega\electronvolt}$.
% mittlere_reichweite =  1.63+/-0.15 cm
% zugehörige Energie =  3.02+/-0.19 MeV


\begin{figure}[H]
    \centering
    \includegraphics[height = 8.5cm]{build/reichweite_6cm_rate.pdf}
    \caption[short]{Die Zählrate $N$ als Funktion der effektiven Länge $x$ beim Abstand von \qty{6}{\cm}.}
    \label{fig:rate_6cm}
\end{figure}

\noindent


Wird die Energie $E$ gegen die effektive Länge $x$ geplottet, ergibt sich Abbildung \ref{fig:energie_6cm.pdf}.
Hier ergibt eine lineare Ausgleichsrechnung analog zu \eqref{eq:lin} die Parameter $a = \qty{-0.260 +- 0.080}{\mega\electronvolt\per\cm}$
und $b = \qty{4.28 +- 0.10}{\mega\electronvolt}$.
Der Energieverlust beträgt also 
\begin{align}
    - \frac{dE}{dx} = -a = \qty{260 +- 80}{\kilo\electronvolt\per\cm}.
\end{align}

% a = -0.260 +- 0.080
% b = 4.277 +- 0.100
% energieverlust = (0.00260 +- 0.00080) MeV/m
% energieverlust = (260.29 +- 79.79) keV/cm

\begin{figure}[H]
    \centering
    \includegraphics[height = 8.5cm]{build/reichweite_6cm_energymax.pdf}
    \caption[short]{Die Energie $E$ als Funktion der effektiven Länge $x$ beim Abstand von \qty{6}{\cm}.}
    \label{fig:energie_6cm}
\end{figure}



\subsection[]{Zweite Messreihe: 7 cm}
