\section{Auswertung}

\subsection{Gaußsche Fehlerfortpflanzung}

Wenn zu Messdaten die Standardabweichung bekannt ist, und mit diesen Messdaten weiter gerechnet werden soll,
wird die Gaußsche Fehlerfortpflanzung verwendet. 
Angenommen, es gibt $k$ Messwerte $x_i [i \in \mathbb{N}, i \leq k]$ mit den Standardabweichungen $\Delta x_i$
und eine abgeleitete Größe $f(x_i)$.
Dann ist der Fehler von $f$
\begin{align}
    \Delta f(x_i) = \sqrt{
    \left(\frac{\partial f}{\partial x_1} \Delta x_1\right)^2%
     + \left(\frac{\partial f}{\partial x_2} \Delta x_2\right)^2%
     + \dots%
     + \left(\frac{\partial f}{\partial x_k} \Delta x_k\right)^2%
    }.
    \label{eq:gauspflanz}
\end{align} 
Im Ergebnis ergibt sich der Mittelwert von $f$ mit der errechneten Abweichung $\overline{f} \pm \Delta f $.
Um Rechenfehler zu vermeiden, wird das Python \cite[]{python} Paket \texttt{uncertainties} \cite[][]{uncertainties} verwendet.
Hier wird die Fehlerfortpflanzung automatisch verrechnet, wenn die Variablen als \texttt{ufloat} definiert werden.

\subsection{Reichweite der Alpha-Strahlung}
In den beiden Messreihen errechnet sich die effektive Länge $x$ gemäß Gleichung \ref{eq:effektive_laenge}.
Da für den Druck jeweils ein Ablesefehler von \qty[]{5}{\milli\bar} angenommen wird, ergibt sich somit ein Fehler von 
\qty[]{3}{\mm} auf die Länge.
Für die Zählrate gibt es jeweils den Fehler $\Delta N = \sqrt{N}$.
Um von der Position \enquote{Channel} des Energiemaximums auf die entsprechende Energie zu kommen, wird der Dreisatz angewandt.
Dabei entspricht die Energie bei \qty{0}{\milli\bar} etwa \qty[]{4}{\mega\electronvolt}. 


\subsubsection[]{Erste Messreihe: 6 cm}
Die gemessenen und resultierenden Messgrößen bei einem Abstand von \qty[]{6}{\cm} sind in Tabelle \ref{tab:6cm} zu sehen.
Da die am Messgerät eingestellte Schwelle zwischen den Channels 696 und 697 liegt, werden die entsprechenden Werte
in der Auswertung nicht weiter berücksichtigt, um keine Ergebnisse zu verfälschen.
In der Tabelle sind die vernachlässigten Werte eingeklammert.

\begin{table}[H]
    \centering
    \caption{Druck $p$, effektive Länge $x$, Channel $C$, Energie $E$ sowie Zählrate $N$ bei einem Abstand von \qty[]{6}{\cm}.}
    \label{tab:6cm}
    \begin{tabular}{
        S[table-format=3.0] %druck
        S[table-format = 1.2] % x = eff laenge
        S[table-format=3.0] @{${}\pm{}$} S[table-format = 2.0] %N +- Fehler
        S[table-format=3.0] %channel
        S[table-format=1.2] % energie
    }
    \toprule
    {$p / \unit[]{\milli\bar}$} & {$x / \unit[]{\cm}$}
    & \multicolumn{2}{c}{$N / (1/\unit[]{\second})$} 
    & {$C$} & {$E / \unit[]{\mega\electronvolt}$} \\
    \midrule
     0  & 0.00 & 164 & 13 & 768 & 4.00 \\ 
     50 & 0.30 & 151 & 12 & 830 & 4.32 \\
    100 & 0.59 & 153 & 12 & 824 & 4.29 \\
    150 & 0.89 & 159 & 13 & 783 & 4.08 \\ 
    200 & 1.18 & 136 & 12 & 775 & 4.04 \\
    260 & 1.54 & 132 & 11 & 754 & 3.93 \\
    300 & 1.78 & 128 & 11 & 719 & 3.74 \\
    350 & 2.07 &  81 &  9 & 699 & 3.64 \\
    400 & 2.37 &  30 &  5 & {$(697)$} & {$(\num{3.63})$} \\
    450 & 2.67 &  14 &  4 & {$(697)$} & {$(\num{3.63})$} \\ 
    500 & 2.96 &   7 &  3 & {$(696)$} & {$(\num{3.62})$} \\
    560 & 3.32 &   6 &  2 & {$(696)$} & {$(\num{3.62})$} \\ 
    600 & 3.55 &   1 &  1 & {$(696)$} & {$(\num{3.62})$} \\
    650 & 3.85 &   2 &  2 & {$(696)$} & {$(\num{3.62})$} \\
    \bottomrule     
    \end{tabular}
\end{table}

\noindent
Wird $N$ gegen $x$ geplottet, ergibt sich Abbildung \ref{fig:rate_6cm}.
Mit Hilfe der Funktion \texttt{ODR} aus dem Python Paket \texttt{scipy} \cite[]{scipy}, 
die auf der Methode der kleinsten Quadrate beruht, 
wird eine Ausgleichsrechnung gemäß
\begin{align}
    N = a \cdot x + b
    \label{eq:lin}
\end{align}
durchgeführt.
Es ergeben sich die Parameter $a = \qty{-42.15 +- 6.69}{\per\cm\per\second}$ und $b = \qty{150.59 +- 22.35}{\per\second}$.
Hieraus lässt sich die mittlere Reichweite
\begin{align}
    R_\text{m} = \frac{N_\text{max} / 2 - b}{a} = \qty{1.63 +- 0.15}{\cm}
\end{align}
mit der maximalen Zählrate aus Tabelle \ref{tab:6cm} ermitteln.
Dies entspricht gemäß Gleichung \ref*{eq:reichweite} der Energie $E_\alpha = \qty{3.02 +- 0.19}{\mega\electronvolt}$.
% mittlere_reichweite =  1.63+/-0.15 cm
% zugehörige Energie =  3.02+/-0.19 MeV


\begin{figure}[H]
    \centering
    \includegraphics[height = 8.5cm]{build/reichweite_6cm_rate.pdf}
    \caption[]{Die Zählrate $N$ als Funktion der effektiven Länge $x$ beim Abstand von \qty{6}{\cm}.}
    \label{fig:rate_6cm}
\end{figure}

\noindent
Wird die Energie $E$ gegen die effektive Länge $x$ geplottet, ergibt sich Abbildung \ref{fig:energie_6cm}.
Hier ergibt eine lineare Ausgleichsrechnung analog zu \eqref{eq:lin} die Parameter $a = \qty{-0.260 +- 0.080}{\mega\electronvolt\per\cm}$
und $b = \qty{4.28 +- 0.10}{\mega\electronvolt}$.
Der Energieverlust beträgt also 
\begin{align}
    - \frac{dE}{dx} = -a = \qty{260 +- 80}{\kilo\electronvolt\per\cm}.
\end{align}

% a = -0.260 +- 0.080
% b = 4.277 +- 0.100
% energieverlust = (0.00260 +- 0.00080) MeV/m
% energieverlust = (260.29 +- 79.79) keV/cm

\begin{figure}[H]
    \centering
    \includegraphics[height = 8.5cm]{build/reichweite_6cm_energymax.pdf}
    \caption[]{Die Energie $E$ als Funktion der effektiven Länge $x$ beim Abstand von \qty{6}{\cm}.}
    \label{fig:energie_6cm}
\end{figure}



\subsection[]{Zweite Messreihe: 7 cm}
Die Auswertung verläuft komplett analog zur ersten Messreihe.
Die Zählrate geht bei diesem Abstand schneller gegen Null, sodass es hier keine Konflikte mit der am Messgerät eingestellten Schwelle gibt.
Es können also alle Werte in der Auswertung verwendet werden.
Die Messgrößen sind in Tabelle \ref{tab:7cm} zu sehen.

\begin{table}[H]
    \centering
    \caption{Druck $p$, effektive Länge $x$, Channel $C$, Energie $E$ sowie Zählrate $N$ bei einem Abstand von \qty[]{7}{\cm}.}
    \label{tab:7cm}
    \begin{tabular}{
        S[table-format=3.0] %druck
        S[table-format = 1.2] % x = eff laenge
        S[table-format=3.0] @{${}\pm{}$} S[table-format = 2.0] %N +- Fehler
        S[table-format=3.0] %channel
        S[table-format=1.2] % energie
    }
    \toprule
    {$p / \unit[]{\milli\bar}$} & {$x / \unit[]{\cm}$}
    & \multicolumn{2}{c}{$N / (1/\unit[]{\second})$} 
    & {$C$} & {$E / \unit[]{\mega\electronvolt}$} \\
    \midrule
     0  & 0.00 & 110 & 10 & 928 & 4.00 \\ 
     50 & 0.30 & 106 & 10 & 879 & 3.79 \\ 
    100 & 0.59 & 103 & 10 & 855 & 3.69 \\ 
    150 & 0.89 &  97 & 10 & 824 & 3.55 \\ 
    200 & 1.18 &  84 &  9 & 768 & 3.31 \\ 
    250 & 1.48 &  65 &  8 & 715 & 3.08 \\ 
    300 & 1.78 &  33 &  6 & 719 & 3.10 \\ 
    350 & 2.07 &   4 &  2 & 713 & 3.07 \\ 
    \bottomrule     
    \end{tabular}
\end{table}

\noindent
In Abbildung \ref{fig:rate_7cm} ist eine graphische Darstellung zwischen $x$ und $N$ zu sehen.
In diesem Fall ergeben sich die Parameter $a = \qty{-62.23 +- 6.85}{\per\cm\per\second}$ und $b= \qty{136.86 +- 12.44}{\per\second}$.
Daraus resultieren die Werte $R_\text{m} = \qty{1.32+-0.08}{\cm}$ und $E_α = \qty{2.62+-0.11}{\mega\electronvolt}$.

\begin{figure}[H]
    \centering
    \includegraphics[height = 8.5cm]{build/reichweite_7cm_rate.pdf}
    \caption[]{Die Zählrate $N$ als Funktion der effektiven Länge $x$ beim Abstand von \qty{7}{\cm}.}
    \label{fig:rate_7cm}
\end{figure}

\noindent
Der $x$-$E$-Plot ist in Abbildung \ref{fig:energie_7cm} zu sehen. 
Hier ergeben sich die Parameter $a = \qty{-0.482 +- 0.046}{\mega\electronvolt\per\cm}$ und $b= \qty{3.948 +- 0.058}{\mega\electronvolt}$.
Daraus folgt der Energieverlust $- {dE}/{dx} = \qty{482 +- 46}{\kilo\electronvolt\per\cm}$.


\begin{figure}[H]
    \centering
    \includegraphics[height = 8.5cm]{build/reichweite_7cm_energymax.pdf}
    \caption[]{Die Energie $E$ als Funktion der effektiven Länge $x$ beim Abstand von \qty{7}{\cm}.}
    \label{fig:energie_7cm}
\end{figure}





\subsection[]{Statistik des Zerfalls}
In Tabelle \ref{tab:stat} sind die gemessenen Zerfälle je 10 Sekunden bei einem Druck von \qty{0}{\milli\bar} zu sehen.

\begin{table}[H]
    \centering
    \caption{Intervallnummer $k$ und Anzahl der Zerfälle $N$ bei \qty{0}{\milli\bar}.}
    \label{tab:stat}
    \sisetup{table-format=2.0}
    \begin{tabular}{
        S %i von  1- 25
        S[table-format=4.0] %N von  1- 25
        S %i von 26- 50
        S[table-format=4.0] %N von 26- 50        
        S %i von 51- 75
        S[table-format=4.0] %N von 51- 75
        S[table-format=3.0] %i von 76-100
        S[table-format=4.0] %N von 76-100
    }
    \toprule
    {$k$} & {$N [1/(10\unit[]{\second})]$}
    & {$k$} & {$N [1/(10\unit[]{\second})]$}
    & {$k$} & {$N [1/(10\unit[]{\second})]$}
    & {$k$} & {$N [1/(10\unit[]{\second})]$} \\
    \cmidrule(lr){1-2}\cmidrule(lr){3-4}\cmidrule(lr){5-6}\cmidrule(lr){7-8}
     1 &  905 & 26 & 1022 & 51 &  941 &  76 & 1070 \\
     2 & 1017 & 27 & 1009 & 52 & 1027 &  77 &  929 \\
     3 &  983 & 28 & 1037 & 53 & 1014 &  78 &  950 \\
     4 & 1075 & 29 & 1035 & 54 &  980 &  79 &  913 \\
     5 &  955 & 30 &  974 & 55 & 1040 &  80 &  997 \\
     6 & 1004 & 31 & 1005 & 56 &  987 &  81 & 1044 \\
     7 &  916 & 32 &  985 & 57 &  995 &  82 &  965 \\
     8 &  952 & 33 & 1010 & 58 & 1082 &  83 & 1029 \\
     9 & 1021 & 34 &  936 & 59 &  957 &  84 &  981 \\
    10 & 1008 & 35 &  949 & 60 & 1066 &  85 &  956 \\
    11 &  989 & 36 &  964 & 61 & 1024 &  86 &  956 \\
    12 &  933 & 37 &  947 & 62 & 1002 &  87 &  964 \\
    13 & 1065 & 38 & 1057 & 63 & 1095 &  88 & 1023 \\
    14 & 1015 & 39 & 1069 & 64 & 1029 &  89 &  995 \\
    15 & 1038 & 40 & 1015 & 65 & 1032 &  90 & 1073 \\
    16 & 1043 & 41 & 1001 & 66 & 1007 &  91 & 1009 \\
    17 &  973 & 42 &  962 & 67 &  968 &  92 &  959 \\
    18 &  931 & 43 &  912 & 68 &  920 &  93 & 1008 \\
    19 & 1029 & 44 &  960 & 69 & 1011 &  94 & 1017 \\
    20 &  985 & 45 & 1002 & 70 & 1036 &  95 & 1025 \\
    21 & 1016 & 46 &  927 & 71 &  972 &  96 &  979 \\
    22 & 1020 & 47 &  973 & 72 &  977 &  97 &  949 \\
    23 & 1004 & 48 &  954 & 73 & 1037 &  98 & 1013 \\
    24 &  967 & 49 & 1028 & 74 &  951 &  99 &  956 \\
    25 &  977 & 50 & 1065 & 75 &  960 & 100 &  993 \\
    \bottomrule     
    \end{tabular}
\end{table}

\noindent
Es ergeben sich der Mittelwert $\overline{N} = 995 /(10\unit[]{\second})$ mit der Standardabweichung $\Delta N = 43/(10\unit[]{\second})$.
Anhand dieser Werte wird ein Histogramm erstellt, in dem zusätzlich eine Gauß- und eine Poisson-Verteilung zu sehen sind.
Das Histogramm ist in Abbildung \ref{fig:stat} zu sehen.
% ohne wurzel: 994.82 +- 42.75380217009945
% Mittelwert & Standardabw: 994.82 +- 3.15
% gerundete Werte: 995 +- 3
% Breite: 2.338973259083296

\begin{figure}[H]
    \centering
    \includegraphics[height = 8.5cm]{build/statistik.pdf}
    \caption[]{Histogramm der Messwerte mit einer Gauß- und einer Poissonverteilung.}
    \label{fig:stat}
\end{figure}
