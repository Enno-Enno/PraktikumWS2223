\section{Ziel}
In diesem Versuch sollen mithilfe von Neutronen verschiedene Proben aus Vanadium und Silber
 Radioaktiv aktiviert werden.
Von den erzeugten radioaktiven Isotopen dieser Proben wird anschließen die Halbwertszeit ermittelt.

\section[Theoretische Grundlagen]{Theoretische Grundlagen\footnote[1]{Unter Verwendung von \cite[]{man:v702}}.}

\subsection{Halbwertszeit}
Die Halbwertszeit eines radioaktiven Materials ist ein statistischer Wert, der bei einer großen Anzahl an Teilchen
die Zeit bestimmt, bis die Hälfte der Teilchen zerfallen ist.
Die Zahl der Noch nicht zerfallenen Kerne in einer Probe wird durch 
\begin{align}
    N(t)= N_0 e^{-\lambda t}
\end{align}
beschrieben.
Die Halbwertszeit $T$ der Probe ergibt sich damit aus
\begin{align}
    1/2 N_0 = N_0 e^{-\lambda T} \rightarrow T = \ln(2/\lambda)
\end{align}


\subsection{Anregung der Proben}
