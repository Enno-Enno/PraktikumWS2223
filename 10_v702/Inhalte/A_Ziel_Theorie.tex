\section{Ziel}
In diesem Versuch sollen mithilfe von Neutronen verschiedene Proben aus Vanadium und Silber
 Radioaktiv aktiviert werden.
Von den erzeugten radioaktiven Isotopen dieser Proben wird anschließen die Halbwertszeit ermittelt.

\section[Theoretische Grundlagen]{Theoretische Grundlagen\footnote[1]{Unter Verwendung von \cite[]{man:v702}}.}

\subsection{Halbwertszeit}
Die Halbwertszeit eines radioaktiven Materials ist ein statistischer Wert, der bei einer großen Anzahl an Teilchen
die Zeit bestimmt, bis die Hälfte der Teilchen zerfallen ist.
Die Zahl der Noch nicht zerfallenen Kerne in einer Probe wird durch 
\begin{align}
    N(t)= N_0 e^{-\lambda t}
\end{align}
beschrieben.
Die Halbwertszeit $T$ der Probe ergibt sich damit aus
\begin{align}
    1/2 N_0 = N_0 e^{-\lambda T} \Leftrightarrow T = \ln(2)/\lambda
    \label{eq:T_aus_lambda}
\end{align}
Um die Halbwertszeit aus den gemessenen Zerfällen in einem Zeitintervall $\Delta t$
zu ermitteln werden sich die Eigenschaften der e-Funktion zunutze gemacht.
Es gilt
\begin{align}
    \ln N_{\Delta t} = \ln{N_0 (1- e^{-\lambda \Delta t})} - \lambda t
    \label{eq:lambda_aus_N}
\end{align}
Den wert für Lambda ergibt sich aus einer linearen Ausgleichrechnung, da
\begin{align*}
    \ln{N_0 (1- e^{-\lambda \Delta t})} = \text{const.}
\end{align*}

\subsection{Anregung der Proben}

\noindent
In diesem Versuch werden drei Radioaktive Isotope aus in der Natur stabilen Metallen erzeugt.
\begin{align}
    \ce{^{51}V}, & \ce{^{107}Ag},  \ce{^{109}Ag}
\end{align}

Dies wird erreicht in dem das Verhältnis von Neutronen und Protonen in den Kernen des Elements verändert wird.
Hierzu werden sich Kernreaktionen mit Neutonen zunutze gemacht.
Eine Kernreaktion eines Atomkerns A mit einem Neuton beginnt mit der Absorption des Neutons im Atomkern.
Den neuen so entstandenen angeregten Kern A* nennt man Zwischenkern oder Compoundkern.
Er ist durch die Bindungsenergie des Neutrons auf einem höheren Energielevel als der einfache Atomkern.
Diese Energie verteilt sich auf die gesamten Nukleonen, weshalb die Energie des Atomkerns meist nicht mehr ausreicht
um das ganze Nukleon wieder abzustoßen.
Das ist vor allem der Fall wenn die Neutonen eine geringe Geschwindigkeit haben.
Stattdessen wird die Energie in Form eines $\gamma$-Quants emittiert wonach das Atom wieder in einem nicht angeregten Zustand befindet.
Es läuft folgende Reaktion ab:
\begin{align*}
    \ce{^{m}_{z}A} + \ce{^{1}_{0}n} \rightarrow \ce{^{m+1}_{z}A*} \rightarrow \ce{^{m+1}_{z} A} + \gamma .
\end{align*}
Dieser neu entstandene Kern $\ce{^{m+1}_{z} A}$ ist oft nicht stabil, weil er mehr Neutronen enthält
als er in einem stabilen Zustand haben kann.
Er wandelt in einem $\ce{\beta^{-}}$-Zerfall in ein Neutron in ein Proton um und emittiert 
Elektron $\ce{\beta^{-}}$ und Antineutrino $\overline{\nu}_\text{e}$ mit einer Kinetischen Energie $E_\text{kin}$:
\begin{align*}
    \ce{^{m+1}_{z} A} \rightarrow \ce{^{m+1}_{z+1} C} + \beta^{-} + E_\text{kin} + \overline{\nu}_\text{e}
\end{align*}

\noindent
In diesem Versuch werden die folgende Kernreaktionen gemessen:
\begin{align*}
    \ce{^{51}_{23} V} &\rightarrow \ce{^{52}_{23} V} \rightarrow \ce{^{52}_{24} Cr} + \beta^{-} + E_\text{kin} + \overline{\nu}_\text{e}
    \ce{^{107}_{47} Ag} &\rightarrow \ce{^{108}_{47} Ag} \rightarrow \ce{^{108}_{48} Cd} + \beta^{-} + E_\text{kin} + \overline{\nu}_\text{e}
    \ce{^{109}_{47} Ag} &\rightarrow \ce{^{110}_{47} Ag} \rightarrow \ce{^{110}_{48} Cd} + \beta^{-} + E_\text{kin} + \overline{\nu}_\text{e}
\end{align*}

\subsubsection{Halbwertszeit bei zwei Silberisotopen}
Bei dem Zerfall der angeregten Silberprobe ist zu beachten, dass die Probe in etwas zu 
gleichen Teilen zwei verschiedene Isotope enthält.
Die Halbwertszeit dieser Isotope ist nur feststellbar, weil die Halbwertszeit des einen
des $\ce{^{108}Ag}$ Isotops deutlich länger als die Halbwertszeit von $\ce{^{110}Ag}$ ist.
Im Halblogarithmischen Diagramm folgt die Zerfallszahl ab einer gewissen zeit $t*$ 


%%% To Do
% Relevante Kernreaktionen
% Wirkungsquerschnitt für schnelle und langsame Neutronen
% Erzeugung Niederenergetischer Neutronen