\section{Ziel}
In diesem Versuch sollen mithilfe von Neutronen verschiedene Proben aus Vanadium und Silber
 Radioaktiv aktiviert werden.
Von den erzeugten radioaktiven Isotopen dieser Proben wird anschließen die Halbwertszeit ermittelt.

\section[Theoretische Grundlagen]{Theoretische Grundlagen\footnote[1]{Unter Verwendung von \cite[]{man:v702}}.}

\subsection{Halbwertszeit}
Die Halbwertszeit eines radioaktiven Materials ist ein statistischer Wert, der bei einer großen Anzahl an Teilchen
die Zeit bestimmt, bis die Hälfte der Teilchen zerfallen ist.
Die Zahl der Noch nicht zerfallenen Kerne in einer Probe wird durch 
\begin{align}
    N(t)= N_0 e^{-\lambda t}
\end{align}
beschrieben.
Die Halbwertszeit $T$ der Probe ergibt sich damit aus
\begin{align}
    1/2 N_0 = N_0 e^{-\lambda T} \Leftrightarrow T = \ln(2/\lambda)
    \label{eq:T_aus_lambda}
\end{align}
Um die Halbwertszeit aus den gemessenen Zerfällen in einem Zeitintervall $\Delta t$
zu ermitteln werden sich die Eigenschaften der e-Funktion zunutze gemacht.
Es gilt
\begin{align}
    \ln N_{\Delta t} = \ln{N_0 (1- e^{-\lambda \Delta t})} - \lambda t
    \label{eq:lambda_aus_N}
\end{align}
Den wert für Lambda ergibt sich aus einer linearen Ausgleichrechnung, da
\begin{align*}
    \ln{N_0 (1- e^{-\lambda \Delta t})} = \text{const.}
\end{align*}

\subsection{Anregung der Proben}

In diesem Versuch werden drei Radioaktive Isotope von natürlich Stabilen Elementen erzeugt.
\begin{align}
    \ce{^{51}V}, & \ce{^{107}Ag} & \ce{^{109}Ag}
\end{align}

Dies wird erreicht in dem das Verhältnis von Neutronen und Protonen in den Kernen des Elements verändert wird.
Hierzu werden sich Kernreaktionen mit Neutonen zunutze gemacht.
Eine Kernreaktion eines Atomkerns A mit einem Neuton beginnt mit der Absorption des Neutons im Atomkern.
Den neuen so entstandenen angeregten Kern A* nennt man Zwischenkern oder Compoundkern.
Er ist durch die Bindungsenergie des Neutrons auf einem höheren Energielevel als der einfache Atomkern.
Diese Energie verteilt sich auf die gesamten Nukleonen, weshalb die Energie des Atomkerns meist nicht mehr ausreicht
um ein ganzes Nukleon abzustoßen.