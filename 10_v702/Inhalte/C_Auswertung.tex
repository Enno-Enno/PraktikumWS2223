\section{Auswertung}

\subsection{Mittelwerte und Fehler}
\textbf{ABSCHNITT ERGÄNZEN}
\subsubsection{Gaußsche Fehlerfortpflanzung}
Wenn zu Messdaten die Standardabweichung bekannt ist, und mit diesen Messdaten weiter gerechnet werden soll,
wird die Gaußsche Fehlerfortpflanzung verwendet. 
Angenommen, es gibt $k$ Messwerte $x_i [i \in \mathbb{N}, i \leq k]$ mit den Standardabweichungen $\Delta x_i$
und eine abgeleitete Größe $f(x_i)$.
Dann ist der Fehler von $f$
\begin{align}
    \Delta f(x_i) = \sqrt{
    \left(\frac{\partial f}{\partial x_1} \Delta x_1\right)^2%
     + \left(\frac{\partial f}{\partial x_2} \Delta x_2\right)^2%
     + \dots%
     + \left(\frac{\partial f}{\partial x_k} \Delta x_k\right)^2%
    }.
    \label{eq:gauspflanz}
\end{align} 
Im Ergebnis ergibt sich der Mittelwert von $f$ mit der errechneten Abweichung $\overline{f} \pm \Delta f $.
Um Rechenfehler zu vermeiden, wird das Python \cite[]{python} Paket \texttt{uncertainties} \cite[][]{uncertainties} verwendet.
Hier wird die Fehlerfortpflanzung automatisch verrechnet, wenn die Variablen als \texttt{ufloat} definiert werden.

\subsection{Der Nulleffekt}
Übe eine Zeitspanne von insgesamt $\qty[]{560}{\second}$ wird der Nulleffekt gemessen.
Dabei wird die Zeit in 16 gleich große Intervalle $x$ der Länge $\Delta t = \qty[]{35}{\second}$ eingeteilt. 
Die Werte sind in Tabelle \ref{tab:nulleffekt} zu sehen.

\begin{table}
    \centering
    \caption[short]{Intervallnummer $x$ und Zerfallsanzahl $y$ beim Nulleffekt.}
    \label{tab:nulleffekt}
    \sisetup{table-format=2.0}
    \begin{tabular}{S S S S}
        \toprule
        {x} & {y} & {x} & {y} \\
        \cmidrule(lr){1-2}\cmidrule(lr){3-4}
        1 & 19 &  9 & 14 \\
        2 & 17 & 10 & 12 \\
        3 & 15 & 11 & 14 \\
        4 & 15 & 12 & 14 \\
        5 & 13 & 13 &  7 \\
        6 & 14 & 14 & 18 \\
        7 &  9 & 15 & 11 \\
        8 & 21 & 16 & 16 \\
        \bottomrule
    \end{tabular}
\end{table}



\subsection{Vanadium Probe}
Die gemessenen Werte $x$ und $y$ des Vanadium Zerfalls sind in Tabelle \ref{tab:vanadium} einzusehen.

\begin{table}
    \centering
    \caption[short]{Intervallnummer $x$ und Zerfallsanzahl $y$ beim Zefall von Vanadium.}
    \label{tab:vanadium}
    \sisetup{table-format=2.0}
    \begin{tabular}{S S[table-format=3.0] S S S S}
        \toprule
        {x} & {y} & {x} & {y} & {x} & {y} \\
        \cmidrule(lr){1-2} \cmidrule(lr){3-4} \cmidrule(lr){5-6}
        1 & 205 & 11 & 76 & 21 & 36 \\
        2 & 221 & 12 & 72 & 22 & 38 \\
        3 & 150 & 13 & 67 & 23 & 36 \\
        4 & 139 & 14 & 59 & 24 & 33 \\
        5 & 116 & 15 & 56 & 25 & 37 \\
        6 & 112 & 16 & 46 & 26 & 27 \\
        7 & 108 & 17 & 59 & 27 & 21 \\
        8 & 115 & 18 & 37 & 28 & 24 \\
        9 &  96 & 19 & 45 & 29 & 23 \\
       10 & 100 & 20 & 44 & 30 & 32 \\
        \bottomrule
    \end{tabular}
\end{table}

\subsection{Silber Probe}

\subsubsection{Messreihe 1}

\begin{table}
    \centering
    \caption[short]{Intervallnummer $x$ und Zerfallsanzahl $y$ der ersten Messreihe mit Silber.}
    \label{tab:silber1}
    \sisetup{table-format=2.0}
    \begin{tabular}{S S[table-format=3.0] S S S S S S S S}
        \toprule
        {x} & {y} & {x} & {y} & {x} & {y} & {x} & {y} & {x} & {y} \\
        \cmidrule(lr){1-2} \cmidrule(lr){3-4} \cmidrule(lr){5-6}  \cmidrule(lr){7-8} \cmidrule(lr){9-10}
        1 & 192 & 11 & 29 & 21 & 16 & 31 & 15 & 41 &  8 \\
        2 & 153 & 12 & 34 & 22 &  8 & 32 & 16 & 42 &  9 \\
        3 & 119 & 13 & 28 & 23 & 17 & 33 & 11 & 43 &  9 \\
        4 &  94 & 14 & 24 & 24 & 23 & 34 & 13 & 44 &  9 \\
        5 &  82 & 15 & 20 & 25 & 10 & 35 & 11 & 45 &  8 \\
        6 &  71 & 16 & 21 & 26 & 12 & 36 & 18 & 46 & 10 \\
        7 &  55 & 17 & 18 & 27 & 10 & 37 & 15 & 47 &  8 \\
        8 &  55 & 18 & 20 & 28 & 17 & 38 & 11 & 48 &  8 \\
        9 &  42 & 19 & 29 & 29 & 15 & 39 & 14 & 49 & 13 \\
       10 &  49 & 20 & 21 & 30 & 18 & 40 & 16 & 50 & 10 \\
        \bottomrule
    \end{tabular}
\end{table}

\subsubsection{Messreihe 2}

\begin{table}
    \centering
    \caption[short]{Intervallnummer $x$ und Zerfallsanzahl $y$ der zweiten Messreihe mit Silber.}
    \label{tab:silber2}
    \sisetup{table-format=2.0}
    \begin{tabular}{S S[table-format=3.0] S S S S S S S S}
        \toprule
        {x} & {y} & {x} & {y} & {x} & {y} & {x} & {y} & {x} & {y} \\
        \cmidrule(lr){1-2} \cmidrule(lr){3-4} \cmidrule(lr){5-6} \cmidrule(lr){7-8} \cmidrule(lr){9-10}
        1 & 156 & 11 & 23 & 21 & 21 & 31 & 15 & 41 & 12 \\
        2 & 112 & 12 & 19 & 22 & 18 & 32 & 16 & 42 &  8 \\
        3 & 116 & 13 & 28 & 23 & 22 & 33 & 11 & 43 &  6 \\
        4 &  78 & 14 & 21 & 24 & 10 & 34 & 13 & 44 & 10 \\
        5 &  69 & 15 & 23 & 25 & 15 & 35 & 11 & 45 &  3 \\
        6 &  61 & 16 & 15 & 26 &  8 & 36 & 18 & 46 & 10 \\
        7 &  53 & 17 & 19 & 27 & 11 & 37 & 15 & 47 & 15 \\
        8 &  45 & 18 & 20 & 28 & 10 & 38 & 11 & 48 &  3 \\
        9 &  41 & 19 & 23 & 29 & 13 & 39 & 14 & 49 & 13 \\
       10 &  34 & 20 & 16 & 30 & 19 & 40 & 16 & 50 &  5 \\
        \bottomrule
    \end{tabular}
\end{table}







 
 
 
 
 
 
 
 
 
 
