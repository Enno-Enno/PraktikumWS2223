\subsection{Auswertung}

\subsection{Der Nulleffekt}
Übe eine Zeitspanne von insgesamt $\qty[]{560}{\second}$ wird der Nulleffekt gemessen.
Dabei wird die Zeit in 16 gleich große Intervalle der Länge $\delta t = \qty[]{35}{\second}$ eingeteilt. 
Die Werte sind in Tabelle ......................... zu sehen.

\begin{table}
    \centering
    \sisetup{table-format=2.0}
    \begin{tabular}{S S S S}
        \toprule
        {} & {Anzahl} & {} & {Anzahl} \\
        \cmidrule(lr){1-2}\cmidrule(lr){3-4}
        1 & 19 &  9 & 14 \\
        2 & 17 & 10 & 12 \\
        3 & 15 & 11 & 14 \\
        4 & 15 & 12 & 14 \\
        5 & 13 & 13 &  7 \\
        6 & 14 & 14 & 18 \\
        7 &  9 & 15 & 11 \\
        8 & 21 & 16 & 16 \\
    \end{tabular}
    \caption[short]{Zerfallsanzahl je Intervallnummer beim Nulleffekt.}
    \label{tab:nulleffekt}
\end{table}
