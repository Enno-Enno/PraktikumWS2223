\section{Auswertung}

\subsection{Mittelwerte und Fehler}
Das arithmetische Mittel $\overline{c}$ und die Standardabweichung $\Delta c$ einer Messreihe mit $N$ Werten $c_k$ errechnet sich gemäß der Formeln
\begin{align}
    \overline{c} &= \frac{1}{N} \sum_{k=1}^{N} c_k, & \Delta c = \sqrt{ \sum_{k=1}^{N} \left(\overline{c} - c_k \right)^2 }.
    \label{eq:mittelstand}
\end{align}

\subsubsection{Gaußsche Fehlerfortpflanzung}
Wenn zu Messdaten die Standardabweichung bekannt ist, und mit diesen Messdaten weiter gerechnet werden soll,
wird die Gaußsche Fehlerfortpflanzung verwendet. 
Angenommen, es gibt $k$ Messwerte $x_i [i \in \mathbb{N}, i \leq k]$ mit den Standardabweichungen $\Delta x_i$
und eine abgeleitete Größe $f(x_i)$.
Dann ist der Fehler von $f$
\begin{align}
    \Delta f(x_i) = \sqrt{
    \left(\frac{\partial f}{\partial x_1} \Delta x_1\right)^2%
     + \left(\frac{\partial f}{\partial x_2} \Delta x_2\right)^2%
     + \dots%
     + \left(\frac{\partial f}{\partial x_k} \Delta x_k\right)^2%
    }.
    \label{eq:gauss}
\end{align} 
Im Ergebnis ergibt sich der Mittelwert von $f$ mit der errechneten Abweichung $\overline{f} \pm \Delta f $.
Um Rechenfehler zu vermeiden, wird das Python \cite[]{python} Paket \texttt{uncertainties} \cite[][]{uncertainties} verwendet.
Hier wird die Fehlerfortpflanzung automatisch verrechnet, wenn die Variablen als \texttt{ufloat} definiert werden.





\subsection{Der Nulleffekt}
\label{sec:ausw_null}
Über eine Zeitspanne von insgesamt $\qty[]{560}{\second}$ wird der Nulleffekt gemessen.
Dabei wird die Zeit in 16 gleich große Intervalle $x$ der Länge $\Delta t = \qty[]{35}{\second}$ eingeteilt, in denen jeweils die 
Zerfallszahl $y$ gemesssen wird. 
Die Werte gemessenen Werte sind in Tabelle \ref{tab:nulleffekt} zu sehen.

\begin{table}[H]
    \centering
    \caption[short]{Intervallnummer $x$ und Zerfallsanzahl $y$ beim Nulleffekt.}
    \label{tab:nulleffekt}
    \sisetup{table-format=2.0}
    \begin{tabular}{S S S S}
        \toprule
        {x} & {y} & {x} & {y} \\
        \cmidrule(lr){1-2}\cmidrule(lr){3-4}
        1 & 19 &  9 & 14 \\
        2 & 17 & 10 & 12 \\
        3 & 15 & 11 & 14 \\
        4 & 15 & 12 & 14 \\
        5 & 13 & 13 &  7 \\
        6 & 14 & 14 & 18 \\
        7 &  9 & 15 & 11 \\
        8 & 21 & 16 & 16 \\
        \bottomrule
    \end{tabular}
\end{table}

\noindent
Werden von diesen Werten Mittelwert und Standardabweichung ausgerechnet, ergibt sich gemäß Formel \eqref{eq:mittelstand} 
eine Zerfallsanzahl von $N_\text{u} = \num[]{0.4089 +- 0.0988}$ pro Sekunde.
Bei der Messung des Vanadium Zerfalls wurde ein Zeitintervall von $\Delta t_\text{V} = \qty[]{35}{\second}$ gewählt, 
wodurch sich in diesem Fall der Nulleffekt von $N_\text{u,V} = \num[]{14.3125 +- 3.4590}$ ergibt.
Analog folgt beim Silber Zerfall mit einem Zeitintervall von $\Delta t_\text{Ag} = \qty[]{10}{\second}$ ein Nulleffekt von 
$N_\text{u,Ag} = \num[]{4.0893 +- 0.9883}$.

\noindent
Da bei der Zerfallszahl nur ganze Zahlen sinnvoll sind, werden diese Werte weiter auf $N_\text{u,V} = \num[]{14 +- 3}$ bzw.
$N_\text{u,Ag} = \num[]{4 +- 1}$ gerundet.
% nulleffekt beim vanadium zerfall je 35 sekunden:  14.3125 +-  3.459023525505428
% nulleffekt beim silber zerfall je 10 sekunden:  4.089285714285714 +-  0.9882924358586936


\subsection{Vanadium Probe}
Die gemessene Zerfallsanzahl $y$ in Abhängigkeit der Intervallnummer $x$ des Vanadium Zerfalls vor Abzug des Nulleffekts sind in Tabelle \ref{tab:vanadium} einzusehen.
Dabei gilt im Folgenden stets für die vergangene Zeit $t = x \cdot \Delta t$ ,wobei in diesem Versuchteil
$\Delta t_\text{V} = \qty[]{35}{\second}$ gilt.
\begin{table}[H]
    \centering
    \caption[short]{Intervallnummer $x$ und Zerfallsanzahl $y$ beim Zefall von Vanadium.}
    \label{tab:vanadium}
    \sisetup{table-format=2.0}
    \begin{tabular}{S S[table-format=3.0] S S S S}
        \toprule
        {x} & {y} & {x} & {y} & {x} & {y} \\
        \cmidrule(lr){1-2} \cmidrule(lr){3-4} \cmidrule(lr){5-6}
        1 & 205 & 11 & 76 & 21 & 36 \\
        2 & 221 & 12 & 72 & 22 & 38 \\
        3 & 150 & 13 & 67 & 23 & 36 \\
        4 & 139 & 14 & 59 & 24 & 33 \\
        5 & 116 & 15 & 56 & 25 & 37 \\
        6 & 112 & 16 & 46 & 26 & 27 \\
        7 & 108 & 17 & 59 & 27 & 21 \\
        8 & 115 & 18 & 37 & 28 & 24 \\
        9 &  96 & 19 & 45 & 29 & 23 \\
       10 & 100 & 20 & 44 & 30 & 32 \\
        \bottomrule
    \end{tabular}
\end{table}

\noindent
Nach Abzug des Nulleffekts $N_\text{u,V}$ ergibt sich die Zahl $N_{\Delta t,V}(t) = y- N_\text{u,V}$ der im Zeitintervall $\Delta t_\text{V}$
gemessenen Zerfälle.
Wird $N_\text{u,V}$ in Abhängigkeit des betrachteten Zeitintervalls halblogarithmisch geplottet, ergibt sich Abbildung \ref{fig:vanadium_log}.
Die darin eingezeichneten Fehler $\Delta N_{\Delta t} = \num[]{3.4590}$ wurden gemäß der Gaußschen Fehlerfortpflanzung \eqref{eq:gauss} berechnet.
Gemäß Formel \eqref{eq:lambda_aus_N} wird eine lineare Ausgleichsrechnung 
\begin{align}
    \ln\left(N_{\Delta t}(t)\right) = a \cdot t + b
    \label{eq:linear_eq}
\end{align}
mittels der \texttt{scipy} \cite[]{scipy} Funktion \texttt{curve\_fit} durchgeführt.
Die Ausgleichsgerade ist ebenfalls in Abbildung \ref{fig:vanadium_log} zu sehen.
Anhand der somit bestimmten Konstanten
\begin{align*}
    a &= (\num{-0.1060 +- 0.0053}) \, \frac{1}{35 \unit{\second}}  & b &= \num{5.3690 +- 0.0376}
\end{align*}
kann die gesuchte Konstante 
\begin{align}
    N(0) \cdot \left(1- \exp(- \lambda \Delta t)\right) = \exp(b) = \num[]{215 +- 8}
\end{align}
sowie die Zerfallskonstante 
\begin{align}
    \lambda = - a = \qty[]{0.0030+-0.0002}{\per\second}
\end{align}
bestimmt werden.
Aus letzterer folgt die Halbwertszeit 
\begin{align}
    T = \qty[]{229 +- 12}{\second} = \qty[]{3.81 +- 0.19}{\minute}
\end{align}
gemäß Gleichung \eqref{eq:T_aus_lambda}.
% a = -0.10603542 ± 0.00534514
% b = 5.36902879 ± 0.03755019
% lamda in 1/s:  0.00303+/-0.00015 /s
% halbwertszeit in sekunden:  229+/-12 s
% halbwertszeit in minuten:  3.81+/-0.19 min
% const = N(0)*(1-exp(...)) =  215+/-8


\begin{figure}[H]
    \centering
    \includegraphics[height=8cm]{build/c02_vanadium_log.pdf}
    \caption[]{Halblogarithmische Darstellung des Vanadium Zerfalls in Abhängigkeit von $\Delta t_\text{V}$.}
    \label{fig:vanadium_log}
\end{figure}

\noindent
Da die Fehlerbalken durch die halblogarithmische Darstellung verzerrt erscheinen, ist für eine bessere Veranschaulichung zusätzlich ein linearer
Plot in Abbildung \ref{fig:vanadium_err} zu sehen.

\begin{figure}[H]
    \centering
    \includegraphics[height=8cm]{build/c02_vanadium_err.pdf}
    \caption[]{Lineare Darstellung des Vanadium Zerfalls in Abhängigkeit von $\Delta t_\text{V}$.}
    \label{fig:vanadium_err}
\end{figure}













\subsection{Silber Probe}

\subsubsection{Messreihe 1}
\label{sec:silber_mr1}
Die Intervallnummer $x$ und Zerfallsanzahl $y$ (vor Abzug des durchschnittlichen Nulleffekts, vgl. Abschnitt \ref{sec:ausw_null}) sind in Tabelle \ref{tab:silber1} zu sehen.

\begin{table}[H]
    \centering
    \caption[short]{Intervallnummer $x$ und Zerfallsanzahl $y$ der ersten Messreihe mit Silber.}
    \label{tab:silber1}
    \sisetup{table-format=2.0}
    \begin{tabular}{S S[table-format=3.0] S S S S S S S S}
        \toprule
        {x} & {y} & {x} & {y} & {x} & {y} & {x} & {y} & {x} & {y} \\
        \cmidrule(lr){1-2} \cmidrule(lr){3-4} \cmidrule(lr){5-6}  \cmidrule(lr){7-8} \cmidrule(lr){9-10}
        1 & 192 & 11 & 29 & 21 & 16 & 31 & 15 & 41 &  8 \\
        2 & 153 & 12 & 34 & 22 &  8 & 32 & 16 & 42 &  9 \\
        3 & 119 & 13 & 28 & 23 & 17 & 33 & 11 & 43 &  9 \\
        4 &  94 & 14 & 24 & 24 & 23 & 34 & 13 & 44 &  9 \\
        5 &  82 & 15 & 20 & 25 & 10 & 35 & 11 & 45 &  8 \\
        6 &  71 & 16 & 21 & 26 & 12 & 36 & 18 & 46 & 10 \\
        7 &  55 & 17 & 18 & 27 & 10 & 37 & 15 & 47 &  8 \\
        8 &  55 & 18 & 20 & 28 & 17 & 38 & 11 & 48 &  8 \\
        9 &  42 & 19 & 29 & 29 & 15 & 39 & 14 & 49 & 13 \\
       10 &  49 & 20 & 21 & 30 & 18 & 40 & 16 & 50 & 10 \\
        \bottomrule
    \end{tabular}
\end{table}

\noindent
Analog zu oben ergibt sich nach Abzug des Nulleffekts die effektive Zerfallzahl 
$N_{\Delta t,Ag}(t) = y- N_\text{u,Ag}$ je Intervall $\Delta t_\text{Ag} = \qty[]{10}{\second}$,
die erneut halblogarithmisch geplottet wird, vgl. Abbildung \ref{fig:silber1_log}.
Hier ergeben sich die eingezeichneten Fehler gemäß der Gaußschen Fehlerfortpflanzung \eqref{eq:gauss} zu $\Delta N_{\Delta t} = \num{0.9883}$.
Um die unterschiedlichen Halbwertszeiten der beiden Isotope im natürlich vorkommenden Silber ermitteln zu können, wird für einen Wert ab 
$t^* = 30 \cdot \Delta t_\text{Ag} = \qty[]{300}{\second}$ eine lineare Ausgleichsrechnung gemäß \eqref{eq:linear_eq} durchgeführt.
Dadurch können die gesuchten Konstanten des langlebigen Isotops ermittelt werden, da hier das kurzlebige Isotop nahezu vollständig zerfallen ist.
Es sei angemerkt, dass aufgrund der Werteschwankungen im späteren Zeitbereich der Wert $t^*$ so festgelegt wurde, 
dass zum einen die zugehörige Ausgleichsgerade nicht zu steil verläuft,
und zum anderen die resultierenden Abweichungen der gesuchten Größen nicht allzu groß sind.
Außerdem wurde darauf geachtet, dass die unten beschriebene Summe der beiden Auslgeichsfunktionen verhältnismäßig nahe an den Messwerten liegt.
In diesem Fall ergeben sich die Konstanten zu 
\begin{align}
    a &= (\num{-0.0374 +- 0.0113}) \, \frac{1}{10 \unit{\second}}  & b &= \num{3.6702 +- 0.4201}.
\end{align}
Hieraus ergeben sich analog zu oben die Werte 
\begin{align}
    N(0) \cdot \left(1- \exp(- \lambda \Delta t)\right) = \exp(b) = \num[]{39 +- 16}
\end{align}
sowie die Zerfallskonstante 
\begin{align}
    \lambda = - a = \qty[]{0.0037 +- 0.0011}{\per\second}.
\end{align}
Aus letzterer folgt die Halbwertszeit 
\begin{align}
    T = \qty[]{190 +- 60}{\second} = \qty[]{3.1 +- 0.9}{\minute}
\end{align}
gemäß Gleichung \eqref{eq:T_aus_lambda}.

% parameter langlebig:
% a = -0.03742470 ± 0.01125532
% b = 3.67018605 ± 0.42005173
% lamda_l in 1/s:  0.0037+/-0.0011 /s
% halbwertszeit_l in sekunden:  (1.9+/-0.6)e+02 s
% halbwertszeit_l in minuten:  3.1+/-0.9 min
% const_l = N(0)*(1-exp(...)) =  39+/-16




\noindent
Zur Bestimmung der Halbwertszeit des kurzlebigen Isotops werden nun die Messwerte bis $t_\text{max} = 4 \cdot \Delta t_\text{Ag} = \qty[]{40}{\second}$ betrachtet,
da in diesem Bereich besonders viele Zerfälle eben dieses Isotops gemessen werden.
Zur Bestimmung der Kenngrößen ist es in diesem Fall notwendig, die Zerfälle des langlebigen Isotops von $N_{\Delta t,Ag}$ abzuziehen.
Für die Zerfälle des langlebigen Isotops werden dafür die einzelnen Werte $x \leq t_\text{max}$ in die Ausgleichsfunktion \eqref{eq:linear_eq} eingesetzt, wodurch
sich $N_{\Delta t,lang}(x)$ ergibt.
Für $\ln\left(N_{\Delta t,kurz}(x)\right)$ mit $N_{\Delta t,kurz}(x) = N_{\Delta t,Ag}(x) - N_{\Delta t,lang}(x)$ wird anschließend analog zu oben eine lineare Ausgleichsrechnung durchgeführt.
Es ergeben sich
\begin{align}
    a &= (\num{-0.3105 +- 0.0160}) \, \frac{1}{10 \unit{\second}}  & b &= \num{5.3268 +- 0.0290}.
\end{align}
Weiterhin folgt
\begin{align}
    N(0) \cdot \left(1- \exp(- \lambda \Delta t)\right) = \exp(b) = \num[]{206 +- 6}
\end{align}
sowie die Zerfallskonstante 
\begin{align}
    \lambda = - a = \qty[]{0.0310 +- 0.0016}{\per\second},
\end{align}
aus welcher wiederum die Halbwertszeit
\begin{align}
    T = \qty[]{22.3 +- 1.1}{\second} = \qty[]{0.372 +- 0.029}{\minute}
\end{align}
folgt.

% parameter kurzlebig:
% a = -0.31047072 ± 0.01597358
% b = 5.32683420 ± 0.02897929
% lamda_k in 1/s:  0.0310+/-0.0016 /s
% halbwertszeit_k in sekunden:  22.3+/-1.1 s
% halbwertszeit_k in minuten:  0.372+/-0.019 min
% const_k = N(0)*(1-exp(...)) =  206+/-6

\noindent
Schließlich können mit Hilfe der beiden Ausgleichsfunktionen die Werte $N_{\Delta t,kurz}(t^*)$ und $N_{\Delta t,lang}(t^*)$
verglichen werden, um zu zeigen, dass ab $t^*$ tatsächlich hauptsächlich das langlebige Isotop zerfällt.
Es ergibt sich
\begin{align*}
    N_{\Delta t,kurz}(t^*) &= \num[]{0.019 +- 0.009} & N_{\Delta t,lang}(t^*) &= \num[]{13 +- 7}, 
\end{align*}
woraus eine prozentuale Abweichung von $\num[]{99.85 +- 0.10} \, \%$ folgt.
% N_kurz =  0.019+/-0.009
% N_lang =  13+/-7
% N_abw =  0.9985+/-0.0010

\noindent
Des Weiteren kann die Summenkurve $N_{\Delta t,kurz}(t) + N_{\Delta t,lang}(t)$ halblogarithmisch dargestellt werden, was ebenfalls
in Abbildung \ref{fig:silber1_log} zu sehen ist.


\begin{figure}[H]
    \centering
    \includegraphics[height=8cm]{build/c03_silber1_log.pdf}
    \caption[]{Halblogarithmische Darstellung des ersten Silber Zerfalls in Abhängigkeit von $\Delta t_\text{Ag}$.}
    \label{fig:silber1_log}
\end{figure}
















\subsubsection{Messreihe 2}
Die Intervallnummer $x$ und die Zerfallsanzahl $y$ sind in Tabelle \ref{tab:silber2} zu sehen.
Die eingeklammerten Werte wurden in der Auswertung vernachlässigt,
da sie bzw. ihre untere Grenze (Mittelwert - Standardabweichung) unterhalb des Nulleffekts liegen (vgl. Abschnitt \ref{sec:ausw_null})
und in der logarithmischen Darstellung zu Komplikationen führen würden.

\begin{table}[H]
    \centering
    \caption[short]{Intervallnummer $x$ und Zerfallsanzahl $y$ der zweiten Messreihe mit Silber.}
    \label{tab:silber2}
    \sisetup{table-format=2.0}
    \begin{tabular}{S S[table-format=3.0] S S S S S S S S}
        \toprule
        {x} & {y} & {x} & {y} & {x} & {y} & {x} & {y} & {x} & {y} \\
        \cmidrule(lr){1-2} \cmidrule(lr){3-4} \cmidrule(lr){5-6} \cmidrule(lr){7-8} \cmidrule(lr){9-10}
        1 & 156 & 11 & 23 & 21 & 21 & 31 & 15 & 41 & 12 \\
        2 & 112 & 12 & 19 & 22 & 18 & 32 & 16 & 42 &  8 \\
        3 & 116 & 13 & 28 & 23 & 22 & 33 & 11 & 43 &  6 \\
        4 &  78 & 14 & 21 & 24 & 10 & 34 & 13 & 44 & 10 \\
        5 &  69 & 15 & 23 & 25 & 15 & 35 & 11 & {$(45)$} &  {$(3)$} \\
        6 &  61 & 16 & 15 & 26 &  8 & 36 & 18 & 46 & 10 \\
        7 &  53 & 17 & 19 & 27 & 11 & 37 & 15 & 47 & 15 \\
        8 &  45 & 18 & 20 & 28 & 10 & 38 & 11 & {$(48)$} &  {$(3)$} \\
        9 &  41 & 19 & 23 & 29 & 13 & 39 & 14 & 49 & 13 \\
       10 &  34 & 20 & 16 & 30 & 19 & 40 & 16 & {$(50)$} &  {$(5)$} \\
        \bottomrule
    \end{tabular}
\end{table}

\noindent
Die Auswertung verläuft hier vollständig analog zur Auswertung der ersten Messreihe mit Silber, vgl. Abschnitt \ref{sec:silber_mr1}.
Für die Festlegung von $t^*$ sei angemerkt, 
dass für $x \geq 31$ die untere Grenze von $T$ (d.h. Mittelwert - Standardabweichung) negativ wird und ab 
$x = 33$ der Mittelwert der Halbwertszeit selbst negativ wird.
Für $ x <= 29$ steigen sowohl der Mittelwert als auch die Standardabweichung der Halbwertszeit im Vergleich zu Messung 1 so deutlich, 
dass $t^* = 30 \cdot \Delta t_\text{Ag} = \qty[]{300}{\second}$ gewählt wird.

\noindent
Für das langlebige Isotop folgen damit die Werte 
\begin{align}
    a &= (\num{-0.0235 +- 0.0127}) \, \frac{1}{10 \unit{\second}}  & b &= \num{3.1325 +- 0.4780}.
\end{align}
Es folgt die Konstante
\begin{align}
    N(0) \cdot \left(1- \exp(- \lambda \Delta t)\right) = \exp(b) = \num[]{23 +- 11}
\end{align}
sowie 
\begin{align}
    \lambda = - a = \qty[]{0.0023 +- 0.0013}{\per\second},
\end{align}
woraus sich 
\begin{align}
    T = \qty[]{300 +- 160}{\second} = \qty[]{4.9 +- 2.7}{\minute}
\end{align}
ableiten lässt.
% parameter langlebig:
% a = -0.02346305 ± 0.01272034
% b = 3.13246545 ± 0.47801445
% lamda_l in 1/s:  0.0023+/-0.0013 /s
% halbwertszeit_l in sekunden:  (3.0+/-1.6)e+02 s
% halbwertszeit_l in minuten:  4.9+/-2.7 min
% const_l = N(0)*(1-exp(...)) =  23+/-11

\noindent
Zur Bestimmung von $t_\text{max}$ werden die folgenden Fälle betrachtet:
Für $x \leq 6$ ist die Standardabweichung der Halbwertszeit sehr groß, zwischen $7 \leq x \leq 12$ sind die Mittelwerte sehr nah bei einander.
Allerdings sinkt in diesem Bereich die Standardabweichung etwas.
Ab $x = 13$ werden die Werte für $N_{\Delta t,kurz}(x) = N_{\Delta t,Ag}(x) - N_{\Delta t,lang}(x)$ negativ,
was eine halblogarithmische Darstellung erneut unmöglich macht.
Folglich wird $t_\text{max} = 12 \cdot \Delta t_\text{Ag} = \qty[]{120}{\second}$ gewählt.

\noindent
Für das kurzlebige Isotop folgt somit 
\begin{align}
    a &= (\num{-0.2494 +- 0.0181}) \, \frac{1}{10 \unit{\second}}  & b &= \num{5.0984 +-0.0585}.
\end{align}
Hieraus ergeben sich die gesuchten Werte
\begin{align}
    N(0) \cdot \left(1- \exp(- \lambda \Delta t)\right) = \exp(b) = \num[]{164 +- 10}
\end{align}
sowie 
\begin{align}
    \lambda = - a = \qty[]{0.0249 +- 0.0018}{\per\second},
\end{align}
und schließlich
\begin{align}
    T = \qty[]{27.8 +- 2.0}{\second} = \qty[]{0.463 +- 0.034}{\minute}.
\end{align}
% parameter kurzlebig:
% a = -0.24937942 ± 0.01808827
% b = 5.09844481 ± 0.05853057
% lamda_k in 1/s:  0.0249+/-0.0018 /s
% halbwertszeit_k in sekunden:  27.8+/-2.0 s
% halbwertszeit_k in minuten:  0.463+/-0.034 min
% const_k = N(0)*(1-exp(...)) =  164+/-10

\noindent
Ferner gilt in dieser Messreihe
\begin{align*}
    N_{\Delta t,kurz}(t^*) &= \num[]{0.09 +- 0.05} & N_{\Delta t,lang}(t^*) &= \num[]{11 +- 7}, 
\end{align*}
woraus die prozentuale Abweichung von $\num[]{99.2 +- 0.7} \, \%$ folgt.
% N_kurz =  0.09+/-0.05
% N_lang =  11+/-7
% N_abw =  0.992+/-0.007

\noindent
Die beiden Ausgleichsgeraden und ihre Summe sind mitsamt der Messwerte des zweiten Silberzerfalls in Abbildung \ref{fig:silber2_log} geplottet.
Die eingezeichneten Fehler betragen wie zuvor $\Delta N_{\Delta t} = \num{0.9883}$.


\begin{figure}[H]
    \centering
    \includegraphics[height=8cm]{build/c04_silber2_log.pdf}
    \caption[]{Halblogarithmische Darstellung des zweiten Silber Zerfalls in Abhängigkeit von $\Delta t_\text{Ag}$.}
    \label{fig:silber2_log}
\end{figure}