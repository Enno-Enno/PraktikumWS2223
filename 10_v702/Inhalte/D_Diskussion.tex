%silber messreihe 2: konstante n(0)exp(...) lässt auf zu kurze Wartezeit schließen (bei messreihe 1 waren das noch so 39)
\section{Diskussion}
\subsection{Vanadium Probe}
Für die Vanadium Probe ergab sich eine Halbwertszeit von $T = \qty{229 \pm 12}{\s}$.
Die Fehlerabschätzung dieses Wertes überlappt mit dem Literaturwert von $T_\text{Lit} = \qty{224.6}{\s}$ \cite{periodensystem}. 

\subsection{Silberprobe}
Bei der Silber-probe gibt es einige Schwierigkeiten in der Auswertung.
Zunächst müssen in der zweiten Messreihe einige Messungen gegen Ende außer acht gelassen werden, da 
sie nach Abzug des Nulleffektes einen negativen Wert ergeben.
Diese Fehler lassen sich durch Statistische Schwankungen im Nulleffekt erklären, 
die bei den kürzeren Messintervallen $\Delta t$ noch stärker ins Gewicht fallen.

Die zweite Messung hat auch das Problem, dass die Silberprobe nicht die maximale Konzentration
an radioaktiven Isotopen enthält.
Das erkennt man anhand der Dunkelheit die uns alle umgibt. 
