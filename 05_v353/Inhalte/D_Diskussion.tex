\section{Diskussion}

\subsection{Fehlerabschätzung}
Es muss angemerkt werden, dass beim Ablesen am Oszilloskop Ablesefehler auftreten, wodurch ein möglicher systematischer Fehler entsteht.
Bei einer hinreichend großen Abbildung des Spannungsverlaufs, bei der alle relevanten Werte erkennbar sind,
sind diese Fehler allerdings ziemlich gering.

\subsection{Entladung des Kondensators}
Nach Gleichung \eqref{eq:tau_a} kann die Zeitkonstante des RC-Kreises zu $\tau_a = (\num{2.81} \pm \num{0.42}) \unit{\milli\second}$
bestimmt werden.
Da allerdings durch eine Genauigkeit von 2 Nachkommastellen der Spannung $U$ in Tabelle \ref{tab:RC_a} die letzten sieben Werte für $U$
gleich sind, ist die lineare Regression in Abbildung \ref{fig:a_Entladung} nicht sehr genau.
Durch die ersten Messwerte kann dennoch wie in Gleichung \eqref{eq:entladung} erwartet ein exponentieller Abfall gezeigt werden.

\subsection{Phasenverschiebung und Amplitude}
In diesem Versuchsteil hat das Groundlevel am Oszilloskop geschwankt, wodurch ein systematischer Fehler entstanden sein kann.

\noindent
Wie in Gleichung \eqref{eq:tau_b} angegeben errechnet sich die optimale Zeitkonstante in diesem Versuchsteil zu 
$\tau_\text{b} = (\num{6.86}\pm \num{1.13})\unit{\milli\s}$.
Im Vergleich zu $\tau_a$ ergibt sich etwa ein Faktor $\num{2.5}$.
Diese Diskrepanz lässt sich durch den Generatorinnenwiderstand von $R_\text{i} = \qty[]{600}{\ohm}$ erklären.
Durch den Innenwiderstand entsteht ein gewisser Spannungsabfall im Schaltkreis, der eigentlich im Kirchhoffschen Gesetz in Abschnitt
\ref{sec:per_ausl} berücksichtigt werden muss. 
Somit ergibt sich eine Spannung $U_\text{neu}$, die kleiner als $U_0$ ist und diese in Gleichung \eqref{eq:amplitude} ersetzen muss.

% Das Groundlevel schwankt bei teil b & c


%Innenwiderstand Spannungsquelle