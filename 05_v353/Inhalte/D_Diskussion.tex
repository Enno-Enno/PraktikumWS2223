\section{Diskussion}

\subsection{Fehlerabschätzung}
Es muss angemerkt werden, dass beim Ablesen am Oszilloskop Ablesefehler auftreten, wodurch ein möglicher systematischer Fehler entsteht.
Bei einer hinreichend großen Abbildung des Spannungsverlaufs, bei der alle relevanten Werte erkennbar sind,
sind diese Fehler allerdings ziemlich gering.

\subsection{Entladung des Kondensators}
Nach Gleichung \eqref{eq:tau_a} kann die Zeitkonstante des RC-Kreises zu $\tau_a = (\num{2.81} \pm \num{0.42}) \unit{\milli\second}$
bestimmt werden.
Da allerdings durch eine Genauigkeit von 2 Nachkommastellen der Spannung $U$ in Tabelle \ref{tab:RC_a} die letzten sieben Werte für $U$
gleich groß sind, ist die lineare Regression in Abbildung \ref{fig:a_Entladung} nicht sehr genau.
Durch die ersten Messwerte kann dennoch wie in Gleichung \eqref{eq:entladung} erwartet ein exponentieller Abfall gezeigt werden.



\subsection{Phasenverschiebung und Amplitude}
In diesem Versuchsteil hat das Groundlevel am Oszilloskop geschwankt.
Auch wenn beim Ablesen hierauf geachtet wurde, kann durch das Schwanken dennoch ein systematischer Fehler entstanden sein.

\noindent
Wie in Gleichung \eqref{eq:tau_b} angegeben errechnet sich die optimale Zeitkonstante in diesem Versuchsteil zu 
$\tau_\text{b} = (\num{6.86}\pm \num{1.13})\unit{\milli\s}$.
Im Vergleich zu $\tau_a$ ergibt sich etwa ein Faktor $\num{2.5}$.
Diese Diskrepanz lässt vermutlich sich durch den Generatorinnenwiderstand von $R_\text{i} = \qty[]{600}{\ohm}$ erklären.
Durch den Innenwiderstand entsteht ein gewisser Spannungsabfall im Schaltkreis, der eigentlich im Kirchhoffschen Gesetz in Abschnitt
\ref{sec:per_ausl} berücksichtigt werden muss. 
Somit ergibt sich eine Spannung $U_\text{neu}$, die kleiner als $U_0$ ist und diese in Gleichung \eqref{eq:amplitude} ersetzen muss.

\noindent
Dies geht mit der Tatsache einher, dass für die Ausgleichsrechnung in Abschnitt \ref{sec:phase_ampl} mit einem Faktor 
$\num{0.4} = \frac{1}{\num{2.5}}$ als Dämpfung multipliziert werden muss.
Die Dämpfung entspricht also genau dem Spannungsunterschied durch den Innenwiderstand $R_\text{i}$.
Mit der angepassten Ausgleichsrechnung in Abbildung \ref{fig:b_plot} kann somit die Gleichung \eqref{eq:amplitude} 
als Frequenzabhängigkeit der Amplitude verfiziert werden.

\noindent
Mit der Abbildung \ref{fig:c_plot} kann auch die Gleichung \eqref{eq:phasenverschiebung}, die die frequenzabhängige Phasenverschiebung 
beschreibt, verifizieren.
Der Korrekturfaktor $a$ lässt sich auch hier durch die Dämpfung erklären.
Die Fehlerbalken der berechneten Zeitkonstante $\tau_\text{c} = (\num{6.87} \pm \num{0.94}) \unit{\milli\s}$ mitsamt des Fehlers
in Gleichung \eqref{eq:tau_c} liegt vollständig in dem angegebenen Fehlerbereich der optimalen Konstanten $\tau_b$.
Es kann also von einer hohen Genauigkeit für die Zeitkonstante ausgegangen werden.

\noindent
Anhand des Polarplots \ref{fig:d_polarplot} kann der Zusammenhang zwischen Phasenverschiebung $\phi$ und Amplitude $A$ visualisiert werden.
%Ich schreib hier noch nichts weiteres zu weil Gleichung (12) in der Auswertung nicht verwendet wurde.




\subsection{RC-Kreis als Integrator}
Anhand der \textbf{REFERENZEN} lässt sich für eine ausreichend hohe Frequenz die Funktion eines RC-Kreises als Integrator verifizieren.
% Hier bin ich verwirrt was du getan hast weil C03_RC_Integrator.tex nicht eingebunden und unvollständig ist...
