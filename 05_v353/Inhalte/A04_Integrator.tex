\subsection{Der RC-Kreis als Integrator}
Eine Schaltung wie sie in Abbildung \ref{fig:schwingend} zu sehen ist kann unter der Voraussetzung
\begin{align}
    \omega >> \frac{1}{\tau}
    \label{eq:voraussetzung_tp} 
\end{align}
die angelegte Wechselspannung $U(t)$ integrieren.
Da im Idealfall nur Frequenzen unterhalb einer Grenzfrequenz beitragen, wird dies im Allgemeinen auch Tiefpass genannt.
Ausgehend vom zweiten Kirchhoffschen Gesetz sowie Gleichung \eqref{eq:stromstaerke} ergibt sich
\begin{align}
    U(t) = \tau \frac{\diff{U_\text{C}}}{\diff{t}} + U_\text{C}.
\end{align}
Aus der Voraussetzung \eqref{eq:voraussetzung_tp} folgt
\begin{align}
    |U_\text{C}| &<< |U_\text{R}|, & |U_\text{C}| &<< |U|,
\end{align}
wodurch für die Generatorspannung 
\begin{align*}
    U(t) = \tau \frac{\diff{U_\text{C}}}{\diff{t}}
\end{align*}
genähert werden kann.
Somit folgt schließlich
\begin{align}
    U_\text{C}(t) = \frac{1}{\tau} \int_0^t U(t') \diff{t'}.
\end{align}