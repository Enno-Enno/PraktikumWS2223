\subsection{Der RC-Kreis als Integrator}
\label{sec:durchf_c}
Im dritten Versuchsteil soll die Funktion eines RC-Kreises als Integrator unter der Voraussetzung \eqref{eq:voraussetzung_tp} verifiziert werden.
Die hierfür eingestellte Frequenz beträgt $\omega~=~\qty{2001}{\hertz}$.
Die Schaltung ist analog zu Abbildung \ref{fig:graph_phasenverschiebung}, wobei die beiden Triggersignale DC gekoppelt werden.  
Es werden jeweils eine Rechteck-, Sinus- und Dreiecksspannung auf das RC-Glied gegeben.
Von den beiden auf dem Zweikanal-Oszilloskop angezeigten Spannungen wird für jede angeschlossene Spannungsform ein Foto gemacht.