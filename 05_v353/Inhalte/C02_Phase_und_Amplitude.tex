\subsection{Phasenverschiebung und Amplitude im RC-Kreis}
\label{sec:phase_ampl}
In Tabelle \ref{tab:RC_b_c} werden die Messergebnisse aufgelistet. 
Die Spalte \enquote{Kästchenwerte} enthält die Einstellungen des Oszilloskops.
Je nach Einheit sind dort die Einstellungen Timediv und Voltdiv für die aktuelle und die folgenden Zeilen notiert.
%
\begin{table}
    \centering
    \sisetup{table-format=1.1}
    \begin{tabular}{
        c
        S[table-format=3.1]
        S
        S
        S[table-format=2.2]
        S[table-format=1.2]
        S[table-format=1.2]
        S[table-format=1.4]
    }
    \toprule
    {Kästchenwerte} & 
    {$\omega / \unit{\hertz}$} &
    {$U_0 / \text{Käst.}$} &
    {$\Delta t / \text{Käst.}$} &
    {$T / \unit{\milli\s}$} &
    {$U_0 / \unit{\volt} $} & 
    {$\Delta t / \unit{\milli\s}$} &
    {$\Phi / 2 \pi$}\\
    \midrule
           
    $\qty{0.5}{\milli\s}$\, $\qty{0.5}{\volt} $ & 30.2  & 1.2 & 0.2  & 33.1 & 0.60 & 1.00 & 0.0302 \\
    $\qty{0.2}{\milli\s}$                       & 60.1  & 1.1 & 0.4  & 16.6 & 0.55 & 0.80 & 0.0481 \\
                                                & 99.9  & 1.0 & 0.3  & 10.0 & 0.50 & 0.60 & 0.0599 \\
                                                & 159.9 & 0.8 & 0.2  & 6.3  & 0.40 & 0.40 & 0.0640 \\
                                                & 200.5 & 0.7 & 0.2  & 5.0  & 0.35 & 0.40 & 0.0802 \\
    $\qty{0.2}{\volt}$                          & 250.3 & 1.5 & 0.2  & 4.0  & 0.30 & 0.40 & 0.1001 \\
                                                & 299.9 & 1.3 & 0.2  & 3.3  & 0.26 & 0.40 & 0.1200 \\
    $\qty{0.5}{\milli\s}$                       & 351.8 & 1.3 & 0.6  & 2.8  & 0.24 & 0.30 & 0.1055 \\
                                                & 407.1 & 1.1 & 0.6  & 2.5  & 0.22 & 0.30 & 0.1221 \\
                                                & 500.2 & 0.9 & 0.5  & 2.0  & 0.18 & 0.25 & 0.1251 \\
                                                & 600   & 0.8 & 0.4  & 1.7  & 0.16 & 0.20 & 0.1200 \\
                                                & 700   & 0.6 & 0.4  & 1.4  & 0.12 & 0.20 & 0.1400 \\
    $\qty{0.1}{\volt}$                          & 800   & 1.2 & 0.3  & 1.25 & 0.12 & 0.15 & 0.1200 \\
                                                & 900   & 1.0 & 0.3  & 1.1  & 0.10 & 0.15 & 0.1350 \\
    $\qty{0.2}{\milli\s}$                       & 1001  & 0.9 & 0.7  & 1.0  & 0.09 & 0.14 & 0.1401 \\
    \bottomrule
    \end{tabular}
\end{table}
%
Die Amplituden $A(\omega)$ werden durch die Amplitude der generierten Schwingung $U_0$ geteilt und in Abbildung \ref{fig:b_plot}
in Abhängigkeit der Frequenz $\omega$ dargestellt.
Um den zusammenhang für $A(\omega)$ darzustellen kann Gleichung \ref{eq:amplitude} verwendet werden.
Nach $A(\omega) / U_0$ umgestellt ergibt sich
\begin{align*}
    \frac{A(\omega)}{U_0} = \frac{1}{\sqrt{1 + \omega^2  \tau^2}}
\end{align*}
Mit der \texttt{scipy.optimize} Funktion \texttt{curvefit} \cite{scipy} wird der optimale Wert für $\tau$ ermittelt.
Es ergibt sich 
\begin{align}
    \tau_\text{b} = (\num{6.86}\pm \num{1.13})\unit{\milli\s} 
    \label{eq:tau_b}
\end{align}
Die durch die einfache Ausgleichsrechnung entstehende Kurve (Abb. \ref{fig:b_plot}) ist um einen konstanten Faktor zu groß.
Wenn diese Kurve mit dem Faktor \num{0.4} multipliziert wird passt sie wiederum sehr gut zu den Messergebnissen.
Da hier wohl eine Dämpfung gleichmäßig auf alle Messergebnisse gewirkt hat,
ist dadurch keine direkte Verzerrung des Wertes für $\tau$ zu erwarten.
%
\begin{figure}
    \centering
    \includegraphics{build/C02_aufg_b.pdf}
    \caption{Graphische Darstellung Amplitude in Abhängigkeit von der Frequenz.}
    \label{fig:b_plot}
\end{figure}

In Abbildung \ref{fig:c_plot} wird die Phasenverschiebung $\Phi$ in Abhängigkeit von der Frequenz $f$ dargestellt.
In der Ausgleichsrechnung für diesen Zusammenhang wird die Gleichung \refeq{eq:phasenverschiebung} verwendet.
Da in dieser Gleichung davon ausgegangen wird, dass die Phasenunterschiede als negative Zahlen notiert wurden,
und dass keine Dämpfung stattfindet wird ein Korrekturfaktor $a$ eingeführt.
Die Gleichung, die sich ergibt lautet
\begin{align*}
    \Phi(\omega) = a \cdot \arctan(- \omega \tau_\text{c}).
\end{align*}
Es ergeben sich die Parameter
\begin{align}
    \tau_\text{c} &= (\num{6.87} \pm \num{0.94}) \unit{\milli\s} & a &= \num{-0.5976186} \pm \num{0.02437771}
    \label{eq:tau_c}
\end{align}
% 
\begin{figure}
    \centering
    \includegraphics{build/C02_aufg_c.pdf}
    \caption{Die Phasenverschiebung in Abhängigkeit von der Frequenz}
    \label{fig:c_plot}
\end{figure}


In Abbildung \ref{fig:d_polarplot} wird die mit $U_0$ normierte Amplitude mit der Phasenverschiebung $\phi$ in einem Polarplot dargestellt.
Aus den vorherigen Regressionen lässt sich eine Ausgleichskurve an dem Polarplot zusammenführen.
% In Gleichung \refeq{eq:sinusgleichung} wurde bereits gezeigt dass sich die Phasenverschiebung und die Amplitude ineinander umrechnen lassen.

\begin{figure}
    \centering
    \includegraphics{build/C02_aufg_c.pdf}
    \caption{Polarplot mit der Kombination aus $A(\omega)$ und $\Phi$ und den jeweiligen Regressionen}
    \label{fig:d_polarplot}
\end{figure}


% Aus Gleichung \textbf{sinusgleichung} erhält man für diesen Zusammenhang die Relation
% \begin{align*}
    % \frac{A(\omega)}{U_0} = - \frac{\sin(\Phi)}{\omega \tau}
% \end{align*}
% Mit der berechneten 