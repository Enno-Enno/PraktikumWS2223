\subsection{Entladung des Kondensators}
Wie in Abschnitt \ref{sec:durchf_a} beschreiben werden die Werte des Oszillatorbildes aufgenommen. 
In Tabelle \ref{tab:RC_a} werden die Spannungen zunächst als Kästcheneinheit aufgeschrieben.
Ein Kästchen auf dem Oszilloskop entspricht $\qty{0.2}{\volt}$.
%
\begin{table}
    \centering
    \caption{Der Entladevorgang des RC-kreises}
    \label{tab:RC_a}
    \sisetup{table-format=1.1}
    \begin{tabular}[]{
        S
        S
        S
        S[table-format=1.2]
        S[table-format=1.3]
    }
    {$t / \text{Kästchen}$} &
    {$U / \text{Kästchen}$} & 
    {$t / \unit{\milli\s}$} &
    {$U / \unit{\volt}$}    &
    {$(U - U(\infty)) / \unit{\volt}$}\\
         0   &   2.7 & 0.0  &  0.54 & 1.302 \\
         0.2 &   0.0 & 0.4  &  0.00 & 0.762 \\
         0.6 &  -2.3 & 1.2  & -0.46 & 0.302 \\
         0.8 &  -3.0 & 1.6  & -0.60 & 0.162 \\
         1.0 &  -3.3 & 2.0  & -0.66 & 0.102 \\
         1.2 &  -3.4 & 2.4  & -0.68 & 0.082 \\
         1.4 &  -3.5 & 2.8  & -0.70 & 0.062 \\
         1.6 &  -3.6 & 3.2  & -0.72 & 0.042 \\
         1.8 &  -3.7 & 3.6  & -0.74 & 0.022 \\
         2.0 &  -3.8 & 4.0  & -0.76 & 0.002 \\
         2.2 &  -3.8 & 4.4  & -0.76 & 0.002 \\
         2.4 &  -3.8 & 4.8  & -0.76 & 0.002 \\
         3.0 &  -3.8 & 6.0  & -0.76 & 0.002 \\
         3.6 &  -3.8 & 7.2  & -0.76 & 0.002 \\
         4.0 &  -3.8 & 8.0  & -0.76 & 0.002 \\
         5.0 &  -3.8 & 10.0 & -0.76 & 0.002 \\
    \end{tabular}
\end{table}
%
Um eine bessere Auswertung zu ermöglichen werden in Abbildung \ref{fig:a_Entladung} von den Messergebnissen 
das Endladungsniveau mit $U(\infty) = \qty[]{-0.762}{\volt}$ abgezogen.
Das ist legitim, da jedes mögliche Spannungsniveau als Nullniveau gewählt werden kann.
Außerdem wird der Logarithmus der Spannungen gebildet um den erwarteten exponentiellen Verlauf mit einer linearen Regression anzunähern.
Hierzu wird die \texttt{numpy} Funktion \texttt{polyfit} verwendet.
%Vielleicht kommt dieser Formelblock auch wieder raus
\begin{align*}
    U(t)&= U_0 \exp \left(\frac{-t}{RC}\right) + U(\infty) \\
    \ln(U(t)) &= \ln(U_0) \frac{-t}{RC} + \ln(U(\infty))
\end{align*}
% params [-0.74102744 -0.96447492]
% abweichung: [0.11292009 0.53204419]
Die lineare Regression ergibt eine Funktion der Form
\begin{align*}
    y  = A t - B
\end{align*}
mit
\begin{align*}
    A = - \ln(U_0) \cdot \frac{1}{RC}
\end{align*}


\begin{figure}
    \centering
    \includegraphics{build/C01_aufg_a.pdf}
    \caption{Graphische Darstellung des Entladevorgangs der Spule}
    \label{fig:a_Entladung}
\end{figure}
