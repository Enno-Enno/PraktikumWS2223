\subsection[]{Die Reynoldszahl}
Um zu ermitteln, ob es sich um eine laminare Strömung handelt, wird jeweils die dimensionslose Reynoldszahl $Re$ für die kleine und große Kugel bestimmt
(vgl. \eqref{eq:reynold}).
Bei höheren Temperaturen können Verwirbelungen entstehen, die dafür sorgen, dass keine laminare Strömung mehr vorliegt.
Das würde bei der Messreihe in Abschnitt \ref{sec:andra_gl} zu Abweichungen von den erwarteten Ergebnissen führen.
Um das zu vermeiden wird die Reynoldszahl für die höchste Temperatur der Messreihe berechnet.
Die Reynoldszahlen für die kleineren Temperaturen sollten kleiner sein, da die Geschwindigkeiten niedriger und die Viskositäten höher sind.
Zur Berechnung fehlen noch Mittelwert und Standardabweichung der Fallgeschwindigkeit $\overline{v}$ jeweils für oben und unten
Die Dichte von Wasser bei Zimmertemperatur beträgt $\rho_{\qty{19}{\celsius}} = \qty{0.99841}{\g \per \cm^3}$ (vgl. \ref{sec:visk_wasser_zt}),
bei $\qty{52}{\celsius}$ beträgt sie $\rho_{\qty{52}{\celsius}} = \qty{0.988}{\g \per \cm^3}$.
Diese werden mit der Gaußschen Fehlerfortpflanzung \eqref{eq:gauspflanz} verrechnet.
%
\begin{table}
    \caption[]{Reynoldszahl $Re$ in Abhängigkeit der Fallgeschwindigkeit $\overline{v}$}
    \label{tab:reynold}
    \centering
    \sisetup{table-format=1.3}
    \begin{tabular}[]{S S S @{${}\pm{}$} S S[table-format=3.0] @{${}\pm{}$} S[table-format=2.0]}
        \toprule
        {Kugel} & {Strecke} & \multicolumn{2}{c}{$\overline{v} / \left(\unit{\cm\per\s}\right)$} & \multicolumn{2}{c}{$Re$} \\
        \midrule
        {klein} & {oben} & 0.811 & 0.010 & 549 & 14 \\
         & {unten} & 0.816 & 0.012 & 555 & 17 \\
        {groß} & {oben} & 0.143 & 0.001 & 97 & 2 \\
         & {unten} & 0.140 & 0.003 & 96 & 3 \\
        {groß \qty{52}{\celsius}} & {oben} & \multicolumn{2}{S}{0.256} & 304 & 5 \\
         & {unten} & \multicolumn{2}{S}{0.261} & 325 & 9 \\
        \bottomrule
    \end{tabular}
\end{table}
%
Es zeigt sich: Alle Werte für die Reynoldszahl sind unterhalb der kritischen Grenze von 2300.

% Kugel % Strecke % MW G % S G % MW Rey % S Rey
%Reynold:
%Geschwindigkeit klein oben:  0.811+/-0.010
%Geschwindigkeit klein unten:  0.816+/-0.012
%Geschwindigkeit gross oben:  0.1428+/-0.0012
%Geschwindigkeit gross unten:  0.1404+/-0.0030

%rey klein oben:  549+/-14
%rey klein unten:  555+/-17
%rey gross oben:  96.5+/-1.5
%rey gross unten:  95.5+/-2.5
