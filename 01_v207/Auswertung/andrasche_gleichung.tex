
\subsection[]{Messreihe 3: Überprüfung der Andradeschen Gleichung}
\begin{figure}
    \centering
    \includegraphics[width=\textwidth]{build/Messreihe3.pdf}
    \caption{Die Fallzeit nach \ref{tab:grKu_steigendeTemp}.}
    \label{fig:groKu_steigendeTemp}
\end{figure}

\begin{table}
    \centering
    \sisetup{table-format=1.2}
    \begin{tabular}[]{S[table-format=2.0] S[table-format=0.5] S @{${}\pm{}$} S[table-format=1.3] S @{${}\pm{}$} S[table-format=1.3]}
        \toprule
        {$T /\unit{\celsius}$} & {$\rho_{\text{Wasser}}$ \cite{geschke}} & \multicolumn{2}{c}{$ \eta / \unit{\milli\Pa\s}$} & \multicolumn{2}{c}{$ \eta / \unit{\milli\Pa\s}$} \\
            26  &  0.99679   & 1.03  & 0.021 &        1.02 & 0.027  \\
            27  &  0.99652   & 1.01  & 0.018 &        1.01 & 0.026  \\
            30  &  0.99565   & 0.96  & 0.017 &        0.95 & 0.026  \\
            32  &  0.99565   & 0.93  & 0.028 &        0.91 & 0.028  \\
            35  &  0.994     & 0.86  & 0.015 &        0.85 & 0.030  \\
            38  &  0.9922    & 0.83  & 0.025 &        0.81 & 0.023  \\
            40  &  0.9922    & 0.79  & 0.016 &        0.77 & 0.020  \\
            43  &  0.9902    & 0.75  & 0.04  &        0.73 & 0.04   \\
            48  &  0.988     & 0.69  & 0.012 &        0.69 & 0.019  \\
            52  &  0.988     & 0.66  & 0.020 &        0.63 & 0.028  \\
    \end{tabular}
    \label{tab:viskositaeten_temp}
    \caption{Die Viskosität in Abhängigkeit der Temperatur}
\end{table}
\begin{figure}
    \centering
    \includegraphics[width=\textwidth]{build/plot.pdf}
    \caption{Die Viskositäten nach \textbf{Tabelle 6}.}
    \label{fig:groKu_steigendeTemp_eta}
\end{figure}

%Formt man die Andradesche Gleichung \eqref{eq:andrade} so um, dass auf beiden Seiten der Logarithmus gezogen wird, ergibt sich
%
%\begin{align}
%    \ln {\left( \eta \right)} = \frac{B}{T} + \ln{\left( A \right)}.
%\end{align}
%Mit den zuvor \ref{tab:viskositaeten_temp} berechneten Werten für die Viskosität kann mittels der Numerical Python Funktion \texttt{polyfit}
%eine Ausgleichsgerade erstellt werden. Dadurch lassen sich die Konstanten $A$ und $B$ ermitteln.

Für die regression der in \ref{tab:viskositaeten_temp} angegebenen und in \ref{fig:groKu_steigendeTemp_eta} dargestellten Größen
hat uns leider die Zeit gefehlt. 

\subsection[]{Die Reynoldszahl}
Um zu ermitteln, ob es sich um eine laminare Strömung handelt, wird jeweils die dimensionslose Reynoldszahl $Re$ für die kleine und große Kugel bestimmt
(vgl. Abschnitt \ref{sec:vorbereitung}).
Zur Berechnung fehlen noch Mittelwert und Standardabweichung der Fallgeschwindigkeit $\overline{v}$ jeweils für oben und unten
(Dichte von Wasser bei Zimmertemperatur: $\rho = \qty{0.99841}{\g \per \cm^3}$, s.o.).

\begin{table}
    \caption[]{Reynoldszahl $Re$ in Abhängigkeit der Fallgeschwindigkeit $\overline{v}$}
    \label{tab:reynold}
    \centering
    \begin{tabular}[]{S S S[table-format=1.4] @{${}\pm{}$} S[table-format=1.4] S[table-format=4.1] @{${}\pm{}$} S[table-format=3.1]}
        \toprule
        {Kugel} & {Strecke} & \multicolumn{2}{c}{$\overline{v} / \left(\unit{\cm\per\s}\right)$} & \multicolumn{2}{c}{$Re$} \\
        \bottomrule
        {klein} & {oben} & 0.811 & 0.010 & 549 & 14 \\
         & {unten} & 0.816 & 0.012 & 555 & 17 \\
        {groß} & {oben} & 0.1428 & 0.0012 & 96.5 & 1.5 \\
         & {unten} & 0.1404 & 0.0030 & 95.5 & 2.5 \\
    \end{tabular}
\end{table}

% Kugel % Strecke % MW G % S G % MW Rey % S Rey
%Reynold:
%Geschwindigkeit klein oben:  0.811+/-0.010
%Geschwindigkeit klein unten:  0.816+/-0.012
%Geschwindigkeit gross oben:  0.1428+/-0.0012
%Geschwindigkeit gross unten:  0.1404+/-0.0030

%rey klein oben:  549+/-14
%rey klein unten:  555+/-17
%rey gross oben:  96.5+/-1.5
%rey gross unten:  95.5+/-2.5
