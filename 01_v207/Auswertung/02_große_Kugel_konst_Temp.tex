\subsection{Messreihe 2: große Kugel, konstante Temperatur}
\begin{table}[]
    \caption{Große Kugel bei Zimmertemperatur; Fallhöhe = 5 cm}
    \label{tab:grKu_Zitemp}
    \centering
    \sisetup{table-format=2.2}
    \begin{tabular}{S S}
        \toprule
        \multicolumn{2}{c}{Fallzeit $ t / \unit{\s}$}\\
        {oben} & {unten}\\
        \midrule
            34.78 &  34.75 \\
            34.65 &  34.94 \\
            35.00 &  36.90 \\
            35.22 &  35.63 \\
            35.47 &  35.78 \\
        \bottomrule

    \end{tabular}
\end{table}

In der zweiten Messreihe werden die Mittelwerte und Standardmessfehler analog zu \ref{sec:auswertung_mr1} berechnet.
Die $N=5$ Messwerte befinden sich in \ref{tab:grKu_Zitemp}


%%%%Für die große Kugel erhält man ganz analog zu \ref{sec:auswertung_mr1} für oben die Werte $t_{\text{gr,oben}} = \left(35.024 \pm 0.296\right) \, \unit{\s}$ 
%%%%bzw. für unten $t_{\text{gr,unten}} = \left(35.600 \pm 0.759\right) \, \unit{\s}$.
%%%%
%%%%Zusammen mit den zuvor bestimmten Viskositäten bei Zimmertemperatur $\eta_{\text{kl, oben}}$ und $\eta_{\text{kl, unten}}$ ergibt sich dann eine Apparaturkonstante von
%%%%$K_{\text{gr, oben}} = \left(0.0236 \pm 0.0004 \right) \unit{\milli\Pa \cubic\cm\per\g}$ für oben bzw.
%%%%$K_{\text{gr, unten}} = \left(0.0231 \pm 0.0006 \right) \unit{\milli\Pa \cubic\cm}$ für unten.

%K oben und gross:  (2.36+/-0.04)e-05
%K unten und gross:  (2.31+/-0.06)e-05

