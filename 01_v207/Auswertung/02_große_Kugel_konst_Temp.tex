\subsection{Messreihe 2: große Kugel, konstante Temperatur}%
%
\subsubsection[]{Fallzeiten der großen Kugel}%
\label{sec:auswertung_mr2}%
\begin{table}[]%
    \caption{Fallzeiten der großen Kugel bei Zimmertemperatur}%
    \label{tab:grKu_Zitemp}%
    \centering%
    \sisetup{table-format=2.2}%
    \begin{tabular}{S S}%
        \toprule%
        \multicolumn{2}{c}{Fallzeit $ t_\text{groß} / \unit{\s}$}\\%
        {oben} & {unten}\\%
        \midrule%
            34.78 &  34.75 \\%
            34.65 &  34.94 \\%
            35.00 &  36.90 \\%
            35.22 &  35.63 \\%
            35.47 &  35.78 \\%
        \bottomrule%
    \end{tabular}%
\end{table}%
%
\noindent
In der zweiten Messreihe werden die Mittelwerte und Standardmessfehler analog zu \ref{sec:auswertung_mr1} berechnet.
Hierfür werden für beide Fallrichtungen je $N=5$ Messwerte auf einer Strecke von \qty[]{5}{\cm} erhoben.
Die Messwerte befinden sich in Tabelle \ref{tab:grKu_Zitemp} und die zugehörigen Mittelwerte sowie deren mittels
\eqref{eq:gauspflanz} bestimmten Standardabweichungen in Tabelle \ref{tab:mittel_fallzeit_gross}.
%
\begin{table}
    \centering
    \caption{Durchschnittliche Fallzeiten für die große Kugel}
    \label{tab:mittel_fallzeit_gross}
    \sisetup{table-format=2.2}
    \begin{tabular}[]{c S @{${}\pm{}$} S}
        \toprule
        Ausrichtung   & \multicolumn{2}{c}{$\overline{t_\text{groß}} / \unit{\s}  $} \\
        \midrule
        oben  & \num{35.02} &  \num{0.30}    \\
        unten & \num{35.60} &  \num{0.76}    \\
        \bottomrule
        \end{tabular}
\end{table}
%
\subsubsection[]{Die Apparatekonstante}
%\begin{align}
%    \label{eq:empirisch}
%    \eta = K \left(\rho_K -\rho_{Fl}\right) \cdot t
%\end{align}
Zur Bestimmung der Apparatekonstanten wird Formel \eqref{eq:empirisch} nach der Apparatekonstanten $K$ um, ergibt sich:
\begin{align*}
    K = \frac{\eta}{\left(\rho_\text{K} - \rho_\text{Fl}\right) t}
\end{align*}
Zusammen mit den zuvor bestimmten durchschnittlichen Fallzeiten $\overline{t_\text{groß}}$ (vgl. Tabelle \ref{tab:mittel_fallzeit_gross})
und Viskositäten $\overline{\eta_{\text{klein}}}$ sowie den Dichten der kleinen Kugel $\rho_\text{K}$ und des Wassers $\rho_\text{Fl}$
ergeben sich für beide Fallrichtungen die Apparatekonstanten in Tabelle \ref{tab:mittel_Apparatekonstante}.
Auch hier berechnen sich die Standardabweichungen mittels \eqref{eq:gauspflanz}.
% Vielleicht noch die umgestellte Formel für die Apparatekonstante aufschreiben.
%
\begin{table}
    \centering
    \caption{Apparatekonstanten für die große Kugel}
    \label{tab:mittel_Apparatekonstante}
    \sisetup{table-format=1.4}
    \begin{tabular}[]{c S @{${}\pm{}$} S}
        \toprule
        Ausrichtung   & \multicolumn{2}{c}{$\overline{K_\text{groß}} / \left(\unit{\milli\Pa \cubic\cm\per\g}\right)  $} \\
        \midrule
        oben  & \num{0.0236} &  \num{0.0004}    \\
        unten & \num{0.0231} &  \num{0.0006}    \\
        \bottomrule
        \end{tabular}
\end{table}
%
%%%%Für die große Kugel erhält man ganz analog zu \ref{sec:auswertung_mr1} für oben die Werte $t_{\text{gr,oben}} = \left(35.024 \pm 0.296\right) \, \unit{\s}$ 
%%%%bzw. für unten $t_{\text{gr,unten}} = \left(35.600 \pm 0.759\right) \, \unit{\s}$.
%%%%
%%%%Zusammen mit den zuvor bestimmten Viskositäten bei Zimmertemperatur $\eta_{\text{kl, oben}}$ und $\eta_{\text{kl, unten}}$ ergibt sich dann eine Apparatekonstante von
%%%%$K_{\text{gr, oben}} =  \left(0.0236 \pm 0.0004 \right) \unit{\milli\Pa \cubic\cm\per\g}$ für oben bzw.
%%%%$K_{\text{gr, unten}} = \left(0.0231 \pm 0.0006 \right) \unit{\milli\Pa \cubic\cm}$ für unten.
%
%K oben und gross:  (2.36+/-0.04)e-05
%K unten und gross:  (2.31+/-0.06)e-05