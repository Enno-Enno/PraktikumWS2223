\subsection{Kenngrößen der Glaskugeln}

Die Standard-Messunsicherheit der Schieblehre wird mit \qty{0.01}{\mm} angegeben.
Die Dichte $\rho$ errechnet sich aus dem Verhältnis zwischen Masse $m$ und Volumen
$V=\frac{4 \pi}{3} \left(\frac{d}{2}\right)^3$ und $\rho = \frac{m}{V}$.
Die Messunsicherheiten werden mithilfe der Gaußschen Fehlerfortpflanzung berechnet.
Es gilt:
\begin{align*}
    \rho &= \frac{m}{V} = \frac{3 m}{4 \pi} \left(  \frac{2}{d} \right)^{3} = \frac{6 m}{\pi} d^{-3} \\%
    \frac{\partial \rho}{\partial d} &= -\frac{18 m}{\pi} d^{-4}  \\%
    %
    \intertext{Da für die Masse keine Standardabweichung gegeben ist,%
     erhält die Gaußsche Fehlerfortpflanzung nur ein Argument}%
    \Delta \rho &= \sqrt{\left(\frac{\partial \rho}{\partial d} \right)^{2} \Delta d^{2} }%
    =\sqrt{\left(-\frac{18 m}{\pi} d^{-4} \right)^{2}} \Delta d \\%
\end{align*}%
%
Einsetzen der Unsicherheiten und der gemessen Werte aus Tabelle \ref{tab:kenngroessen} ergibt die Werte in Tabelle \ref{tab:Dichten}.%

\begin{table}
    \caption{}
    \label{tab:Dichten}
    \centering
    \begin{tabular}[]{S S[table-format=1.4] S[table-format=2.2] @{${}\pm{}$} S[table-format=0.2] S[table-format=1.4] @{${}\pm{}$} S[table-format=0.2]}
        \toprule
        {Kugel} & {$m / \unit{\g}$} & \multicolumn{2}{c}{$d / \unit{\mm}$}  & \multicolumn{2}{c}{{$\rho / \left( \unit{\g \per \cubic\cm}\right)$}} \\
        \midrule
        {klein} & 4.4531 & 15.61 & 0.01 & 2.2359 & 0.005\\
        {groß}  & 2.9528 & 15.78 & 0.01 & 2.4073 & 0.005\\
        \bottomrule 

    \end{tabular}
\end{table}
Um Rechenfehler zu vermeiden werden für die konkrete Berechnung der Messunsicherheiten 
das Python-Paket uncertainties \cite{uncertainties} \cite{python} verwendet.


\subsection{Messreihe 1: kleine Kugel, konstante Temperatur}
\label{sec:auswertung_mr1}

Für die kleine Kugel oben ergibt sich mit Tabelle \ref{tab:klKu_Zitemp} 
also für die Fallzeit ein arithmetisches Mittel von%
%
\begin{align*}
    \overline{t}_{\text{kl, oben}} &= \frac{1}{10} \sum_{k=1}^{10} {t_{\text{kl, oben}, k}} = \qty[]{12.324}{\s}\\%
%
\intertext{mit einer Standardabweichung von}%
%
    \Delta \overline{t}_{\text{kl, oben}} &= 
    \sqrt[]{\frac{1}{N \left(N-1\right)} \sum_{k=1}^{10} \left(t_{\text{kl, oben}, k} - \overline{t}_{\text{kl, oben}}\right)^2}
    = \qty[]{0.155}{\s}\\%
    %
\end{align*}
Analog folgt für unten eine Fallzeit von der Mittelwert $\overline{t}_{\text{kl, unten}} = \qty[]{12.248}{\s}$
und die Standardabweichung $\Delta\overline{t}_{\text{kl, unten}} = \qty[]{0.184}{\s}$.

%Durchschnittszeiten kleine Kugel:
%oben: 12.32400 mit Fehler:  0.15500
%unten: 12.24800 mit Fehler:  0.18405

%Durchschnittszeiten grosse Kugel:
%oben: 35.02400 mit Fehler:  0.29574
%unten: 35.60000 mit Fehler:  0.75913

Für die kleine Kugel beträgt nach \cite*[]{va207} die Apparaturkonstante $K_{\text{kl}} = \qty[]{0.07640}{\milli\Pa \cubic\cm}$.
Nach \cite[]{geschke} beträgt bei $T = \qty{19}{\degreeCelsius}$ die Dichte von Wasser $\rho = \qty{0.99841}{\g \per \cm^3}$.
Mit Gleichung \eqref{eq:empirisch} folgt somit für die Viskosität oben $\eta_{\text{kl, oben}} = \left(1.165 \pm 0.015\right) 10^{-3} \, \unit{\pascal \s}$
und unten $\eta_{\text{kl, unten}} = \left(1.158 \pm 0.018\right) 10^{-3} \, \unit{\pascal \s}$.

%eta klein oben:  0.001165+/-0.000015
%eta klein unten:  0.001158+/-0.000018
