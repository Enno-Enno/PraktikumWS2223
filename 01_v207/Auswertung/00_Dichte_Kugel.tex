\subsection{Kenngrößen der Glaskugeln}



Wenn zu Messdaten die Standardabweichung bekannt ist, und mit diesen Messdaten weiter gerechnet werden soll,
wird die Gaußsche Fehlerfortpflanzung verwendet. 
Angenommen wir haben $k$ Messwerte $x_i ~[i \in \mathbb{N}, i \leq k]$ mit den Standardabweichungen $\Delta x_i$
und eine abgeleitete Größe $f(x_i)$ dann ist der Fehler von $f$
\begin{align}
    \Delta f(x_i) = \sqrt{
    \left(\frac{\partial f}{\partial x_1} \Delta x_1\right)^2%
     + \left(\frac{\partial f}{\partial x_2} \Delta x_2\right)^2%
     + \dots%
     + \left(\frac{\partial f}{\partial x_k} \Delta x_k\right)^2%
    }.
    \label{eq:gauspflanz}
\end{align} 
Im Ergebnis gibt es dann den Mittelwert von $f$ mit der errechneten Abweichung $\overline{f} \pm \Delta f $.
%Um Rechenfehler zu vermeiden wird das Python Paket \texttt{uncertainties} \cite[][]{uncertainties} verwendet.
%Hier wird die Fehlerfortpflanzung automatisch verrechnet, wenn man die Variablen als \texttt{ufloat} definiert.
%Die Gaußsche Fehlerfortpflanzung \eqref{eq:gauspflanz} wird hier das erste Mal verwendet.
Die Standard-Messunsicherheit der Schieblehre wird mit $\qty{0.01}{\mm}$ angegeben.
Die Dichte $\rho$ errechnet sich aus dem Verhältnis zwischen Masse $m$ und Volumen.
\begin{align*}
    V &= \frac{4 \pi}{3} \left(\frac{d}{2}\right)^3 & \rho &= \frac{m}{V} \\
    \frac{\partial \rho}{\partial d} &= -\frac{18 m}{\pi} d^{-4} %
\end{align*}
%%%%%
%\rho &= \frac{m}{V} = \frac{3 m}{4 \pi} \left(  \frac{2}{d} \right)^{3} = \frac{6 m}{\pi} d^{-3} \\%
%
Da für die Masse keine Standardabweichung gegeben ist, enthält die Gaußsche Fehlerfortpflanzung nur einen Summanden unter der Wurzel.
\begin{align*}
    \Delta \rho &= \sqrt{\left(\frac{\partial \rho}{\partial d} \right)^{2} \Delta d^{2} }%
    =\sqrt{\left(-\frac{18 m}{\pi} d^{-4} \right)^{2} \Delta d^{2} }%
\end{align*}%
Für die Dichten $\rho$ der Kugeln folgen somit die Werte in Tabelle \ref{tab:Dichten}.



\begin{table}
    \caption{Die Dichten $\rho$ mit Messfehlern}
    \label{tab:Dichten}
    \centering
    \begin{tabular}[]{S S[table-format=1.4] S[table-format=2.2] @{${}\pm{}$} S[table-format=0.2] S[table-format=1.4] @{${}\pm{}$} S[table-format=0.2]}
        \toprule
        {Kugel} & {$m / \unit{\g}$} & \multicolumn{2}{c}{$d / \unit{\mm}$}  & \multicolumn{2}{c}{{$\rho / \left( \unit{\g \per \cubic\cm}\right)$}} \\
        \midrule
        {klein} & 4.4531 & 15.61 & 0.01 & 2.2359 & 0.005\\
        {groß}  & 2.9528 & 15.78 & 0.01 & 2.4073 & 0.005\\
        %                        \pm             \pm
        \bottomrule 

    \end{tabular}
\end{table}
%Um Rechenfehler zu vermeiden werden für die konkrete Berechnung der Messunsicherheiten 
%das Python-Paket uncertainties \cite{uncertainties} \cite{python} verwendet.


