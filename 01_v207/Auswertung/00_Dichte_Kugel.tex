\subsection{Kenngrößen der Glaskugeln}

Die Standard-Messunsicherheit der Schieblehre wird mit $\qty{0.01}{\mm}$ angegeben.
Die Dichte $\rho$ errechnet sich aus dem Verhältnis zwischen Masse $m$ und Volumen
\begin{align*}
    V = \frac{4 \pi}{3} \left(\frac{d}{2}\right)^3 & \rho = \frac{m}{V}
\end{align*}
Die Messunsicherheiten werden mithilfe der Gaußschen Fehlerfortpflanzung \ref{eq:gauspflanz} berechnet.
%Es gilt:
%%%%%
%\begin{align*}
%    \rho &= \frac{m}{V} = \frac{3 m}{4 \pi} \left(  \frac{2}{d} \right)^{3} = \frac{6 m}{\pi} d^{-3} \\%
%    \frac{\partial \rho}{\partial d} &= -\frac{18 m}{\pi} d^{-4}  \\%
%    %
%    \intertext{Da für die Masse keine Standardabweichung gegeben ist,%
%     erhält die Gaußsche Fehlerfortpflanzung nur ein Argument}%
%    \Delta \rho &= \sqrt{\left(\frac{\partial \rho}{\partial d} \right)^{2} \Delta d^{2} }%
%    =\sqrt{\left(-\frac{18 m}{\pi} d^{-4} \right)^{2}} \Delta d \\%
%\end{align*}%
%
%Einsetzen der Unsicherheiten und der gemessen Werte aus Tabelle \ref{tab:kenngroessen} ergibt die Werte in Tabelle \ref{tab:Dichten}.%

\begin{table}
    \caption{}
    \label{tab:Dichten}
    \centering
    \begin{tabular}[]{S S[table-format=1.4] S[table-format=2.2] @{${}\pm{}$} S[table-format=0.2] S[table-format=1.4] @{${}\pm{}$} S[table-format=0.2]}
        \toprule
        {Kugel} & {$m / \unit{\g}$} & \multicolumn{2}{c}{$d / \unit{\mm}$}  & \multicolumn{2}{c}{{$\rho / \left( \unit{\g \per \cubic\cm}\right)$}} \\
        \midrule
        {klein} & 4.4531 & 15.61 & 0.01 & 2.2359 & 0.005\\
        {groß}  & 2.9528 & 15.78 & 0.01 & 2.4073 & 0.005\\
        %                        \pm             \pm
        \bottomrule 

    \end{tabular}
\end{table}
%Um Rechenfehler zu vermeiden werden für die konkrete Berechnung der Messunsicherheiten 
%das Python-Paket uncertainties \cite{uncertainties} \cite{python} verwendet.


