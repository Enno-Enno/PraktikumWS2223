\subsection{Messreihe 1: kleine Kugel, konstante Temperatur}
\label{sec:auswertung_mr1}

\begin{table}[]
    \caption{Kleine Kugel bei Zimmertemperatur; Fallhöhe = 10 cm}
    \label{tab:klKu_Zitemp}
    \centering
    \sisetup{table-format=2.2}
    \begin{tabular}{S S}
        \toprule
        \multicolumn{2}{c}{Fallzeit $ t / \unit{\s}$}\\
        {oben} & {unten}\\
        \midrule
        12.21 &  12.66 \\
        12.40 &  12.00 \\
        12.25 &  12.32 \\
        12.65 &  12.06 \\
        12.16 &  12.32 \\
        12.13 &  12.09 \\
        12.50 &  12.15 \\
        12.38 &  12.18 \\
        12.34 &  12.38 \\
        12.22 &  12.32 \\
        \bottomrule

    \end{tabular}
\end{table}
%
%
Für die kleine Kugel oben ergibt sich mit Tabelle \ref{tab:klKu_Zitemp} 
also für die Fallzeit ein arithmetisches Mittel von%
%
\begin{align*}
    \overline{t}_{\text{kl, oben}} &= \frac{1}{10} \sum_{k=1}^{10} {t_{\text{kl, oben}, k}} = \qty[]{12.324}{\s}\\%
%
\intertext{mit einer Standardabweichung von}%
%
    \Delta \overline{t}_{\text{kl, oben}} &= 
    \sqrt[]{\frac{1}{N \left(N-1\right)} \sum_{k=1}^{10} \left(t_{\text{kl, oben}, k} - \overline{t}_{\text{kl, oben}}\right)^2}
    = \qty[]{0.155}{\s}\\%
    %
\end{align*}
Analog folgt für unten eine Fallzeit von der Mittelwert $\overline{t}_{\text{kl, unten}} = \qty[]{12.248}{\s}$
und die Standardabweichung $\Delta\overline{t}_{\text{kl, unten}} = \qty[]{0.184}{\s}$.
%
%Durchschnittszeiten kleine Kugel:
%oben: 12.32400 mit Fehler:  0.15500
%unten: 12.24800 mit Fehler:  0.18405
%
%Durchschnittszeiten grosse Kugel:
%oben: 35.02400 mit Fehler:  0.29574
%unten: 35.60000 mit Fehler:  0.75913
%
Für die kleine Kugel beträgt nach \cite*[]{va207} die Apparatekonstante $K_{\text{kl}} = \qty[]{0.07640}{\milli\Pa \cubic\cm}$.
Nach \cite[]{geschke} beträgt bei $T = \qty{19}{\degreeCelsius}$ die Dichte von Wasser $\rho = \qty{0.99841}{\g \per \cm^3}$.
Mit Gleichung \eqref{eq:empirisch} folgt somit für die Viskosität oben $\eta_{\text{kl, oben}} = \left(1.165 \pm 0.015\right) 10^{-3} \, \unit{\pascal \s}$
und unten $\eta_{\text{kl, unten}} = \left(1.158 \pm 0.018\right) 10^{-3} \, \unit{\pascal \s}$.

%eta klein oben:  0.001165+/-0.000015
%eta klein unten:  0.001158+/-0.000018
