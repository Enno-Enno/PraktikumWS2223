\subsection{Messreihe 1: kleine Kugel, konstante Temperatur}
\label{sec:auswertung_mr1}
%
\subsubsection[]{Arithmetisches Mittel und Standardabweichung}
Bei Messungen von denen die Standardmessunsicherheit nicht bekannt ist, 
ist es sinnvoll mehrere Messungen zu machen und von ihnen das arithmetische Mittel zu bilden.
So entsteht eine genauere Messung. % Überdenken
Das arithmetische Mittel einer Messreihe mit $N$ Werten $x_i$ wird folgendermaßen berechnet:
\begin{align}
    \overline{x} = \frac{1}{N} \sum_{i=1}^{N}x_i
    \label{eq:artim_mit}
\end{align}
%
Der Standardmessfehler des arithmetischen Mittels der Messreihe berechnet sich dann wie folgt:
\begin{align}
    \Delta \overline{x} = \sqrt[]{\frac{1}{N \left(N-1\right)} \sum_{k=1}^{N} \left(x_{i} - \overline{x}\right)^2}
    \label{eq:stddev_mit}
\end{align}
%
%
\subsubsection[]{Fallzeiten der kleinen Kugel}
In der ersten Messreihe werden die Fallzeiten für die kleinere Kugel gemessen.
Das Wasser hat eine Zimmertemperatur von \qty{19}{\celsius} 
und die Fallstrecke beträgt \qty[]{10}{\cm}. 
Somit ergeben sich die Daten in Tabelle \ref{tab:klKu_Zitemp}.
\begin{table}%[h!]
    \caption{Fallzeiten der kleinen Kugel bei Zimmertemperatur}
    \label{tab:klKu_Zitemp}
    \centering
    \sisetup{table-format=2.2}
    \begin{tabular}{S S}
        \toprule
        \multicolumn{2}{c}{Fallzeit $ t_\text{klein} / \unit{\s}$}\\
        {oben} & {unten}\\
        \midrule
        12.21 &  12.66 \\
        12.40 &  12.00 \\
        12.25 &  12.32 \\
        12.65 &  12.06 \\
        12.16 &  12.32 \\
        12.13 &  12.09 \\
        12.50 &  12.15 \\
        12.38 &  12.18 \\
        12.34 &  12.38 \\
        12.22 &  12.32 \\
        \bottomrule
    \end{tabular}
\end{table}
%
Das arithmetischen Mittel der Laufzeiten $\overline{t_\text{oben}}$ und $\overline{t_\text{unten}}$
werden nach \eqref{eq:artim_mit} für $N= 10$ Werte aus Tabelle \ref{tab:klKu_Zitemp} berechnet.
Die Standardmessfehler des Mittelwertes werden gemäß \eqref{eq:artim_mit} berechnet.
Die Ergebnisse sind in Tabelle \ref{tab:mittel_fallzeit_klein} einzusehen.
%
\begin{table}[H]
    \centering
    \caption{Durchschnittliche Fallzeiten für die kleine Kugel}
    \label{tab:mittel_fallzeit_klein}
    \sisetup{table-format=2.2}
    \begin{tabular}[]{c S @{${}\pm{}$} S}
        \toprule
        Ausrichtung   & \multicolumn{2}{c}{$\overline{t_\text{klein}} / \unit{\s}  $} \\
        \midrule
        oben  & \num{12.32} &  \num{0.16}    \\
        unten & \num{12.25} &  \num{0.18}    \\
        \bottomrule
        \end{tabular}
\end{table}
%
%
% Alte Version:
%%%%%Für die kleine Kugel oben ergibt sich mit Tabelle \ref{tab:klKu_Zitemp} 
%%%%%also für die Fallzeit ein arithmetisches Mittel von%
%%%%%%
%%%%%\begin{align*}
%%%%%    \overline{t_{\text{kl, oben}}} &= \frac{1}{10} \sum_{k=1}^{10} {t_{\text{kl, oben}, k}} = \qty[]{12.324}{\s}\\%
%%%%%%
%%%%%\intertext{mit einer Standardabweichung von}%
%%%%%%
%%%%%    \Delta \overline{t_{\text{kl, oben}}} &= 
%%%%%    \sqrt[]{\frac{1}{90} \sum_{k=1}^{10} \left(t_{\text{kl, oben}, k} - \overline{t}_{\text{kl, oben}}\right)^2}
%%%%%    = \qty[]{0.155}{\s}\\%
%%%%%    %
%%%%%\end{align*}
%%%%%Analog folgt für unten eine Fallzeit von der Mittelwert $\overline{t}_{\text{kl, unten}} = \qty[]{12.248}{\s}$
%%%%%und die Standardabweichung $\Delta\overline{t}_{\text{kl, unten}} = \qty[]{0.184}{\s}$.
%
%Durchschnittszeiten kleine Kugel:
%oben: 12.32400 mit Fehler:  0.15500
%unten: 12.24800 mit Fehler:  0.18405
%
%Durchschnittszeiten grosse Kugel:
%oben: 35.02400 mit Fehler:  0.29574
%unten: 35.60000 mit Fehler:  0.75913
%
%\newpage
\subsubsection[]{Viskosität des Wassers}
Für die kleine Kugel beträgt nach \cite{va207} die Apparatekonstante $K_{\text{klein}} = \qty[]{0.07640}{\milli\Pa \cubic\cm\per\gram}$.
Nach \cite[][290]{geschke} beträgt bei $T = \qty{19}{\degreeCelsius}$ die Dichte von Wasser $\rho_\text{Fl} = \qty{0.99841}{\g \per \cm^3}$.
Mit Gleichung \eqref{eq:empirisch} folgen somit für die Viskosität $\overline{\eta_{\text{klein}}}$ des Wassers die Werte in Tabelle \ref{tab:visk_kl_Zitemp},
wobei die Fehler gemäß \eqref{eq:gauspflanz} berechnet wurden. 
%%%% Vielleicht muss das an eine frühere Stelle, ich denke aber nicht
% Da für die beiden unterschiedlichen Ausrichtungen des Viskosimeters kleine Unterschiede zu erwarten sind,
% werden die Ergebnisse in $x_\text{oben}$ und in $x_\text{unten}$ aufgeteilt.
% Dies steht dafür auf welcher Seite (oben oder unten) die Schläuche von der Wärmeinheit des 
% Viskosimeters \textbf{Referenz zu Abbildung einfügen} bei der jeweiligen Messung sind.
%
\begin{table}
    \centering
    \caption{Durchschnittliche Viskositäten des Wassers für die kleine Kugel}
    \label{tab:visk_kl_Zitemp}
    \sisetup{table-format=1.3}
    \begin{tabular}[]{c S @{${}\pm{}$} S}
        \toprule
        Ausrichtung   & \multicolumn{2}{c}{$\overline{\eta_{\text{klein}}} / \unit{\milli\pascal\s}  $} \\
        \midrule
        oben & \num{1.165} &  \num{0.015}    \\
        unten & \num{1.158} &  \num{0.018}    \\
        \bottomrule
        \end{tabular}
\end{table}
%Mit Gleichung \eqref{eq:empirisch} folgt somit für die Viskosität oben $\eta_{\text{kl, oben}} = \left(1.165 \pm 0.015\right) 10^{-3} \, \unit{\pascal \s}$
%und unten 
%$\eta_{\text{kl, unten}} = \left(1.158 \pm 0.018\right) 10^{-3} \, \unit{\pascal \s}$.
%%%%%%%%%%%%%% Eigentlich wäre das hier in einer Tabelle viel schöner.
%eta klein oben:  0.001165+/-0.000015
%eta klein unten:  0.001158+/-0.000018 
