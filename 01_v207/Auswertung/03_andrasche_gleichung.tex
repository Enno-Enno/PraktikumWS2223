
\subsection[]{Messreihe 3: Überprüfung der Andradeschen Gleichung}
\label{sec:andra_gl}

%\begin{figure}
%    \centering
%    \includegraphics[width=\textwidth]{build/Messreihe3.pdf}
%    \caption{Die Fallzeit nach \ref{tab:grKu_steigendeTemp}.}
%    \label{fig:groKu_steigendeTemp}
%\end{figure}
\subsubsection[]{Fallzeiten und Viskositäten}
\begin{table}[H]
    \caption{Fallzeiten der große Kugel bei variabler Temperatur}
    \label{tab:grKu_steigendeTemp}
    \centering
    \sisetup{table-format=2.2}
    \begin{tabular}{S[table-format=2.0] S[table-format=2.2] S  S  S }
        \toprule
        & \multicolumn{2}{c}{Fallzeit $ t / \unit{\s}$} & \multicolumn{2}{c}{ Mittelwert $ \overline{t} / \unit{\s}$} \\
        {$T / \unit{\celsius}$} & {oben} & {unten}  & {oben} & {unten}\\
        \midrule
            26 & 31.37 &  31.22 &   31.02   &  31.21    \\
               & 30.66 &  31.19 &           &           \\
            27 & 30.43 &  31.12 &   30.40   &  30.88    \\
               & 30.37 &  30.63 &           &           \\
            30 & 29.03 &  29.78 &   28.93   &  29.16    \\
               & 28.82 &  28.53 &           &           \\
            32 & 27.63 &  27.96 &   27.87   &  28.03    \\
               & 28.10 &  28.10 &           &           \\
            35 & 25.82 &  25.81 &   25.75   &  25.99    \\
               & 25.68 &  26.16 &           &           \\
            38 & 24.69 &  24.81 &   24.89   &  24.89    \\
               & 24.93 &  24.97 &           &           \\
            40 & 23.81 &  23.78 &   23.56   &  23.44    \\
               & 23.31 &  23.09 &           &           \\
            43 & 22.28 &  22.37 &   22.32   &  22.26    \\
               & 22.35 &  22.15 &           &           \\
            48 & 20.57 &  21.00 &   20.45   &  20.89    \\
               & 20.32 &  20.78 &           &           \\
            52 & 19.65 &  19.09 &   19.56   &  19.13    \\
               & 19.46 &  19.16 &           &           \\
        \bottomrule  
    \end{tabular}
\end{table}
%artm_t_oben  [31.015 30.4   28.925 27.865 25.75  24.81  23.56  22.315 20.445 19.555]
%artm_t_unten [31.205 30.875 29.155 28.03  25.985 24.89  23.435 22.26  20.89  19.125]
%sdev_t_oben  [0.355 0.115 0.03  0.67  0.105 0.595 0.235 1.14  0.07  0.495]
%sdev_t_unten [0.015 0.035 0.245 0.425 0.625 0.285 0.07  1.145 0.175 0.675]
In dieser Messreihe wird die Temperatur $T$ schrittweise erhöht um die Temperaturabhängigkeit der Viskosität herauszufinden.
Die Fallhöhe beträgt \qty[]{5}{\cm}.
Zunächst werden nach \eqref{eq:artim_mit} die Mittelwerte für die für die Fallzeiten $\overline{t}$ jeder Temperatur berechnet,
siehe Tabelle \ref{tab:grKu_steigendeTemp}.
Der Mittelwert beider Werten wird je auf die zweite Nachkommastelle gerundet,
da die Messdaten nur zwei Nachkommastellen haben.
Mit der empirischen Gleichung \eqref{eq:empirisch} werden die Viskositäten des Wassers $\eta$ beider Fallrichtungen für diese Messreihe berechnet.
Die hierfür benötigte temperaturabhängige Dichte des Wassers wird aus \cite[][290]{geschke} übernommen. 
Bei den Temperaturen, die nicht in \cite[][290]{geschke} enthalten sind wird auf die nächste angegebene Temperatur gerundet.
% Für die verschiedenen Temperaturen sind dort auch Literaturwerte für die Viskosität $\eta_\text{Geschke}$ gegeben. 
% In Tabelle \ref{tab:viskositaeten_temp} werden diese zum Vergleich aufgelistet.
%
\begin{table}
    \caption{Die Viskosität in Abhängigkeit von der Temperatur}
    \label{tab:viskositaeten_temp}
    \centering
    \sisetup{table-format=1.3}
    \begin{tabular}[]{S[table-format=2.0] %
        S[table-format=1.5] %
        S[table-format=1.3] %
        S[table-format=1.3] %
        S[table-format=1.3]}
        \toprule
        {$T /\unit{\celsius}$} 
        & {$\rho_{\text{Wasser}}/ (\unit{\g \per \cm^3})$ \cite{geschke}} %
        & {$ \eta_\text{oben} / \unit{\milli\Pa\s}$} %
        & {$ \eta_\text{unten} / \unit{\milli\Pa\s}$} \\%
        % & {$ \eta_\text{Geschke} / \unit{\milli\Pa\s}$ \cite{geschke}}\\
        \midrule
            26  &  0.99679   & 1.032   &        1.017   \\ % & 0.874
            27  &  0.99652   & 1.012   &        1.006   \\ % & 0.855
            30  &  0.99565   & 0.963   &        0.951   \\ % & 0.801
            32  &  0.99565   & 0.928   &        0.914   \\ % & 0.801
            35  &  0.9940    & 0.859   &        0.848   \\ % & 0.723
            38  &  0.9922    & 0.828   &        0.813   \\ % & 0.656
            40  &  0.9922    & 0.787   &        0.766   \\ % & 0.656
            43  &  0.9902    & 0.746   &        0.729   \\ % & 0.599
            48  &  0.988     & 0.685   &        0.685   \\ % & 0.549
            52  &  0.988     & 0.655   &        0.627   \\ % & 0.549
        \bottomrule
    \end{tabular}
\end{table}
% Variante der Tabelle mit Standardfehler
%%%\begin{table}
%%%    \centering
%%%    \sisetup{table-format=1.3}
%%%    \begin{tabular}[]{S[table-format=2.0] S[table-format=0.5] S @{${}\pm{}$} S[table-format=1.3] S @{${}\pm{}$}  S[table-format=1.3] S[table-format=1.3]}
%%%        \toprule
%%%        {$T /\unit{\celsius}$} & {$\rho_{\text{Wasser}}$ \cite{geschke}} & \multicolumn{2}{c}{$ \eta_\text{oben} / \unit{\milli\Pa\s}$} & \multicolumn{2}{c}{$ \eta_\text{unten} / \unit{\milli\Pa\s}$} & {$ \eta_\text{Geschke} / \unit{\milli\Pa\s}$\cite{geschke}}\\
%%%        \midrule
%%%            26  &  0.99679   & 1.032  & 0.021 &        1.017 & 0.027 & 0.874 \\
%%%            27  &  0.99652   & 1.012  & 0.018 &        1.006 & 0.026 & 0.855 \\
%%%            30  &  0.99565   & 0.963  & 0.017 &        0.951 & 0.026 & 0.801 \\
%%%            32  &  0.99565   & 0.928  & 0.028 &        0.914 & 0.028 & 0.801 \\
%%%            35  &  0.994     & 0.859  & 0.015 &        0.848 & 0.030 & 0.723 \\
%%%            38  &  0.9922    & 0.828  & 0.025 &        0.813 & 0.023 & 0.656 \\
%%%            40  &  0.9922    & 0.787  & 0.016 &        0.766 & 0.020 & 0.656 \\
%%%            43  &  0.9902    & 0.75   & 0.04  &        0.73  & 0.04  & 0.599 \\
%%%            48  &  0.988     & 0.685  & 0.012 &        0.685 & 0.019 & 0.549 \\
%%%            52  &  0.988     & 0.655  & 0.020 &        0.627 & 0.028 & 0.549 \\
%%%        	%                        \pm                   \pm 
%%%        \bottomrule
%%%    \end{tabular}
%%%    \caption{Die Viskosität in Abhängigkeit von der Temperatur}
%%%    \label{tab:viskositaeten_temp}
%%%\end{table}
%%%


\subsubsection[]{Die Andradesche Gleichung}
Wird die Andradesche Gleichung \eqref{eq:andrade} so umgeformt, dass auf beiden Seiten der Logarithmus gezogen wird, ergibt sich
\begin{align}
    \ln {\left( \eta \right)} = \frac{B}{T} + \ln{\left( A \right)}.
\end{align}
Die logarithmischen Werte der Viskosität sollten also nach der theoretischen Erwartung linear mit dem inversen der Temperatur zusammenhängen.
Von den zuvor berechneten Werten in Tabelle \ref{tab:viskositaeten_temp} wird der natürliche Logarithmus geplottet, vgl. Abbildung \ref{fig:groKu_steigendeTemp_eta_fit}.
Mithilfe der \texttt{numpy} \cite[]{numpy} Funktion \texttt{polyfit} wird eine lineare Regression durchgeführt.
Durch das Prinzip der kleinsten Abstandsquadrate entsteht so eine Funktion der Form
\begin{align*}
    f(x) = B x + C 
\end{align*}
für $x= \frac{1}{T}$, $f(x)= \ln (\eta)$ und $\ln(A) = C$
Für die Parameter ergeben sich die Werte:
\begin{align*}
    B_\text{oben} &= \num{1761.27} & C_\text{oben} &= \num{-5.85}   & A_\text{oben}  &=  \num{2.86e-3} \\
    B_\text{unten} &= \num{1824.47} & C_\text{unten} &= \num{-6.08} & A_\text{unten} &=  \num{2.30e-3}
\end{align*}
%
\begin{figure}[H]
    \centering
    \includegraphics[width=\textwidth]{build/plot.pdf}
    \caption{Die errechneten Viskositäten in einer logarithmischen Darstellung mit einem linearen Fit}
    \label{fig:groKu_steigendeTemp_eta_fit}
\end{figure}
%