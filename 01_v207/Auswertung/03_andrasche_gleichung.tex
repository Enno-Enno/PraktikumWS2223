
\subsection[]{Messreihe 3: Überprüfung der Andradeschen Gleichung}

\begin{figure}
    \centering
    \includegraphics[width=\textwidth]{build/Messreihe3.pdf}
    \caption{Die Fallzeit nach \ref{tab:grKu_steigendeTemp}.}
    \label{fig:groKu_steigendeTemp}
\end{figure}

\begin{table}[]
    \caption{Große Kugel bei variabler Temperatur T; Fallhöhe = 5 cm}
    \label{tab:grKu_steigendeTemp}
    \centering
    \sisetup{table-format=2.2}
    \begin{tabular}{S[table-format=2.0] S[table-format=2.2] S  S @{${}\pm{}$}S[table-format=1.2] S @{${}\pm{}$} S[table-format=1.2]}
        \toprule
        & \multicolumn{2}{c}{Fallzeit $ t / \unit{\s}$} & \multicolumn{4}{c}{ $ \overline{t} / \unit{\s}$} \\
        {T / \unit{\celsius}} & {oben} & {unten}  & \multicolumn{2}{c}{oben} & \multicolumn{2}{c}{unten}\\
        \midrule
            26 & 31.37 &  31.22 &   31.0   & 0.4        &  31.21  & 0.02  \\
               & 30.66 &  31.19 &          &            &         &       \\
            27 & 30.43 &  31.12 &   30.4   & 0.1        &  30.88  & 0.04  \\
               & 30.37 &  30.63 &          &            &         &       \\
            30 & 29.03 &  29.78 &   28.93  & 0.03       &  29.16  & 0.25  \\
               & 28.82 &  28.53 &          &            &         &       \\
            32 & 27.63 &  27.96 &   27.87  & 0.67       &  28.03  & 0.425 \\
               & 28.10 &  28.10 &          &            &         &       \\
            35 & 25.82 &  25.81 &   25.75  & 0.105      &  25.9   & 0.6   \\
               & 25.68 &  26.16 &          &            &         &       \\
            38 & 24.69 &  24.81 &   24.8   & 0.6        &  24.89  & 0.29  \\
               & 24.93 &  24.97 &          &            &         &       \\
            40 & 23.81 &  23.78 &   23.56  & 0.235      &  23.44  & 0.07  \\
               & 23.31 &  23.09 &          &            &         &       \\
            43 & 22.28 &  22.37 &   22.32  & 1.14       &  22.26  & 1.15  \\
               & 22.35 &  22.15 &          &            &         &       \\
            48 & 20.57 &  21.00 &   20.45  & 0.07       &  20.89  & 0.18  \\
               & 20.32 &  20.78 &          &            &         &       \\
            52 & 19.65 &  19.09 &   19.56  & 0.5        &  19.1   & 0.7   \\
               & 19.46 &  19.16 &          &            &         &       \\
        \bottomrule  
    \end{tabular}
\end{table}
%artm_t_oben  [31.015 30.4   28.925 27.865 25.75  24.81  23.56  22.315 20.445 19.555]
%artm_t_unten [31.205 30.875 29.155 28.03  25.985 24.89  23.435 22.26  20.89  19.125]
%sdev_t_oben  [0.355 0.115 0.03  0.67  0.105 0.595 0.235 1.14  0.07  0.495]
%sdev_t_unten [0.015 0.035 0.245 0.425 0.625 0.285 0.07  1.145 0.175 0.675]
In dieser Messreihe wird die Temperatur schrittweise erhöht um die Temperaturabhängigkeit der Viskosität herauszufinden.
Zunächst werden nach \eqref{eq:artim_mit} die Mittelwerte für die Temperaturen berechnet.
Die 
Die empirische Gleichung für die Viskosität \ref{eq:empirisch} enthält die Dichte des Wassers, die auch temperaturabhängig ist.
In \ref{tab:viskositaeten_temp} werden die Dichten des Wassers aus \cite[][290]{geschke} übernommen. 
Bei den Temperaturen, die nicht in \cite[][290]{geschke} enthalten sind wird auf die nächste angegebene Temperatur gerundet.
%Idee für die Diskussion hier villeicht noch Vergelichswerte aus dem Geschke hinzufügen.
\begin{table}
    \centering
    \sisetup{table-format=1.3}
    \begin{tabular}[]{S[table-format=2.0] S[table-format=0.5] S @{${}\pm{}$} S[table-format=1.3] S @{${}\pm{}$}  S[table-format=1.3] S[table-format=1.3]}
        \toprule
        {$T /\unit{\celsius}$} & {$\rho_{\text{Wasser}}$ \cite{geschke}} & \multicolumn{2}{c}{$ \eta_\text{oben} / \unit{\milli\Pa\s}$} & \multicolumn{2}{c}{$ \eta_\text{unten} / \unit{\milli\Pa\s}$} & {$ \eta_\text{Geschke} / \unit{\milli\Pa\s}$\cite{geschke}}\\
        \midrule
            26  &  0.99679   & 1.032  & 0.021 &        1.017 & 0.027 & 0.874 \\
            27  &  0.99652   & 1.012  & 0.018 &        1.006 & 0.026 & 0.855 \\
            30  &  0.99565   & 0.963  & 0.017 &        0.951 & 0.026 & 0.801 \\
            32  &  0.99565   & 0.928  & 0.028 &        0.914 & 0.028 & 0.801 \\
            35  &  0.994     & 0.859  & 0.015 &        0.848 & 0.030 & 0.723 \\
            38  &  0.9922    & 0.828  & 0.025 &        0.813 & 0.023 & 0.656 \\
            40  &  0.9922    & 0.787  & 0.016 &        0.766 & 0.020 & 0.656 \\
            43  &  0.9902    & 0.75   & 0.04  &        0.73  & 0.04  & 0.599 \\
            48  &  0.988     & 0.685  & 0.012 &        0.685 & 0.019 & 0.549 \\
            52  &  0.988     & 0.655  & 0.020 &        0.627 & 0.028 & 0.549 \\
        	%                        \pm                   \pm 
        \bottomrule
    \end{tabular}
    \caption{Die Viskosität in Abhängigkeit von der Temperatur}
    \label{tab:viskositaeten_temp}
\end{table}

\begin{figure}
    \centering
    \includegraphics[width=\textwidth]{build/plot.pdf}
    \caption{Die errechneten Viskositäten in einer logarithmischen Darstellung mit einem linearen Fit}
    \label{fig:groKu_steigendeTemp_eta_fit}
\end{figure}
%
\noindent
Formt man die Andradesche Gleichung \eqref{eq:andrade} so um, dass auf beiden Seiten der Logarithmus gezogen wird, ergibt sich
\begin{align}
    \ln {\left( \eta \right)} = \frac{B}{T} + \ln{\left( A \right)}.
\end{align}
Die Werte für die Viskosität sollten also nach der theoretischen Erwartung linear mit dem inversen der Temperatur zusammenhängen.
Von den zuvor berechneten Werten \ref{tab:viskositaeten_temp} wird der natürliche Logarithmus geplottet \ref{fig:groKu_steigendeTemp_eta_fit}.
Mithilfe der \texttt{numpy} Funktion \texttt{polyfit} wird eine lineare Regression durchgeführt.
Durch das Prinzip der kleinsten Abstandsquadrate entsteht so eine Funktion der Form
\begin{align*}
    f(x) = ax + b
\end{align*}
Für die Parameter ergeben sich die Werte:
\begin{align*}
    a_\text{oben} &= 1761.27 & b_\text{oben} &= -5.85 \\
    A
\end{align*}
Dadurch lassen sich die Konstanten $A$ und $B$ ermitteln.


\subsection[]{Die Reynoldszahl}
Um zu ermitteln, ob es sich um eine laminare Strömung handelt, wird jeweils die dimensionslose Reynoldszahl $Re$ für die kleine und große Kugel bestimmt
(vgl. Abschnitt \ref{sec:vorbereitung}).
Zur Berechnung fehlen noch Mittelwert und Standardabweichung der Fallgeschwindigkeit $\overline{v}$ jeweils für oben und unten
(Dichte von Wasser bei Zimmertemperatur: $\rho = \qty{0.99841}{\g \per \cm^3}$, s.o.).

\begin{table}
    \caption[]{Reynoldszahl $Re$ in Abhängigkeit der Fallgeschwindigkeit $\overline{v}$}
    \label{tab:reynold}
    \centering
    \begin{tabular}[]{S S S[table-format=1.4] @{${}\pm{}$} S[table-format=1.4] S[table-format=4.1] @{${}\pm{}$} S[table-format=3.1]}
        \toprule
        {Kugel} & {Strecke} & \multicolumn{2}{c}{$\overline{v} / \left(\unit{\cm\per\s}\right)$} & \multicolumn{2}{c}{$Re$} \\
        \bottomrule
        {klein} & {oben} & 0.811 & 0.010 & 549 & 14 \\
         & {unten} & 0.816 & 0.012 & 555 & 17 \\
        {groß} & {oben} & 0.1428 & 0.0012 & 96.5 & 1.5 \\
         & {unten} & 0.1404 & 0.0030 & 95.5 & 2.5 \\
    \end{tabular}
\end{table}

% Kugel % Strecke % MW G % S G % MW Rey % S Rey
%Reynold:
%Geschwindigkeit klein oben:  0.811+/-0.010
%Geschwindigkeit klein unten:  0.816+/-0.012
%Geschwindigkeit gross oben:  0.1428+/-0.0012
%Geschwindigkeit gross unten:  0.1404+/-0.0030

%rey klein oben:  549+/-14
%rey klein unten:  555+/-17
%rey gross oben:  96.5+/-1.5
%rey gross unten:  95.5+/-2.5
