\section{Messdaten}
%\subsection{Kenngrößen der Kugeln}

Anfangs wurden die relevanten Kenngrößen der Kugeln bestimmt.
Die Massen sind vorgegeben.
Die Zimmertemperatur betrug stets $T_{Zt}=\qty[]{19}{\degreeCelsius}$.

%%%%%%%%\begin{table}[]
%%%%%%%%    \caption[]{Kenngrößen der kleinen und der großen Kugel. Masse $m$, Durchmesser $d$ und Dichte $\rho$.}
%%%%%%%%    \label{tab:kenngroessen}
%%%%%%%%    \centering
%%%%%%%%    \begin{tabular}[]{S S[table-format=1.4] S[table-format=2.2] @{${}\pm{}$} S[table-format=0.2]}
%%%%%%%%        \toprule
%%%%%%%%        {Kugel} & {$m / \unit{\g}$} & \multicolumn{2}{c}{$d / \unit{\mm}$}   \\
%%%%%%%%        \midrule
%%%%%%%%        {klein} & 4.4531 & 15.61 & 0.01 \\
%%%%%%%%        {groß}  & 2.9528 & 15.78 & 0.01 \\
%%%%%%%%        \bottomrule 
%%%%%%%%
%%%%%%%%    \end{tabular}
%%%%%%%%\end{table}
%ich fand das so als Tabelle etwas schöner hihi :D
%ich weiß nur nicht warum ziwschen Durchmesser und Fehler so eine Lücke ist...
%Lösung: Die Überschrift war zu breit. Mit nur d/mm passt es besser


%\subsection{Messreihe 1: kleine Kugel, konstante Temperatur}

\begin{table}[]
    \caption{Kleine Kugel bei Zimmertemperatur; Fallhöhe = 10 cm}
    \label{tab:klKu_Zitemp}
    \centering
    \sisetup{table-format=2.2}
    \begin{tabular}{S S}
        \toprule
        \multicolumn{2}{c}{Fallzeit $ t / \unit{\s}$}\\
        {oben} & {unten}\\
        \midrule
        12.21 &  12.66 \\
        12.40 &  12.00 \\
        12.25 &  12.32 \\
        12.65 &  12.06 \\
        12.16 &  12.32 \\
        12.13 &  12.09 \\
        12.50 &  12.15 \\
        12.38 &  12.18 \\
        12.34 &  12.38 \\
        12.22 &  12.32 \\
        \bottomrule

    \end{tabular}
\end{table}



\begin{table}[]
    \caption{Große Kugel bei Zimmertemperatur; Fallhöhe = 5 cm}
    \label{tab:grKu_Zitemp}
    \centering
    \sisetup{table-format=2.2}
    \begin{tabular}{S S}
        \toprule
        \multicolumn{2}{c}{Fallzeit $ t / \unit{\s}$}\\
        {oben} & {unten}\\
        \midrule
            34.78 &  34.75 \\
            34.65 &  34.94 \\
            35.00 &  36.90 \\
            35.22 &  35.63 \\
            35.47 &  35.78 \\
        \bottomrule

    \end{tabular}
\end{table}

%\subsection{Messreihe 3: große Kugel, steigende Temperatur}


\begin{table}[]
    \caption{Große Kugel bei variabler Temperatur T; Fallhöhe = 5 cm}
    \label{tab:grKu_steigendeTemp}
    \centering
    \sisetup{table-format=2.2}
    \begin{tabular}{S[table-format=2.0] S S S[table-format=2.0] S S}
        \toprule
        & \multicolumn{2}{c}{Fallzeit $ t / \unit{\s}$} & & \multicolumn{2}{c}{Fallzeit $ t / \unit{\s}$} \\
        {T / \unit{\celsius}} & {oben} & {unten} & {T / \unit{\celsius}} & {oben} & {unten}\\
        \midrule
            26 & 31.37 &  31.22 &  38 & 24.69 &  24.81 \\
               & 30.66 &  31.19 &     & 24.93 &  24.97 \\
            27 & 30.43 &  31.12 &  40 & 23.81 &  23.78 \\
               & 30.37 &  30.63 &     & 23.31 &  23.09 \\
            30 & 29.03 &  29.78 &  43 & 22.28 &  22.37 \\
               & 28.82 &  28.53 &     & 22.35 &  22.15 \\
            32 & 27.63 &  27.96 &  48 & 20.57 &  21.00 \\
               & 28.10 &  28.10 &     & 20.32 &  20.78 \\
            35 & 25.82 &  25.81 &  52 & 19.65 &  19.09 \\
               & 25.68 &  26.16 &     & 19.46 &  19.16 \\   
        \bottomrule
    \end{tabular}
\end{table}
