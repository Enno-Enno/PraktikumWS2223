\section{Messdaten}
%\subsection{Kenngrößen der Kugeln}

Anfangs wurden die relevanten Kenngrößen der Kugeln bestimmt.
Die Massen sind vorgegeben.
Die Zimmertemperatur betrug stets $T_{Zt}=\qty[]{19}{\degreeCelsius}$.

%%%%%%%%\begin{table}[]
%%%%%%%%    \caption[]{Kenngrößen der kleinen und der großen Kugel. Masse $m$, Durchmesser $d$ und Dichte $\rho$.}
%%%%%%%%    \label{tab:kenngroessen}
%%%%%%%%    \centering
%%%%%%%%    \begin{tabular}[]{S S[table-format=1.4] S[table-format=2.2] @{${}\pm{}$} S[table-format=0.2]}
%%%%%%%%        \toprule
%%%%%%%%        {Kugel} & {$m / \unit{\g}$} & \multicolumn{2}{c}{$d / \unit{\mm}$}   \\
%%%%%%%%        \midrule
%%%%%%%%        {klein} & 4.4531 & 15.61 & 0.01 \\
%%%%%%%%        {groß}  & 2.9528 & 15.78 & 0.01 \\
%%%%%%%%        \bottomrule 
%%%%%%%%
%%%%%%%%    \end{tabular}
%%%%%%%%\end{table}
%ich fand das so als Tabelle etwas schöner hihi :D
%ich weiß nur nicht warum ziwschen Durchmesser und Fehler so eine Lücke ist...
%Lösung: Die Überschrift war zu breit. Mit nur d/mm passt es besser


%\subsection{Messreihe 1: kleine Kugel, konstante Temperatur}





%\subsection{Messreihe 3: große Kugel, steigende Temperatur}


