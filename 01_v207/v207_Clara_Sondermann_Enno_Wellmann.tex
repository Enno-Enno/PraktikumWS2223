\input{../header.tex}

\title{Versuch 207: Kugelfall-Viskosimeter nach Höppler}
\date{Durchführung: 22.11.2022, Abgabe: 29.11.22}


\begin{document}
\maketitle

\tableofcontents
\newpage



\section{Ziel}
In diesem Versuch wird die dynamische Viskosität\footnote{ Nicht zu verwechseln mit der \textit{kinematischen} Viskosität.
Im folgenden wird die \textit{dynamische} Viskosität mit Viskosität bezeichnet.} $\eta$ 
von destilliertem Wasser in Abhängigkeit der Temperatur mit Hilfe des Kugelfall-Viskosimeters nach Höppler ermittelt.
Des Weiteren wird die Apperatekonstante $K$ einer Kugel bestimmt, die eine Proportionalitätskonstante zwischen Viskosität $\eta$ und Fallzeit $t$ liefert.
Außerdem wird anhand der Reynoldszahl $Re$ ermittelt, ob das Fluid im Viskosimeters laminar oder turbulent strömt. 



\section{Grundlagen}
\label{sec:grundlagen}
Reale Stoffe (d.h. Flüssigkeiten oder Gase) sind \textit{strömend}, wenn sie einen makroskopischen Impuls haben.
Die thermische Bewegung der einzelnen Moleküle kann dabei vernachlässigt werden.
Die Strömung kann mit Hilfe der Strömungsgeschwindigkeit $\symbf{u}\left(\symbf{r},t\right)$ beschrieben werden.
Als \textit{Stromlinie} oder \textit{Stromfaden} wird die Ortskurve bezeichnet, die ein Flüssigkeitselement, etwa ein Korkstück, durchläuft.
Eine Stromlinie wird durch Reibungskräfte im Medium beeinflusst \cite[]{demtroeder}.
Ein Maß für die Stärke der inneren Reibung bzw. die Zähigkeit des Stoffes ist die \textit{dynamische Viskosität} $\eta$ mit der SI-Einheit $\unit{\Pa \s}$.
Diese ist eine temperaturabhängige Materialkonstante \cite[]{geschke}.
Hierbei wird zwischen idealen bzw. nicht-viskosen Stoffen unterschieden, für die $\eta \simeq 0$ gilt, und viskosen Stoffen,
deren Strömungsverhalten in laminar und turbulent eingeteilt wird (vgl. Abschnitt \ref{sec:vorbereitung}).
\noindent
Die Reibung, die eine Kugel mit Radius $r$ und Geschwindigkeit $v$ in einem Fluid erfährt wird mit Hilfe der Stokesschen Reibung 
\noindent
\begin{align}
    F_R = 6 \pi \eta v r    
\end{align}
\noindent
beschrieben.
Sie wirkt mit der Auftriebskraft $\symbf{F}_A$ der Schwerkraft $\symbf{F}_g$ entgegen. 
Mit steigender Geschwindigkeit $v$ wächst die Reibungskraft $\symbf{F}_R$, bis sich ein Kräftegleichgewicht einstellt
und $v$ einen konstanten Wert annimmt.
Dabei wird die Viskosität des Fluids mit der empirischen Formel 
\begin{align}
    \label{eq:empirisch}
    \eta = K \left(\rho_K -\rho_{Fl}\right) \cdot t
\end{align}
\noindent
ermittelt, wobei $K$ eine Apparatekonstante ist, die Fallhöhe und Kugelgeometrie berücksichtigt,
$\rho_K$ bzw. $\rho_{Fl}$ die Dichte der Kugel bzw. des Fluids sind und $t$ die Fallzeit darstellt.
\noindent
%
Durch die \textit{Andradesche Gleichung} wird die Temperaturabhängigkeit der Viskosität von Flüssigkeiten dargestellt.
\begin{equation}
\label{eq:andrade}
    \eta \left(T\right) = A \exp{\left(\frac{B}{T}\right)}
\end{equation}
$A$ und $B$ sind dabei Konstanten \cite*[]{va207}.





\section{Vorbereitungsaufgaben}
\subsection{Laminare Strömung und Reynoldszahl}
\label{sec:vorbereitung}
Von einer laminaren Strömung wird gesprochen, wenn in einem realen Stoff
die einzelnen Stromfäden ohne sich zu beeinflussen nebeneinander liegen. Das bedeutet, dass es keine Wirbel oder
Turbulenzen innerhalb des Stoffes gibt und sich die einzelnen Schichten mit Strömungsgeschwindigkeiten nicht vermischen.
Die \textit{Reynoldszahl} $Re$ ist ein Maß dafür, ob es sich um eine laminare oder turbulente Strömung handelt.
\noindent
\begin{align}
    Re = \frac{\rho \overline{v} R}{\eta}
\end{align}
Dabei ist $\rho$ die Dichte des Fluids, $\overline{v}$ die relative mittlere Geschwindigkeit zwischen Fluid und Körper und
$R$ die charakteristische Länge.
Wenn die Reynoldszahl unterhalb einer kritischen Grenze $Re_c$ liegt, kann von einer laminaren Strömung gesprochen werden.
Die charakteristische Länge $R$ ist hier der Durchmesser der Kugel und die kritische Grenze der Reynoldszahl $Re_c$ beträgt ca. 2300 \cite*[]{geschke}.


\subsection{Temperaturabhängige Viskosität und Dichte von destilliertem Wasser}
Die Dichte $\rho$ von Wasser hat ihren Maximalwert bei $T_{max} = \qty[]{4}{\degreeCelsius}$ \cite*[]{geschke}.
Außerdem ist $\rho$ im festen Aggregatszustand geringer als im flüssigen Zustand. 
Diese beiden Tatsachen werden auch als \enquote{Dichteanomalie des Wassers} bezeichnet\cite*[]{demtroeder}.
Die Viskosität $\rho$ fällt exponentiell mit dem Kehrwert der Temperatur $\frac{1}{T}$ (vgl. \eqref{eq:andrade}).



\section[]{Versuchsaufbau}
Das Kugelfall-Viskosimeter nach Höppler besteht aus einem zylinderförmigen Glasrohr,
das mit einer Flüssigkeit gefüllt wird (hier destilliertes Wasser), deren Viskosität bestimmt werden soll. 
In dem befüllten Rohr lässt man eine Kugel fallen, deren Durchmesser nahe am Rohrdurchmeser liegt.
Es wurden hier zwei Kugeln mit unterschiedlichen Durchmessern verwendet. 
Um unkontrollierte Stöße mit der Rohrwand und Wirbelbildungen zu vermeiden, wird das Rohr etwas geneigt.
Am Rohr selbst befinden sich insgesamt drei Markierungen, die je einen Abstand von 5 Zentimetern zu einander haben.

Das Fallrohr befindet sich in einem Wasserbad, durch das mit Hilfe eines angeschlossenen Thermostates die Temperatur im Rohr geregelt werden kann.
Am Thermostat befindet sich ein Thermometer, sodass die Temperatur im Viskosimeter abgelesen werden kann.

\section{Durchführung}

Zu Beginn des Versuchs wurden die Durchmesser der Kugeln mit Hilfe einer Schieblehre bestimmt, damit man mit den vorgegebenen Massen die Dichten ermitteln kann.
Außerdem wurde die Zimmertemperatur erfasst.
Anschließend wurde der obere Verschluss geöffnet und das Fallrohr mit destilliertem Wasser befüllt. 
Danach wurde zunächst die kleinere Kugel in das Rohr gelassen.
Hierbei war darauf zu achten, dass sich keine Luftbläschen bildeten.
Die Bläschen, die nach dem Befüllen und Einlassen der Kugel vorhanden waren, wurden mit einer Bürste bzw. einem Glaskolben entfernt.
Anschließend wurde der Deckel verschlossen und darauf geachtet, dass sich keine weitere Luft im Fallrohr befand.

Bereits am Anfang des Versuchs war Wasser im Wasserbad vorhanden und die Temperatur wurde auf Zimmertemperatur reguliert,
damit für die ersten beiden Versuchsreihen die Temperatur im Rohr konstant gehalten werden konnte.

Die erste Versuchsreihe bestand darin, dass bei Zimmertemperatur die kleine Kugel fallen gelassen wurde.
Nachdem sich eine konstante Geschwindigkeit durch das in Abschnitt \ref{sec:grundlagen} genannte Kräftegleichgewicht eingestellt hatte,
wurde die Fallzeit bei einer Strecke von 10 Zentimetern (oberste bis unterste Markierung)
mit Hilfe einer Stoppuhr gemessen.
Nach jedem Fall wurde das Rohr um 180° und erneut gemessen.
Je nach Orientierung werden diese Messwerte im Folgenden mit \enquote{oben} bzw. \enquote{unten} betitelt.
% Anführungszeichen funktionieren nur mit \enquote{}
Dies wurde 10 Mal wiederholt, sodass sich insgesamt 20 Werte (je 10 oben und unten) ergaben.

Die zweite Messreihe verlief analog zur ersten mitsamt der Vorbereitungen, nur dass nun die größere Kugel auf einer Strecke von 5 Zentimetern betrachtet wurde.
Um die Kugeln auszutauschen, musste der Verschluss geöffnet werden, wodurch das destillierte Wasser aus dem Rohr entweichen konnte.
Auch hier mussten also beim erneuten befüllen etwaige Luftbläschen von Rohr und Kugel werden.
Außerdem wurden die Messungen hier nur 5 Mal wiederholt.

Die dritte Messreihe bestand darin, dass mit dem Thermostat die Temperatur langsam erhöht wurde
und dabei die Fallzeit der größeren Kugel auf 5 Zentimetern gemessen wurde.
Da die gleiche Kugel wie zuvor verwendet wurde, musste das Rohr nicht aufgeschraubt oder aufgefüllt werden.
Für jede Temperatur wurden je zwei Werte für oben und unten aufgenommen und es wurde für insgesamt 10 unterschiedliche Temperaturen gemessen.
Hierbei war zu beachten, dass sich erst die gewünschte Temperatur im Fallrohr einstellte, bevor man für die höhere Temperatur gemessen hat, 
um Messunsicherheiten zu vermeiden. Die Wartezeit betrug häufig ca. 3-5 Minuten. Maximal sollte eine Temperatur von 50°C erreicht werden.


%%%%%%%%%%%%%%%%%%%%%%%%%%%%% Section Messdaten entfernen und in der Auswertung einfügen.

\section{Messdaten}
%\subsection{Kenngrößen der Kugeln}

Anfangs wurden die relevanten Kenngrößen der Kugeln bestimmt.
Die Massen sind vorgegeben.
Die Zimmertemperatur betrug stets $T_{Zt}=\qty[]{19}{\degreeCelsius}$.

%%%%%%%%\begin{table}[]
%%%%%%%%    \caption[]{Kenngrößen der kleinen und der großen Kugel. Masse $m$, Durchmesser $d$ und Dichte $\rho$.}
%%%%%%%%    \label{tab:kenngroessen}
%%%%%%%%    \centering
%%%%%%%%    \begin{tabular}[]{S S[table-format=1.4] S[table-format=2.2] @{${}\pm{}$} S[table-format=0.2]}
%%%%%%%%        \toprule
%%%%%%%%        {Kugel} & {$m / \unit{\g}$} & \multicolumn{2}{c}{$d / \unit{\mm}$}   \\
%%%%%%%%        \midrule
%%%%%%%%        {klein} & 4.4531 & 15.61 & 0.01 \\
%%%%%%%%        {groß}  & 2.9528 & 15.78 & 0.01 \\
%%%%%%%%        \bottomrule 
%%%%%%%%
%%%%%%%%    \end{tabular}
%%%%%%%%\end{table}
%ich fand das so als Tabelle etwas schöner hihi :D
%ich weiß nur nicht warum ziwschen Durchmesser und Fehler so eine Lücke ist...
%Lösung: Die Überschrift war zu breit. Mit nur d/mm passt es besser


%\subsection{Messreihe 1: kleine Kugel, konstante Temperatur}

\begin{table}[]
    \caption{Kleine Kugel bei Zimmertemperatur; Fallhöhe = 10 cm}
    \label{tab:klKu_Zitemp}
    \centering
    \sisetup{table-format=2.2}
    \begin{tabular}{S S}
        \toprule
        \multicolumn{2}{c}{Fallzeit $ t / \unit{\s}$}\\
        {oben} & {unten}\\
        \midrule
        12.21 &  12.66 \\
        12.40 &  12.00 \\
        12.25 &  12.32 \\
        12.65 &  12.06 \\
        12.16 &  12.32 \\
        12.13 &  12.09 \\
        12.50 &  12.15 \\
        12.38 &  12.18 \\
        12.34 &  12.38 \\
        12.22 &  12.32 \\
        \bottomrule

    \end{tabular}
\end{table}



\begin{table}[]
    \caption{Große Kugel bei Zimmertemperatur; Fallhöhe = 5 cm}
    \label{tab:grKu_Zitemp}
    \centering
    \sisetup{table-format=2.2}
    \begin{tabular}{S S}
        \toprule
        \multicolumn{2}{c}{Fallzeit $ t / \unit{\s}$}\\
        {oben} & {unten}\\
        \midrule
            34.78 &  34.75 \\
            34.65 &  34.94 \\
            35.00 &  36.90 \\
            35.22 &  35.63 \\
            35.47 &  35.78 \\
        \bottomrule

    \end{tabular}
\end{table}

%\subsection{Messreihe 3: große Kugel, steigende Temperatur}


\begin{table}[]
    \caption{Große Kugel bei variabler Temperatur T; Fallhöhe = 5 cm}
    \label{tab:grKu_steigendeTemp}
    \centering
    \sisetup{table-format=2.2}
    \begin{tabular}{S[table-format=2.0] S S S[table-format=2.0] S S}
        \toprule
        & \multicolumn{2}{c}{Fallzeit $ t / \unit{\s}$} & & \multicolumn{2}{c}{Fallzeit $ t / \unit{\s}$} \\
        {T / \unit{\celsius}} & {oben} & {unten} & {T / \unit{\celsius}} & {oben} & {unten}\\
        \midrule
            26 & 31.37 &  31.22 &  38 & 24.69 &  24.81 \\
               & 30.66 &  31.19 &     & 24.93 &  24.97 \\
            27 & 30.43 &  31.12 &  40 & 23.81 &  23.78 \\
               & 30.37 &  30.63 &     & 23.31 &  23.09 \\
            30 & 29.03 &  29.78 &  43 & 22.28 &  22.37 \\
               & 28.82 &  28.53 &     & 22.35 &  22.15 \\
            32 & 27.63 &  27.96 &  48 & 20.57 &  21.00 \\
               & 28.10 &  28.10 &     & 20.32 &  20.78 \\
            35 & 25.82 &  25.81 &  52 & 19.65 &  19.09 \\
               & 25.68 &  26.16 &     & 19.46 &  19.16 \\   
        \bottomrule
    \end{tabular}
\end{table}


\section{Auswertung}
\subsection{Fehlerrechnung}
\subsubsection{Gaußsche Fehlerforpflanzung}
Wenn zu Messdaten die Standardabweichung bekannt ist, und mit diesen Messdaten weiter gerechnet werden soll,
wird die Gaußsche Fehlerfortpflanzung verwendet. 
Angenommen wir haben $k$ Messwerte $x_i [i \in \mathbb{N}, i \leq k]$ mit den Standardabweichungen $Delta x_i$
und eine abgeleitete Größe $f(x_i)$ dann ist der Fehler von $f$
\begin{align}
    \Delta f(x_i) = \sqrt{}
\end{align} 

\subsection{Die Güteziffer}
Um die Güteziffer $\nu$ zu bestimmen, muss gemäß \eqref{eq:guetePraxis} zunächst
die zeitliche Änderung des Temperaturverlaufs ermittelt werden.
Dies geschieht, indem die die Messdaten der Temperaturen beider Reservoire aus Tabelle \ref{tab:messdaten} geplottet werden.
Anschließend wird mit Hilfe einer nicht-linearen Ausgleichsrechnung eine Funktion $T \left(t\right)$ ermittelt, die den Verlauf 
für das jeweilige Reservoir beschreiben soll.
Diese Funktion wird nach der Zeit differenziert, sodass man $\nu$ für je vier unterschiedliche Temperaturen bestimmen kann.
Für die vier Berechnungen der Güteziffer werden Tabelle \ref{tab:messdaten} jeweils die Temperaturen $T_{\text{k}}$ und $T_{\text{w}}$ nach 
$t_1 = \qty[]{3}{\s}$, $t_2 = \qty[]{9}{\s}$, $t_3 = \qty[]{15}{\s}$ und $t_4 = \qty[]{21}{\s}$ entnommen.


\subsubsection{Plot der Temperaturverläufe}
Plottet man die Temperaturverläufe der beiden Reservoire in Abhängigkeit von der Zeit ergibt sich Abbildung \ref{fig:temperaturverlauf}.

\begin{figure}
    \includegraphics[]{build/plot_temp_verlauf.pdf}
    \caption[]{Zeitabhängige Temperaturverläufe beider Reservoire}
    \label{fig:temperaturverlauf}
\end{figure}


\subsubsection[]{Ermittlung der Ausgleichsfunktionen}
Um die Ausgleichskurve zu approximieren, sind folgende Funktionen möglich:
\begin{align}
    T \left(t\right) &= A t^2 + B t + C \label{eq:ausgleichsfunktion_1} \\
    T \left(t\right) &= \frac{A}{1 + B t^{\alpha}} \label{eq:ausgleichsfunktion_2} \\
    T \left(t\right) &= \frac{A t^{\alpha}}{1 + B t^{\alpha}} + C \label{eq:ausgleichsfunktion_3}
\end{align}
Hierbei sind $A$, $B$, $C$ und $\alpha$ die zu bestimmenden Parameter, wobei $1 \leq \alpha \leq 2$ gelten soll.
Mit Hilfe der Python \cite[]{python} Funktion \texttt{curve\_fit} aus dem Paket \texttt{scipy.optimize} 
\cite[]{scipy} lassen sich die nicht-linearen Ausgleichsrechnungen durchführen.
\begin{figure}
    \includegraphics[]{build/plot_ausgleich_1.pdf}
    \caption[]{Ausgleichskurve zu Funktion \eqref{eq:ausgleichsfunktion_1}}
    \label{fig:ausgleichsplot_1}
\end{figure}
\begin{figure}
    \includegraphics[]{build/plot_ausgleich_2.pdf}
    \caption[]{Ausgleichskurve zu Funktion \eqref{eq:ausgleichsfunktion_2}}
    \label{fig:ausgleichsplot_2}
\end{figure}
\begin{figure}
    \includegraphics[]{build/plot_ausgleich_3.pdf}
    \caption[]{Ausgleichskurve zu Funktion \eqref{eq:ausgleichsfunktion_3}}
    \label{fig:ausgleichsplot_3}
\end{figure}
Vergleicht man die Plots \ref{fig:ausgleichsplot_1}, \ref{fig:ausgleichsplot_2} und \ref{fig:ausgleichsplot_3} mit einander,
erkennt man unmittelbar, dass die Approximation \eqref{eq:ausgleichsfunktion_2} auch nach einstellen der Startparameter am ungenausten ist.
\eqref{eq:ausgleichsfunktion_1} und \eqref{eq:ausgleichsfunktion_3} erscheinen zunächst in etwa den gleichen Exaktheitsgrad zu haben.
Im Folgenden wird mit \eqref{eq:ausgleichsfunktion_3} weiter gerechnet, da diese besonders für kleine Zeiten näher an den eigentlichen Messdaten liegt
als \eqref{eq:ausgleichsfunktion_1}.
Die entsprechenden Parameter betragen dann für das kalte Reservoir
\begin{align*}
    A_{\text{k}} &= -0.9113 \pm 0.0742 & B_{\text{k}} &= 0.0204 \pm 0.0008 \\
    C_{\text{k}} &= 294.6660 \pm 0.1221 & \alpha_{\text{k}} &= 1.1946 \pm 0.0378 \\
    %error kalt:  [0.07422951 0.00082891 0.12214775 0.03782461]
    \intertext{und für das warme Reservoir}
    A_{\text{w}} &= 1.1391 \pm 0.0429 & B_{\text{w}} &= 0.0229  \pm 0.0005 \\
    C_{\text{w}} &= 294.0467 \pm 0.0704 & \alpha_{\text{w}} &= 1.2211 \pm 0.0175,
\end{align*}
%error warm:  [0.04285291 0.0004705  0.07042233 0.01747473]
wobei die Fehler der Kovarianzmatrix entnommen wurden.


\subsubsection[]{Bestimmung der Differentialquotienten}
\label{sec: dif_quot}
Differenziert man \eqref{eq:ausgleichsfunktion_3} nach der Zeit, erhält man mittels Quotientenregel
\begin{align*}
    \frac{\symup{d}T}{\symup{d}t} = \frac{\alpha A t^{\alpha -1}}{\left(1 + B t^{\alpha}\right)^2}.
\end{align*}
%
Setzt man für beide Reservoire jeweils die obigen Parameter und entsprechenden Zeiten ein, erhält man 
\begin{align*}
    \frac{\symup{d}}{\symup{d}t} T_\text{k} \left(t_1\right) &= \left( -1.16 \pm 0.12 \right)\, \unit[]{\kelvin\per\min}  &
    \frac{\symup{d}}{\symup{d}t} T_\text{w} \left(t_1\right) &= \left(  1.50 \pm 0.07 \right)\, \unit[]{\kelvin\per\min}   \\
    \frac{\symup{d}}{\symup{d}t} T_\text{k} \left(t_2\right) &= \left( -1.02 \pm 0.12 \right)\, \unit[]{\kelvin\per\min}  &
    \frac{\symup{d}}{\symup{d}t} T_\text{w} \left(t_2\right) &= \left(  1.27 \pm 0.07 \right)\, \unit[]{\kelvin\per\min} \\
    \frac{\symup{d}}{\symup{d}t} T_\text{k} \left(t_3\right) &= \left( -0.80 \pm 0.09 \right)\, \unit[]{\kelvin\per\min} &
    \frac{\symup{d}}{\symup{d}t} T_\text{w} \left(t_3\right) &= \left(  0.96 \pm 0.05 \right)\, \unit[]{\kelvin\per\min} \\
    \frac{\symup{d}}{\symup{d}t} T_\text{k} \left(t_4\right) &= \left( -0.63 \pm 0.06 \right)\, \unit[]{\kelvin\per\min}  &
    \frac{\symup{d}}{\symup{d}t} T_\text{w} \left(t_4\right) &= \left(  0.72 \pm 0.03 \right)\, \unit[]{\kelvin\per\min}.
\end{align*}
%

\subsubsection[]{Berechnung der Güteziffer}
Gemäß Formel \eqref{eq:gueteTheorie} ergeben sich folgende ideale Güteziffern $\nu_\text{ideal}$ in Tabelle \ref{tab:gueteziffern}.
%
Mit Hilfe der spezifischen Wärmekapazität von Wasser $c_{\text{w}} = \qty{4190}{\joule\per\kg\per\kelvin}$ \cite[]{leifi}, der gegebenen Wärmekapazität
der Kupferspirale $m_{\text{K}} c_{\text{K}} = \qty{750}{\joule\per\kelvin}$ und der zuvor bestimmten Differentialquotienten
kann bei bei einer Wassermenge von $\qty[]{3}{\liter}$ pro Behälter die reale Güteziffer $\nu_{\text{real}}$ gemäß \eqref{eq:guetePraxis} bestimmt werden.
Hierfür wird die gemittelte Leistungsaufnahme des Kompressors
\begin{align*}
    N = \frac{1}{n} \sum_{m=1}^{n} P_m.
\end{align*}
benötigt.
Man erhält
\begin{align*}
    N_1  &=  \qty[]{7200}{\J\per\min} &
    N_2  &=  \qty[]{7360}{\J\per\min} &
    N_3  &=  \qty[]{7376}{\J\per\min} &
    N_4  &=  \qty[]{7240}{\J\per\min}
\end{align*}
Sodass sich schließlich  $\nu_\text{real}$ in Tabelle \ref{tab:gueteziffern} ergibt.
Man erkennt große Diskrepanzen zwischen idealen und realen Werten,
die rechnerisch als prozentuale Abweichungen der Mittelwerte $\Delta \nu$ in der selben Tabelle angegeben sind.

\begin{table}
    \caption[]{Ideale und reale Güteziffern mit ihrer Abweichung}
    \label{tab:gueteziffern}
    \sisetup{table-format = 2.2}
    \begin{tabular}{S S S @{${}\pm{}$} S S}
        \toprule
        {$t /\unit[]{\min}$} & {$\nu_\text{ideal}$} & \multicolumn{2}{c}{$\nu_\text{real}$} & {$\Delta \nu$} \\
        \midrule
        3  & 45.9 & 2.77 & 0.13 & 93.90 \\
        9  & 14.2 & 2.30 & 0.12 & 83.80 \\
        15 & 9.3  & 1.73 & 0.08 & 81.72 \\
        21 & 7.4  & 1.33 & 0.06 & 82.43 \\ 
        \bottomrule 
    \end{tabular}
    \centering
\end{table}
%Abweichung nu:  [0.93899782 0.83802817 0.8172043  0.82432432]

%nu real 0:  2.77+/-0.13
%nu real 1:  2.30+/-0.12
%nu real 2:  1.73+/-0.08
%nu real 3:  1.33+/-0.06




\subsection[]{Der Massendurchsatz}
Um den Massendurchsatz zu bestimmen, werden die zuvor in Abschnitt \ref{sec: dif_quot} Differentialquotienten des kalten Reservoirs benötigt.
Des Weiteren muss die Verdampfungswärme $L$ des Transportmediums, hier: $\ce{Cl2F2C}$, für Formel \textbf{REFERENZ!!!} ermittelt werden.


\subsubsection[]{Die Verdampfungswärme}
Zur Bestimmung der Verdampfungswärme $L$ werden die Wertepaare $\left(p_{\text{k}}, \frac{1}{T_{\text{k}}}\right)$ logarithmisch dargestellt
und anschließend eine lineare Ausgleichsrechnung zur Bestimmung von $L$ durchgeführt.
Dies baut auf der Formel 
\begin{align*}
    p = p_0 \exp{\left(-\frac{L}{RT}\right)}
\end{align*}
aus Versuch \cite[]{man:v203} auf.
$p_0$ ist dabei der Umgebungsdruck bei Zimmertemperatur und $R$ die Gaskonstante.
Sobald die Steigung $m$ der Ausgleichsgeraden ermittelt wurde, kann somit $L = -R \cdot m$ bestimmt werden.
Laut der Wetterstation \cite*[][]{wetterstation} betrug der Luftdruck am Tag des Experiments
$p_0 = \qty[]{1018.1}{\hecto\pascal} \simeq \qty[]{1.0}{\bar}$.
Mittels der Python \cite[]{python} Funktion \texttt{polyfit} aus dem Paket \texttt{numpy} \cite[]{numpy} wird die lineare Regression durchgeführt und
der Plot \ref{fig:ausgleichsgerade} erstellt.
Es ergibt sich die Steigung $m = \left(-2001.8141\pm 111.7750 \right)$ und ein Achsenabschnitt von $n = \left( 8.3639 \pm 0.3961 \right)$, 
wobei letzterer für die Rechnungen nicht relevant ist. 
Folglich gilt $L = $ %%%%%%%%%%%%%%%%%%%%%%%%%%%%%%%%%%%%%%%%%%%%%%%%%%%%%%%%%%%%%%%%%%%%%%%%%%%%%%%%%%
\begin{figure}
    \includegraphics[]{build/plot_verdampfungswaerme.pdf}
    \caption[]{lineare Ausgleichsgerade zur Bestimmung von $L$}
    \label{fig:ausgleichsgerade}
\end{figure}

%Steigung m= -2001.814130953005
%achsenabschn b=  8.363939543080686
%#Fehler:  [111.77504319   0.39608049]





%fragen: p0, standardabweichung?

\section{Diskussion}

Die realen Güteziffern sind in etwa um den Faktor 10 kleiner als die idealen (siehe \ref{tab:gueteziffern}).
Das hängt mit verschiedenen Schritten zusammen, an denen Energie verloren geht.
Zunächst wird die elektrische Energie, die von dem Kompressor verwendet wird
nicht verlustfrei in mechanische Energie umgewandelt.
Die Rohre und Behälter der Wärmepumpe sind zwar isoliert, 
Wärmeenergie geht trotzdem nach außen verloren.
In \ref{sec:mech_Leistung} wurde die mechanische Leistung anhand der gemessenen Wirkung im Reservoir k   ermittelt.
Die dabei ermittelte Mechanische Leistung war um einen Faktor 100 kleiner als die elektrische Leistung.
Auch hier gibt es Wirkungsverluste in allen Zwischenschritten, die eine Erklärung sein können.

Außerdem können die statistischen Messfehler noch größer sein als bisher ermittelt.
An den Geräten gab es oft keine Angaben zum typischen Messfehler,
weshalb diese nicht in die Abschätzung des Messfehlers eingehen konnten.
Eines der Manometer hatte eine um $\qty{1}{\bar}$ verschobene Messanzeige.
Diese Verschiebung wurde in den Rechnungen berücksichtigt.
Ob es weitere systematische Messfehler gibt kann ohne weiteres nicht gesagt werden.

Bei der Mechanischen Leistung in \ref{sec:mech_Leistung} ist ein negativer Wert herausgekommen. 
Das hängt mit der Perspektive zusammen aus dem man den Massendurchsatz bzw. die Leistung betrachtet.
Wenn man sie von dem k-Teil der Wärmepumpe betrachtet dann wird diesem Gas, Druck und damit Energie entzogen,
eine negative Leistung ist sinnvoll.
Für den Vergleich mit der elektrischen Leistung der Pumpe ist allerdings der Betrag der Leistung sinnvoller.


\printbibliography

\section*{Anhang}

\begin{minipage}[t]{0.4\textwidth}
    \includegraphics[height=5cm, page=1]{scans_messdaten/v308_Hysteresekurve.pdf}
\end{minipage}
\begin{minipage}[t]{0.4\textwidth}
    \includegraphics[height=5cm, keepaspectratio, page=2]{scans_messdaten/v308_Hysteresekurve.pdf}
\end{minipage}

\begin{minipage}[t]{0.4\textwidth}
    \includegraphics[height=5cm, page=3]{scans_messdaten/v308_Hysteresekurve.pdf}
\end{minipage}
\begin{minipage}[t]{0.4\textwidth}
    \includegraphics[height=5cm, page=4]{scans_messdaten/v308_Hysteresekurve.pdf}
\end{minipage}

\begin{minipage}[t]{0.4\textwidth}
    \includegraphics[height=5cm, page=5]{scans_messdaten/v308_Hysteresekurve.pdf}
\end{minipage}

\begin{minipage}[t]{0.4\textwidth}
    \includegraphics[height=5cm, page=6]{scans_messdaten/v308_Hysteresekurve.pdf}
\end{minipage}
\begin{minipage}[t]{0.4\textwidth}
    \includegraphics[height=5cm, page=7]{scans_messdaten/v308_Hysteresekurve.pdf}
\end{minipage}

\begin{minipage}[t]{0.4\textwidth}
    \includegraphics[height=5cm, page=8]{scans_messdaten/v308_Hysteresekurve.pdf}
\end{minipage}
\begin{minipage}[t]{0.4\textwidth}
    \includegraphics[height=5cm, page=9]{scans_messdaten/v308_Hysteresekurve.pdf}
\end{minipage}


\begin{minipage}[t]{0.4\textwidth}
    \includegraphics[height=5cm, page=1]{scans_messdaten/v308_lange_kurze_Spule.pdf}
\end{minipage}
\begin{minipage}[t]{0.4\textwidth}
    \includegraphics[height=5cm, keepaspectratio, page=2]{scans_messdaten/v308_lange_kurze_Spule.pdf}
\end{minipage}

\begin{minipage}[t]{0.4\textwidth}
    \includegraphics[height=5cm, page=3]{scans_messdaten/v308_lange_kurze_Spule.pdf}
\end{minipage}
\begin{minipage}[t]{0.4\textwidth}
    \includegraphics[height=5cm, page=4]{scans_messdaten/v308_lange_kurze_Spule.pdf}
\end{minipage}

\begin{minipage}[t]{0.4\textwidth}
    \includegraphics[height=5cm, page=5]{scans_messdaten/v308_lange_kurze_Spule.pdf}
\end{minipage}

\begin{minipage}[t]{0.4\textwidth}
    \includegraphics[height=5cm, page=6]{scans_messdaten/v308_lange_kurze_Spule.pdf}
\end{minipage}



\begin{minipage}[t]{0.4\textwidth}
    \includegraphics[height=5cm, page=1]{scans_messdaten/v308_spulenpaar.pdf}
\end{minipage}
\begin{minipage}[t]{0.4\textwidth}
    \includegraphics[height=5cm, keepaspectratio, page=2]{scans_messdaten/v308_spulenpaar.pdf}
\end{minipage}

\begin{minipage}[t]{0.4\textwidth}
    \includegraphics[height=5cm, page=3]{scans_messdaten/v308_spulenpaar.pdf}
\end{minipage}
\begin{minipage}[t]{0.4\textwidth}
    \includegraphics[height=5cm, page=4]{scans_messdaten/v308_spulenpaar.pdf}
\end{minipage}

\begin{minipage}[t]{0.4\textwidth}
    \includegraphics[height=5cm, page=5]{scans_messdaten/v308_spulenpaar.pdf}
\end{minipage}
%
%\includegraphics[height=5cm]{scans_messdaten/v308_Hysteresekurve.pdf}
%\includegraphics[height=5cm]{scans_messdaten/v308_lange_kurze_Spule.pdf}
%\includegraphics[height=5cm]{scans_messdaten/v308_spulenpaar.pdf}
%

\end{document}