\section{Ziel}
In diesem Versuch wird die dynamische Viskosität\footnote{ Nicht zu verwechseln mit der \textit{kinematischen} Viskosität.
Im folgenden wird die \textit{dynamische} Viskosität mit Viskosität bezeichnet.} $\eta$ 
von destilliertem Wasser in Abhängigkeit der Temperatur mit Hilfe des Kugelfall-Viskosimeters nach Höppler ermittelt.
Des Weiteren wird die Apperatekonstante $K$ einer Kugel bestimmt, die eine Proportionalitätskonstante zwischen Viskosität $\eta$ und Fallzeit $t$ liefert.
Außerdem wird anhand der Reynoldszahl $Re$ ermittelt, ob das Fluid im Viskosimeters laminar oder turbulent strömt. 



\section{Grundlagen}
\label{sec:grundlagen}
Reale Stoffe (d.h. Flüssigkeiten oder Gase) sind \textit{strömend}, wenn sie einen makroskopischen Impuls haben.
Die thermische Bewegung der einzelnen Moleküle kann dabei vernachlässigt werden.
Die Strömung kann mit Hilfe der Strömungsgeschwindigkeit $\symbf{u}\left(\symbf{r},t\right)$ beschrieben werden.
Als \textit{Stromlinie} oder \textit{Stromfaden} wird die Ortskurve bezeichnet, die ein Flüssigkeitselement, etwa ein Korkstück, durchläuft.
Eine Stromlinie wird durch Reibungskräfte im Medium beeinflusst \cite[]{demtroeder}.
Ein Maß für die Stärke der inneren Reibung bzw. die Zähigkeit des Stoffes ist die \textit{dynamische Viskosität} $\eta$ mit der SI-Einheit $\unit{\Pa \s}$.
Diese ist eine temperaturabhängige Materialkonstante \cite[]{geschke}.
Hierbei wird zwischen idealen bzw. nicht-viskosen Stoffen unterschieden, für die $\eta \simeq 0$ gilt, und viskosen Stoffen,
deren Strömungsverhalten in laminar und turbulent eingeteilt wird (vgl. Abschnitt \ref{sec:vorbereitung}).
\noindent
Die Reibung, die eine Kugel mit Radius $r$ und Geschwindigkeit $v$ in einem Fluid erfährt wird mit Hilfe der Stokesschen Reibung 
\noindent
\begin{align}
    F_\text{R} = 6 \pi \eta v r    
\end{align}
\noindent
beschrieben.
Sie wirkt mit der Auftriebskraft $\symbf{F}_A$ der Schwerkraft $\symbf{F}_g$ entgegen. 
Mit steigender Geschwindigkeit $v$ wächst die Reibungskraft $\symbf{F}_R$, bis sich ein Kräftegleichgewicht einstellt
und $v$ einen konstanten Wert annimmt.
Dabei wird die Viskosität des Fluids mit der empirischen Formel 
\begin{align}
    \label{eq:empirisch}
    \eta = K \left(\rho_\text{K} -\rho_\text{Fl}\right) \cdot t
\end{align}
\noindent
ermittelt, wobei $K$ eine Apparatekonstante ist, die Fallhöhe und Kugelgeometrie berücksichtigt,
$\rho_K$ bzw. $\rho_{Fl}$ die Dichte der Kugel bzw. des Fluids sind und $t$ die Fallzeit darstellt.
\noindent
%
Durch die \textit{Andradesche Gleichung} wird die Temperaturabhängigkeit der Viskosität von Flüssigkeiten dargestellt.
\begin{equation}
\label{eq:andrade}
    \eta \left(T\right) = A \exp{\left(\frac{B}{T}\right)}
\end{equation}
$A$ und $B$ sind dabei Konstanten \cite*[]{va207}.

