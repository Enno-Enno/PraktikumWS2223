\section[]{Versuchsaufbau}
Das Kugelfall-Viskosimeter nach Höppler besteht aus einem zylinderförmigen Glasrohr,
das mit einer Flüssigkeit gefüllt wird (hier destilliertes Wasser), deren Viskosität bestimmt werden soll. 
In dem befüllten Rohr lässt man eine Kugel fallen, deren Durchmesser nahe am Rohrdurchmeser liegt.
Es wurden hier zwei Kugeln mit unterschiedlichen Durchmessern verwendet. 
Um unkontrollierte Stöße mit der Rohrwand und Wirbelbildungen zu vermeiden, wird das Rohr etwas geneigt.
Am Rohr selbst befinden sich insgesamt drei Markierungen, die je einen Abstand von 5 Zentimetern zu einander haben.

Das Fallrohr befindet sich in einem Wasserbad, durch das mit Hilfe eines angeschlossenen Thermostates die Temperatur im Rohr geregelt werden kann.
Am Thermostat befindet sich ein Thermometer, sodass die Temperatur im Viskosimeter abgelesen werden kann.

\section{Durchführung}

Zu Beginn des Versuchs wurden die Durchmesser der Kugeln mit Hilfe einer Schieblehre bestimmt, damit man mit den vorgegebenen Massen die Dichten ermitteln kann.
Außerdem wurde die Zimmertemperatur erfasst.
Anschließend wurde der obere Verschluss geöffnet und das Fallrohr mit destilliertem Wasser befüllt. 
Danach wurde zunächst die kleinere Kugel in das Rohr gelassen.
Hierbei war darauf zu achten, dass sich keine Luftbläschen bildeten.
Die Bläschen, die nach dem Befüllen und Einlassen der Kugel vorhanden waren, wurden mit einer Bürste bzw. einem Glaskolben entfernt.
Anschließend wurde der Deckel verschlossen und darauf geachtet, dass sich keine weitere Luft im Fallrohr befand.

Bereits am Anfang des Versuchs war Wasser im Wasserbad vorhanden und die Temperatur wurde auf Zimmertemperatur reguliert,
damit für die ersten beiden Versuchsreihen die Temperatur im Rohr konstant gehalten werden konnte.

Die erste Versuchsreihe bestand darin, dass bei Zimmertemperatur die kleine Kugel fallen gelassen wurde.
Nachdem sich eine konstante Geschwindigkeit durch das in Abschnitt \ref{sec:grundlagen} genannte Kräftegleichgewicht eingestellt hatte,
wurde die Fallzeit bei einer Strecke von 10 Zentimetern (oberste bis unterste Markierung)
mit Hilfe einer Stoppuhr gemessen.
Nach jedem Fall wurde das Rohr um 180° und erneut gemessen.
Je nach Orientierung werden diese Messwerte im Folgenden mit \enquote{oben} bzw. \enquote{unten} betitelt.
% Anführungszeichen funktionieren nur mit \enquote{}
Dies wurde 10 Mal wiederholt, sodass sich insgesamt 20 Werte (je 10 oben und unten) ergaben.

Die zweite Messreihe verlief analog zur ersten mitsamt der Vorbereitungen, nur dass nun die größere Kugel auf einer Strecke von 5 Zentimetern betrachtet wurde.
Um die Kugeln auszutauschen, musste der Verschluss geöffnet werden, wodurch das destillierte Wasser aus dem Rohr entweichen konnte.
Auch hier mussten also beim erneuten befüllen etwaige Luftbläschen von Rohr und Kugel werden.
Außerdem wurden die Messungen hier nur 5 Mal wiederholt.

Die dritte Messreihe bestand darin, dass mit dem Thermostat die Temperatur langsam erhöht wurde
und dabei die Fallzeit der größeren Kugel auf 5 Zentimetern gemessen wurde.
Da die gleiche Kugel wie zuvor verwendet wurde, musste das Rohr nicht aufgeschraubt oder aufgefüllt werden.
Für jede Temperatur wurden je zwei Werte für oben und unten aufgenommen und es wurde für insgesamt 10 unterschiedliche Temperaturen gemessen.
Hierbei war zu beachten, dass sich erst die gewünschte Temperatur im Fallrohr einstellte, bevor man für die höhere Temperatur gemessen hat, 
um Messunsicherheiten zu vermeiden. Die Wartezeit betrug häufig ca. 3-5 Minuten. Maximal sollte eine Temperatur von 50°C erreicht werden.
