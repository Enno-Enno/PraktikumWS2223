\section{Vorbereitungsaufgaben}
\subsection{Laminare Strömung und Reynoldszahl}
\label{sec:vorbereitung}
Von einer laminaren Strömung wird gesprochen, wenn in einem realen Stoff
die einzelnen Stromfäden ohne sich zu beeinflussen nebeneinander liegen. Das bedeutet, dass es keine Wirbel oder
Turbulenzen innerhalb des Stoffes gibt und sich die einzelnen Schichten mit Strömungsgeschwindigkeiten nicht vermischen.
Die \textit{Reynoldszahl} $Re$ ist ein Maß dafür, ob es sich um eine laminare oder turbulente Strömung handelt.
\noindent
\begin{align}
    Re = \frac{\rho \overline{v} R}{\eta}
\end{align}
Dabei ist $\rho$ die Dichte des Fluids, $\overline{v}$ die relative mittlere Geschwindigkeit zwischen Fluid und Körper und
$R$ die charakteristische Länge.
Wenn die Reynoldszahl unterhalb einer kritischen Grenze $Re_c$ liegt, kann von einer laminaren Strömung gesprochen werden.
Die charakteristische Länge $R$ ist hier der Durchmesser der Kugel und die kritische Grenze der Reynoldszahl $Re_c$ beträgt ca. 2300 \cite*[]{geschke}.


\subsection{Temperaturabhängige Viskosität und Dichte von destilliertem Wasser}
Die Dichte $\rho$ von Wasser hat ihren Maximalwert bei $T_{max} = \qty[]{4}{\degreeCelsius}$ \cite*[]{geschke}.
Außerdem ist $\rho$ im festen Aggregatszustand geringer als im flüssigen Zustand. 
Diese beiden Tatsachen werden auch als \enquote{Dichteanomalie des Wassers} bezeichnet\cite*[]{demtroeder}.
Die Viskosität $\rho$ fällt exponentiell mit dem Kehrwert der Temperatur $\frac{1}{T}$ (vgl. \eqref{eq:andrade}).

