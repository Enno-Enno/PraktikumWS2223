\section{Diskussion}

\subsection[]{Messunsicherheiten}
\textbf{Kugeln:}
Die Massen der Kugeln wurden ohne Ungenauigkeit angegeben, sodass sie keinen Einfluss auf indirekte Messgrößen haben.
Da allerdings nicht bekannt ist, von wann die Werte stammen, kann es durchaus sein, dass im Laufe der Zeit Schwankungen entstanden sind
(etwa durch Herunterfallen der Kugeln, Vertauschungen, etc.).
Die mittels der Schieblehre gemessenen Durchmesser der beiden Kugeln sind hinreichend genau.

\textbf{Temperatur:}
Dem Thermometer konnte man nur ganze Zahlen entnehmen und es war kein Fehler angegeben, sodass man hier von einer relativ hohen Ungenauigkeit ausgehen kann.
Am Thermostat war es nur bis auf ca. $\qty{3}{\degreeCelsius}$ möglich die Temperatur einzustellen und durch die Wartezeit nach dem Einstellen der Temperatur
handelt es sich um einen wohl sehr fehleranfälligen Teil des Versuchs.

\textbf{Zeitmessung:}
Die Stoppuhr kann die Zeit bis auf die 2. Nachkommastelle genau stoppen, was an sich ausreichend ekakt ist.
Allerdings können hier durch den Betrachtungswinkel, und die Reflexe der Experimentierenden ebenfalls Schwankungen auftreten.

\textbf{Viskosimeter:}
%Durch die leicht unterschiedlichen Neigungswinkel bei den oberen und unteren Messungen lassen sich mögliche Unterschiede zwischen den einzelnen oberen und
%unteren Messreihen erklären.
Durch die leicht unterschiedlichen Neigungswinkel im Viskosimeter sind leichte unterschiede in den Ergebnissen vorhanden.
Diese sind allerdings nicht zu gravierend, da die Fehlerintervalle für die Viskosität sich in
 \ref{tab:visk_kl_Zitemp}, \ref{tab:mittel_fallzeit_gross} und \ref{tab:mittel_Apparatekonstante} überschneiden.
Auch ist nicht bekannt, was die Genauigkeit zwischen den markierten Abständen am Fallrohr ist, was ebenfalls zu einer weiteren Fehlerquelle führen kann.

\textbf{Systematischer Fehler:} 
Der Vergleich in Tabelle \ref{tab:viskositaeten_temp} mit den Literaturwerten zeigt, 
dass die gemessenen Werte konstant \num{0.1} bis \qty{0.2}{\milli\Pascal\s} größer sind als die Literaturwerte. 
Es ist davon auszugehen, dass es einen systematischen Fehler gibt, der die Ergebnisse für die
Viskosität verfälscht.
Mögliche Fehlerquelle kann z.B. eine fehlerhaft angegebene Apparatekonstante sein.
Ein anderer möglicher Fehler ist eine falsche Massenangabe für die große oder kleine Kugel.

%vergleich mit Literaturwert

\subsection[]{Die Andradesche Gleichung}
Der lineare Fit in Abbildung \ref{fig:groKu_steigendeTemp_eta_fit} konnte für beide Messreihen die
Ergebnisse ausreichend genau abbilden. 
Die Andradesche Gleichung ist also in diesem Experiment gültig.

\subsection[]{Die Reynoldszahl}
Anhand der ermittelten Werte $Re$ deutlich unterhalb von 2300 kann man annehmen, dass es sich bei den Messreihen um laminare Strömung handelt.
Trotzdem fällt eine große Diskrepanz zwischen der Reynoldszahl für die kleine und der für die große Kugel auf.
Rein rechnerisch hängt dies in erster Linie mit den großen Unterschieden der Geschwindigkeiten zusammen.
Diese wiederum begründet sich vor allem mit der um den etwa Faktor 2 unterschiedliche Gerätekonstante.
Da $v$ proportional zur Apparatekonstante $K$ ist lässt sich die unterschiedliche Geschwindigkeit 
durch die unterschiedlichen Beschaffenheiten des Geräts (z.B. geringerer Abstand zwischen Kugeloberfläche und Wand) begründen.
%Es lässt sich also annehmen, dass durch die unterschiedlichen Gerätekonstanten und Dichten andere Strömungungen entstehen, sodass 
%die große Kugel langsamer fällt als die kleine Kugel.

%\subsection[]{Bemerkungen}
%Zusammenfassend ist der Versuch gut umzusetzen aber auch an manchen Punkten fehleranfällig.
%Durch ein besseres Thermostat ließe sich nicht nur eine große Fehlerquelle senken,
%sondern auch noch lange Wartezeiten zwischen den Messungen sowie fallende Konzentration seitens der Experimentierenden vorbeugen.

% Dinge über die systematische Abweichung vom Geschke bei den Viskositäten  



%Fehler von messreihe 3 durch nur 2 Messungen jeweils

%vergleichen mit literaturwerten

%messunsicherheiten kürzer(?)
