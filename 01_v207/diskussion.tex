\section{Diskussion}

\subsection[]{Messunsicherheiten}
\textbf{Kugeln:}
Die Massen der Kugeln wurden ohne Ungenauigkeit angegeben, sodass sie keinen Einfluss auf indirekte Messgrößen haben.
Da allerdings nicht bekannt ist, von wann die Werte stammen, kann es durchaus sein, dass im Laufe der Zeit Schwankungen entstanden sind
(etwa durch Herunterfallen der Kugeln, Vertauschungen, etc.).
Die mittels der Schieblehre gemessenen Durchmesser der beiden Kugeln sind hinreichend genau.

\textbf{Temperatur:}
Dem Thermometer konnte man nur ganze Zahlen entnehmen und es war kein Fehler angegeben, sodass man hier von einer relativ hohen Ungenauigkeit ausgehen kann.
Am Thermostat war es nur bis auf ca. $\qty{3}{\degreeCelsius}$ möglich die Temperatur einzustellen und durch die Wartezeit nach dem Einstellen der Temperatur
handelt es sich um einen wohl sehr fehleranfälligen Teil des Versuchs.

\textbf{Zeitmessung:}
Die Stoppuhr kann die Zeit bis auf die 2. Nachkommastelle genau stoppen, was an sich ausreichend ekakt ist.
Allerdings können hier durch den Betrachtungswinkel, die Reflexe, ... der Experimentierenden ebenfalls Schwankungen auftreten.

\textbf{Viskosimeter:}
Durch die leicht unterschiedlichen Neigungswinkel bei den oberen und unteren Messungen lassen sich mögliche Unterschiede zwischen den einzelnen oberen und
unteren Messreihen erklären.
Auch ist nicht bekannt, was die Genauigkeit zwischen den markierten Abständen am Fallrohr ist, was ebenfalls zu einer weiteren Fehlerquelle führt.

%vergleich mit Literaturwert

\subsection[]{Die Andradesche Gleichung}
Anhand der Messdaten können wir feststellen, dass eine höhere Temperatur zu einer niedrigeren Viskosität führt.
Über die Gültigkeit der Andradeschen Gleichung kann keine Aussage getroffen werden.

\subsection[]{Die Reynoldszahl}
Anhand der ermittelten Werte $Re$ deutlich unterhalb von 2300 kann man annehmen, dass es sich bei den Messreihen um laminare Strömung handelt.
Trotzdem fällt eine große Diskrepanz zwischen der Reynoldszahl für die kleine und der für die große Kugel auf.
Rein rechnerisch hängt dies in erster Linie mit den großen Unterschieden der Geschwindigkeiten zusammen.
Es lässt sich also annehmen, dass durch die unterschiedlichen Gerätekonstanten und Dichten andere Strömungsentstehen, sodass 
die große Kugel langsamer fällt als die kleine Kugel.

\subsection[]{Bemerkungen}
Zusammenfassend ist der Versuch gut umzusetzen aber auch an manchen Punkten fehleranfällig.
Durch ein besseren Thermostat ließe sich nicht nur eine große Fehlerquelle senken,
sondern auch noch lange Wartezeiten zwischen den Messungen sowie fallende Konzentration seitens der Experimentierenden vorbeugen.



%Fehler von messreihe 3 durch nur 2 Messungen jeweils

%vergleichen mit literaturwerten

%messunsicherheiten kürzer(?)