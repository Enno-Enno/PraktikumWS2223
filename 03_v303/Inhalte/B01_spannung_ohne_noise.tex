%FRAGEN:
%Noch mehr auf Verifikation eingehen?
%
%
%
%
%
%
\subsection{Schaltung ohne Noise Generator}
\label{sec:ohne_noise}
\subsubsection{Verifikaton des Gleichrichters}
\label{sec:gleichrichter}
Mit dem linken Ausgang des Funktionsgenerators lassen sich die Spannungsamplituden variieren,
während der rechten Ausgang ohne die Beschriftung \enquote{phase shift} eine konstante Spannung erzeugt.
%
Im nächsten Schritt wird der Schaltplan \textbf{SCHALTPLAN REFERENZIEREN} aufgebaut,
um die Funktionsweise eines phasenempfindlichen Gleichrichters zu verifizieren.
Dabei wird der Noise Generator gebrückt.
Dies findet ohne Verwendung des Tiefpasses statt.
Es wird ein sinusförmiges Signal $U_\text{sig}$ von \qty[]{1}{\kilo\hertz} und $U_0 = \qty[]{10}{\milli\volt}$ am Oszillator Output eingestellt.
Dieses Signal wird mit einem Sinus- / Referenzsignal $U_\text{ref}$ identischer Frequenz gemischt.
Der Gain Regulierer des Verstärkers hat einen Wert von 1 und der Gain Regulierer des Lock-in Detektors einen Wert von 20.
%% FRAGE: MUSS MAN DAS IN DER RECHNUNG BERÜCKSICHTIGEN?
Mit dem Phasenschieber werden insgesamt fünf verschiedene Phasenverschiebungen $\phi$ eingestellt,
die in Abbildung \ref{fig:phasenunterschiede_ohne_noise} einzusehen sind.
Vergleicht man diese Ergebnisse mit \textbf{REFERENZ ABBILDUNG!!!}, lässt sich
die Funktionsweise eines phasenempfindlichen Gleichrichters verifizieren.
%
\begin{figure}[H]%
    \begin{subfigure}{0.5\textwidth}%
    \centering%
    \includegraphics[width = 7.3cm]{./Oszilloskop Bilder/png/5.2/1 MAP002.png}%
    \caption{$\phi = \qty[]{0}{\degree}$}%
    \label{fig:phase1}%
    \end{subfigure}%
    %
    \hfill% Fills available space in the center -> space between figures
    \begin{subfigure}{0.5\textwidth}%
    \centering%
    \includegraphics[width = 7.3cm]{./Oszilloskop Bilder/png/5.2/2 MAP003.png}%
    \caption{$\phi = \qty[]{45}{\degree}$}%
    \label{fig:phase2}%
    \end{subfigure}%
    %
    \hfill
    \begin{subfigure}{0.5\textwidth}%
    \centering%
    \includegraphics[width = 7.3cm]{./Oszilloskop Bilder/png/5.2/3 MAP004.png}%
    \caption{$\phi = \qty[]{90}{\degree}$}%
    \label{fig:phase3}%
    \end{subfigure}%
    %
    \hfill% Fills available space in the center -> space between figures
    \begin{subfigure}{0.5\textwidth}%
    \centering%
    \includegraphics[width = 7.3cm]{./Oszilloskop Bilder/png/5.2/4 MAP005.png}%
    \caption{$\phi = \qty[]{135}{\degree}$}%
    \label{fig:phase4}%
    \end{subfigure}%
    %
    \hfill
    \begin{subfigure}{0.5\textwidth}%
    \centering%
    \includegraphics[width = 7.3cm]{./Oszilloskop Bilder/png/5.2/5 MAP006.png}%
    \caption{$\phi = \qty[]{180}{\degree}$}%
    \label{fig:phase5}%
    \end{subfigure}%
    %
    \caption{Spannungsverläufe für unterschiedliche Phasen ohne Integration und ohne Noise Generator}%
    \label{fig:phasenunterschiede_ohne_noise}%
\end{figure}%
%  
%
%
\subsubsection{Spannung in Abhängigkeit der Phase nach Integration ohne Geräusche}
\label{sec:integration_ohne_noise}
Schaltet man den Tiefpass hinzu, ergeben sich je nach Phase $\phi$ die Spannungsamplituden $U_\text{out}$ in Tabelle \ref{tab:u_out_tp_ohne_noise}.
%
\begin{table}
    \centering
    \caption[]{Ausgangsspannung nach Integration ohne Geräuschsignal}
%    \caption[]{Amplitude der Ausgangsspannung nach Integration durch den Tiefpass}
    \label{tab:u_out_tp_ohne_noise}
    \sisetup{table-format=3.0}
    \begin{tabular}[]{S c S[table-format=2.2]}
        \toprule
        {$\phi / \unit[]{\degree}$} & {$\phi / \unit[]{\radian}$} & {$U_\text{out} / \unit[]{\volt}$} \\
        \midrule
           0 &     0          & -0.10 \\ % Spannungsamplitude von channel 1 in mV: 871.2
          45 & $    \pi / 4 $ & -0.10 \\ % Spannungsamplitude von channel 1 in mV: 712.8
          90 & $    \pi / 2 $ &  0.04 \\ % Spannungsamplitude von channel 1 in mV: 653.4
         135 & $ 3  \pi / 4 $ &  0.18 \\ % Spannungsamplitude von channel 1 in mV: 613.8
         180 & $    \pi     $ &  0.28 \\ % Spannungsamplitude von channel 1 in mV: 455.4
         225 & $ 5  \pi / 4 $ &  0.26 \\ % Spannungsamplitude von channel 1 in mV: 950.4
         270 & $ 3  \pi / 2 $ &  0.14 \\ % Spannungsamplitude von channel 1 in mV: 1190  
         300 & $ 5  \pi / 3 $ &  0.06 \\ % Spannungsamplitude von channel 1 in mV: 990.0
         315 & $ 7  \pi / 4 $ &  0.00 \\ % Spannungsamplitude von channel 1 in mV: 1170  
         330 & $ 11 \pi / 6 $ & -0.06 \\ % Spannungsamplitude von channel 1 in mV: 1030  
        \bottomrule
    \end{tabular} 
\end{table}
%

\noindent
Da die Spannungswerte gemäß \textbf{REFERENZ!!!!} $2 \pi$-periodisch sein sollen, wird eine Ausgleichsrechnung mit der Funktion 
\begin{align}
    \label{eq:ausgleich_tp_ohne_noise}
    U(\phi) = A \cos{\left(B \phi + C\right)} + D 
\end{align}
durchgeführt, was mittels der Funktion \texttt{curve\_fit} aus dem Python \cite{python} Paket \texttt{scipy.optimize} \cite[]{scipy} realisiert wird.
Es ergibt sich Plot \ref{eq:ausgleich_tp_ohne_noise} mit den Parametern
\begin{align*}
    A_\text{out} &= \left(\num[]{-0.2002} \pm \num[]{0.0055}\right) \, \unit{\volt} & B_\text{out} &=  \num[]{0.9938} \pm \num[]{0.0208} \\
    C_\text{out} &= \num[]{-0.2768} \pm \num[]{0.0779} & D_\text{out} &= \left(\num[]{0.0869} \pm \num[]{0.0054}\right) \, \unit[]{\volt},
\end{align*}
wobei die Fehler durch die Kovarianzmatrix berechnet werden.
% cosinus für av_ch2:
% a = -0.20018493
% b = 0.99376117
% c = -0.27683892
% d = 0.08687120
%fehler:
%[0.00549263 0.02075618 0.07790146 0.0054004 ]
%
\begin{figure}[H]
    \includegraphics[]{build/B01_ausgleichsplot.pdf}
    \caption[]{Ausgleichsfunktion \eqref{eq:ausgleich_tp_ohne_noise} zu den Werten aus Tabelle \ref{tab:u_out_tp_ohne_noise}}
    \label{fig:ausgleichsplot1}
\end{figure}

\noindent
Im Vergleich mit Gleichung \textbf{REFERENZ!!!} ergibt sich für die Amplitude der Eingangsspannung $U_{0,\text{out}}$,
wenn man noch durch die obigen Gain Faktoren 1 und 20 teilt:
\begin{align*}
    U_{0,\text{out}} = \frac{\pi}{2} \frac{1}{1 \cdot 20} A_\text{out} = \left(-15.7 \pm 0.4\right) \, \unit[]{\milli\volt}
\end{align*}
Dies entspricht einem Verhältnis von $\Delta U_{0,\text{out}} = \left(57 \pm 4\right) \, \%$
im Vergleich zu $U_0 = \qty[]{10}{\milli\volt}$.

%#delta u_0 =  1.57+/-0.04

