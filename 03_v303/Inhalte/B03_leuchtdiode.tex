\subsection{Photodetektorschaltung}
Es wird der Schaltplan \ref{fig:Lock-In-LED} aufgebaut und die Leuchtdiode mit einer Rechteckspannung der Frequenz \qty[]{214.26}{\hertz}  moduliert.
Der Gain Regulierer am Preamplifier wird auf 10 gestellt, der Regulierer am Lock-in / Amplitude Detector auf 20.
Die Leuchtdiode steht auf der Schiene bei $x_\text{LED} = \qty[]{5}{\cm}$ während der Photodetektor verschoben wird, 
sodass für den Abstand $r$ zwischen LED und Detektor $r = x_\text{PD} - x_\text{LED}$ gilt.
Die Werte für Abstand $r$ und Spannung $U_\text{r}$ sind in Tabelle \ref{tab:abstand_spanung} einzusehen.
\begin{table}
    \centering
    \caption{Spannung in Abhängigkeit des Abstandes}
    \label{tab:abstand_spanung}
    \begin{tabular}[]{S[table-format = 3.0] S[table-format = 3.0] S[table-format = 1.3]}
        \toprule
        {$x_\text{PD} / \unit[]{\cm}$} & {$r / \unit[]{\cm}$} & {$U_\text{r} / \unit[]{\volt}$} \\
        \midrule
         10 &   5 & 4.49  \\
         15 &  10 & 1.34  \\
         20 &  15 & 0.48  \\
         25 &  20 & 0.24  \\
         30 &  25 & 0.225 \\
         40 &  35 & 0.125 \\
         50 &  45 & 0.082 \\
         60 &  55 & 0.070 \\
         70 &  65 & 0.056 \\
         80 &  75 & 0.045 \\
         90 &  85 & 0.035 \\
        100 &  95 & 0.030 \\
        110 & 105 & 0.025 \\
        120 & 115 & 0.021 \\
        130 & 125 & 0.020 \\
        140 & 135 & 0.019 \\
        150 & 145 & 0.017 \\
        \bottomrule        
    \end{tabular}
\end{table}

\noindent
Es kann von einer Spannungsabnahme proportional zu $r^{-2}$ ausgegangen werden.
Mit der Funktion \texttt{curve\_fit} des Python \cite[]{python} Pakets \texttt{scipy.optimize} \cite[]{scipy} wird eine nicht-lineare Ausgleichsrechnung
der Funktion 
\begin{align}
    \label{eq:ausgleich2}
    U(r) = C \frac{1}{r^{2}} + D
\end{align}
durchgeführt.
Es ergeben sich die Parameter 
\begin{align*}
    C &= \left(\num[]{0.0113} \pm \num[]{0.0001}\right)\unit{\volt\meter\squared} & D &= \left(\num[]{0.0254} \pm \num[]{0.0142}\right)\unit{\volt} 
\end{align*}
%#a = 0.01125850
%#b = 0.02540392
%#fehler:  [0.00014112 0.01423447]
sowie Plot \ref{fig:ausgleich3_plot}.
%
\begin{figure}
    \includegraphics[]{build/B03_ausgleichsplot.pdf}
    \caption[]{Ausgleichsfunktion \eqref{eq:ausgleich2} zu den Werten aus Tabelle \ref{tab:abstand_spanung}}
    \label{fig:ausgleich3_plot}
\end{figure}

\noindent
Sowohl in Tabelle \ref{tab:abstand_spanung} als auch in Abbildung \ref{fig:ausgleich3_plot} ist erkennbar, 
dass kein maximaler Abstand $r_\text{max}$ erreicht wird, bei dem das Licht nicht mehr nachgewiesen werden kann.
Das hängt damit zusammen, dass die Schiene zu kurz ist.


%% In Diskussion: Umgebungslicht wird als noise gefiltert