\section{Diskussion}
%auf Aufgabenstellung (Verifikation Gleichrichter und Lock-in-Verstärker sowie photodetektorschaltung eingehen)
\subsection{Gleichrichter}
Anhand der Abbildung \ref{fig:phasenunterschiede_ohne_noise} lässt sich die Funktionsweise eines Gleichrichters verifizieren.
Dies ist insbesondere an dem Vergleich für $\phi = \qty[]{0}{\degree}$ erkennbar:
Der Signalverlauf in Abb. \ref{fig:phase1} hat die gleiche Form wie in der dritten Zeile von Abb. \ref{fig:signalverlaeufe}.


%U_0 weicht etwas ab weil Fehler nirgends gegeben, ungenaue einstellungen am Gerät, ... -> systematischer Fehler
% Systematischer Fehler Faktor 1.5 

\subsection{Messfehler}
In \refname{}

\subsection{Photodetektorschaltung}
