\section{Diskussion}
\subsection{Gleichrichter}
Anhand der Abbildung \ref{fig:phasenunterschiede_ohne_noise} lässt sich die Funktionsweise eines Gleichrichters verifizieren.
Dies ist insbesondere an dem Vergleich für $\phi = \qty[]{0}{\degree}$ erkennbar:
Der Signalverlauf in Abb. \ref{fig:phase1} hat die gleiche Form wie in der dritten Zeile von Abb. \ref{fig:signalverlaeufe}.

%Verifikation Lock-in-Verstärker 

%U_0 weicht etwas ab weil Fehler nirgends gegeben, ungenaue einstellungen am Gerät, ... -> systematischer Fehler
% Systematischer Fehler Faktor 1.5 

\subsection{Messfehler}
In \refname{}

\subsection{Photodetektorschaltung}
Durch die nicht-lineare Ausgleichsfunktion \eqref{eq:ausgleich2} und der Abb. \ref{fig:ausgleich3_plot} wird gezeigt, 
dass die detektierte Spannung $U_\text{r}$ proportional zu $1/r^2$ ist.
Die Rauschunterdrückung des Lock-in-Verstärkers kann aufgrund der geringen Abweichungen im Plot ebenfalls gezeigt werden:
Bei der Durchführung des Versuchs war die Deckenbeleuchtung eingeschaltet und Tageslicht fiel durch Fenster ein und war durch Bewölkung unterschiedlich intensiv.
Diese Störsignale wurden erfolgreich vom Lock-in-Verstärker heraus gefiltert.
Ein maximaler Abstand zur Erfassung des Lichts kann aufgrund der geringen Länge der Schiene nicht ermittelt werden.