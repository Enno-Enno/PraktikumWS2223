\section{Diskussion}
%auf Aufgabenstellung (Verifikation Gleichrichter und Lock-in-Verstärker sowie photodetektorschaltung eingehen)
\subsection{Gleichrichter}
Anhand der Abbildung \ref{fig:phasenunterschiede_ohne_noise} lässt sich die Funktionsweise eines Gleichrichters verifizieren.
Dies ist insbesondere an dem Vergleich für $\phi = \qty[]{0}{\degree}$ erkennbar:
Der Signalverlauf in Abb. \ref{fig:phase1} hat nahezu die gleiche Form wie 
die gleichgerichte Funktion in der dritten Zeile von Abb. \ref{fig:signalverlaeufe}.
Leichte Abweichungen lassen sich möglicherweise durch kleine Verzögerungen in den Bauteilen erklären, die eine kleine
Phasenverschiebung hervorrufen können.

\subsection{Tiefpass}

Die Messungen für $U_\text{out}$ am Ausgang des Tiefpasses 
werden in beiden Messreihen durch den generierten Fit ausreichend gut angenähert.
Das vermischen des Signals mit künstlichem Rauschen kann in der hier gewählten Konfiguration das Ergebnis nicht verfälschen.

\subsection{Messfehler}

In Abschnitt \ref{sec:integration_ohne_noise} weicht die gemessene Größe von $U_0$ um etwa $\qty[]{57}{\percent}$ nach oben von der eingestellten Größe ab.
In Abschnitt \ref{sec:fit_mit_noise} gibt es Eine Abweichung um $\qty[]{57}{\percent}$ in die Entgegengesetzte Richtung.
Zwischen den Messungen wurde ein Verstärker und der Rauschgenerator dazugeschaltet.
Diese Technischen Elemente können eine Erklärung für die Abweichungen in den Messungen sein.
Für die gleiche Größe der Abweichungen in beide Richtungen kann keine richtige Erklärung gefunden werden.
Es wären weitere Experimente nötig um diese konsistenten Messfehler zu erklären.
Die Messungen innerhalb einer Messreihe stimmen allerdings untereinander mit den Erwartungen überein, 
weshalb die Messergebnisse als in sich konsistent betrachtet werden können.


%U_0 weicht etwas ab weil Fehler nirgends gegeben, ungenaue Einstellungen am Gerät, ... -> systematischer Fehler
% Systematischer Fehler Faktor 1.5 
\subsection{Photodetektorschaltung}

Bei der Messung mit der Leuchtdiode gibt es einige Technische Herausforderungen die kleinen Signalspannungen auf dem Oszilloskop zu messen.
Dafür muss in mehreren Schritten der Messbereich für die präziseren Messungen heruntergesetzt werden.
Mit diesen empfindlichen Messungen konnte keine Entfernung gefunden werden, bei der kein Signal mehr gemessen werden konnte. 
% Schlechte Idee, mehr Arbeit: 
% Die Distanz, bei der in dem gröbsten Messbereich das Signal nicht mehr sicher nachweisbar gewesen wäre beträgt 30 cm
Der Fit der Funktion in der Form $1/x^2$ ist typisch für ein Signal im dreidimensionalen Raum,
das sich in $r$ Richtung ausbreitet und in zwei Dimensionen verteilt.
Die generierte Funktion kann die gemessenen Werte hinreichend gut abbilden.
Es gibt allerdings hier systematische Messfehler in der Messung des Abstandes, 
weil die Zahlen auf der Messleiste nicht direkt mit den Positionen der Leuchtdiode des Photodetektors übereinstimmen.
Es konnte allerdings gezeigt werden dass trotz starker störsignale durch Sonnenlicht und Raumbeleuchtung, 
das gewünschte Signal isoliert und gemessen werden konnte.