\section{Diskussion}
\subsection{Temperatur}
Die am Temperaturregler eingestellten Temperaturen $T$ schwanken in einem Bereich von ca. $\qty[]{10}{\celsius}$.
Da $T$ allerdings nicht in Abhängigkeit der jeweils gemessenen Spannung $U$ dokumentiert wurde, 
sondern immer nur zwischendurch in unregelmäßigen Abständen abgelesen wurde, kann dieser Bereich nicht genauer eingegrenzt werden.
%Augenscheinlich war $T$ eine gewisse Zeit lang unterhalb des Mittelwertes, während sie in anderen Zeitintervallen
%oberhalb des Mittelwertes war.
Augenscheinlich haben sowohl $T$ als auch die Zeitintervalle eines Temperaturbereichs fast willkürlich fluktuiert.
Somit gibt es keine näheren Informationen über die Temperaturverteilung.
Für eine etwas bessere Abschätzung wurde zwar stets der Mittelwert der (folgich insgesamt ungenau angegebenen) Temperaturintervalle verwendet,
eine etwaige Fehlerrechnung hätte aufgrund des Mangels an Daten allerdings keine große Aussagekraft, weshalb auf diese verzichtet wurde.

\noindent
Mit den in Abschnitt \ref{sec:dampfdruck_auswertung} berechneten Werten für den Dampfdruck $p_\text{sät}$ 
kann nachgewiesen werden, dass die mittlere Weglänge $\overline{w}$ im Versuchsteil A nicht zum Franck-Hertz-Effekt führt.
%Temperatur sehr ungenau
%U_B maximal bei 55V
%alle möglichen regler sind arbitrary units