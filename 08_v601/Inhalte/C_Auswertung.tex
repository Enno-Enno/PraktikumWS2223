\section{Auswertung}

\subsection{Der Dampfdruck}
Um die Messungen am x-y Schreiber auszuwerten müssen zunächst der Dampfdruck des Quecksilbers (Gleichung \textbf{Referenz})
und die mittlere freie Weglänge der Elektronen (Gleichung \textbf{Referenz}) basierend auf der Temperatur bestimmt werden.
Es ist nicht möglich die Temperatur in den Geräten präzise einzustellen. Die Temperaturen schwanken jeweils in einem Bereich von etwa
\qty{10}{\celsius}. 
Für die Berechnungen werden die unteren Grenzen der Temperaturverteilung genommen was zu einer hohen Einschätzung der mittleren
freien Wegstrecke führt.
\begin{table}
    \centering
        \begin{tabular}{c 
            S[table-format = 3.0] 
            S[table-format = 3.0] 
            S 
            S}
        \toprule
        {} &
        {$T/\unit{\celsius}$}&
        {$T/\unit{\celsius}$}&
        {$p_\text{sät}/ \unit{\milli\bar} $}&
        {$\overline{w}/ \unit{cm}$}\\
        \midrule
        A  & 24  & 297 & 0.005  & 0.598   \\
        B  & 145 & 418 & 3.948  & 0.7 e-3 \\
        C  & 157 & 430 & 6.247  & 0.5 e-3 \\
        D  & 185 & 458 & 16.606 & 0.2 e-3 \\      
        \bottomrule
    \end{tabular}
\end{table}

Bei Messung A ist die mittlere freie Wegstrecke der Elektronen in einer vergleichbaren Größenordnung zu der Länge der Röhre.
Es ist zu erwarten, dass Elektronen im Durchlauf der Röhre nur selten mit den Hg Atomen interagieren.
 

Um die Abbremsspannung $U_A$ zu bestimmen, die mit der 