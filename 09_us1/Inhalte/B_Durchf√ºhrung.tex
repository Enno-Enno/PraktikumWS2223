\section{Aufbau}
Der Versuch wird mit einem Ultraschallechoskop, welches nur im Impulsbetrieb betrieben werden kann, 
je einer Ultraschallsonde der Frequenz $\qty[]{1}{\mega\hertz}$ und $\qty[]{2}{\mega\hertz}$
und einem Computer zur Datenaufnahme- und analyse durchgeführt.
Die Sonden sind empfindlich, weshalb sie vorsichtig behandelt werden sollten.
Der Schalter \texttt{REFLEC./TRANS.} wird auf \texttt{REFLEC.} gestellt, um das Impuls-Echo-Verfahren verwenden zu können.
Das Durchschallungsverfahren wird in diesem Versuch nicht benötigt.
Die Sende- und Empfangsleistung sind während der Durchführung auf einem konstanten Wert eingestellt.
Es gibt einen zu untersuchenden Acrylblock und ein Brustmodell.
Als Kontaktmittel steht beim Acrylblock bidestiliertes Wasser zur Verfügung, beim Brustmodell wird ein Kontaktgel verwendet.



\noindent
Mit dem Programm \enquote{AScan} am Rechner können die Daten angezeigt, ausgewertet und gespeichert werden.
Außerdem kann die Scanart gewählt werden.
Beim A-Scan können die Signale als Funktion der Zeit oder der Eindringtiefe angezeigt werden.
Zur Bestimmung der Tiefe muss die Schallgeschwindigkeit im jeweiligen Medium eibgegeben werden.
Mittels der FFT-Taste können die gemessenen Frequenzen angegeben werden.




\section{Durchführung}
Mit Hilfe einer Schieblehre werden zunächst die Maße des Acrylblocks mitsamt der Lage und Größe der Löcher bestimmt und notiert.
Eine Darstellung des Blocks mit einer Nummerierung der Löcher ist in Abbildung \ref{fig:acrylblock} zu sehen.

\begin{figure}[H]
    \centering
    \includegraphics*[height=3.5cm]{Abbildungen/acrylblock.png}
    \caption{Darstellung des Acrylblocks \cite[]{man:us1}.}
    \label{fig:acrylblock}
\end{figure}

\subsection{Schallgeschwindigkeit in Acryl}
Im ersten Versuchsteil soll die Schallgeschwindigkeit in Acryl mittels eines A-Scans bestimmt werden.
Der Block wird auf ein Papiertuch gelegt und das Wasser als Kontaktmittel auf die obere Fläche gegeben.
Es werden die Schalllaufzeiten für insgesamt sieben Löcher bei einer Frequenz von $\qty[]{2}{\mega\hertz}$ untersucht, 
indem die Sonde auf die jeweilige Stelle am Block gehalten wird, die sich oberhalb des Lochs befindet. 
Im Anschluss wird der Block umgedreht und der Vorgang wiederholt, sodass sich insgesamt 14 Messwerte ergeben.
Durch die Anpassungsschicht der Sonde tritt hierbei eine systematische Abweichung auf, die ebenfalls zu bestimmen ist.


\subsection{Lage und Größe der Löcher}
In diesem Versuchsteil sollen Position und Durchmesser der Löcher bestimmt werden.
Analog zur ersten Reihe wird der Block wieder mit Wasser beträufelt und die gleiche Sonde verwendet.
Auch hier werden für jedes Loch ein A-Scan gemacht, zuerst von oben und nach Rotation und Beträufelung des Blocks von unten.


\subsection{Auflösungsvermögen}
Beim dritten Versuchsteil werden an den beiden benachbarten Fehlstellen des Blocks (vgl. Abbildung \ref{fig:acrylblock})
A-Scans mit beiden Sonden durchgeführt und ihre Grafiken gespeichert, um zu ermitteln, welche der beiden Frequenzen eine höher Auflösung hat.

\subsection{B-Scan am Acrylblock}
In diesem Teil soll die Lage der Bohrungen mittels eines B-Scans und der $\qty[]{2}{\mega\hertz}$ Sonde von beiden Seiten ermittelt werden.
Erneut wird Wasser auf den Block gegeben und die Sonde aufgesetzt.
Anschließend wird der B-Scan gestartet und die Sonde möglichst gleichmäßig bewegt, um ein zweidimensionales Bild am Computer zu erzeugen.
Die Scans werden erneut von beiden Seiten durchgeführt.

\subsection{Untersuchung des Brustmodells}
Ein Brustmodell soll mit der $\qty[]{2}{\mega\hertz}$ Sonde auf Tumore untersucht werden.
Das bereitstehende Brustmodell hat zwei Tumore.
Der erste ist eine Zyste, also ein mit Flüssigkeit gefüllter Hohlraum.
Der zweite Tumor besteht aus festem Gewebe.
Die grobe Position der beiden Tumor wird zunächst mit der Hand ertastet.
Anschließend wird das Kontaktgel aufgetragen und für jeden Tumor mindestens ein B-Scan durchgeführt, der entlang einer geraden Linie verläuft
und den Tumor möglichst gut abbildet.
Die Bilder werden erneut exportiert.
