\section{Auswertung}
\subsection{Schallgeschwindigkeit im Acrylblock}

\begin{table}
    \centering
    \sisetup{table-format=2.2}
    \begin{tabular}{S S S}
        \toprule
        {$x_A/\unit{\mm} $} & {$D/\unit{\mm}$} & {$x_B/\unit{\mm}$}\\
        \midrule
        15.40   &  10.00    & 55.00         \\
        71.00   &  2.85     & 6.55          \\
        62.80   &  2.90     & 14.70         \\
        55.00   &  2.85     & 22.55         \\
        47.10   &  2.85     & 30.45         \\
        38.90   &  2.95     & 38.55         \\
        30.75   &  3.90     & 45.75         \\
        22.10   &  4.95     & 53.35         \\
        13.30   &  5.90     & 61.20         \\
        61.15   &  1.45     & 17.80         \\
        59.50   &  1.45     & 19.45         \\
        \bottomrule
    \end{tabular}
    \caption{Abmessungen des Acrylblocks}
    \label{tab:messungen}
\end{table}
Die Entfernungen der Fehlstellen vom Rand des Acrylblocks von der Seite A werden in Tabelle \ref{tab:messungen} mit $x_A$ bezeichnet.
Anhand des Lochdurchmessers $D$ wird und des Blockdurchmessers $x_\text{Block} = \qty{80.40}{\mm}$ wird der 
Abstand zum Rand des Blocks auf der B-Seite errechnet.
Diese Messungen werden in \ref{tab:schallzeit} den Ergebnissen des A-Scans gegenübergestellt.
Die Laufzeit des Schalls zur Fehlstelle und zurück wird mit $t_A$ und $t_B$ angegeben.
Zur Berechnung der Schallgeschwindigkeiten $c_A$ und $c_B$ wird von dieser Laufzeit \qty{1}{\micro\s} abgezogen,
um die Verzögerung im Wasser und in der Sonde auszugleichen. 
Dieser Wert entsteht durch vergleichen der Ergebnisse von verschiedenen verschiebungskonstanten mit dem Literaturwert der Schallgeschwindigkeit
im Acrylglas.
\begin{table}
    \centering
    \sisetup{table-format = :.0}
    \begin{tabular}[pos]{S S S S S}
        \toprule
        {Nr.}& {$t_A/\unit{\micro\s}$} & {$t_B/\unit{\micro\s}$} & {$c_a/\unit{\meter\per\second}$} & {$c_b/\unit{\meter\per\second}$} \\
        \midrule
        1       &  11    & 41    & 2800          & 2683          \\
        3       &  46    & 11    & 2730          & 2673          \\
        4       &  40    & 17    & 2750          & 2653          \\
        5       &  34    & 22    & 2771          & 2768          \\
        6       &  28    & 29    & 2779          & 2659          \\
        7       &  22    & 34    & 2795          & 2691          \\
        8       &  16    & 39    & 2763          & 2736          \\
        \bottomrule
    \end{tabular}
    \caption{Laufzeiten und Schallgeschwindigkeiten im Acrylblock \textbf{korrigieren!}}
    \label{tab:schallzeit}
\end{table}
Aus den Werten in Tabelle \ref{tab:schallzeit} wird der Mittelwert $\overline{c}$ und die Standardabweichung $\Delta c$ der Messungen mit den bekannten Formeln berechnet.
Es ergibt sich eine Schallgeschwindigkeit von $c = \qty{2730 +- 50}{\meter\per\second}$.

\subsection{Messung der Fehlstellen mit dem A-Scan}
Um die Tiefe der Fehlstellen korrekt zu bestimmen muss die Korrektur, die bei der Laufzeit für die Schallgeschwindigkeit aufgestellt wurde
mit eingerechnet werden. 
Dazu wird die Strecke $s_1 = \frac{1}{2}\qty{2730}{\meter\per\second} \cdot \qty{1e-6}{\s}= \qty{1.365}{\milli\meter}$ 
von der im Computerprogramm berechneten Strecke abgezogen.
Die abgelesene Tiefe des Blocks $s_\text{Block}$ kann so von \qty{81.5}{\mm} auf \qty{80.14}{mm} korrigiert werden was deutlich
näher am wahren Wert $x_\text{Block} = \qty{80.40}{\mm}$ liegt.
Mit den korrigierten Messwerten für die Tiefe des Blocks und die tiefen der Fehlstellen von den Seiten A und B lassen sich die
Durchmesser der Fehlstellen berechnen.
In Tabelle \ref{tab:a-scan} werden die gemessenen Tiefen $s_A$ und $s_B$ und die berechneten Durchmesser $D_s$ den 
mit der Schieblehre gemessenen Durchmessern $D$ gegenübergestellt. 
Bei der Messung für Loch Nr. 2 von Seite A konnte kein Wert erhoben werden, da das Loch 1 im Weg war.
\begin{table}
    \centering
    \sisetup{table-format=2.1}
    \begin{tabular}[pos]{S[table-format = 1.0] S S S S S S}
        \toprule
        {Nr} & {$s_\text{A}/\unit{mm}$} & {$s_\text{A, korr}/\unit{mm}$} & {$s_\text{B}/\unit{mm}$} & {$s_\text{B, korr}/\unit{mm}$} & {$D_\text{s}/\unit{mm}$} & {$D/\unit{mm}$}\\
        \midrule
        1       &  16.5          & 15.1          & 56.5          & 55.1          & 9.9   & 10.0  \\
        2       &  {-}           & {-}           & 8.5           & 7.1           & {-}   & 2.9   \\
        3       &  64.0          & 62.6          & 16.5          & 15.1          & 2.4   & 2.9   \\
        4       &  56.5          & 55.1          & 24.0          & 22.6          & 2.4   & 2.9   \\
        5       &  48.5          & 47.1          & 32.0          & 30.6          & 2.4   & 2.9   \\
        6       &  40.0          & 38.6          & 40.0          & 38.6          & 2.9   & 3.0   \\
        7       &  31.5          & 30.1          & 47.5          & 46.1          & 3.9   & 3.9   \\
        8       &  23.0          & 21.6          & 55.0          & 53.6          & 4.9   & 5.0   \\
        9       &  14.5          & 13.1          & 62.5          & 61.1          & 5.9   & 5.9   \\
        10      &  62.5          & 61.1          & 19.0          & 17.6          & 1.4   & 1.4   \\
        11      &  61.0          & 59.6          & 21.0          & 19.6          & 0.9   & 1.4   \\
        \bottomrule
    \end{tabular}
    \caption{Messung der Fehlstellen mithilfe des A-Scans}
    \label{tab:a-scan}
\end{table}