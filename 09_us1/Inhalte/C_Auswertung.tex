\section{Auswertung}
\subsection{Schallgeschwindigkeit im Acrylblock}

\begin{table}
    \centering
    \sisetup{table-format=2.2}
    \begin{tabular}{S S S}
        \toprule
        {$x_A/\unit{\mm} $} & {$D/\unit{\mm}$} & {$x_B/\unit{\mm}$}\\
        \midrule
        15.40   &  10.00    & 55.00         \\
        71.00   &  2.85     & 6.55          \\
        62.80   &  2.90     & 14.70         \\
        55.00   &  2.85     & 22.55         \\
        47.10   &  2.85     & 30.45         \\
        38.90   &  2.95     & 38.55         \\
        30.75   &  3.90     & 45.75         \\
        22.10   &  4.95     & 53.35         \\
        13.30   &  5.90     & 61.20         \\
        61.15   &  1.45     & 17.80         \\
        59.50   &  1.45     & 19.45         \\
        \bottomrule
    \end{tabular}
    \caption{Abmessungen des Acrylblocks}
    \label{tab:messungen}
\end{table}
Die Entfernungen der Fehlstellen vom Rand des Acrylblocks von der Seite A werden in Tabelle \ref{tab:messungen} mit $x_A$ bezeichnet.
Anhand des Lochdurchmessers $D$ wird und des Blockdurchmessers $x_\text{Block} = \qty{80.40}{\mm}$ wird der 
Abstand zum Rand des Blocks auf der B-Seite errechnet.
Diese Messungen werden in \ref{tab:schallzeit} den Ergebnissen des A-Scans gegenübergestellt.
Die Laufzeit des Schalls zur Fehlstelle und zurück wird mit $t_A$ und $t_B$ angegeben.
Zur Berechnung der Schallgeschwindigkeiten $c_A$ und $c_B$ wird von dieser Laufzeit \qty{1}{\micro\s} abgezogen,
um die Verzögerung im Wasser und in der Sonde auszugleichen. 
Dieser Wert entsteht durch vergleichen der Ergebnisse von verschiedenen verschiebungskonstanten mit dem Literaturwert der Schallgeschwindigkeit
im Acrylglas.
\begin{table}
    \centering
    \sisetup{table-format = :.0}
    \begin{tabular}[pos]{S S S S S}
        \toprule
        {Nr.}& {$t_A$} & {$t_B$} & {$c_a$} & {$c_b$} \\
        \midrule
        1       &  11    & 41    & 2800          & 2683          \\
        3       &  46    & 11    & 2730          & 2673          \\
        4       &  40    & 17    & 2750          & 2653          \\
        5       &  34    & 22    & 2771          & 2768          \\
        6       &  28    & 29    & 2779          & 2659          \\
        7       &  22    & 34    & 2795          & 2691          \\
        8       &  16    & 39    & 2763          & 2736          \\
        \bottomrule
    \end{tabular}
    \caption{Laufzeiten und Schallgeschwindigkeiten im Acrylblock \textbf{korrigieren!}}
    \label{tab:schallzeit}
\end{table}
Aus den Werten in Tabelle \ref{tab:schallzeit} wird der Mittelwert $\overline{c}$ und die Standardabweichung $\Delta c$ der Messungen mit den bekannten Formeln berechnet.
Es ergibt sich eine Schallgeschwindigkeit von $c = \qty{2730 +- 50}{\meter\per\second}$.

\subsection{Messung der Fehlstellen mit Ultraschall}
Um die Tiefe der Fehlstellen korrekt zu bestimmen muss die Korrektur, die bei der Laufzeit für die Schallgeschwindigkeit aufgestellt wurde
mit einberechnet werden 