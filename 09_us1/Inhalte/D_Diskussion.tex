\section{Diskussion}
\subsection{Schieblehre}
\label{sec:disk_schieb}
Die Schlieblehre ist mit einem angegeben Fehler von $\qty[]{0.05}{\milli\meter}$ an sich hinreichend genau.
Allerdings hat sich die Messung der Lage und Größe der Fehlstellen aufgrund des Teils sehr geringen Durchmessers als schwierig erwiesen, 
da es hier insbesondere bei kleineren Fehlstellen nicht möglich war, die Schieblehre komplett gerade zu halten,
weshalb von einer höheren Abweichung als dem eigentlich angegeben Wert auszugehen ist.
Etwaige Abweichungen von anderen per Ultraschall gemessenen und errechneten Werten lassen sich also hierdurch erklären.

\subsection{Schallgeschwindigkeit in Acryl}
Die Schallgeschwindigkeit in Acryl liegt nach Abschnitt \ref{sec:ausw_schallgeschw} bei einem Wert von $c = \qty{2730 +- 50}{\meter\per\second}$.
Der Mittelwert entspticht nach \cite[]{schall-acryl} exakt dem Literaturwert, die relativen Abweichungen der Standardabweichung betragen
$\frac{|\Delta c - c_\text{Lit}|}{c_\text{Lit}} = \num{1.83} \%$, woraus sich eine sehr hohe Messgenauigkeit ableiten lässt.

\subsection{Fehlstellen des A-Scans}
In Abschnitt \ref{sec:ausw_fehlste} konnte die Tiefe des Blocks zu $x_\text{Scan} = \qty{80.14}{\mm}$ bestimmt werden, 
was näher am mit der Schieblehre gemessenen Wert $x_\text{Block} = \qty{80.40}{\mm}$ liegt als die zuvor gemessenen $x_\text{alt} = \qty{81.5}{\mm}$.
Der neue Wert hat eine relative Abweichung von $\frac{|x_\text{Scan} - x_\text{Block}|}{x_\text{Block}} = \num{0.32} \%$,
was ebenfalls für eine sehr hohe Genauigkeit der Messungen spricht.
Die Genauigkeit der Schieblehre kann die  Schallmessung hier allerdings nicht erreichen.
Dies spiegelt sich auch in den Werten in Tabelle \ref{tab:a-scan} wieder.

\subsection{Auflösungsvermögen}
Die Auflösung der Sonde mit einer Frequenz von $\qty[]{1}{\mega\hertz}$ hat sich in Abschnitt \ref{sec:ausw_aufl} nicht als hinreichend 
genau erwiesen, da hier die beiden Fehlstellen in der Messung nicht von einander getrennt betrachtet werden können.
Die Sonde mit $\qty[]{2}{\mega\hertz}$ erweist sich als exakter, da hier zwei Maxima erkannt werden können.

\subsection{B-Scan des Acrylblocks}
Die Messung mit dem B-Scan in Abschnitt \ref{sec:ausw_B} ermöglicht eine zweidimensionale Darstellung des Acrylblocks. 
Die Reihenfolge der Fehlstellen wird so darstellbar.
Die Abstände der Fehlstellen senkrecht zur Messrichtung können nicht gemessen werden,
weil die Geschwindigkeit mit der die Sonde über den Block bewegt wurde möglicherweise unregelmäßig ist.
Für die Messgenauigkeit ergibt sich kein erkennbarer Effekt.

\subsection{Untersuchung des Brustmodells}
Im Brustmodell konnten die beiden Tumore zwar gescant werden, allerdings sind ihre Bereiche nur grob zu erkennen,
was vermutlich mit der Einstellung des Ultraschallechoskops zusammenhängt.
Wären dessen Parameter anders gewählt worden, wäre möglicherweise eine höhere Auflösung möglich und die Tumore wären besser zu erkennen.
Dennoch konnte festgestellt werden, dass Tumor 1 aufgrund der kleineren Ausschläge die Zyste und Tumor 2 folglich der Tumor aus 
festem Gewebe sein muss.
Es sei angemerkt, dass es aufgrund der Beschaffenheit des Modells mitsamt der Neigung und Flexibilität nicht so einfach ist,
die Sonde glechmäßig auf einer Linie zu bewegen. 
Dadurch lassen sich die \enquote{Schnitte} und augenscheinlichen Neigungen des Untergrundes in den enstehenden Bildern erklären.
% b-scan nicht ganz gleichmäßig bewegt?
% b-scan an Brust wegen Wölbung vlt ungenau