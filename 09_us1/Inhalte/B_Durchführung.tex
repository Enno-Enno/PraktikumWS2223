\section{Aufbau}
Der Versuch wird mit einem Ultraschallechoskop, welches nur im Impulsbetrieb betrieben werden kann, 
je einer Ultraschallsonde der Frequenz $\qty[]{1}{\mega\hertz}$ und $\qty[]{2}{\mega\hertz}$
und einem Computer zur Datenaufnahme- und analyse durchgeführt.
Die Sonden sind empfindlich, weshalb sie vorsichtig behandelt werden sollten.
Der Schalter \texttt{REFLEC./TRANS.} wird auf \texttt{REFLEC.} gestellt, um das Impuls-Echo-Verfahren verwenden zu können.
Das Durchschallungsverfahren wird in diesem Versuch nicht benötigt.
Die Sende- und Empfangsleistung sind während der Durchführung auf einem konstanten Wert eingestellt.
Es gibt einen zu untersuchenden Acrylblock und ein Brustmodell.
Als Kontaktmittel steht beim Acrylblock bidestiliertes Wasser zur Verfügung, beim Brustmodell wird ein Kontaktgel verwendet.



\noindent
Mit dem Programm \enquote{AScan} am Rechner können die Daten angezeigt, ausgewertet und gespeichert werden.
Außerdem kann die Scanart gewählt werden.
Beim A-Scan können die Signale als Funktion der Zeit oder der Eindringtiefe angezeigt werden.
Zur Bestimmung der Tiefe muss die Schallgeschwindigkeit im jeweiligen Medium eibgegeben werden.
Mittels der FFT-Taste könne ndie gemessenen Frequenzen angegeben werden.




\section{Durchführung}
Mit Hilfe einer Schieblehre werden zunächst die Maße des Acrylblocks mitsamt der Lage und Größe der Löcher bestimmt und notiert.
Eine Darstellung des Blocks mit einer Nummerierung der Löcher ist in Abbildung \ref{fig:acrylblock} zu sehen.

\begin{figure}[H]
    \centering
    \includegraphics*[height=3.5cm]{Abbildungen/acrylblock.png}
    \caption*{Darstellung des Acrylblocks \cite[]{man:us1}.}
    \label{fig:acrylblock}
\end{figure}

\noindent
Im ersten Versuchsteil soll die Schallgeschwindigkeit in Acryl bestimmt werden.
Hierzu werden die Schalllaufzeiten für insgesamt sieben Löcher untersucht.
Durch die Anpassungsschicht der Sonde tritt hierbei eine systematische Abweichung auf, die zu bestimmen ist.

