\section[Theorie]{Theorie\footnote{Unter Verwendung der Quellen \cite[]{man:v308} und \cite[]{demtroeder2}.}}

\subsection{Grundlagen}
Bewegen sich elektrische Ladungen, so erzeugen sie Magnetfelder.
Diese Felder können vektoriell durch die magnetischen Feldstärke $\symbf{H}$ quantifiziert werden.
Ein wichtiges Mittel zur Veranschaulichung sind hierbei die Magnetfeldlinien.
Sie verlaufen stets tangential zu den Vektoren, wobei die Dichte der Linien die Stärke des Feldes widerspiegelt.
Wichtig ist hierbei, dass die Feldlinien immer geschlossen verlaufen.
In Abbildung \ref{fig:ger_stromdurchfl_draht} ist exemplarisch das Magnetfeld eines geraden Stromdurchflossenen Drahtes aufgeführt.
\begin{figure}[H]
    \centering
    \includegraphics*[height = 6cm]{./abbildungen/stromdurchfl_gerader_draht.png}
    \caption[]{Magnetfeldlinien eines geraden stromdurchflossenen Drahtes \cite[S. 86]{demtroeder2}.}
    \label{fig:ger_stromdurchfl_draht}
\end{figure}
%
\noindent
Mit Hilfe des Biot-Savart-Gesetzes 
\begin{align}
    \label{eq:biot_savart}
    \symbf{H} = \frac{I}{4 \pi} \oint_\Gamma \frac{\diff{\symbf{s}} \times \symbf{r}}{r^3}
\end{align}
lassen sich die Magnetfelder beliebiger stormdurchflossener Leiterschleifen \Gamma ermitteln.
Die Feldlinien sind dabei senkrecht zum Strom $I$ verlaufende konzentrische Kreise\footnote{Nach Konvetion
bilden die Richtung des Stroms und die Richtung der Feldlinien Rechtsschrauben.}.
\noindent
Eine weitere wichtige Größe ist die magnetische Flussdichte
\begin{align}
    \label{eq:B_mu_H}
    \symbf{B} = \mu_0 \cdot \mu_\text{r} \cdot \symbf{H},
\end{align}
die gilt, wenn die durch Elektronenbewegungen entstehenden magnetischen Momente aufgrund der 
Wärmebewegung statistisch verteilt sind.
Die Vakuum-Permeabilität $\mu_0 = 4 \pi \cdot 10^{-7}$ und die materialabhängige, dimensionslose relative Permeabilität $\mu_\text{r}$
werden dabei häufig zur Permeabilität $\mu = \mu_0 \cdot \mu_\text{r}$ zusammengefasst.

\subsection{Magnetfeld einer Spule}
Mit Hilfe der Gleichungen \eqref{eq:biot_savart} und \eqref{eq:B_mu_H} ergibt sich das Magnetfeld einer Spule mit einer Windung und Radius $R$ zu
\begin{align}
    \symbf{B}(x) = \frac{\mu_0 I}{2} \frac{R^2}{\left(R^2 + x^2\right)^{3/2}} \cdot \hat{\symbf{x}},
\end{align}
wobei x den senkrechten Abstand vom Spulenmittelpunkt beschreibt, vgl. auch Abbildung~\ref{fig:spule_eine_Windung}.
%
\begin{figure}[H]
    \centering
    \includegraphics*[height = 6cm]{./abbildungen/spule_eine_windung.png}
    \caption[]{Skizze zur Spule mit einer Windung \cite{man:v308}.}
    \label{fig:spule_eine_Windung}
\end{figure}
\noindent
Dementsprechend steigt der Betrag des magnetischen Flusses mit der Anzahl der Windungen $n$.
Bei stromdurchflossenen Solenoiden, d.h. langen Spulen, liegt in der Mitte ein homogenes Magnetfeld vor.
Das bedeutet, dass die magnetfeldlinien parallel verlaufen.
Für eine Spule der Länge $l$ mit Windungszahl $n$ ergibt sich in der Spulenmitte eine magnetische Flussdichte von 
\begin{align}
    \label{eq:homogen_lange_spule}
    B = \mu \frac{n}{l} I,
\end{align}
wobei die relative Permeabilität $mu_\text{r}$ ins Gewicht fällt, falls die Spule mit einem Kern ausgestattet ist.
Außerhalb der Spule nimmt das Feld ab und die Feldlinien gehen aus einander, vgl. Abbildung \ref{fig:solenoid}.
\begin{figure}[H]
    \centering
    \includegraphics*[height = 6cm]{./abbildungen/Solenoid.png}
    \caption[]{Feldinienverlauf eines Solenoiden \cite[S. 86]{demtroeder2}.}
    \label{fig:solenoid}
\end{figure}

\noindent
%%%%%%weiter mit Toroidspule

%Hall-Sonde
