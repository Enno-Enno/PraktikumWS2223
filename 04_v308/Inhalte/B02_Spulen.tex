\subsection{Kurze und lange Spule}
\label{sec:kurze_und_lange_Spule}

\begin{table}
    \caption{Kenngrößen der Spulen}
    \label{tab:B02_Kenngroessen_Spulen}
    \begin{tabular}[]{
        c
        S[table-format=3.0]
        S[table-format=2.0]
        S[table-format=1.2]
        S[table-format=1.1]
    }
        \toprule
                    & {Windungszahl $N$} & {Länge $l / \unit{\cm}$} & {Mittlerer Durchmesser $D / \unit{\cm}$} & {Stromstärke $I / \unit{\ampere}$}\\
        \midrule
        lange Spule &       300          &          16              &           4.1                            &            1.0                    \\ 
        kurze Spule &       3400         &          9               &           8.75                           &            0.6                    \\ 
        \bottomrule
    \end{tabular}
\end{table}

In diesem Versuch werden die Magnetfelder einer langen und einer kurzen Spule ausgemessen und in einem x-B-Diagramm aufgestellt.
Bei den Messungen wurde die Longitudinalsonde auf die richtige Höhe mittig von der Spule eingestellt 
und bei der $\qty{50}{\cm}$ Marke auf der Messlatte aufgestellt. % In Durchführung festgeklebte Messlatte vlt erwähnen.
Die Spule wird anschließend so weit wie es geht nach links über die Sonde gezogen und mit einem konstanten Strom betrieben.
Die Position des linken Rands der Spule $x_\text{absolut}$ wird an der Messlatte abgelesen.
Um den Abstand von der Sonde zu der Spule besser darstellen zu können wird die $x$ Achse so gelegt, 
dass $x = 0$ ist, wenn die Sonde in der Mitte der Spule ist.
Für die Transformation gilt also bei der langen Spule $x = x_\text{absolut} - \qty{50}{\cm} + \qty{8}{\cm}$ 

\begin{table}
    \centering
    \caption{Messergebnisse bei der kurzen Spule.}
    
    
    \begin{tabular}[]{
        S[table-format=2.0] 
        S[table-format=2.2] 
        S[table-format=2.0] 
        S[table-format=1.2]}
        \toprule
        {$x / \unit{\centi\meter}$} & 
        {$B / \unit{\milli\tesla}$} & 
        {$x / \unit{\centi\meter}$} & 
        {$B / \unit{\milli\tesla}$}\\
        \midrule
        45 & -19.65   &    62 & -0.80 \\
        46 & -19.25   &    64 & -0.57 \\
        47 & -18.29   &    66 & -0.42 \\
        49 & -14.37   &    68 & -0.32 \\
        50 & -11.87   &    70 & -0.25 \\
        51 & -9.43    &    72 & -0.19 \\
        52 & -7.38    &    74 & -0.15 \\
        53 & -9.43    &    76 & -0.12 \\
        54 & -4.46    &    78 & -0.10 \\
        55 & -3.45    &    80 & -0.08 \\
        56 & -2.69    &    82 & -0.07 \\
        57 & -2.15    &    84 & -0.06 \\
        58 & -1.75    &    86 & -0.05 \\
        59 & -1.38    &    88 & -0.04 \\
        60 & -1.16    &    90 & -0.03 \\
        \bottomrule
          
        
        
        
        
        
        
        
        
        
        
        
        
        
        
\end{tabular}
\end{table}
