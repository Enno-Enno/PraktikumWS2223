\subsection{Kurze und lange Spule}
\label{sec:kurze_und_lange_Spule}
Der Vollständigkeit halber sind die in Abschnitt \ref{sec:durchführung_kurze_lange_spule} genannten Kenngrößen der Spulen sowie die an ihnen
eingestellten Stromstärken in Tabelle \ref{tab:B02_Kenngroessen_Spulen} aufgeführt.
%
\begin{table}[H]
    \centering
    \caption{Die Kenngrößen der Spulen.}
    \label{tab:B02_Kenngroessen_Spulen}
    \begin{tabular}[]{
        c
        S[table-format=4.0]
        S[table-format=2.0]
        S[table-format=1.1]
        S[table-format=1.1]
    }
        \toprule
        {Spule} & {$n$} & {$l / \unit{\cm}$} & {$D / \unit{\cm}$} & {$I / \unit{\ampere}$} \\
        \midrule
        lang &  300 & 16 & 4.1 & 1.0\\ 
        kurz & 3400 &  9 & 8.5 & 0.6\\ 
        \bottomrule
    \end{tabular}
\end{table}
\noindent
Wie in Abschnitt \ref{sec:durchführung_kurze_lange_spule} beschrieben steht die Hall-Sonde bei $x_\text{Sonde} = \qty[]{50}{\cm}$.
Am linken Rand der Spulen $x_\text{abgel}$ die Position wird abgelesen.
Um den Abstand von der Sonde zu der Spule besser darstellen zu können wird die $x$ Achse so gelegt, 
dass $x = 0$ ist, wenn die Sonde in der Mitte der Spule ist.
Es gilt also bei einer Spulenlänge $l$ die Transformation  $x = x_\text{abgel} - x_\text{Sonde} + \frac{l}{2}$. 

\subsubsection{Kurze Spule}
\label{sec:kurze_spule}
Die abgelesenen Positionen $x_\text{abgel}$, die Abstände $x$ zwischen Spulenrand und Sonde 
sowie die gemessenen magnetischen Flussdichten $B$ sind in Tabelle \ref{tab:messergebnisse_kurze_spule} angegeben.

\begin{table}[H]
    \centering
    \caption{Abgeleser Abstand, Distanz und magnetische Flussdichte der kurzen Spule.}
    \label{tab:messergebnisse_kurze_spule}
    \begin{tabular}[]{
        S[table-format=2.0] 
        S[table-format=2.1] 
        S[table-format=3.2] 
        S[table-format=2.0]
        S[table-format=2.1]  
        S[table-format=2.2]}
        \toprule
        {$x_\text{abgel} / \unit{\centi\meter}$} &
        {$x / \unit{\centi\meter}$} & 
        {$B / \unit{\milli\tesla}$} &
        {$x_\text{abgel} / \unit{\centi\meter}$} & 
        {$x / \unit{\centi\meter}$} & 
        {$B / \unit{\milli\tesla}$} \\
        \midrule
        45 & -0.5 & -19.65   &    62 & 16.5 & -0.80 \\
        46 &  0.5 & -19.25   &    64 & 18.5 & -0.57 \\
        47 &  1.5 & -18.29   &    66 & 20.5 & -0.42 \\
        49 &  3.5 & -14.37   &    68 & 22.5 & -0.32 \\
        50 &  4.5 & -11.87   &    70 & 24.5 & -0.25 \\
        51 &  5.5 & -9.43    &    72 & 26.5 & -0.19 \\
        52 &  6.5 & -7.38    &    74 & 28.5 & -0.15 \\
        53 &  7.5 & -5.65    &    76 & 30.5 & -0.12 \\
        54 &  8.5 & -4.46    &    78 & 32.5 & -0.10 \\
        55 &  9.5 & -3.45    &    80 & 34.5 & -0.08 \\
        56 & 10.5 & -2.69    &    82 & 36.5 & -0.07 \\
        57 & 11.5 & -2.15    &    84 & 38.5 & -0.06 \\
        58 & 12.5 & -1.75    &    86 & 40.5 & -0.05 \\
        59 & 13.5 & -1.38    &    88 & 42.5 & -0.04 \\
        60 & 14.5 & -1.16    &    90 & 44.5 & -0.03 \\
        \bottomrule 
    \end{tabular}
\end{table}

\noindent
Werden die Werte aus Tabelle \ref{tab:messergebnisse_kurze_spule} geplottet, ergibt sich Abbildung \ref{fig:plot_kurze_spule}.
Dabei gibt die vertikale Linie den Wechsel zwischen dem Inneren und Äußeren der Spule an.
%
\begin{figure}[H]
    \centering
    \includegraphics{build/B02_kurze_spule.pdf}
    \caption[]{$x$-$B$-Diagramm aus Tabelle \ref{tab:messergebnisse_kurze_spule} der kurzen Spule.}
    \label{fig:plot_kurze_spule}
\end{figure}
%

\subsubsection{Lange Spule}
Analog zum Vorgehen im vorigen Abschnitt \ref{sec:kurze_spule} werden auch hier die Werte für $x_\text{abgel}$, $x$ und $B$
in Tabelle \ref{tab:messergebnisse_lange_spule} aufgelistet und in Abbildung \ref{fig:plot_lange_spule} geplottet.
Die vertikale Linie gibt erneut den Wechsel zwischen Innerem und Äußerem der Spule an.

\begin{table}[H]
    \centering
    \caption[]{Abgeleser Abstand, Distanz und magnetische Flussdichte der langen Spule.}
    \label{tab:messergebnisse_lange_spule}
    \begin{tabular}[]{S[table-format=2.0] S[table-format=1.0] S[table-format=2.2] S[table-format=2.1] S[table-format=2.1] S[table-format=2.2]}
        \toprule
        {$x_\text{abgel} / \unit{\centi\meter}$} &{$x / \unit{\centi\meter}$} & {$B / \unit{\milli\tesla}$} & {$x_\text{abgel} / \unit{\centi\meter}$} &{$x / \unit{\centi\meter}$} & {$B / \unit{\milli\tesla}$} \\
        \midrule 
        42   &  0   & -2.30 & 50   &  8   & -0.24 \\ 
        43   &  1   & -2.27 & 51   &  9   & -0.13 \\ 
        44   &  2   & -2.21 & 52   & 10   & -0.09 \\ 
        45   &  3   & -2.08 & 53   & 11   & -0.06 \\ 
        46   &  4   & -1.85 & 54   & 12   & -0.04 \\ 
        47   &  5   & -1.42 & 55   & 13   & -0.03 \\ 
        48   &  6   & -0.80 & 57.5 & 15.5 & -0.02 \\ 
        49   &  7   & -0.44 & 60   & 18   & -0.01 \\  
        \bottomrule
    \end{tabular}
\end{table}

%
\begin{figure}[H]
    \centering
    \includegraphics{build/B02_lange_spule.pdf}
    \caption[]{$x$-$B$-Diagramm aus Tabelle \ref{tab:messergebnisse_lange_spule} der langen Spule.}
    \label{fig:plot_lange_spule}
\end{figure}
