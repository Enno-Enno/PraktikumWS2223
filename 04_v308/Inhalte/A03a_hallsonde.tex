\subsection[Hall-Sonde]{Die Hall-Sonde\footnote{Unter Verwendung der Quelle \cite{man:v308}.}}
Hall-Sonden sind Messgerät, die der Bestimmung der Magnetfeldstärke dienen.
Sie haben an ihrer Spitze ein Leiterplättchen, an dem ein Steuerstrom angelegt wird,
der senkrecht zum angelegten Magnetfeld verlaufen soll.
Dafür gibt es longitudinale und transversale Sonden, um je nach Situation messen zu können.
Allgemein gilt hierbei, dass das Leiterplätchen senkrecht zu den magnetischen Feldlinien orientiert sein soll.
Dementsprechend ist darauf zu achten, dass die Sonde möglichst senkrecht zum Feld orientiert ist.
Dadurch, dass Strom und Magnetfeld senkrecht auf einander stehen, wirkt eine Lorentzkraft auf die Ladungen.
Somit entsteht ein elektrischer Verschiebungsstrom.
Folglich wird eine Spannung, die sogenannte Hallspannung, aufgebaut.
Ist der angelegte Steuerstrom konstant, so ist die Hallspannung proportional zu Strom und magnetischer Flussdichte und
dadurch ein Maß für die Stärke des Magnetfeldes.
Dieser Effekt nennt sich auch Hall-Effekt.

