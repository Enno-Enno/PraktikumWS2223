\subsection{Spulenpaar}
In diesem Versuch soll herausgefunden werden welcher Spulenabstand bei zwei gleichen Spulen
am besten dafür geeignet ist ein homogenes Magnetfeld zu erzeugen.
Hierzu werden zwei Spulen mit einer Windungszahl $n= 100$ je Spule, einem mittleren Spulendurchmesser $d = \qty{125}{\mm}$
und einer Spulenbreite von $b = \qty[]{33}{\mm}$ mit einem konstanten Strom durchflossen.
Der Abstand der Spulen wird auf $\qty[]{24}{cm}$, $\qty[]{12}{cm}$ und $\qty[]{6}{cm}$ eingestellt.
In jeder Konfiguration wird das Magnetfeld senkrecht zur Spulenfläche innerhalb und außerhalb der Spulen gemessen.
% Gucken 

\subsubsection{Messungen zwischen den Spulen}
Zwischen den Spulen wird soll ein homogenes Magnetfeld erzeugt werden.
%Die Formel \textbf{Referenz zu Helmholtz-Formel} ergibt das Magnetfeld, dass in einer Helmholtz-Spule erwatet wird
Um die Homogenität des Magnetfelds zu bewerten wird der Mittelwert $\overline{B}$ der Messdaten gebildet.
Die Funktion $f(x) = \overline{B}$  wird als linearer Fit für die Messwerte dargestellt.


\input{Inhalte/B03a_Tabellen}


Die Standardabweichung der Messwerte wirkt so als Abschätzungsgröße der Homogenität des Magnetfeldes.
Der Mittelwert von $N$ Messwerten wird mit
\begin{align}
    \overline{B} = \frac{1}{N} \sum_{i=1}^{N} B_{i}
\end{align} 
berechnet.
Die Standardabweichung dieser Messwerte berechnet sich mit
\begin{align}
    \Delta B = \sqrt{\frac{1}{N-1} \sum_{i=1}{N}\left(B_{i}- \overline{B}\right)^2}
\end{align}

Die Standardabweichung ist nur ein gutes Maß für die Homogenität des Magnetfeldes, 
wenn die Orte der Messungen gleichmäßig zwischen den Spulen verteilt sind bzw. einen hinreichend großen Bereich abdecken.
Wenn das nicht vorausgesetzt wird könnte man zum Beispiel die Messungen außerhalb der Spulen in den ersten vier Millimetern
alle Messwerte aufnehmen und aus einer geringen Standardabweichung auf ein homogenes Feld außerhalb der Spulen schließen,
auch wenn da ein schwächerwerdenes Feld wirkt.


In Abbildung \ref{fig:B03_innen} werden die gemessenen Magnetfelder nach $x$ graphisch dargestellt.
Mit $x$ wird dabei der Abstand des Messgerätes von der Mitte der der linken Spule bezeichnet.
\begin{figure}
    \centering
    \caption{Die Magnetfelder innerhalb der Spulen}
    \label{fig:B03_innen}
    \includegraphics[width=\textwidth]{build/B03_innen_Spulenpaar.pdf}
\end{figure}
