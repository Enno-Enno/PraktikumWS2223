\section{Durchführung}
Es finden insgesamt drei Versuchsteile statt. 
Im ersten Teil wird die Hysteresekurve eines Eisenkerns in einer Toroidspule bestimmt.
Beim zweiten Teil wird das Magnetfeld von einer kurzen und einer langen Spule in Abhängigkeit des Abstands untersucht.
Im letzten Teil werden für unterschiedliche Spulenabstände das Magnetfeld zwischen und hinter einem Spulenpaar in Abhängigkeit des Abstands ermittelt.
Hierfür werden zunächst alle nötigen Kenngrößen der einzelnen Spulen bestimmt bzw. mit den Angaben in der Anleitung \cite[]{man:v308} abgeglichen.
Radius und Länge der Spulen werden dabei mit einem Maßband gemessen,
Windungszahl und maximal zulässige Stromstärke werden den Beschriftungen der Spule und der Anleitung entnommen.
Des Weiteren stehen je nach Versuchsteil verschiedene Spannungstypen und Hallsonden zur Verfügung.
Vor jedem Versuch werden die Experimente im spannungslosen Zustand aufgebaut.
Beim Einschalten der Spannungsgeräte ist darauf zu achten, dass sowohl Strom $I$ als auch Spannung $U$ auf 0 geregelt sind.
Beim Einstellen ist auf die maximal zulässige stromstärke zu achten.

\subsection[Die Hall-Sonde]{Die Hall-Sonde\footnote{Unter Verwendung der Quelle \cite{man:v308}.}}
Hall-Sonden sind Messgerät, die der Bestimmung der Magnetfeldstärke dienen.
Sie haben an ihrer Spitze ein Leiterplättchen, an dem ein Steuerstrom angelegt wird,
der senkrecht zum angelegten Magnetfeld verlaufen soll.
Dafür gibt es longitudinale und transversale Sonden, um je nach Situation messen zu können.
Allgemein gilt hierbei, dass das Leiterplätchen senkrecht zu den magnetischen Feldlinien orientiert sein soll.
Dadurch, dass Strom und Magnetfeld senkrecht auf einander stehen, wirkt eine Lorentzkraft auf die Ladungen.
Somit entsteht ein elektrischer Verschiebungsstrom.
Folglich wird eine Spannung, die sogenannte Hallspannung, aufgebaut.
Ist der angelegte Steuerstrom konstant, so ist die Hallspannung proportional zu Strom und magnetischer Flussdichte und
dadurch ein Maß für die Stärke des Magnetfeldes.
Dieser Effekt nennt sich auch Hall-Effekt.



\subsection{Hysteresekurve}
\label{sec:A_Hysterese_Drcf}
In dieser Versuchsreihe wird die Hysteresekurve mitsamt der Neukurve eines Eisenkerns in einer Toroidspule 
in Abhängigkeit der Stromstärke $I$ ermittelt.
Anhand der gemessenen Daten sollen Sättigungsmagnetisierung, Remanenz und Koerzitivkraft der Hysteresekurve bestimmt werden.
Außerdem sind differentielle Permeabilität für $H = 0$ und Sättigungswert der Neukruve zu ermitteln.

\noindent
Die Ringspule hat einen Luftspalt, damit an dieser Stelle eine transversale Hall-Sonde das Magnetfeld messen kann.
Der Anleitung wird eine Windungszahl von $n = 595$ und eine Breite des Luftspaltes von \qty[]{3}{\mm} entnommen.
Der Aufbau der Toroidspule ist weitestgehend vorbereitet, es muss nur das Spannungsgerät mit 2 Kabeln angeschlossen und die Hall-Sonde orientiert werden.
Da der Eisenkern noch eine gewisse Restmagnetisierung von vorherigen Versuchen haben kann,
wird zunächst eine Wechselspannungsquelle angeschlossen und ein starkes Wechselfeld eingestellt, um das Material zu entmagnetisieren.

\noindent
Im nächsten Schritt wird die eigentliche Spannungsquelle an die Spule angeschlossen.
Für eine Spannung $U = \qty[]{0}{\volt}$ und eine Stromstärke $I = \qty[]{0}{\ampere}$ wird der Wert der magnetischen Flussdichte abgelesen, 
die auch nach der Entmagnetisierung noch vorhanden ist.
Daraufhin wird die Stromstärke $I$ schrittweise um \qty[]{0.5}{\ampere} erhöht bis bei einem Wert von \qty[]{10}{\ampere} die Sättigung erreicht wird.
Dabei ist darauf zu achten, dass die Stromstärke nur in eine Richtung erhöht wird, um die Ergebnisse nicht zu verfälschen.
Falls also beim Erhöhen der Werte $I$ etwas zu groß eingestellt wird, darf es nicht niedriger geregelt werden.
Für jede eingestellte Stromstärke $I$ wird die magnetische Flussdichte $B$ in Abhängigkeit von $I$ notiert.
Mit Hilfe der hier gemessenen Daten können Neukurve sowie alle geforderten zugehörigen Größen ermittelt werden.

\noindent
Um die Kenngrößen der Hysteresekurve zu bestimmen, wird mit der gleichen Schrittweite $I$ stückweise bis zu \qty[]{0}{\ampere} gesenkt.
Anschließend wird umgepolt, wobei vor und nach dem Umpolen die magnetische Flussdichte für \qty[]{0}{\ampere} notiert wird.
Nach dem Umpolen wird erneut die Stromstärke bis \qty[]{10}{\ampere} erhöht und die eingestellten Werte von $I$ werden beim Notieren
mit einem negativen Vorzeichen versehen.
Anhand dieser Daten kann der obere Teil der Hysteresekurve bestimmt werden.
Der Vorgang wird wiederholt, damit auch der untere Teil der Kurve ermittelt werden kann.
Ein Foto des Versuchsaufbaus ist in Abbildung \ref{fig:hysterese_aufbau} zu sehen.


\begin{figure}[H]
    \centering
    \includegraphics[]{abbildungen/toroidspule mit kern.png}
    \caption{Versuchsaufbau zur Bestimmung der Hysteresekurve \cite[]{man:v308}.}
    \label{fig:hysterese_aufbau}
\end{figure}

%Um die magnetische Flußdichte eines Toroides mit Eisenkern zu messen, muß ein Luftspalt eingefugt werden.



