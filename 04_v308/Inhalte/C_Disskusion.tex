\section{Diskussion}


\subsection{Fehlerabschätzung}
Allgemein muss angemerkt werden, dass es an den Messgeräten keinerlei Angaben von Messfehlern gibt,
sodass diese nicht in die Fehlerabschätzung mit eingehen können.

\noindent
Mögliche systematische Fehler in den Messungen können durch das Eisen im Beton des Gebäudes auftreten.
Außerdem kann durch eine schiefe Anbringung der Hall-Sonden bzw. ein ungewolltes Verdrehen während der Messungen
ein weiterer systematischer Fehler entstehen. 


\subsection{Hysteresekurve}
Die Hysteresekurve kann bei der darstallung im B-I Diagramm wiedererkannt werden.
Die Remanenz und die Koerzitivkraft können nachgewiesen werden.
Die geringe Anzahl an Messwerten und Umpolungseffekte machen es allerdings schwierig diese Werte genau zu bestimmen.
Systematische Unsicherheit gibt es auch, weil der Eisenkern nicht vollständig entmagnetisiert werden konnte.
Ein Effekt, der die Messung systematisch stören könnte ist der technisch notwendige Luftspalt von 3mm dicke im Eisenkern der Spule.
Durch ihn werden die Weißschen Bezirke voneinander ein Stück weit entkoppelt und die Sättigungsmagnetisierung wird möglicherweise geschwächt.
Mit einem Polynom 5. grades können die Messwerte innerhalb des Messintervalls ausreichend gut angenähert werden.
Außerhalb des Messbereichs ist allerdings keine Aussage zu dem Zusammenhang des Polynoms und den zu erwartenden Messwerten möglich.
Ob die Sättigungsmagnetisierung im Messintervall tatsächlich erreicht wurde kann zum Beispiel nicht gesagt werden.
Der ermittelte Wert der Sättigungsmagnetisierung ist also nur die untere Schranke des tatsächlichen Wertes.



%Breite des luftspalt -> Randeffekte?
%umpolung diskrepanz, vorhandener offset

\subsection{Lange und kurze Spule}
Die beiden Plots \ref{fig:plot_kurze_spule} und \ref{fig:plot_lange_spule} unterstützen die die Tatsache, 
dass sich die Feldlinien außerhalb der Spule auffächern.
Allerdings gibt es große Diskrepanzen zur Formel \eqref{eq:spule_n_windungen}.
Ganz explizit sind die theoretischen Werte betragsmäßig größer als die experimentellen Werte.
Diese Unterschiede lassen sich dadurch erklären, dass in der Theorie davon ausgegangen wird, dass sich die Windungen an der gleichen Stelle befinden.
Aufgrund der räumlichen Ausdehnung der beiden Spulen ist dies allerdings nicht gegeben und fällt besonders bei der langen Spule ins Gewicht.
Am Plot \ref{fig:plot_kurze_spule} lässt sich durch die Steigung ebenfalls erkennen, dass im Inneren der kurzen Spule kein homogenes Feld vorhanden ist.
Anhand von Plot \ref{fig:plot_lange_spule} kann man nur für Abstände nahe der Spulenmitte von einem homogenen Magnetfeld ausgehen.
Für Abstände, die weiter von der Mitte entfernt sind, muss man von Verlusten durch Randeffekte ausgehen, die bei steigendem Abstand zunehmen.


\subsection{Spulenpaar}
Bei dem Spulenpaar wird erwartet, dass die Anordnung des Helmholtzspulenpaars ein maximal homogenes Magnetfeld im zwischen  den Spulen erzeugt.
In dieser Messung konnte das nur bedingt gezeigt werden.
Die Spulenparameter sind so gewählt, dass nur ein sehr kleiner Messbereich von $\qty{0.3}{cm}$  auszumessen ist.
Die Standardabweichung als maß für die Homogenität kann also nicht zweifelsfrei dieses Magnetfeld als homogen bezeichnen.
Es ist allerdings möglich die Standardabweichung als maß der Inhomogenität zu verwenden. 
Als solche kann gezeigt werden, dass die Konfigurationen mit einem Größeren Abstand als bei der Helmholtzspulenkonfiguration
eine größere Inhomogenität des Feldes haben als das was wir als inhomogenität bei dem Helmholtzspulenpaar messen konnten.

Auch hier gibt es zufällige Messunsicherheiten.
Die Sonde kann in der Fassung gedreht werden, was die Messergebnisse für die in Spulenrichtung verlaufenden Magnetfelder verändern kann.
Außerdem ist die eine Seite der Haltevorrichtung auf  Rollen gelegt, was die genauigkeit weiter einschränkt.

%Parameter der Spule nicht gut gewählt (zu kleiner Radius für Helmholtz...)
%Schiene der Sonde wackelig -> muss festgehalten werden
%Sonde kann beim Verschieben gedreht werden.


