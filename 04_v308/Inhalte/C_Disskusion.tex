\section{Diskussion}


\subsection{Fehlerabschätzung}
Allgemein muss angemerkt werden, dass es an den Messgeräten keinerlei Angaben von Messfehlern gibt,
sodass diese nicht in die Fehlerabschätzung mit eingehen können.

\noindent
Mögliche systematische Fehler in den Messungen können durch das Eisen im Beton des Gebäudes auftreten.


\subsection{Hysteresekurve}
%Breite des luftspalt -> Randeffekte?
%umpolung diskrepanz, vorhandener offset

\subsection{Lange und kurze Spule}
Die beiden Plots \ref{fig:plot_kurze_spule} und \ref{fig:plot_lange_spule} unterstützen die die Tatsache, 
dass sich die Feldlinien außerhalb der Spule auffächern.
Allerdings gibt es große Diskrepanzen zur Formel \eqref{eq:spule_n_windungen}.
Ganz explizit sind die theoretischen Werte betragsmäßig größer als die experimentellen Werte.
Diese Unterschiede lassen sich dadurch erklären, dass in der Theorie davon ausgegangen wird, dass sich die Windungen an der gleichen Stelle befinden.
Aufgrund der räumlichen Ausdehnung der beiden Spulen ist dies allerdings nicht gegeben und fällt besonders bei der langen Spule ins Gewicht.
Am Plot \ref{fig:plot_kurze_spule} lässt sich durch die Steigung ebenfalls erkennen, dass im Inneren der kurzen Spule kein homogenes Feld vorhanden ist.
Anhand von Plot \ref{fig:plot_lange_spule} kann man nur für Abstände nahe der Spulenmitte von einem homogenen Magnetfeld ausgehen.
Für Abstände, die weiter von der Mitte entfernt sind, muss man von Verlusten durch Randeffekte ausgehen, die bei steigendem Abstand zunehmen.


\subsection{Spulenpaar}
%Parameter der Spule nicht gut gewählt (zu kleiner Radius für Helmholtz...)
%Schiene der Sonde wackelig -> muss festgehalten werden
%Sonde kann beim Verschieben gedreht werden.


