\subsection{Hysteresekurve}
In diesem Versuch wird die Magnetisierung eines Eisenkerns betrachtet.
Wie in Abschnitt \ref{sec:A_Hysterese_Drcf} beschrieben wird der Strom nach und nach variiert 
und den Magnetfeldern gegenübergestellt.
Wie in \ref{eq:vereinfachungsgleichung} dargestellt ist die Magnetisierung des eisenkerns nahezu äquivalent zu dem tatsächlich messbaren Magnetfeld.
Das äußere Magnetfeld berechnet sich wie in Gleichung \ref{eq:toroid} mit
\begin{align}
    H = \frac{B}{\mu}
    H = \frac{n}{2 \pi r_\text{T}}. I
\end{align}
Bei der gegebenen Toroidspule beträgt der Radius $r_\text{T}= \qty{13.5}{cm}$ und die Windungszahl $n = 595$
Da $H$ proportional zu $I$ ist wird das Magnetfeld in Tabelle \ref{tab:hysterese_werte}
und in \ref{fig:Hysteresekurve_werte} gegen die Stromstärke aufgestellt.



%Für die Hysteresekurve wird allerdings die Magnetfeldstärke im Vakuum $H$ gebraucht.
%Diese berechnet sich wie in Gleichung \ref{eq:toroid} mit
%\begin{align*}
%    H = \frac{B}{\mu_0}
%    H =  \frac{1}{\mu_0} \frac{n}{2 \pi r_\text{T}} I
%\end{align*}
%Bei der gegebenen Toroidspule beträgt der Radius $r_\text{T}= \qty{13.5}{cm}$ und die Windungszahl $n = 595$
%Das Magnetische Moment berechnet sich analog zu Gleichung \ref{eq:Magnetisierung}
%\begin{align*}
%    M = \frac{B}{\mu_0} - H
%\end{align*}


%%% Tabelle mit Werten


\begin{table}
    \centering
    
    \OverfullCenter{
    \begin{tabular}{
        S[table-format=1.1]
        S[table-format=3.1]
        S[table-format=4.1]
        S[table-format=1.1]
        S[table-format=3.1]
        S[table-format=4.1]
        S[table-format=1.1]
        S[table-format=3.1]
        S[table-format=4.1]
    }
    \toprule
    \multicolumn{3}{c}{Neukruve} &
    \multicolumn{3}{c}{Absinkende Stromstärke} &
    \multicolumn{3}{c}{Steigende Stromstärke} \\
    {$I/ \unit{\ampere}$} & {$B\simeq \mu_0 M/ \unit{\milli\tesla}$} & {$H / (\unit{\ampere \per \meter})$} &   
    {$I/ \unit{\ampere}$} & {$B\simeq \mu_0 M/ \unit{\milli\tesla}$} & {$H / (\unit{\ampere \per \meter})$} &     
    {$I/ \unit{\ampere}$} & {$B\simeq \mu_0 M/ \unit{\milli\tesla}$} & {$H / (\unit{\ampere \per \meter})$}      \\
    \midrule
        0.0  &  21.0   &  0.0     &  9.5  &  717.0  & 6663.9    &    -9.5  &  -684.0 & -6663.9   \\
        0.5  &  55.0   &  350.73  &  9.0  &  708.0  & 6313.1    &    -9.0  &  -676.0 & -6313.1   \\
        1.0  &  134.0  &  701.46  &  8.5  &  698.0  & 5962.4    &    -8.5  &  -667.0 & -5962.4   \\
        1.5  &  225.0  &  1052.2  &  8.0  &  688.0  & 5611.7    &    -8.0  &  -657.0 & -5611.7   \\
        2.0  &  312.0  &  1402.9  &  7.5  &  677.0  & 5261.0    &    -7.5  &  -646.0 & -5261.0   \\
        2.5  &  381.0  &  1753.7  &  7.0  &  665.0  & 4910.2    &    -7.0  &  -635.0 & -4910.2   \\
        3.0  &  433.0  &  2104.4  &  6.5  &  654.0  & 4559.5    &    -6.5  &  -623.0 & -4559.5   \\
        3.5  &  478.0  &  2455.1  &  6.0  &  640.0  & 4208.8    &    -6.0  &  -614.0 & -4208.8   \\
        4.0  &  512.0  &  2805.8  &  5.5  &  625.0  & 3858.0    &    -5.5  &  -596.0 & -3858.0   \\
        4.5  &  542.0  &  3156.6  &  5.0  &  609.0  & 3507.3    &    -5.0  &  -580.0 & -3507.3   \\
        5.0  &  567.0  &  3507.3  &  4.5  &  592.0  & 3156.6    &    -4.5  &  -561.0 & -3156.6   \\
        5.5  &  592.0  &  3858.0  &  4.0  &  573.0  & 2805.8    &    -4.0  &  -543.0 & -2805.8   \\
        6.0  &  613.0  &  4208.8  &  3.5  &  550.0  & 2455.1    &    -3.5  &  -522.0 & -2455.1   \\
        6.5  &  634.0  &  4559.5  &  3.0  &  527.0  & 2104.4    &    -3.0  &  -496.0 & -2104.4   \\
        7.0  &  651.0  &  4910.2  &  2.5  &  496.0  & 1753.7    &    -2.5  &  -467.0 & -1753.7   \\
        7.5  &  666.0  &  5261.0  &  2.0  &  458.0  & 1402.9    &    -2.0  &  -434.0 & -1402.9   \\
        8.0  &  682.0  &  5611.7  &  1.5  &  408.0  & 1052.2    &    -1.5  &  -377.0 & -1052.2   \\
        8.5  &  695.0  &  5962.4  &  1.0  &  323.0  & 701.5     &    -1.0  &  -301.0 & -701.5    \\
        9.0  &  709.0  &  6313.1  &  0.5  &  228.0  & 350.7     &    -0.5  &  -210.0 & -350.7    \\
        9.5  &  720.0  &  6663.9  &  0.0  &  142.0  & 0.0       &     0.0  &  -117.0 & -0.0      \\
        10.0 &  732.0  &  7014.6  &  0.0  &  16.0   & -0.0      &     0.0  &  -21.0  &  0.0      \\
             &         &          & -0.5  &  -5.0   & -350.7    &     0.5  &  -4.0   &  350.7    \\
             &         &          & -1.0  &  -66.0  & -701.5    &     1.0  &  77.0   &  701.5    \\
             &         &          & -1.5  &  -153.0 & -1052.2   &     1.5  &  167.0  &  1052.2   \\
             &         &          & -2.0  &  -230.0 & -1402.9   &     2.0  &  244.0  &  1402.9   \\
             &         &          & -2.5  &  -305.0 & -1753.7   &     2.5  &  321.0  &  1753.7   \\
             &         &          & -3.0  &  -364.0 & -2104.4   &     3.0  &  383.0  &  2104.4   \\
             &         &          & -3.5  &  -413.0 & -2455.1   &     3.5  &  433.0  &  2455.1   \\
             &         &          & -4.0  &  -457.0 & -2805.8   &     4.0  &  477.0  &  2805.8   \\
             &         &          & -4.5  &  -490.0 & -3156.6   &     4.5  &  511.0  &  3156.6   \\
             &         &          & -5.0  &  -519.0 & -3507.3   &     5.0  &  539.0  &  3507.3   \\
             &         &          & -5.5  &  -546.0 & -3858.0   &     5.5  &  566.0  &  3858.0   \\
             &         &          & -6.0  &  -567.0 & -4208.8   &     6.0  &  591.0  &  4208.8   \\
             &         &          & -6.5  &  -589.0 & -4559.5   &     6.5  &  611.0  &  4559.5   \\
             &         &          & -7.0  &  -607.0 & -4910.2   &     7.0  &  627.0  &  4910.2   \\
             &         &          & -7.5  &  -625.0 & -5261.0   &     7.5  &  645.0  &  5261.0   \\
             &         &          & -8.0  &  -640.0 & -5611.7   &     8.0  &  661.0  &  5611.7   \\
             &         &          & -8.5  &  -656.0 & -5962.4   &     8.5  &  676.0  &  5962.4   \\
             &         &          & -9.0  &  -670.0 & -6313.1   &     9.0  &  691.0  &  6313.1   \\
             &         &          & -9.5  &  -684.0 & -6663.9   &     9.5  &  703.0  &  6663.9   \\
             &         &          & -10.0 &  -697.0 & -7014.6   &     10.2 &  720.0  &  7154.9   \\                                   
    \bottomrule
    \end{tabular}
    }
    \caption{Die Messwerte für die Spule mit dem Eisenkern.}
    \label{tab:hysterese_werte}
\end{table}
%%% Graphische Darstellung mit curve-Fit 

Die Remanenz $B_\text{r}$ und die Koerzitivkraft $H_\text{c}$können aus der Tabelle \ref{tab:hysterese_werte} bestimmt werden.
Für jeden dieser Werte gibt es mehrere Datenpunkte, die für die Bestimmung relevant sind.
Bei $I=0$ verändert sich das für die Remanenz relevante Messergebnis beim Umpolen der Stromzufuhr.
Die Koerzitivkraft kann auch nicht immer abgelesen werden, wenn keine Stromeinstellung getroffen wird bei der Magnetfeld genau null ist.
Relevant sind dann die beiden benachbarten Werte.
In Tabelle \ref{tab:B01a_relevant} werden alle relevanten Werte aufgeführt.
\begin{table}
    \centering
    \begin{tabular}[]{
        c
        S[table-format=1.1]
        S[table-format=2.1]
        S[table-format=3.1]
    }
    \toprule
        \multicolumn{3}{c}{Remanenz}\\
        & {$I/ \unit{\ampere}$} &{$B\simeq \mu_0 M/ \unit{\milli\tesla}$} & {$H / (\unit{\ampere \per \meter})$} \\
    \midrule
    Absinkende Stromstärke    & 0.0  &  142.0  & 0.0 \\
                              & 0.0  &  16.0   & 0.0 \\
    Steigende Stromstärke     & 0.0  &  -117.0 & 0.0 \\
                              & 0.0  &  -21.0  & 0.0 \\
    \midrule
        \multicolumn{3}{c}{Koerzitivkraft}\\
        & {$I/ \unit{\ampere}$} &{$B\simeq \mu_0 M/ \unit{\milli\tesla}$} & {$H / (\unit{\ampere \per \meter})$}\\
        \midrule
    Absinkende Stromstärke    &  0.0  &  16.0  &   0.0   \\
                              & -0.5  &  -5.0  &  -350.7 \\
    Steigende Stromstärke     &  0.5  &  -4.0  &  350.7  \\
                              &  1.0  &  77.0  &  701.5  \\
    \bottomrule
    \end{tabular}
    \caption{Werte, die für die Remanenz und die Koerzitivkraft relevant sind.}
    \label{tab:B01a_relevant}
\end{table}

Für die Remanenz wird der Mittelwert von den Beträgen der relevanten Magnetfeldstärken berechnet.
Die  


\begin{figure}
    \includegraphics[]{build/B01_Hysteresekurve.pdf}
    \caption{Die Hysteresekurve, mit Polynom-Fit für die Neukurve.}
    \label{fig:Hysteresekurve_werte}
\end{figure}

