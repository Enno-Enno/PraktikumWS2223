\subsection{Hysteresekurve}
In diesem Versuch wird die Magnetisierung eines Eisenkerns betrachtet.
Wie in Abschnitt \ref{sec:A_Hysterese_Drcf} beschrieben wird der Strom nach und nach variiert 
und den Magnetfeldern gegenübergestellt.
Für die Hysteresekurve wird allerdings die Magnetfeldstärke im Vakuum $H$ gebraucht.
Diese berechnet sich wie in Gleichung \ref{eq:toroid} mit
\begin{align*}
    H =  \mu_0 \frac{n}{2 \pi r_\text{T}} I
\end{align*}
Bei der gegebenen Toroidspule beträgt der Radius $r_\text{T}= \qty{13.5}{cm}$ und die Windungszahl $n = 595$
Das Magnetische Moment berechnet sich analog zu Gleichung \ref{} 


%%% Tabelle mit Werten


%%% Graphische Darstellung mit curve-Fit 
