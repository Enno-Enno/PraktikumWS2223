\section{Diskussion}
In diesem Versuch konnte in der Zählrohrcharakteristik das Plateau der Geiger-Müller_Bereichs identifiziert werden.
Für die Steigung $s$ wurden durch die beiden verschiedenen Messmethoden die Werte $s_\text{P} &= (0.032 \pm 0.004) \, \unit{\per\volt\per\second}$ und $s_\text{L} = (0.045 \pm 0.001) \, \unit{\per\volt\per\second}$.
Die Fehlerbalken dieser Messungen überschneiden sich nicht. 
Das kann mit verschieden systematischen Messfehlern in beiden Messungen zusammenhängen.

Die Totzeit  wurde mit $\tau_\text{calc} = \qty[]{120 +- 60}{\micro\second}$ bzw $\tau_\text{oszi} = \qty[]{95 +- 7}{\micro\second}$ bestimmt.
Die Fehlerbalken dieser beiden Messungen überlappen.
Die Messung mit dem Oszilloskop stellt sich heirbei aber als die genauere Messung heraus.