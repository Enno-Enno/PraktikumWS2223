\input{../header.tex}

\author{Herr Enno Wellmann \and Frau Clara Sondermann} %Ok Frau Clara...

\title{Versuch: Hooke'sches Gesetz}

\begin{document}
\maketitle


\section[short]{Ziel}
In diesem Experiment ermitteln wir die Federkonstante nach dem Hooke'schen Gesetz.


\section{Grundlagen}
Eine Feder, die um $ \Delta x $ aus ihrer Ruhelage $ x_0 $ ausgelenkt wird, erfährt eine Kraft $F$ 
proportional zu $ \Delta x $.
Die Federkonstante $ D $ ist dabei eine eindeutige Kenngröße für die Feder.
\begin{align}
                    F &= D \cdot \Delta x \\
    \Leftrightarrow D &= \frac{F}{\Delta x}
\end{align}


\section{Durchführung}
Wir verwenden das interaktive Bildschirmexperiment der Universität Duisburg-Essen\footnote{\url{http://kallisto.didaktik.physik.uni-due.de/IBEs/Hooke.php}}.
Mit der Maus kann man die Auslenkung $ \Delta x $ der Feder an einem Lineal einstellen. 
Ein angeschlossenes Newtonmeter misst die Kraft, die von der Feder bei der gewählten Auslenkung ausgeht.
Für 10 verschiedene $ \Delta x $ wurde die Kraft $ F $ jeweils einmal gemessen.


\section{Messdaten}
\begin{table}
    \centering
    \caption{Kraft $ F $ nach $ \Delta x $ }
    \sisetup{table-format=1.2}
    \begin{tabular}{S S S}
        \toprule
        {$ \Delta x  [\unit{\cm}]$}  & {$ F [\unit{\newton}]  $}   & {$ D  [\unit{\newton\per\cm}] $} \\
        \midrule
        %0             &  0.00      & {--}   \\
        %ansonsten hätten wir ja 10 Werte 
        5             &  0.15      & 0.0300 \\
        10            &  0.29      & 0.0290 \\
        15            &  0.44      & 0.0293 \\
        20            &  0.59      & 0.0295 \\
        25            &  0.74      & 0.0296 \\
        30            &  0.89      & 0.0297 \\
        35            &  1.04      & 0.0297 \\
        40            &  1.19      & 0.0297 \\
        45            &  1.34      & 0.0298 \\
        50            &  1.49      & 0.0298 \\
        

    \end{tabular}
\end{table}

Anmerkung: Die Messdaten und alle folgenden Ergebnisse sind bis auf die vierte Nachkommastelle gerundet.
Wir erhalten folgende Messdaten:

\section{Auswertung}
\subsection{Mittelwert}
Wir bilden den Mittelwert $\overline{D}$ der ausgerechneten Werte für die Federkonstante.
\begin{align}
    \overline{D} =\frac{1}{10} \cdot \sum_{k=1}^{10} D_k = \frac{1}{10} \cdot \qty{0.2961}{\newton\per\cm} = \qty{0.0296}{\newton\per\cm}
\end{align}

\subsection{Lineare Ausgleichsrechnung}
Nun berechnen wir die lineare Regression der Federkonstanten $\hat{D}$.
%Dafür müssen zunächst die Mittelwerte der Kraft  $\overline{F}$ und der Auslenkung $\overline{\Delta x}$ bestimmt werden.
Dafür müssen zunächst die folgenden Mittelwerte bestimmt werden:

\begin{align}
    \overline{\Delta x} &=\frac{1}{10} \cdot \sum_{k=1}^{10} \Delta x_k = \frac{1}{10} \cdot \qty{275}{\cm} = \qty{27.5}{\cm} \\
    \overline{F} &=\frac{1}{10} \cdot \sum_{k=1}^{10} F_k = \frac{1}{10} \cdot \qty{8.16}{\newton}= \qty{0.816}{\newton} \\
    \overline{\Delta x F} &=\frac{1}{10} \cdot \sum_{k=1}^{10} \left(\Delta x F\right)_k = \frac{1}{10} \cdot \qty{286.05}{\cm\newton}= \qty{28.605}{\cm\newton} \\
    \overline{\left(\Delta x\right)^2} &=\frac{1}{10} \cdot \sum_{k=1}^{10} \left(\Delta x_k\right)^2 = \frac{1}{10} \cdot \qty{9625}{\cm^2}= \qty{962.5}{\cm^2} \\
\end{align}

%Mittelwert D:  0.0296
%Mittelwert x:  27.5000
%Mittelwert F: 0.8160
%Mittelwert xF: 28.6050
%Mittelwert xsqrd: 962.5000
%linreg für D:  0.0299

Mithilfe der Formel für die lineare Regression folgt dann für $\hat{D}$:

\begin{align}
    \hat{D} = \frac{\overline{\Delta x F} - \overline{\Delta x} \cdot \overline{F}}{\overline{\left(\Delta x\right)^2} - {\overline{\Delta x}}^2}
    = \frac{\qty{28.605}{\cm\newton} - \qty{27.5}{\cm} \cdot \qty{0.816}{\newton}}{\qty{962.5}{\cm^2} - \left({\qty{27.5}{\cm}}\right)^2}
    = \qty[]{0.0299}{\newton\per\cm}
\end{align}

\subsection{Diskussion der Ergebnisse}
Vergleicht man die Werte für den Mittelwert $\overline{D}$ und die lineare Ausgleichsrechnung $\hat{D}$, so fällt auf, 
dass diese bis auf die dritte Nachkommastelle übereinstimmen.   
Da die Messwerte nur bis auf die maximal zweite Nachkommastelle gegeben waren,
kann man also von einer hohen Genauigkeit für beide Verfahren reden. 
Etwaige Schwankungen entstehen dabei vermutlich durch Rundungsfehler etc.
Auch graphisch erkennt man sehr gut, wie nah $\overline{D}$ und $\hat{D}$ bei einander liegen:

\begin{figure}
    \centering
        \includegraphics[width=\textwidth]{build/plot.pdf}
    \caption{Graphische Darstellung der Versuchsauswertung.}
    \label{fig:plot}
    \end{figure}
    

\end{document}