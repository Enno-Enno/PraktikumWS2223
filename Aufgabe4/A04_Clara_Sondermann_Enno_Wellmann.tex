\input{../header.tex}

\author{Herr Enno Wellmann \and Frau Clara Sondermann} %Ok Frau Clara...
%\date{}
%^Datum einfügen!
\title{Versuch: Hook'sches Gesetz}

\begin{document}
\maketitle

\section[short]{Ziel}
In diesem Experiment ermitteln wir die Federkonstante nach dem hookeschen Gesetz.

\section{Grundlagen}
Eine Feder, die um $ \Delta x $ aus ihrer Ruhelage $ x_0 $ ausgelenkt wird, erfährt eine Kraft $ F $ 
proportional zu $ \Delta x $.
\begin{align}
    F = D \cdot \Delta x
\end{align}
Die Federkonstante $ D $ ist eine eindeutige Kenngröße für die Feder.

\section{Durchführung}
Wir verwenden das interaktive Bildschirmexperiment der Universität Duisburg-Essen (*Quelle einfügen).
Mit der Maus kann man die Auslenkung $ \Delta x $ der Feder an einem Lineal einstellen. 
Ein angeschlossenes Newtonmeter misst die Kraft, die von der Feder bei der gewählten Auslenkung ausgeht.
Für 10 verschiedene $ \Delta x $ wurde die Kraft $ F $ jeweils einmal gemessen.
\section{Messdaten}
\begin{table}
    \centering
    \caption{Kraft $ F $ nach $ \Delta x $ }
    \sisetup{table-format=1.2}
    \begin{tabular}{S S S}
        \toprule
        {$ \Delta x $}  & {$ F $}   & {$ D $} \\
        \midrule
        0             &  0.00      & {--} \\
        5             &  0.15      & {--} \\
        10            &  0.29      & {--} \\
        15            &  0.44      & {--} \\
        20            &  0.59      & {--} \\
        25            &  0.74      & {--} \\
        30            &  0.89      & {--} \\
        35            &  1.04      & {--} \\
        40            &  1.19      & {--} \\
        45            &  1.34      & {--} \\
        50            &  1.49      & {--} \\

        
    \end{tabular}

\end{table}
\end{document}