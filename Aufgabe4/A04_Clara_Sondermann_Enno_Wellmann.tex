\input{../header.tex}

\author{Herr Enno Wellmann \and Frau Clara Sondermann} %Ok Frau Clara...
%\date{}
%^Datum einfügen!
\title{Versuch: Hook'sches Gesetz}

\begin{document}
\maketitle

\section[short]{Ziel}
In diesem Experiment ermitteln wir die Federkonstante nach dem Hook'schen Gesetz.

\section{Grundlagen}
Eine Feder, die um $ \Delta x $ aus ihrer Ruhelage $ x_0 $ ausgelenkt wird, erfährt eine Kraft $ F $ 
proportional zu $ \Delta x $.
\begin{align}
                    F &= D \cdot \Delta x \\
    \Leftrightarrow D &= \frac{F}{\Delta x}
\end{align}
Die Federkonstante $ D $ ist eine eindeutige Kenngröße für die Feder.
Die Formel nach $ D $ umgestellt lautet $  $

\section{Durchführung}
Wir verwenden das interaktive Bildschirmexperiment der Universität Duisburg-Essen (*Quelle einfügen).
Mit der Maus kann man die Auslenkung $ \Delta x $ der Feder an einem Lineal einstellen. 
Ein angeschlossenes Newtonmeter misst die Kraft, die von der Feder bei der gewählten Auslenkung ausgeht.
Für 10 verschiedene $ \Delta x $ wurde die Kraft $ F $ jeweils einmal gemessen.










\section{Messdaten}
\begin{table}
    \centering
    \caption{Kraft $ F $ nach $ \Delta x $ }
    \sisetup{table-format=1.2}
    \begin{tabular}{S S S}
        \toprule
        {$ \Delta x  [\unit{\cm}]$}  & {$ F [\unit{\newton}]  $}   & {$ D  [\unit{\newton\per\cm}] $} \\
        \midrule
        0             &  0.00      & {--}   \\
        5             &  0.15      & 0.0300 \\
        10            &  0.29      & 0.0290 \\
        15            &  0.44      & 0.0293 \\
        20            &  0.59      & 0.0295 \\
        25            &  0.74      & 0.0296 \\
        30            &  0.89      & 0.0297 \\
        35            &  1.04      & 0.0297 \\
        40            &  1.19      & 0.0297 \\
        45            &  1.34      & 0.0298 \\
        50            &  1.49      & 0.0298 \\
        

    \end{tabular}
\end{table}

Anmerkung: Die Messdaten und alle folgenden Ergebnisse sind bis auf die vierte Nachkommastelle gerundet.

\section{Auswertung}
Wir bilden den Mittelwert der ausgerechneten Werte für $D$
\begin{align}
    \overline{D} =\frac{1}{10} \sum_{k} D_k = \frac{\qty{0.2961}{\newton\per\cm}}{10} = \qty{0.0296}{\newton\per\cm}
\end{align}

\end{document}