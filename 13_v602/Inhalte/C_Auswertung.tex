\section{Auswertung}
\subsection{Fehlerfortpflanzung}
Bei Rechnungen mit Fehlerbehafteten Größen wird der Fehler im Ergebnis mit der Gaußschen Fehlerfortpflanzung berechnet
\begin{align}
    \Delta f(x_i) = \sqrt{
    \left(\frac{\partial f}{\partial x_1} \Delta x_1\right)^2%
     + \left(\frac{\partial f}{\partial x_2} \Delta x_2\right)^2%
     + \dots%
     + \left(\frac{\partial f}{\partial x_k} \Delta x_k\right)^2%
    }.
    \label{eq:gauss}
\end{align}
\subsection{Überprüfung der Bragg-Bedingung}
Um die Bragg-Bedingung zu überprüfen wird ein Kristall bei einem Winkel $\theta = \qty{14}{\degree}$ mit Röntgenstrahlung bestrahlt.
In Abbildung \ref{fig:01} und in Tabelle \ref{tab:c01_bragg} ist ein Peak bei dem Austrittswinkel $\theta= \qty{14.25+-0.05}{\degree}$
zu erkennen. 
Für die Größen der Unsicherheiten wird jeweils in beide Richtungen die Hälfte der Schrittgröße gewählt.
In allen Tabellen wird der Austrittswinkel als der tatsächliche Austrittswinkel $\theta$ dargestellt und nicht als das 
Doppelte dieses Winkels, wie er von der Messmaschine angezeigt wurde.
Der Unterschied zwischen dem Eintrittswinkel und dem Austrittswinkel des Peaks ist also kleiner als \qty{1}{\degree}.
Diese Messung hilft dabei die Winkelmessung des Gerätes zu kalibrieren, die für diesen Versuch hinreichend genau ist.
\begin{table}
    \centering
    \sisetup{table-format=:.1}
    \begin{tabular}{S[table-format=:.0] S S S}
        \toprule
        {$i$} &
        {$U / \unit{\volt}$} &
        {$I/ \unit{\micro\ampere}$} &
        {$\text{grad}_i/ (\unit{\micro\ampere\per\volt})$} \\
        \midrule
        0   & 0.0       & 0.0    & 1.0 \\
        1   & 5.0       & 5.0    & 1.2 \\
        2   & 10.0      & 11.0    & 1.2 \\
        3   & 15.0      & 17.0    & 1.4 \\
        4   & 20.0      & 24.0    & 0.8 \\
        5   & 25.0      & 28.0    & 0.8 \\
        6   & 30.0      & 32.0    & 0.6 \\
        7   & 35.0      & 35.0    & 0.4 \\
        8   & 40.0      & 37.0    & 0.2 \\
        9   & 50.0      & 39.0    & 0.1 \\
        10   & 60.0     & 40.0    & 0.0 \\
        11   & 70.0     & 40.0    & 0.0 \\
        12   & 80.0     & 40.0    & {-} \\
        \bottomrule
    \end{tabular}
    \caption{Messdaten für $I_\text{f}= \qty{2.0}{\ampere}$}
    \label{tab:c01}
\end{table} % \label{tab:c01_bragg}
\begin{figure}
    \centering
    \includegraphics[width=0.76\textwidth]{build/01_plot.pdf}
    \caption{Graph zur Bragg Bedingung}
    \label{fig:01}
\end{figure}

\subsection{Emissionspektrum der Kupfer Röntgenröhre}

In dem groben Emissionsspektrum in Abbildung \ref{fig:02} bzw Tabelle \ref{tab:c02_Cu-Roere_grob} können die
Peaks der $K_\alpha$ und $K_\beta$ Linie bei etwa \qty{19}{\degree} bis
\qty{24}{degree} finden.
In Abbildung \ref{fig:test} wird das Bremsspektrum in einem kleineren Bereich ohne die Peaks der K-kanten dargestellt.
In dieser Darstellung kann der Bremsberg und der Grenzwinkel mit der Maximalen Energie des Spektrums ermittelt werden.
Es ergeben sich die Werte
\begin{align*}
    \theta_\text{Berg}  &= \qty{11.0 \pm 0.05}{\degree} &
    \theta_\text{Grenz} &= \qty{5.4  \pm 0.05}{\degree} 
\end{align*} 
Der Bereich mit den Peaks wird mit der kleineren Schrittweite genauer untersucht.
In Abbildung \ref{fig:03} und Tabelle \ref{tab:c03_Cu-Roere_fein} Peaks bei den Stellen 
\begin{align}
    \theta_\alpha = \qty{22.50 \pm 0.05}{\degree} \text{und}
    \theta_\beta = \qty{ 20.25 \pm 0.05}{\degree}
\end{align}
gefunden.
Diese Winkel lassen sich mit der Formel \eqref{eq:bragg} in Energien umrechnen.
Es gilt
\begin{align*}
    \lambda = 2 d \sin(\theta) & E = h \frac{c}{\lambda}
\end{align*}
% Die Messunsicherheiten der Winkelmessung werden mithilfe der Gaußschen Fehlerfortpflanzung in den Umrechnungen berücksichtigt
% \begin{align}
    % \Delta \lambda(\theta) = \sqrt{
    % \left(\frac{\partial \lambda}{\partial \theta} \Delta \theta \right)^2 }.
    % \label{eq:gauss}
% \end{align}
% $\Delta \lambda(\theta)$ ist hier die Unsicherheit der Wellenlänge in Abhängigkeit von der Unsicherheit des Winkels $\Delta \theta$.
% Analog wird die Unsicherheit von $E$ aus $\Delta \lambda$ berechnet.
Es ergeben sich folgende Werte 
\begin{align*}
    \theta_\alpha:        = \qty{22.50 \pm 0.05}{\degree} & \lambda_\alpha = \qty{154.14 \pm 0.32}{\pico\meter}  & E_\alpha = \qty{8.043 \pm 0.017}{\kilo\eV}  \\
    \theta_\beta:         = \qty{20.25 \pm 0.05}{\degree} & \lambda_\beta  = \qty{139.42 \pm 0.33}{\pico\meter}  & E_\beta  = \qty{8.893 \pm 0.021}{\kilo\eV}   
    \theta_\text{Grenz}:  = \qty{5.4 \pm 0.05}{\degree} & \lambda_\text{Grenz}  = \qty{37.91 \pm 0.35}{\pico\meter}  & E_\text{Grenz}  = \qty{32.71 \pm 0.30}{\kilo\eV}   
\end{align*}
Für die Grenzwellenlänge ist der gemessene Wert $\lambda_\text{Grenz}$ größer als der erwartete Wert von $\lambda_\text{min}= \qty{35.42}{\pico\m}$.

Zusätzlich wird die Halbwertsbreite der Peaks mit den Funktionen \texttt{scipy.signal.find\_peaks}  und \texttt{scipy.signal.peak\_widths} \cite{scipy} ermittelt.
Es ergeben sich für die $K_\alpha$ ergeben sich die Werte
\begin{align*}
    \theta_{1,\text{FWHM}} &= \qty{22.25 \pm 0.05}{\degree} & \theta_{2,\text{FWHM}} = \qty{22.71 \pm 0.05}{\degree} \\
    \Delta \theta &= \qty{0.46 \pm 0.07}{\degree}.
\end{align*}
Für $K_\beta$ ergibt sich
\begin{align*}
    \theta_{1,\text{FWHM}} &= \qty{20.02 \pm 0.05}{\degree} & \theta_{2,\text{FWHM}} = \qty{20.44 \pm 0.05}{\degree} \\
    \Delta \theta &= \qty{0.42 \pm 0.07}{\degree}.
\end{align*}
Daraus kann das Auflösungsvermögen der Messapparatur $A= \frac{\theta_\text{max}}{\Delta \theta}$ berechnen.
Es ergibt
\begin{align*}
    A_\alpha &= \num{49 \pm 8} & A_beta = \num{48 \pm 8}
\end{align*}

\subsubsection{Abschirmkonstanten von Kupfer}
Für die Abschirmkonstanten von Kupfer ergeben sich nach \cite{man:v602} folgende Formeln
\begin{align}
    \sigma_1 = Z_K -\sqrt{\frac{E_\text{K,abs}}{R_\infty}}\\
    \sigma_2 = Z_K -2\sqrt{\frac{E_\text{K,abs}- E_{K,\alpha} }{R_\infty}}\\
    \sigma_3 = Z_K -3\sqrt{\frac{E_\text{K,abs}- E_{K,\alpha} }{R_\infty}}\\
\end{align}
Mit der Rydberg Energie von $R_infty = \qty{13.6}{\eV}$ und der Absorptionskante $E_\text{K,abs} = \qty{8980.5(10)}{\eV}$ \cite{x-ray}
ergeben sich die folgenden Werte für die Absorptionszahlen.
\begin{align}
    \sigma_1 &= \num{3.3031 \pm 0.0014} &
    \sigma_2 &= \num{12.40 \pm 0.15} &
    \sigma_3 &= \num{21.4 \pm 0.9}
\end{align}



\begin{figure}
    \centering
    \includegraphics[width=0.76\textwidth]{build/02_plot.pdf}
    \caption{Spektrum der Kupfer-Röntgenröhre mit Schrittweite von \qty{0.2}{\degree}.}
    \label{fig:02}
\end{figure}
\begin{figure}
    \centering
    \includegraphics[width=0.76\textwidth]{build/test_plot.pdf}
    \caption{Spektrum aus Abbildung \ref{fig:02} ohne den Bereich mit den Peaks.}
    \label{fig:test}
\end{figure}

\begin{figure}
    \centering
    \includegraphics[width=0.76\textwidth]{build/03_plot.pdf}
    \caption{Genaueres Spektrum der Kupfer-Röntgenröhre mit Schrittweite von \qty{0.1}{\degree} und einem eingeschränkten Messbereich.}
    \label{fig:03}
\end{figure}




\subsection{Absorptionsspektren verschiedener Materialien}
Mit den Materialien vor dem Auslesegerät ergeben sich Absorptionsspektren mit einer K-Kante, einem abrupten Abfall
der Messhäufigkeit der Röntgenstrahlung bei Energien über einer Charakteristischen Energie.
Dieser Abfall bezieht sich auf kleiner werdende Winkel für den möglichen Messbereich von $\theta$.
Die K-Kante wird in Abbildung \ref{fig:04} als Mitte der beobachtbaren Flanke berechnet.
Es ergeben sich folgende werte für die Energien und die Wellenlängen der K-Kanten
\begin{align*}
    \text{Zink:       } \theta_k &= \qty{20.05 \pm 0.05}{\degree} & \lambda_k &= \qty{138.10 \pm 0.33}{\pico\meter} & E_k &= \qty{8.978 \pm 0.021}{ \kilo\eV } \\
    \label{eq:absorptionskanten}
    \text{Brom:       } \theta_k &= \qty{13.25 \pm 0.05}{\degree} & \lambda_k &= \qty{92.32 \pm 0.34 }{\pico\meter} & E_k &= \qty{13.43 \pm 0.05 }{ \kilo\eV }\\
    \text{Strontium:  } \theta_k &= \qty{11.05 \pm 0.05}{\degree} & \lambda_k &= \qty{77.20 \pm 0.34 }{\pico\meter} & E_k &= \qty{16.06 \pm 0.07 }{ \kilo\eV }\\
    \text{Zirconium:  } \theta_k &= \qty{10.00 \pm 0.05}{\degree} & \lambda_k &= \qty{69.95 \pm 0.35 }{\pico\meter} & E_k &= \qty{17.73 \pm 0.10}{\kilo\eV}. 
\end{align*}

\noindent
Nach \eqref{eq:abschirmung_k} berechnen sich die Abschirmkonstanten der Materialien zu
\begin{align*}
    \text{Zink      } \sigma_k &= \num{4.517 \pm 0.031} &
    \text{Brom      } \sigma_k &= \num{3.90 \pm 0.06} \\
    \text{Strontium } \sigma_k &= \num{4.04 \pm 0.08} &
    \text{Zirconium } \sigma_k &= \num{4.37 \pm 0.09} 
\end{align*}

\begin{figure}
    \centering
    \includegraphics[width=0.76\textwidth]{build/04_plot.pdf}
    \caption{Absorptionsspektren verschiedener Materialien mit den mittleren K-Kanten in schwarz eingezeichnet.}
    \label{fig:04}
\end{figure}






\begin{table}
    \centering
    \sisetup{table-format=:.1}
    \begin{tabular}{S[table-format=:.0] S S S}
        \toprule
        {$i$} &
        {$U / \unit{\volt}$} &
        {$I/ \unit{\micro\ampere}$} &
        {$\text{grad}_i/ (\unit{\micro\ampere\per\volt})$} \\
        \midrule
        0   & 0.0       & 0.0      & 2.2 \\
        1   & 5.0       & 11.0     & 3.2 \\
        2   & 10.0      & 27.0     & 3.6 \\
        3   & 15.0      & 45.0     & 3.8 \\
        4   & 20.0      & 64.0     & 5.4 \\
        5   & 25.0      & 91.0     & 4.8 \\
        6   & 30.0      & 115.0    & 5.0 \\
        7   & 35.0      & 140.0    & 5.8 \\
        8   & 40.0      & 169.0    & 4.6 \\
        9   & 45.0      & 192.0    & 6.0 \\
        10  & 50.0      & 222.0    & 6.1 \\
        11  & 60.0      & 283.0    & 5.2 \\
        12  & 70.0      & 335.0    & 4.8 \\
        13  & 80.0      & 383.0    & 5.3 \\
        14  & 90.0      & 436.0    & 4.5 \\
        15  & 100.0     & 481.0    & 2.9 \\
        16  & 110.0     & 510.0    & 1.5 \\
        17  & 120.0     & 525.0    & 0.9 \\
        18  & 130.0     & 534.0    & 1.2 \\
        19  & 140.0     & 546.0    & 0.8 \\
        20  & 150.0     & 554.0    & 0.7 \\
        21  & 160.0     & 561.0    & 0.5 \\
        22  & 180.0     & 571.0    & 0.35 \\
        23  & 200.0     & 578.0    & 0.25 \\
        24  & 220.0     & 583.0    & 0.25 \\
        25  & 240.0     & 588.0    & 0.3 \\
        26  & 250.0     & 591.0    & {-} \\
        \bottomrule
    \end{tabular}
    \caption{}
    % \label{}
\end{table} % \label{tab:c02_Cu-Roere_grob}
\begin{table}
    \centering
    \sisetup{table-format=:.1}
    \begin{tabular}{S[table-format=:.0] S S S S[table-format=:.0] S S S}
        \toprule
        {$i$} &
        {$U / \unit{\volt}$} &
        {$I/ \unit{\micro\ampere}$} &
        {$\text{grad}_i/ (\unit{\micro\ampere\per\volt})$} &
        {$i$} &
        {$U / \unit{\volt}$} &
        {$I/ \unit{\micro\ampere}$} &
        {$\text{grad}_i/ (\unit{\micro\ampere\per\volt})$} \\
        \midrule
        0    & 0.0      & 0.0       & 2.8   &        15   & 100.0    & 750.0     & 11.1   \\
        1    & 5.0      & 14.0      & 3.2   &        16   & 110.0    & 861.0     & 12.5   \\
        2    & 10.0     & 30.0      & 3.8   &        17   & 120.0    & 986.0     & 13.0   \\
        3    & 15.0     & 49.0      & 4.6   &        18   & 130.0    & 1116.0    & 9.9    \\
        4    & 20.0     & 72.0      & 6.2   &        19   & 140.0    & 1215.0    & 9.1    \\
        5    & 25.0     & 103.0     & 5.6   &        20   & 150.0    & 1306.0    & 9.4    \\
        6    & 30.0     & 131.0     & 8.0   &        21   & 160.0    & 1400.0    & 8.8    \\
        7    & 35.0     & 171.0     & 7.2   &        22   & 170.0    & 1488.0    & 8.6    \\
        8    & 40.0     & 207.0     & 6.8   &        23   & 180.0    & 1574.0    & 8.7    \\
        9    & 45.0     & 241.0     & 8.2   &        24   & 190.0    & 1661.0    & 9.0    \\
        10   & 50.0     & 282.0     & 8.9   &        25   & 200.0    & 1751.0    & 6.9    \\
        11   & 60.0     & 371.0     & 10.8  &        26   & 210.0    & 1820.0    & 7.0    \\
        12   & 70.0     & 479.0     & 8.8   &        27   & 220.0    & 1890.0    & 7.1    \\
        13   & 80.0     & 567.0     & 11.0  &        28   & 230.0    & 1961.0    & {-}    \\
        14   & 90.0     & 677.0     & 7.3   & & & & \\

        \bottomrule
    \end{tabular}
    \caption{Messdaten für $I_\text{f}= \qty{2.5}{\ampere}$}
    \label{tab:c03}
\end{table} % \label{tab:c03_Cu-Roere_fein}
\begin{table}
    \centering
    \begin{tabular}{S S}
        {$U_A / \unit{\volt}$}{$I/\unit{\ampere}$}\\
        -0.0067   & 6.70 \\
        -0.0451   & 5.10 \\
        -0.0840   & 4.00 \\
        -0.1231   & 3.10 \\
        -0.1624   & 2.40 \\
        -0.2017   & 1.70 \\
        -0.2413   & 1.30 \\
        -0.2809   & 0.90 \\
        -0.3206   & 0.60 \\
        -0.3603   & 0.30 \\
        -0.4007   & 0.75 \\
        -0.4406   & 0.58 \\
        -0.4804   & 0.45 \\
        -0.5203   & 0.32 \\
        -0.5602   & 0.24 \\
        -0.6001   & 0.15 \\
        -0.7000   & 0.03 \\        
    \end{tabular}
    \caption{Messwerte für den Anlaufstrom}
    \label{tab:c04}
\end{table} % \label{tab:c04_absorption}