\section{Diskussion}

\subsection{Die Bragg-Bedingung}
Die Bragg-Bedingung konnte verifiziert werden.
Der eingestellte Winkel von \qty[]{14}{\degree} liegt hinreichend nahe am experimentell bestimmten Wert von \qty{14.25+-0.05}{\degree}
und hat somit eine Abweichung (\num[]{1.8 +- 0.4}) \,\%.
Folglich ist davon auszugehen, dass ein möglicher systematischer Fehler, der die gemessenen Peaks/ Spektren verzerrt, nicht von Bedeutung ist.


\subsection{Das Emissionsplektrum der Röhre}
Die aus den Winkeln $\theta_\alpha = \qty{22.50 \pm 0.05}{\degree}$ und $\theta_\beta = \qty{20.25 \pm 0.05}{\degree}$
 resultierenden Energien $E_\alpha = \qty{8.043 \pm 0.017}{\kilo\eV}$ und $E_\beta  = \qty{8.893 \pm 0.021}{\kilo\eV}$
liegen sehr nahe an den Literaturwerten von $E_{\alpha,Lit} = \qty[]{8.04}{\kilo\electronvolt}$ und
$E_{\beta,Lit} = \qty[]{8.94}{\kilo\electronvolt}$.
Die Abweichungen betragen hier (\num[]{0.04 +- 0.21}) \,\% bzw. (\num[]{0.53 +- 0.23}) \,\%.
% Kupfer_\beta: (20.25+/-0.05)° & (139.42+/-0.33) pm & 8.893+/-0.021 keV \\ 
% Kupfer_\alpha: (22.50+/-0.05)° & (154.14+/-0.32) pm & 8.043+/-0.017 keV \\ 



\subsection{Die Absorptionsspektren}
Für die gemessenen Absorptionskanten ergeben sich die Absorptionskanten in den Gleichungen \eqref{eq:absorptionskanten}.
Verglichen mit den theoretischen Werten aus Tabelle \ref{tab:vorbereitung} ergeben sich Abweichungen von (\num[]{7.06 +- 0.22}) \,\% für Zink,
(\num[]{0.3 +- 0.4}) \,\% für Brom, (\num[]{0.2 +- 0.4}) \,\% für Strontium sowie (\num[]{3.1 +- 0.6}) \,\% für Zirconium.
Es ist deutlich erkennbar, dass auch diese Messungen sehr exakte Ergebnisse aufweisen.


\subsection{Fazit}% (Systematischer) Fehler bei Absorption von Zink
Es ist von Messungen ohne singifikante systematische Fehler auszugehen, da insbesondere die Bragg-Bedingung verifiziert werden konnte
und auch die Ergebnisse insgesamt nur wenig von den theoretischen Werten abweichen.
Wie zu erwarten liegen statistische Schwankungen innerhalb der Messungen vor.
Da diese sich allerdings im Schnitt allerdings wegheben, ist insgesamt von guten Messungen und Messergebnissen auszugehen.
% Zink: (20.05+/-0.05)° & (138.10+/-0.33) pm & 8.978+/-0.021 keV \\
% Brom: (13.25+/-0.05)° & (92.32+/-0.34) pm & 13.43+/-0.05 keV \\
% Strontium: (11.05+/-0.05)° & (77.20+/-0.34) pm & 16.06+/-0.07 keV \\
% Zirconium: (9.55+/-0.05)° & (66.83+/-0.35) pm & 18.55+/-0.10 keV \\