\section{Diskussion}
\subsection{Reflexion}
In Abschnitt \ref{sec:ausw_reflexion} wurde das Reflexionsgesetz betrachtet.
Nach der theoretischen Erwartung müsste die Gerade eine mit der Steigung $m = 1$ und dem y-Achsenabschnitt $b=0$ sein.
Die gemessenen Werte weichen leicht über ihre Standartabweichung hinaus davon ab.
Das kann mit systematischen messfehlern wie einer verschobenen Schablone zusammenhängen.

\subsection{Brechung}
Der in Abschnitt \ref{sec:ausw_brechung} bestimmte Brechungsindex stimmt nahezu vollständig mit dem Literaturwert überein.
Bei den in Abschnitt \ref{sec:ausw_strahlversatz} verwendeten Berechnungsverfahren gibt es für die Variante $s_2$ kleinere Fehlerabschätzungen.
Dementsprechend ist die Variante, bei der die Brechungswinkel über das Snellius-gesetz bestimmt werden die genauere.7

\subsection{Prisma}
Für größere Eintrittswinkel $\alpha$ kann in Abschnitt \ref{sec:ausw_prisma} keine Abweichung zwischen rotem und grünem Licht festgestellt werden.
Wenn der Eintrittswinkel kleiner wird wird die Abweichung allerdings deutlich erkennbar, sodass die Messunsicherheiten
für die beiden Laser sich bei \qty{26}{\degree} nicht mehr überschneiden.
Die unterschiedlichen Laserfarben, werden also unterschiedlich stark abgelenkt.

\subsection{Beugung}
Mithilfe eines Gitters konnten in Abschnitt \ref{sec:ausw_beugung} die Wellenlängen bis auf eine kleine Abweichung korrekt gemessen werden.
Dabei wurde jeweils ein leicht höherer Wert ermittelt, als in der Literatur angegeben wurde.
Das weißt auf einen systematischen Messfehler hin, der sich jetzt aber nicht rekonstruieren lässt.