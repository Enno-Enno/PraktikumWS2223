\section{Auswertung}

\subsection{Mittelwerte und Fehler}
Das arithmetische Mittel $\overline{c}$ und die Standardabweichung $\Delta c$ einer Messreihe mit $N$ Werten $c_k$ errechnet sich gemäß der Formeln
\begin{align}
    \overline{c} &= \frac{1}{N} \sum_{k=1}^{N} c_k, & \Delta c = \sqrt{ \sum_{k=1}^{N} \left(\overline{c} - c_k \right)^2 }.
    \label{eq:mittelstand}
\end{align}

\subsubsection{Gaußsche Fehlerfortpflanzung}
Wenn zu Messdaten die Standardabweichung bekannt ist, und mit diesen Messdaten weiter gerechnet werden soll,
wird die Gaußsche Fehlerfortpflanzung verwendet. 
Angenommen, es gibt $k$ Messwerte $x_i [i \in \mathbb{N}, i \leq k]$ mit den Standardabweichungen $\Delta x_i$
und eine abgeleitete Größe $f(x_i)$.
Dann ist der Fehler von $f$
\begin{align}
    \Delta f(x_i) = \sqrt{
    \left(\frac{\partial f}{\partial x_1} \Delta x_1\right)^2%
     + \left(\frac{\partial f}{\partial x_2} \Delta x_2\right)^2%
     + \dots%
     + \left(\frac{\partial f}{\partial x_k} \Delta x_k\right)^2%
    }.
    \label{eq:gauss}
\end{align} 
Im Ergebnis ergibt sich der Mittelwert von $f$ mit der errechneten Abweichung $\overline{f} \pm \Delta f $.
Um Rechenfehler zu vermeiden, wird das Python \cite[]{python} Paket \texttt{uncertainties} \cite[][]{uncertainties} verwendet.
Hier wird die Fehlerfortpflanzung automatisch verrechnet, wenn die Variablen als \texttt{ufloat} definiert werden.







\subsection{Reflexionsgesetz}
Zur Überprüfung des Reflexionsgesetzes werden mit dem grünen Laser für insgesamt 7 Einfallswinkel $\alpha_1$
die resultierenden Ausfallswinkel $\alpha_2$ gemessen, die durch Reflexion am Spiegel entstehen.
Die gemessenen Winkel sind in Tabelle \ref{tab:reflexion} zu sehen.
Die Messungenauigkeit der Winkel wird dabei auf $\Delta \alpha = \qty[]{1}{\degree}$ geschätzt.

\begin{table}[H]
    \centering
    \caption[short]{Einfallswinkel $\alpha_1$ und Ausfallswinkel $\alpha_2$ bei der Reflexion am Spiegel.}
    \label{tab:reflexion}
    \sisetup{table-format=2.0}
    \begin{tabular}{S S}
        \toprule
        {$\alpha_1 / \unit[]{\degree}$} & {$\alpha_2 / \unit[]{\degree}$} \\
        \midrule
        %\cmidrule(lr){1-2}\cmidrule(lr){3-4}
        15 & 14 \\
        20 & 20 \\
        30 & 31 \\
        40 & 40 \\
        50 & 51 \\
        60 & 61 \\
        70 & 71 \\
        \bottomrule
    \end{tabular}
\end{table}
% # einfall ausfall
% 15 14
% 20 20
% 30 31
% 40 40
% 50 51
% 60 61
% 70 71

\noindent
Gemäß der Formel \textbf{REF! EINFALL=AUSFALL} wird eine Ausgleichsrechung der Form
\begin{align}
    \alpha_2 = m \cdot \alpha_2 + b
\end{align}
mit Hilfe der \texttt{scipy} \cite[]{scipy} Funktion \texttt{curve\_fit} durchgeführt.
Es ergeben sich die Parameter
\begin{align}
    m &= \num[]{1.029 +- 0.011}, & b &= (\num[]{-0.75 +- 0.51}) \, \unit[]{\degree},
\end{align}
wobei die Unsicherheiten der Kovarianzmatrix entnommen werden.
% # m = 1.02889518 +- 0.01129740
% # b = -0.74787535 +- 0.50748549
Die Messdaten inklusive Fehlerbalken sowie ihre Ausgleichsgerade sind in Abbildung \ref{fig:reflexion} zu sehen.

\begin{figure}[H]
    \centering
    \includegraphics[height=8cm]{build/c01_reflexion.pdf}
    \caption[]{Plot der Einfallswinkel $\alpha_1$ und Ausfallswinkel $\alpha_2$.}
    \label{fig:reflexion}
\end{figure}






\subsection{Brechungsgesetz}
In diesem Versuchsteil wird zur Überprüfung des Brechungsgesetzes das Licht des grünen Lasers auf eine planparallele Platte geschickt.
Für insgesamt 7 Einfallswinkel $\alpha$ wird der Brechungswinkel $\beta$ bestimmt.
Die Skala zur Bestimmung von $\beta$ hat eine Genauigkeit von $\qty[]{0.5}{\degree}$.