\section{Auswertung}

\subsection{Mittelwerte und Fehler}
Das arithmetische Mittel $\overline{c}$ und die Standardabweichung $\Delta c$ einer Messreihe mit $N$ Werten $c_k$ errechnet sich gemäß der Formeln
\begin{align}
    \overline{c} &= \frac{1}{N} \sum_{k=1}^{N} c_k, & \Delta c = \sqrt{ \sum_{k=1}^{N} \left(\overline{c} - c_k \right)^2 }.
    \label{eq:mittelstand}
\end{align}

\subsubsection{Gaußsche Fehlerfortpflanzung}
Wenn zu Messdaten die Standardabweichung bekannt ist, und mit diesen Messdaten weiter gerechnet werden soll,
wird die Gaußsche Fehlerfortpflanzung verwendet. 
Angenommen, es gibt $k$ Messwerte $x_i [i \in \mathbb{N}, i \leq k]$ mit den Standardabweichungen $\Delta x_i$
und eine abgeleitete Größe $f(x_i)$.
Dann ist der Fehler von $f$
\begin{align}
    \Delta f(x_i) = \sqrt{
    \left(\frac{\partial f}{\partial x_1} \Delta x_1\right)^2%
     + \left(\frac{\partial f}{\partial x_2} \Delta x_2\right)^2%
     + \dots%
     + \left(\frac{\partial f}{\partial x_k} \Delta x_k\right)^2%
    }.
    \label{eq:gauss}
\end{align} 
Im Ergebnis ergibt sich der Mittelwert von $f$ mit der errechneten Abweichung $\overline{f} \pm \Delta f $.
Um Rechenfehler zu vermeiden, wird das Python \cite[]{python} Paket \texttt{uncertainties} \cite[][]{uncertainties} verwendet.
Hier wird die Fehlerfortpflanzung automatisch verrechnet, wenn die Variablen als \texttt{ufloat} definiert werden.







\subsection{Reflexionsgesetz}
\label{sec:ausw_reflexion}
Zur Überprüfung des Reflexionsgesetzes werden mit dem grünen Laser für insgesamt 7 Einfallswinkel $\alpha_1$
die resultierenden Ausfallswinkel $\alpha_2$ gemessen, die durch Reflexion am Spiegel entstehen.
Die gemessenen Winkel sind in Tabelle \ref{tab:reflexion} zu sehen.
Die Messungenauigkeit der Winkel wird dabei auf $\Delta \alpha = \qty[]{1}{\degree}$ geschätzt.

\begin{table}[H]
    \centering
    \caption[short]{Einfallswinkel $\alpha_1$ und Ausfallswinkel $\alpha_2$ bei der Reflexion am Spiegel.}
    \label{tab:reflexion}
    \sisetup{table-format=2.0}
    \begin{tabular}{S S}
        \toprule
        {$\alpha_1 / \unit[]{\degree}$} & {$\alpha_2 / \unit[]{\degree}$} \\
        \midrule
        %\cmidrule(lr){1-2}\cmidrule(lr){3-4}
        15 & 14 \\
        20 & 20 \\
        30 & 31 \\
        40 & 40 \\
        50 & 51 \\
        60 & 61 \\
        70 & 71 \\
        \bottomrule
    \end{tabular}
\end{table}
% # einfall ausfall
% 15 14
% 20 20
% 30 31
% 40 40
% 50 51
% 60 61
% 70 71

\noindent
Gemäß der Formel \eqref{eq:reflexion} wird eine Ausgleichsrechung der Form
\begin{align}
    \alpha_2 = m \cdot \alpha_2 + b
\end{align}
mit Hilfe der \texttt{scipy} \cite[]{scipy} Funktion \texttt{curve\_fit} durchgeführt.
Es ergeben sich die Parameter
\begin{align}
    m &= \num[]{1.029 +- 0.011}, & b &= (\num[]{-0.75 +- 0.51}) \, \unit[]{\degree},
\end{align}
wobei die Unsicherheiten der Kovarianzmatrix entnommen werden.
% # m = 1.02889518 +- 0.01129740
% # b = -0.74787535 +- 0.50748549
Die Messdaten inklusive Fehlerbalken sowie der Ausgleichsgeraden sind in Abbildung \ref{fig:reflexion} zu sehen.

\begin{figure}[H]
    \centering
    \includegraphics[height=8cm]{build/c01_reflexion.pdf}
    \caption[]{Plot der Einfallswinkel $\alpha_1$ und Ausfallswinkel $\alpha_2$.}
    \label{fig:reflexion}
\end{figure}






\subsection{Brechungsgesetz}
\label{sec:ausw_brechung}
In diesem Versuchsteil wird zur Überprüfung des Brechungsgesetzes das Licht des grünen Lasers auf eine planparallele Platte geschickt.
Für insgesamt 7 Einfallswinkel $\alpha$ wird der Brechungswinkel $\beta$ bestimmt.
Die Skala zur Bestimmung von $\beta$ hat eine Genauigkeit von $\qty[]{0.5}{\degree}$.
Die Einfallswinkel $\alpha$ mit den entsprechenden Brechungswinkeln $\beta$ sind Tabelle \ref{tab:brechung} zu sehen.

\begin{table}[H]
    \centering
    \caption[short]{Einfallswinkel $\alpha$ und Brechungswinkel $\beta$ bei der Brechung an der planparallelen Platte.}
    \label{tab:brechung}
    \sisetup{table-format=2.0}
    \begin{tabular}{S S[table-format=2.1]}
        \toprule
        {$\alpha / \unit[]{\degree}$} & {$\beta / \unit[]{\degree}$} \\
        \midrule
        %\cmidrule(lr){1-2}\cmidrule(lr){3-4}
        15 & 10.0 \\
        20 & 13.5 \\
        30 & 19.5 \\
        40 & 25.5 \\
        50 & 31.0 \\
        60 & 35.5 \\
        70 & 39.0 \\
        \bottomrule
    \end{tabular}
\end{table}
% #alpha beta
% 15 10.0
% 20 13.5
% 30 19.5
% 40 25.5
% 50 31.0
% 60 35.5
% 70 39.0


Zur Bestimmung des Brechungsindexes $n$ wird gemäß Formel \eqref{eq:brechung} eine Ausgleichsrechnung
\begin{align}
    \sin(\beta) = \frac{1}{n} \sin(\alpha) + b
\end{align}
mit der Funktion \texttt{curve\_fit} durchgeführt.
Es ergibt sich der experimentelle Brechungsindex
\begin{align}
    n_\text{Exp} = \num[]{1.4970 +- 0.0063}
\end{align}
sowie die Konstante $b = \num[]{0.0020 +- 0.0020}$.
Der entsprechende Plot ist in Abbildung \ref{fig:brechung} zu sehen.
Im Vergleich zum Literaturwert $n_\text{Lit} = \num[]{1.49}$ nach \cite[]{brechungsindex} ergibt sich eine Abweichung von 
\begin{align}
    \frac{|n_\text{Exp} - n_\text{Lit}|}{n_\text{Lit}} = \num[]{0.5 +- 0.4} \, \%.
\end{align}
Ferner kann gemäß Gleichung \eqref{eq:c_n} die Lichtgeschwindigkeit in Plexiglas zu
\begin{align}
    v_\text{Plex} = (\num[]{2.003 +- 0.008}) \cdot 10^8 \, \unit{\meter\per\second}
\end{align}
bestimmt werden.
% Teil 2: Brechungsgesetz
% n = 1.49702650 +- 0.00626929
% b = 0.00198052 +- 0.00195368
% Brechungsindex: n =  1.497+/-0.006
% Abweichung zur Literatur:  0.005+/-0.004
% Lichtgeschwindigkeit in Plexiglas: v =  (2.003+/-0.008)e+08  m/s =  0.6680+/-0.0028  c 


\begin{figure}[H]
    \centering
    \includegraphics[height=8cm]{build/c02_brechung.pdf}
    \caption[]{Plot der Einfallswinkel $\alpha$ und Brechungswinkel $\beta$.}
    \label{fig:brechung}
\end{figure}






\subsection{Strahlversatz bei der planparallelen Platte}
\label{sec:ausw_strahlversatz}
In diesem Versuchsteil werden die gleichen Messdaten wie im vorigen Teil verwendet, vgl. Tabelle \ref{tab:brechung}.
Der Strahlversatz wird auf zwei verschiedene Methoden bestimmt.
Gemäß Formel \eqref{eq:teo_strahlversatz} wird zunächst abhängig von $\alpha$ und $\beta_1$ der Strahlversatz $s_1$ bestimmt,
wobei die Breite der Platte nach \cite[]{man:v400} $d = \qty[]{5.85}{\cm}$ beträgt.
Die andere Methode beruht darauf, dass gemäß des Brechungsgesetzes nach Snellius \eqref{eq:brechung} zunächst die Brechungswinkel
$\beta_2$ bestimmt werden, die dann anschließend zur Berechnung des Strahlversatzes $s_2$ nach \eqref{eq:teo_strahlversatz} verwendet werden.
Die genannten Werte sowie die Abweichung $\Delta s$ von $s_1$ und $s_2$ sind in Tabelle \ref{tab:strahlversatz} zu sehen.
Die Standardabweichungen der obigen Werte berechnen sich dabei nach der Gaußschen Fehlerfortpflanzung \eqref{eq:gauss}.

\begin{table}[H]
    \centering
    \caption[]{Die Strahlversätze $s_1$ und $s_2$ in Abhängigkeit der entsprechenden Winkel.}
    \label{tab:strahlversatz}
    %\sisetup{table-format=2.0}
    \begin{tabular}{S[table-format=2.0] S[table-format=2.1] %alpha, beta1
        S[table-format=2.2] @{${}\pm{}$} S[table-format=1.2] %beta 2 +- fehler
        S[table-format=2.2] @{${}\pm{}$} S[table-format=1.2] %s1 +- fehler
        S[table-format=2.2] @{${}\pm{}$} S[table-format=1.2] %s2 +- fehler
        S[table-format=2.2] @{${}\pm{}$} S[table-format=2.2]} %delta s +- fehler
        \toprule
        & \multicolumn{3}{c}{Brechungswinkel} & \multicolumn{6}{c}{Strahlversatz}\\ % & \multicolumn{4}{c}{Strahlversätze}
        \cmidrule(lr){2-4} \cmidrule(lr){5-10}
        {$\alpha / \unit[]{\degree}$} & {$\beta_1 / \unit[]{\degree}$} 
        & \multicolumn{2}{c}{{$\beta_2 / \unit[]{\degree}$}}
        & \multicolumn{2}{c}{{$s_1 / \unit[]{\cm}$}} 
        & \multicolumn{2}{c}{{$s_2 / \unit[]{\cm}$}}
        & \multicolumn{2}{c}{{$\Delta s / \%$}} \\
        \midrule
        15 & 10.0 &  9.96 & 0.04 & 0.52 & 0.10 & 0.52 & 0.01 & 00.86 & 19.49 \\
        20 & 13.5 & 13.21 & 0.06 & 0.68 & 0.10 & 0.71 & 0.01 & 04.18 & 14.30 \\
        30 & 19.5 & 19.51 & 0.09 & 1.13 & 0.10 & 1.13 & 0.01 & 00.10 & 08.84 \\
        40 & 25.5 & 25.43 & 0.11 & 1.62 & 0.10 & 1.63 & 0.01 & 00.42 & 05.93 \\
        50 & 31.0 & 30.78 & 0.14 & 2.22 & 0.09 & 2.24 & 0.01 & 00.88 & 04.02 \\
        60 & 35.5 & 35.35 & 0.17 & 2.98 & 0.08 & 2.99 & 0.01 & 00.40 & 02.61 \\
        70 & 39.0 & 38.88 & 0.19 & 3.88 & 0.06 & 3.88 & 0.01 & 00.18 & 01.52 \\
        \bottomrule
    \end{tabular}
\end{table}
% Teil 3: Strahlversatz
% Strahlversatz : s =  
% [0.5177265243031446+/-0.1016890274635045 
%  0.6810564443122777+/-0.10147428935944072
%  1.1309472733837191+/-0.09951092473453996
%  1.6228084471524857+/-0.09600872518565101
%  2.221939290047947 +/-0.08932428325308159
%  2.979866388540225 +/-0.07702490505550862
%  3.876971492684826 +/-0.05782021861223494] cm
% neu gerechnetes beta: β =  
% [0.17376183799045458+/-0.0007350981920326408
%  0.23050205167403742+/-0.0009827704997573978
%  0.34053924999584023+/-0.0014839310629197642
%  0.4438019961178242+/-0.0019910345212903018 
%  0.537174721815124+/-0.0024942530415351227
%  0.6168849138261443+/-0.0029700785203662775
%  0.6786029088706705+/-0.0033768700296112984] rad =  
% [9.955819957289018+/-0.04211802393116764     
%  13.206794730028756+/-0.05630860186606159   
%  19.511461783312082+/-0.0850229869936647   
%  25.427981317032675+/-0.11407787493478454  
%  30.77784442112068+/-0.14291017231763134    
%  35.344902007529555+/-0.17017296403944804   
%  38.88108264359024+/-0.1934804006609447]     deg
% Strahlversatz die Zweite: s =  
% [0.5222185400199149+/-0.004281779028244064
%  0.7107729910151951+/-0.005700010449844021
%  1.129806619908172 +/-0.008461915238928518
%  1.6297187297376368+/-0.010939367518440959
%  2.241737153147955 +/-0.01270626664757905 
%  2.991789803323222 +/-0.013057180380057095
%  3.8838357932126644+/-0.011149617241589785] cm
% Strahlversatz Abweichung :  
% [-0.008601792875065229 +/-0.1948946322580831  
%  -0.04180877309430861  +/-0.14297274687324923 
%   0.0010096006302741068+/-0.08839634923931533 
%  -0.004240168845739101 +/-0.05928918706357141 
%  -0.008831482795477099 +/-0.04024010905345669 
%  -0.003985378508126771 +/-0.026109824150291686
%  -0.0017674023551238513+/-0.015160703557230005]






\subsection{Prisma}
\label{sec:ausw_prisma}
In diesem Versuchsteil soll die Ablenkung $\delta$ des Laserlichts beim Auftreffen auf ein Prisma untersucht werden.
Dabei wird Formel \eqref{eq:prisma} verwendet.
Der hierfür erforderliche Winkel $\beta_1$ lässt sich dabei über das Snelliussche Brechungsgesetz \eqref{eq:brechung} bestimmen,
wobei der Brechungsindex von Kronglas nach \cite[]{brechungsindex} $n_\text{Kron} = \num[]{1.51}$ beträgt.
Ferner wird bei einem Innenwinkel $\gamma = \qty[]{60}{\degree}$ der benötigte Winkel $\beta_2$ aus $\beta_1 + \beta_2 = \gamma$ bestimmt.
Die Winkel sind in Abbildung \ref{fig:prisma} schematisch dargestellt.


\subsubsection{Grüner Laser}
Die Messwerte des Einfallswinkels $\alpha_1$ und Ausfallswinkels $\alpha_2$ sowie
die Brechungswinkel $\beta_1$ und $\beta_2$ mit der Ablenkung $\delta$ bei Verwendung des grünen Lasers sind in Tabelle \ref{tab:ablenkung_gruen} zu sehen.

\begin{table}[H]
    \centering
    \caption[]{Ablenkung $\delta$ in Abhängigkeit der entsprechenden Winkel bei grünem Licht.}
    \label{tab:ablenkung_gruen}
    \begin{tabular}{S[table-format=2.0] S[table-format=2.0] %alpha1, alpha2
        S[table-format=2.1]  %beta 1
        S[table-format=2.1]  %beta 2
        S[table-format=2.1] @{${}\pm{}$} S[table-format=1.1]} %delta +- fehler
        \toprule
        {$\alpha_1 / \unit[]{\degree}$} & {$\alpha_2 / \unit[]{\degree}$} 
        & {$\beta_1 / \unit[]{\degree}$}
        & {$\beta_2 / \unit[]{\degree}$} 
        & \multicolumn{2}{c}{{$\delta / \unit[]{\degree}$}} \\
        \midrule
        60 & 38 & 35.00 & 25.00 & 38.0 & 1.0 \\
        50 & 45 & 30.49 & 29.51 & 35.0 & 1.0 \\
        40 & 56 & 25.19 & 34.81 & 36.0 & 1.0 \\
        30 & 73 & 19.34 & 40.66 & 43.0 & 1.0 \\
        26 & 87 & 16.88 & 43.12 & 53.0 & 1.0 \\
        \bottomrule
    \end{tabular}
\end{table}
% #einfall ausfall
% 60 38
% 50 45
% 40 56
% 30 73
% 26 87
% beta_1 =  [
%     0.6108045927738653 
%     0.5320652808845506 
%     0.4397211502484149
%     0.33749646246254644 
%     0.29455288213066827] rad =  
%     [34.99652527314942 
%      30.485095020127428 
%      25.19416607187213 
%      19.337122899699327
%      16.876636989501694] deg
% beta_2 =  [
%     0.4363929584227323 
%     0.515132270312047 
%     0.6074764009481828
%     0.7097010887340511 
%     0.7526446690659294] rad =  [
%         25.00347472685057 
%         29.51490497987257 
%         34.805833928127875 
%         40.66287710030067
%         43.1233630104983] deg
% delta =  
% [0.6632251157578453+/-0.017453292519943295
%  0.6108652381980153+/-0.017453292519943295
%  0.6283185307179588+/-0.017453292519943295
%  0.7504915783575619+/-0.017453292519943295
%  0.9250245035569946+/-0.017453292519943295] rad =  
% [38.00000000000001+/-1.0  
%  35.0+/-1.0                
%  36.000000000000014+/-1.0
%  43.000000000000014+/-1.0 
% 53.0+/-1.0               ] deg


\subsubsection{Roter Laser}
Die gemessenen und berechneten Winkel bei der Verwendung von rotem Licht sind in Tabelle \ref{tab:ablenkung_rot} zu sehen.
\begin{table}[H]
    \centering
    \caption[]{Ablenkung $\delta$ in Abhängigkeit der entsprechenden Winkel bei rotem Licht.}
    \label{tab:ablenkung_rot}
    \begin{tabular}{S[table-format=2.0] S[table-format=2.0] %alpha1, alpha2
        S[table-format=2.1]  %beta 1
        S[table-format=2.1]  %beta 2
        S[table-format=2.1] @{${}\pm{}$} S[table-format=1.1]} %delta +- fehler
        \toprule
        {$\alpha_1 / \unit[]{\degree}$} & {$\alpha_2 / \unit[]{\degree}$} 
        & {$\beta_1 / \unit[]{\degree}$}
        & {$\beta_2 / \unit[]{\degree}$} 
        & \multicolumn{2}{c}{{$\delta / \unit[]{\degree}$}} \\
        \midrule
        60 & 38 & 35.00 & 25.00 & 38.0 & 1.0 \\
        50 & 45 & 30.49 & 29.51 & 35.0 & 1.0 \\
        40 & 55 & 25.19 & 34.81 & 35.0 & 1.0 \\
        30 & 72 & 19.34 & 40.66 & 42.0 & 1.0 \\
        26 & 85 & 16.88 & 43.12 & 51.0 & 1.0 \\
        \bottomrule
    \end{tabular}
\end{table}
% 60 38
% 50 45
% 40 55
% 30 72
% 26 85
 %beta_1 =  [0.6108045927738653 0.5320652808845506 0.4397211502484149
 % 0.33749646246254644 0.29455288213066827] rad =  [
 %    34.99652527314942 
 %    30.485095020127428 
 %    25.19416607187213 
 %    19.337122899699327
 %    16.876636989501694] deg
 %beta_2 =  [0.4363929584227323 0.515132270312047 0.6074764009481828
 % 0.7097010887340511 0.7526446690659294] rad =  [
 %    25.00347472685057 
 %    29.51490497987257 
 %    34.805833928127875 
 %    40.66287710030067
 %    43.1233630104983] deg
 %delta =  [0.6632251157578453+/-0.017453292519943295
 % 0.6108652381980153+/-0.017453292519943295
 % 0.6108652381980153+/-0.017453292519943295
 % 0.7330382858376183+/-0.017453292519943295
 % 0.8901179185171082+/-0.017453292519943295] rad =  [
 % 38.00000000000001+/-1.0 
 % 35.0+/-1.0             
 % 35.0+/-1.0             
 % 41.99999999999999+/-1.0
 % 51.00000000000001+/-1.0] deg



\subsection{Beugung am Gitter}
\label{sec:ausw_beugung}

In diesem Versuchsteil werden Gitter mit je 600, 300 und 100 Linien pro Millimeter verwendet, um anhand der Beugungsmaxima die Wellenlänge 
$\lambda$ der beiden Laser anhand von Gleichung \eqref{eq:beugung} zu bestimmen.
Dabei errechnet sich die Gitterkonstante $d$ aus dem Kehrwert der Linienanzahl $L$ pro Millimeter.
Die Messdaten des Beugungsmaximums und des Ablenkwinkels $\phi$ sind in der Tabelle \ref{tab:beugung} für das grüne bzw. rote Licht
zusammen mit $d$ und der resultierenden Wellenlänge $\lambda$ aufgetragen.
Dabei wird $\phi$ aus dem Mittelwert der beiden Winkel gebildet, die sich jeweils auf der linken und rechten Seite des 0. Maximums befinden.
Aufgrund der Länge der Winkelskala kann beim Gitter mit 100 Linien pro Millimeter nur bis zum 7. Maximum gemessen werden.
Werden Mittelwert und Standardabweichung der Wellenlänge gemäß \eqref{eq:mittelstand} gebildet, so ergeben sich
\begin{align}
    \lambda_\text{grün} & = (\num[]{544 +- 11}) \, \unit[]{\nano\meter}, & \lambda_\text{rot} & = (\num[]{651 +- 15}) \, \unit[]{\nano\meter}.
\end{align}
Im Vergleich zu den Literaturwerten \cite[]{man:v400} von $\lambda_\text{Lit,grün} = \qty[]{532}{\nano\meter}$ und 
$\lambda_\text{Lit,rot} = \qty[]{635}{\nano\meter}$ entspricht das Abweichungen von 
\begin{align}
    \Delta \lambda_\text{grün} &= (\num[]{2.3 +- 2.0}) \, \%, & \Delta \lambda_\text{rot} &= (\num[]{2.5 +- 2.4}) \, \%.
\end{align}
%grün
% Durchschnittliche Wellenlänge:  (5.44+/-0.11)e-07
% Literaturwert:  5.32e-07
% Abweichung der Wellenlänge:  0.023+/-0.020
%rot
% Durchschnittliche Wellenlänge:  (6.51+/-0.15)e-07
% Literaturwert:  6.350000000000001e-07
% Abweichung der Wellenlänge:  0.025+/-0.024


%huhu hier ist ulli, ihr macht dassupiiiiiiiiiiii :D
%dondt start unbelieving in urselfss 
% liebe grüsse 

% gruenes Licht
% 600.0 Linien pro mm
% zugehörige Gitterkonstante:  1.6666666666666667 μm
% zugehöriger Winkel:  [18.5] deg
% λ gruen bei 600.0 Linien =  [5.288410940084869e-07]
% 300.0 Linien pro mm
% zugehörige Gitterkonstante:  3.3333333333333335 μm
% zugehöriger Winkel:  [ 9.5 19.  29.5] deg
% λ gruen bei 300.0 Linien =  [5.501586862022589e-07 5.426135907619278e-07 5.471372890038524e-07]
% 100.0 Linien pro mm
% zugehörige Gitterkonstante:  10.0 μm
% zugehöriger Winkel:  [
%  3.0  
%  6.5  
%  9.5 
% 12.5 
% 15.5 
% 19.5 
% 22.5 
% 25.5 
% 29.5] deg
% λ gruen bei 100.0 Linien =  [
%  5.233595624294384e-07 
%  5.660160688395336e-07 
%  5.501586862022589e-07
%  5.410990348452572e-07 
%  5.344767521565138e-07 
%  5.563447653896183e-07
%  5.46690617664414e-07 
%  5.38138871010369e-07 
%  5.471372890038524e-07]
\begin{table}
    \centering
    \caption[]{Die Wellenlänge $\lambda$ in Abhängigkeit von $d$, $k$ und $\phi$ beider Laser.}
    \label{tab:beugung}
    \begin{tabular}{
        S[table-format = 3.0]  % Linien pro mm
        S[table-format = 2.2]  % d
        S[table-format = 1.0]  % k
        S[table-format = 2.0]  % phi links
        S[table-format = 2.0]  % phi rechts
        S[table-format = 2.1]  % phi
        S[table-format = 3.2]  % lambda
        S[table-format = 2.0]  % phi links
        S[table-format = 2.0]  % phi rechts
        S[table-format = 2.1]  % phi
        S[table-format = 3.2]} % lambda
        \toprule
        \multicolumn{2}{c}{Gitter} & & \multicolumn{4}{c}{rot} & \multicolumn{4}{c}{grün} \\
        \cmidrule(lr){1-2} \cmidrule(lr){4-7} \cmidrule(lr){8-11}
        {$L / (1/\unit[]{\mm})$} & {$d / \unit[]{\micro\meter}$} & {k} 
        & {$\phi_\text{l} / \unit{\degree}$} & {$\phi_\text{r} / \unit{\degree}$}
        & {$\phi / \unit{\degree}$} & {$\lambda / \unit{\nano\meter}$}
        & {$\phi_\text{l} / \unit{\degree}$} & {$\phi_\text{r} / \unit{\degree}$}
        & {$\phi / \unit{\degree}$} & {$\lambda / \unit{\nano\meter}$} \\
        \midrule
        600 &  1.67 & 1 & 19 & 18 & 18.5 & 528.84 & 23 & 22 & 22.5 & 637.81 \\
        \midrule
        300 &  3.33 & 1 & 10 &  9 &  9.5 & 550.16 & 12 & 11 & 11.5 & 664.56 \\
            &       & 2 & 20 & 18 & 19.0 & 542.61 & 24 & 22 & 23.0 & 651.22 \\ 
            &       & 3 & 30 & 29 & 29.5 & 547.14 & 37 & 35 & 36.0 & 653.09 \\
        \midrule
        100 & 10.00 & 1 &  3 &  3 &  3.0 & 523.36 &  4 &  3 &  3.5 & 610.49 \\
            &       & 2 &  7 &  6 &  6.5 & 566.02 &  8 &  7 &  7.5 & 652.63 \\    
            &       & 3 & 10 &  9 &  9.5 & 550.16 & 12 & 11 & 11.5 & 664.56 \\    
            &       & 4 & 13 & 12 & 12.5 & 541.10 & 15 & 15 & 15.0 & 647.05 \\    
            &       & 5 & 16 & 15 & 15.5 & 534.48 & 19 & 19 & 19.0 & 651.14 \\    
            &       & 6 & 20 & 19 & 19.5 & 556.34 & 24 & 23 & 23.5 & 664.58 \\    
            &       & 7 & 23 & 22 & 22.5 & 546.69 & 28 & 27 & 27.5 & 659.64 \\    
            &       & 8 & 26 & 25 & 25.5 & 538.14 &    &    &      &        \\    
            &       & 9 & 30 & 29 & 29.5 & 547.14 &    &    &      &        \\        
        \bottomrule
    \end{tabular}
\end{table}
% rotes Licht
% 600.0 Linien pro mm
% zugehörige Gitterkonstante:  1.6666666666666667 μm
% zugehöriger Winkel:  [22.5] deg
% λ rot bei 600.0 Linien =  [6.37805720608483e-07]
% 300.0 Linien pro mm
% zugehörige Gitterkonstante:  3.3333333333333335 μm
% zugehöriger Winkel:  [11.5 23.  36. ] deg
% λ rot bei 300.0 Linien =  [6.645597813906573e-07 6.51218547482123e-07 6.530947247694146e-07]
% 100.0 Linien pro mm
% zugehörige Gitterkonstante:  10.0 μm
% zugehöriger Winkel:  [ 
%    3.5  
%    7.5 
%    11.5 
%    15.  
%    19.  
%    23.5
%    27.5] deg
% λ rot bei 100.0 Linien =  [
%  6.104853953485688e-07 
%  6.526309611002579e-07 
%  6.645597813906573e-07
%  6.470476127563019e-07 
%  6.511363089143134e-07 
%  6.645817815420771e-07
%  6.596408760500485e-07]
