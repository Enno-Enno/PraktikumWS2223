\section{Auswertung}

\subsection{Mittelwerte und Fehler}
Das arithmetische Mittel $\overline{c}$ und die Standardabweichung $\Delta c$ einer Messreihe mit $N$ Werten $c_k$ errechnet sich gemäß der Formeln
\begin{align}
    \overline{c} &= \frac{1}{N} \sum_{k=1}^{N} c_k, & \Delta c = \sqrt{ \sum_{k=1}^{N} \left(\overline{c} - c_k \right)^2 }.
    \label{eq:mittelstand}
\end{align}

\subsubsection{Gaußsche Fehlerfortpflanzung}
Wenn zu Messdaten die Standardabweichung bekannt ist, und mit diesen Messdaten weiter gerechnet werden soll,
wird die Gaußsche Fehlerfortpflanzung verwendet. 
Angenommen, es gibt $k$ Messwerte $x_i [i \in \mathbb{N}, i \leq k]$ mit den Standardabweichungen $\Delta x_i$
und eine abgeleitete Größe $f(x_i)$.
Dann ist der Fehler von $f$
\begin{align}
    \Delta f(x_i) = \sqrt{
    \left(\frac{\partial f}{\partial x_1} \Delta x_1\right)^2%
     + \left(\frac{\partial f}{\partial x_2} \Delta x_2\right)^2%
     + \dots%
     + \left(\frac{\partial f}{\partial x_k} \Delta x_k\right)^2%
    }.
    \label{eq:gauss}
\end{align} 
Im Ergebnis ergibt sich der Mittelwert von $f$ mit der errechneten Abweichung $\overline{f} \pm \Delta f $.
Um Rechenfehler zu vermeiden, wird das Python \cite[]{python} Paket \texttt{uncertainties} \cite[][]{uncertainties} verwendet.
Hier wird die Fehlerfortpflanzung automatisch verrechnet, wenn die Variablen als \texttt{ufloat} definiert werden.







\subsection{Reflexionsgesetz}
Zur Überprüfung des Reflexionsgesetzes werden mit dem grünen Laser für insgesamt 7 Einfallswinkel $\alpha_1$
die resultierenden Ausfallswinkel $\alpha_2$ gemessen, die durch Reflexion am Spiegel entstehen.
Die gemessenen Winkel sind in Tabelle \ref{tab:reflexion} zu sehen.
Die Messungenauigkeit der Winkel wird dabei auf $\Delta \alpha = \qty[]{1}{\degree}$ geschätzt.

\begin{table}[H]
    \centering
    \caption[short]{Einfallswinkel $\alpha_1$ und Ausfallswinkel $\alpha_2$ bei der Reflexion am Spiegel.}
    \label{tab:reflexion}
    \sisetup{table-format=2.0}
    \begin{tabular}{S S}
        \toprule
        {$\alpha_1 / \unit[]{\degree}$} & {$\alpha_2 / \unit[]{\degree}$} \\
        \midrule
        %\cmidrule(lr){1-2}\cmidrule(lr){3-4}
        15 & 14 \\
        20 & 20 \\
        30 & 31 \\
        40 & 40 \\
        50 & 51 \\
        60 & 61 \\
        70 & 71 \\
        \bottomrule
    \end{tabular}
\end{table}
% # einfall ausfall
% 15 14
% 20 20
% 30 31
% 40 40
% 50 51
% 60 61
% 70 71

\noindent
Gemäß der Formel \textbf{REF! EINFALL=AUSFALL} wird eine Ausgleichsrechung der Form
\begin{align}
    \alpha_2 = m \cdot \alpha_2 + b
\end{align}
mit Hilfe der \texttt{scipy} \cite[]{scipy} Funktion \texttt{curve\_fit} durchgeführt.
Es ergeben sich die Parameter
\begin{align}
    m &= \num[]{1.029 +- 0.011}, & b &= (\num[]{-0.75 +- 0.51}) \, \unit[]{\degree},
\end{align}
wobei die Unsicherheiten der Kovarianzmatrix entnommen werden.
% # m = 1.02889518 +- 0.01129740
% # b = -0.74787535 +- 0.50748549
Die Messdaten inklusive Fehlerbalken sowie der Ausgleichsgeraden sind in Abbildung \ref{fig:reflexion} zu sehen.

\begin{figure}[H]
    \centering
    \includegraphics[height=8cm]{build/c01_reflexion.pdf}
    \caption[]{Plot der Einfallswinkel $\alpha_1$ und Ausfallswinkel $\alpha_2$.}
    \label{fig:reflexion}
\end{figure}






\subsection{Brechungsgesetz}
In diesem Versuchsteil wird zur Überprüfung des Brechungsgesetzes das Licht des grünen Lasers auf eine planparallele Platte geschickt.
Für insgesamt 7 Einfallswinkel $\alpha$ wird der Brechungswinkel $\beta$ bestimmt.
Die Skala zur Bestimmung von $\beta$ hat eine Genauigkeit von $\qty[]{0.5}{\degree}$.
Die Einfallswinkel $\alpha$ mit den entsprechenden Brechungswinkeln $\beta$ sind Tabelle \ref{tab:brechung} zu sehen.

\begin{table}[H]
    \centering
    \caption[short]{Einfallswinkel $\alpha$ und Brechungswinkel $\beta$ bei der Brechung an der planparallelen Platte.}
    \label{tab:brechung}
    \sisetup{table-format=2.0}
    \begin{tabular}{S S[table-format=2.1]}
        \toprule
        {$\alpha / \unit[]{\degree}$} & {$\beta / \unit[]{\degree}$} \\
        \midrule
        %\cmidrule(lr){1-2}\cmidrule(lr){3-4}
        15 & 10.0 \\
        20 & 13.5 \\
        30 & 19.5 \\
        40 & 25.5 \\
        50 & 31.0 \\
        60 & 35.5 \\
        70 & 39.0 \\
        \bottomrule
    \end{tabular}
\end{table}
% #alpha beta
% 15 10.0
% 20 13.5
% 30 19.5
% 40 25.5
% 50 31.0
% 60 35.5
% 70 39.0


Zur Bestimmung des Brechungsindexes $n$ wird gemäß Formel \textbf{REF! SNELLIUS} eine Ausgleichsrechnung
\begin{align}
    \sin(\beta) = \frac{1}{n} \sin(\alpha) + b
\end{align}
mit der Funktion \texttt{curve\_fit} durchgeführt.
Es ergibt sich der experimentelle Brechungsindex
\begin{align}
    n_\text{Exp} = \num[]{1.4970 +- 0.0063}
\end{align}
sowie die Konstante $b = \num[]{0.0020 +- 0.0020}$.
Der entsprechende Plot ist in Abbildung \ref{fig:brechung} zu sehen.
Im Vergleich zum Literaturwert $n_\text{Lit} = \num[]{1.49}$ nach \cite[]{brechungsindex} ergibt sich eine Abweichung von 
\begin{align}
    \frac{|n_\text{Exp} - n_\text{Lit}|}{n_\text{Lit}} = \num[]{0.5 +- 0.4} \, \%.
\end{align}
Ferner kann gemäß Gleichung \textbf{REF! LICHTGESCHWINDIGKEIT (SNELLIUS)} die Lichtgeschwindigkeit in Plexiglas zu
\begin{align}
    v_\text{Plex} = (\num[]{2.003 +- 0.008}) \cdot 10^8 \, \unit{\meter\per\second}
\end{align}
bestimmt werden.
% Teil 2: Brechungsgesetz
% n = 1.49702650 +- 0.00626929
% b = 0.00198052 +- 0.00195368
% Brechungsindex: n =  1.497+/-0.006
% Abweichung zur Literatur:  0.005+/-0.004
% Lichtgeschwindigkeit in Plexiglas: v =  (2.003+/-0.008)e+08  m/s =  0.6680+/-0.0028  c 


\begin{figure}[H]
    \centering
    \includegraphics[height=8cm]{build/c02_brechung.pdf}
    \caption[]{Plot der Einfallswinkel $\alpha$ und Brechungswinkel $\beta$.}
    \label{fig:brechung}
\end{figure}






\subsection{Strahlversatz bei der planparallelen Platte}
In diesem Versuchsteil werden die gleichen Messdaten wie im vorigen Teil verwendet, vgl. Tabelle \ref{tab:brechung}.
Der Strahlversatz wird auf zwei verschiedene Methoden bestimmt.
Gemäß Formel \textbf{REF! STRAHLENVERSATZ} wird zunächst abhängig von $\alpha$ und $\beta_1$ der Strahlversatz $s_1$ bestimmt,
wobei die Breite der Platte nach \cite[]{man:v400} $d = \qty[]{5.85}{\cm}$ beträgt.
Die andere Methode beruht darauf, dass gemäß des Brechungsgesetzes nach Snellius \textbf{REF! SNELLIUS} zunächst die Brechungswinkel
$\beta_2$ bestimmt werden, die dann anschließend zur Berechnung des Strahlversatzes $s_2$ nach \textbf{REF! STRAHLV} verwendet werden.
Die genannten Werte sowie die Abweichung $\Delta s$ von $s_1$ und $s_2$ sind in Tabelle \ref{tab:strahlversatz} zu sehen.
Die Standardabweichungen der obigen Werte berechnen sich dabei nach der Gaußschen Fehlerfortpflanzung \eqref{eq:gauss}.

\begin{table}[H]
    \centering
    \caption[]{Die Strahlversätze $s_1$ und $s_2$ in Abhängigkeit der entsprechenden Winkel.}
    \label{tab:strahlversatz}
    %\sisetup{table-format=2.0}
    \begin{tabular}{S[table-format=2.0] S[table-format=2.1] %alpha, beta1
        S[table-format=2.2] @{${}\pm{}$} S[table-format=1.2] %beta 2 +- fehler
        S[table-format=2.2] @{${}\pm{}$} S[table-format=1.2] %s1 +- fehler
        S[table-format=2.2] @{${}\pm{}$} S[table-format=1.2] %s2 +- fehler
        S[table-format=2.2] @{${}\pm{}$} S[table-format=2.2]} %delta s +- fehler
        \toprule
        & \multicolumn{3}{c}{Brechungswinkel} & \multicolumn{6}{c}{Strahlversatz}\\ % & \multicolumn{4}{c}{Strahlversätze}
        \cmidrule(lr){2-4} \cmidrule(lr){5-10}
        {$\alpha / \unit[]{\degree}$} & {$\beta_1 / \unit[]{\degree}$} 
        & \multicolumn{2}{c}{{$\beta_2 / \unit[]{\degree}$}}
        & \multicolumn{2}{c}{{$s_1 / \unit[]{\cm}$}} 
        & \multicolumn{2}{c}{{$s_2 / \unit[]{\cm}$}}
        & \multicolumn{2}{c}{{$\Delta s / \%$}} \\
        \midrule
        15 & 10.0 &  9.96 & 0.04 & 0.52 & 0.10 & 0.52 & 0.01 & 00.86 & 19.49 \\
        20 & 13.5 & 13.21 & 0.06 & 0.68 & 0.10 & 0.71 & 0.01 & 04.18 & 14.30 \\
        30 & 19.5 & 19.51 & 0.09 & 1.13 & 0.10 & 1.13 & 0.01 & 00.10 & 08.84 \\
        40 & 25.5 & 25.43 & 0.11 & 1.62 & 0.10 & 1.63 & 0.01 & 00.42 & 05.93 \\
        50 & 31.0 & 30.78 & 0.14 & 2.22 & 0.09 & 2.24 & 0.01 & 00.88 & 04.02 \\
        60 & 35.5 & 35.35 & 0.17 & 2.98 & 0.08 & 2.99 & 0.01 & 00.40 & 02.61 \\
        70 & 39.0 & 38.88 & 0.19 & 3.88 & 0.06 & 3.88 & 0.01 & 00.18 & 01.52 \\
        \bottomrule
    \end{tabular}
\end{table}
% Teil 3: Strahlversatz
% Strahlversatz : s =  
% [0.5177265243031446+/-0.1016890274635045 
%  0.6810564443122777+/-0.10147428935944072
%  1.1309472733837191+/-0.09951092473453996
%  1.6228084471524857+/-0.09600872518565101
%  2.221939290047947 +/-0.08932428325308159
%  2.979866388540225 +/-0.07702490505550862
%  3.876971492684826 +/-0.05782021861223494] cm
% neu gerechnetes beta: β =  
% [0.17376183799045458+/-0.0007350981920326408
%  0.23050205167403742+/-0.0009827704997573978
%  0.34053924999584023+/-0.0014839310629197642
%  0.4438019961178242+/-0.0019910345212903018 
%  0.537174721815124+/-0.0024942530415351227
%  0.6168849138261443+/-0.0029700785203662775
%  0.6786029088706705+/-0.0033768700296112984] rad =  
% [9.955819957289018+/-0.04211802393116764     
%  13.206794730028756+/-0.05630860186606159   
%  19.511461783312082+/-0.0850229869936647   
%  25.427981317032675+/-0.11407787493478454  
%  30.77784442112068+/-0.14291017231763134    
%  35.344902007529555+/-0.17017296403944804   
%  38.88108264359024+/-0.1934804006609447]     deg
% Strahlversatz die Zweite: s =  
% [0.5222185400199149+/-0.004281779028244064
%  0.7107729910151951+/-0.005700010449844021
%  1.129806619908172 +/-0.008461915238928518
%  1.6297187297376368+/-0.010939367518440959
%  2.241737153147955 +/-0.01270626664757905 
%  2.991789803323222 +/-0.013057180380057095
%  3.8838357932126644+/-0.011149617241589785] cm
% Strahlversatz Abweichung :  
% [-0.008601792875065229 +/-0.1948946322580831  
%  -0.04180877309430861  +/-0.14297274687324923 
%   0.0010096006302741068+/-0.08839634923931533 
%  -0.004240168845739101 +/-0.05928918706357141 
%  -0.008831482795477099 +/-0.04024010905345669 
%  -0.003985378508126771 +/-0.026109824150291686
%  -0.0017674023551238513+/-0.015160703557230005]






\subsection{Prisma}
In diesem Versuchsteil soll die Ablenkung $\delta$ des Laserlichts beim Auftreffen auf ein Prisma untersucht werden.
Dabei wird Formel \textbf{REF!!! ABLENKUNG} verwendet.
Der hierfür erforderliche Winkel $\beta_1$ lässt sich dabei über das Snelliussche Brechungsgesetz \textbf{REF!!!} bestimmen,
wobei der Brechungsindex von Kronglas nach \cite[]{brechungsindex} $n_\text{Kron} = \num[]{1.51}$ beträgt.
Ferner wird bei einem Innenwinkel $\gamma = \qty[]{90}{\degree}$ der benötigte Winkel $\beta_2$ aus $\beta_1 + \beta_2 = \gamma$ bestimmt.
Die Winkel sind in Abbildung \textbf{REF! PRISMA ABBILDUNG} schematisch dargestellt.

\subsubsection{Grüner Laser}
Die Messwerte des Einfallswinkels $\alpha_1$ und Ausfallswinkels $\alpha_2$ sowie
die Brechungswinkel $\beta_1$ und $\beta_2$ mit der Ablenkung $\delta$ bei Verwendung des grünen Lasers sind in Tabelle \ref{tab:ablenkung_gruen} zu sehen.

\begin{table}[H]
    \centering
    \caption[]{Ablenkung $\delta$ in Abhängigkeit der entsprechenden Winkel bei grünem Licht.}
    \label{tab:ablenkung_gruen}
    \begin{tabular}{S[table-format=2.0] S[table-format=2.0] %alpha1, alpha2
        S[table-format=2.1]  %beta 1
        S[table-format=2.1]  %beta 2
        S[table-format=2.1] @{${}\pm{}$} S[table-format=1.1]} %delta +- fehler
        \toprule
        {$\alpha_1 / \unit[]{\degree}$} & {$\alpha_2 / \unit[]{\degree}$} 
        & {$\beta_1 / \unit[]{\degree}$}
        & {$\beta_2 / \unit[]{\degree}$} 
        & \multicolumn{2}{c}{{$\delta / \unit[]{\degree}$}} \\
        \midrule
        60 & 38 & 35.0 & 25.0 & 38.0 & 1.0 \\
        50 & 45 & 30.5 & 29.5 & 35.0 & 1.0 \\
        40 & 56 & 25.2 & 34.8 & 36.0 & 1.0 \\
        30 & 73 & 19.3 & 40.7 & 43.0 & 1.0 \\
        26 & 87 & 16.9 & 43.1 & 53.0 & 1.0 \\
        \bottomrule
    \end{tabular}
\end{table}
% #einfall ausfall
% 60 38
% 50 45
% 40 56
% 30 73
% 26 87
% beta_1 =  [
%     0.6108045927738653 
%     0.5320652808845506 
%     0.4397211502484149
%     0.33749646246254644 
%     0.29455288213066827] rad =  
%     [34.99652527314942 
%      30.485095020127428 
%      25.19416607187213 
%      19.337122899699327
%      16.876636989501694] deg
% beta_2 =  [
%     0.4363929584227323 
%     0.515132270312047 
%     0.6074764009481828
%     0.7097010887340511 
%     0.7526446690659294] rad =  [
%         25.00347472685057 
%         29.51490497987257 
%         34.805833928127875 
%         40.66287710030067
%         43.1233630104983] deg
% delta =  
% [0.6632251157578453+/-0.017453292519943295
%  0.6108652381980153+/-0.017453292519943295
%  0.6283185307179588+/-0.017453292519943295
%  0.7504915783575619+/-0.017453292519943295
%  0.9250245035569946+/-0.017453292519943295] rad =  
% [38.00000000000001+/-1.0  
%  35.0+/-1.0                
%  36.000000000000014+/-1.0
%  43.000000000000014+/-1.0 
% 53.0+/-1.0               ] deg


\subsubsection{Roter Laser}
Die gemessenen und berechneten Winkel bei der Verwendung von rotem Licht sind in Tabelle \ref{tab:ablenkung_rot} zu sehen.
\begin{table}[H]
    \centering
    \caption[]{Ablenkung $\delta$ in Abhängigkeit der entsprechenden Winkel bei rotem Licht.}
    \label{tab:ablenkung_rot}
    \begin{tabular}{S[table-format=2.0] S[table-format=2.0] %alpha1, alpha2
        S[table-format=2.1]  %beta 1
        S[table-format=2.1]  %beta 2
        S[table-format=2.1] @{${}\pm{}$} S[table-format=1.1]} %delta +- fehler
        \toprule
        {$\alpha_1 / \unit[]{\degree}$} & {$\alpha_2 / \unit[]{\degree}$} 
        & {$\beta_1 / \unit[]{\degree}$}
        & {$\beta_2 / \unit[]{\degree}$} 
        & \multicolumn{2}{c}{{$\delta / \unit[]{\degree}$}} \\
        \midrule
        60 & 38 & 35.0 & 25.0 & 38.0 & 1.0 \\
        50 & 45 & 30.5 & 29.5 & 35.0 & 1.0 \\
        40 & 55 & 25.2 & 34.8 & 35.0 & 1.0 \\
        30 & 72 & 19.3 & 40.7 & 42.0 & 1.0 \\
        26 & 85 & 16.9 & 43.1 & 51.0 & 1.0 \\
        \bottomrule
    \end{tabular}
\end{table}
% 60 38
% 50 45
% 40 55
% 30 72
% 26 85
 %beta_1 =  [0.6108045927738653 0.5320652808845506 0.4397211502484149
 % 0.33749646246254644 0.29455288213066827] rad =  [
 %    34.99652527314942 
 %    30.485095020127428 
 %    25.19416607187213 
 %    19.337122899699327
 %    16.876636989501694] deg
 %beta_2 =  [0.4363929584227323 0.515132270312047 0.6074764009481828
 % 0.7097010887340511 0.7526446690659294] rad =  [
 %    25.00347472685057 
 %    29.51490497987257 
 %    34.805833928127875 
 %    40.66287710030067
 %    43.1233630104983] deg
 %delta =  [0.6632251157578453+/-0.017453292519943295
 % 0.6108652381980153+/-0.017453292519943295
 % 0.6108652381980153+/-0.017453292519943295
 % 0.7330382858376183+/-0.017453292519943295
 % 0.8901179185171082+/-0.017453292519943295] rad =  [
 % 38.00000000000001+/-1.0 
 % 35.0+/-1.0             
 % 35.0+/-1.0             
 % 41.99999999999999+/-1.0
 % 51.00000000000001+/-1.0] deg



\subsection{Beugung am Gitter}