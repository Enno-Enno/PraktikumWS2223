\section{Auswertung}

\subsection{Kennlinien und Raumkurve}
In den folgenden Messungen wird der Sättigungsstrom und der Gültigkeitsbereich des
Langmuir-Schottkyschen Raumladungsgesetzes \eqref{eq:lang_schott} überprüft.
Dazu wird die Anodenspannung $U_A$ mit den an der Diode gemessenen Strömen $I$ dargestellt.
Zu jedem Wertepaar wird eine Steigung berechnet
\begin{align}
    \text{grad}_{i} = \frac{I_{i+1}-I_{i}}{U_{i+1}- U_{i}}.
\end{align}
Anhand dieser Steigundsdreiecke lässt sich ein Wendepunkt abschätzen, vor dem das Langmuir-Schottkysche Raumladungsgesetz gilt.
Für die Wertepaare wird eine Funktion der Form
\begin{align}
    a \cdot U^{b}
    \label{eq:raum_fit}
\end{align}
gewählt.
Mit dieser Form wird der Exponent dieses Zusammenhangs ermittelt.
Für die Sättigungskurve wird ein exponentieller Zusammenhang der Form
\begin{align}
    -a \cdot e^{b\cdot (U-d)} + c
\end{align}
verwendet.
von besonderem Interesse ist hier nur der Wert für $c$, da er dem Sättigungsstrom $I_s$ entspricht.
In den Messungen bei $I_\text{f} = \qty{2.3}{\ampere}$ und $\qty{2.5}{\ampere}$ lässt sich diese Ausgleichsrechnug nicht mehr
durchführen, da die ströme keinen Sättigungswert erreichen.
Deshalb wird für diese Kurven der Sättigungsstrom anhand der graphischen Darstellung abgeschätzt.


\begin{table}
    \centering
    \sisetup{table-format=:.1}
    \begin{tabular}{S[table-format=:.0] S S S}
        \toprule
        {$i$} &
        {$U / \unit{\volt}$} &
        {$I/ \unit{\micro\ampere}$} &
        {$\text{grad}_i/ (\unit{\micro\ampere\per\volt})$} \\
        \midrule
        0   & 0.0       & 0.0    & 1.0 \\
        1   & 5.0       & 5.0    & 1.2 \\
        2   & 10.0      & 11.0    & 1.2 \\
        3   & 15.0      & 17.0    & 1.4 \\
        4   & 20.0      & 24.0    & 0.8 \\
        5   & 25.0      & 28.0    & 0.8 \\
        6   & 30.0      & 32.0    & 0.6 \\
        7   & 35.0      & 35.0    & 0.4 \\
        8   & 40.0      & 37.0    & 0.2 \\
        9   & 50.0      & 39.0    & 0.1 \\
        10   & 60.0     & 40.0    & 0.0 \\
        11   & 70.0     & 40.0    & 0.0 \\
        12   & 80.0     & 40.0    & {-} \\
        \bottomrule
    \end{tabular}
    \caption{Messdaten für $I_\text{f}= \qty{2.0}{\ampere}$}
    \label{tab:c01}
\end{table}
\begin{table}
    \centering
    \sisetup{table-format=:.1}
    \begin{tabular}{S[table-format=:.0] S S S}
        \toprule
        {$i$} &
        {$U / \unit{\volt}$} &
        {$I/ \unit{\micro\ampere}$} &
        {$\text{grad}_i/ (\unit{\micro\ampere\per\volt})$} \\
        \midrule
        0   & 0.0       & 0.0      & 2.2 \\
        1   & 5.0       & 11.0     & 3.2 \\
        2   & 10.0      & 27.0     & 3.6 \\
        3   & 15.0      & 45.0     & 3.8 \\
        4   & 20.0      & 64.0     & 5.4 \\
        5   & 25.0      & 91.0     & 4.8 \\
        6   & 30.0      & 115.0    & 5.0 \\
        7   & 35.0      & 140.0    & 5.8 \\
        8   & 40.0      & 169.0    & 4.6 \\
        9   & 45.0      & 192.0    & 6.0 \\
        10  & 50.0      & 222.0    & 6.1 \\
        11  & 60.0      & 283.0    & 5.2 \\
        12  & 70.0      & 335.0    & 4.8 \\
        13  & 80.0      & 383.0    & 5.3 \\
        14  & 90.0      & 436.0    & 4.5 \\
        15  & 100.0     & 481.0    & 2.9 \\
        16  & 110.0     & 510.0    & 1.5 \\
        17  & 120.0     & 525.0    & 0.9 \\
        18  & 130.0     & 534.0    & 1.2 \\
        19  & 140.0     & 546.0    & 0.8 \\
        20  & 150.0     & 554.0    & 0.7 \\
        21  & 160.0     & 561.0    & 0.5 \\
        22  & 180.0     & 571.0    & 0.35 \\
        23  & 200.0     & 578.0    & 0.25 \\
        24  & 220.0     & 583.0    & 0.25 \\
        25  & 240.0     & 588.0    & 0.3 \\
        26  & 250.0     & 591.0    & {-} \\
        \bottomrule
    \end{tabular}
    \caption{}
    % \label{}
\end{table}
\begin{table}
    \centering
    \sisetup{table-format=:.1}
    \begin{tabular}{S[table-format=:.0] S S S S[table-format=:.0] S S S}
        \toprule
        {$i$} &
        {$U / \unit{\volt}$} &
        {$I/ \unit{\micro\ampere}$} &
        {$\text{grad}_i/ (\unit{\micro\ampere\per\volt})$} &
        {$i$} &
        {$U / \unit{\volt}$} &
        {$I/ \unit{\micro\ampere}$} &
        {$\text{grad}_i/ (\unit{\micro\ampere\per\volt})$} \\
        \midrule
        0    & 0.0      & 0.0       & 2.8   &        15   & 100.0    & 750.0     & 11.1   \\
        1    & 5.0      & 14.0      & 3.2   &        16   & 110.0    & 861.0     & 12.5   \\
        2    & 10.0     & 30.0      & 3.8   &        17   & 120.0    & 986.0     & 13.0   \\
        3    & 15.0     & 49.0      & 4.6   &        18   & 130.0    & 1116.0    & 9.9    \\
        4    & 20.0     & 72.0      & 6.2   &        19   & 140.0    & 1215.0    & 9.1    \\
        5    & 25.0     & 103.0     & 5.6   &        20   & 150.0    & 1306.0    & 9.4    \\
        6    & 30.0     & 131.0     & 8.0   &        21   & 160.0    & 1400.0    & 8.8    \\
        7    & 35.0     & 171.0     & 7.2   &        22   & 170.0    & 1488.0    & 8.6    \\
        8    & 40.0     & 207.0     & 6.8   &        23   & 180.0    & 1574.0    & 8.7    \\
        9    & 45.0     & 241.0     & 8.2   &        24   & 190.0    & 1661.0    & 9.0    \\
        10   & 50.0     & 282.0     & 8.9   &        25   & 200.0    & 1751.0    & 6.9    \\
        11   & 60.0     & 371.0     & 10.8  &        26   & 210.0    & 1820.0    & 7.0    \\
        12   & 70.0     & 479.0     & 8.8   &        27   & 220.0    & 1890.0    & 7.1    \\
        13   & 80.0     & 567.0     & 11.0  &        28   & 230.0    & 1961.0    & {-}    \\
        14   & 90.0     & 677.0     & 7.3   & & & & \\

        \bottomrule
    \end{tabular}
    \caption{Messdaten für $I_\text{f}= \qty{2.5}{\ampere}$}
    \label{tab:c03}
\end{table}