\section{Auswertung}

\subsection{Kennlinien und Raumkurve}
In den folgenden Messungen wird der Sättigungsstrom und der Gültigkeitsbereich des
Langmuir-Schottkyschen Raumladungsgesetzes \eqref{eq:lang_schott} überprüft.
Dazu wird die Anodenspannung $U_A$ mit den an der Diode gemessenen Strömen $I$ dargestellt.
Zu jedem Wertepaar wird eine Steigung berechnet
\begin{align}
    \text{grad}_{i} = \frac{I_{i+1}-I_{i}}{U_{i+1}- U_{i}}.
\end{align}
Anhand dieser Steigundsdreiecke lässt sich ein Wendepunkt abschätzen, vor dem das Langmuir-Schottkysche Raumladungsgesetz gilt.
Für die Wertepaare wird eine Funktion der Form
\begin{align}
    a \cdot U^{b}
    \label{eq:raum_fit}
\end{align}
gewählt.
Mit dieser Form wird der Exponent dieses Zusammenhangs ermittelt.
Für die Sättigungskurve wird ein exponentieller Zusammenhang der Form
\begin{align}
    -a \cdot e^{b\cdot (U-d)} + c
\end{align}
verwendet.
von besonderem Interesse ist hier nur der Wert für $c$, da er dem Sättigungsstrom $I_s$ entspricht.
In den Messungen bei $I_\text{f} = \qty{2.3}{\ampere}$ und $\qty{2.5}{\ampere}$ lässt sich diese Ausgleichsrechnug nicht mehr
durchführen, da die ströme keinen Sättigungswert erreichen.
Deshalb wird für diese Kurven der Sättigungsstrom anhand der graphischen Darstellung abgeschätzt.


\begin{table}
    \centering
    \sisetup{table-format=:.1}
    \begin{tabular}{S[table-format=:.0] S S S}
        \toprule
        {$i$} &
        {$U / \unit{\volt}$} &
        {$I/ \unit{\micro\ampere}$} &
        {$\text{grad}_i/ (\unit{\micro\ampere\per\volt})$} \\
        \midrule
        0   & 0.0       & 0.0    & 1.0 \\
        1   & 5.0       & 5.0    & 1.2 \\
        2   & 10.0      & 11.0    & 1.2 \\
        3   & 15.0      & 17.0    & 1.4 \\
        4   & 20.0      & 24.0    & 0.8 \\
        5   & 25.0      & 28.0    & 0.8 \\
        6   & 30.0      & 32.0    & 0.6 \\
        7   & 35.0      & 35.0    & 0.4 \\
        8   & 40.0      & 37.0    & 0.2 \\
        9   & 50.0      & 39.0    & 0.1 \\
        10   & 60.0     & 40.0    & 0.0 \\
        11   & 70.0     & 40.0    & 0.0 \\
        12   & 80.0     & 40.0    & {-} \\
        \bottomrule
    \end{tabular}
    \caption{}
    % \label{}
\end{table}
\begin{table}[h]
    \centering
    \sisetup{table-format=:.1}
    \begin{tabular}{S[table-format=:.0] S S S S[table-format=:.0] S S S}
        \toprule
        {$i$} &
        {$U / \unit{\volt}$} &
        {$I/ \unit{\micro\ampere}$} &
        {$\text{grad}_i/ (\unit{\micro\ampere\per\volt})$}&
        {$i$} &
        {$U / \unit{\volt}$} &
        {$I/ \unit{\micro\ampere}$} &
        {$\text{grad}_i/ (\unit{\micro\ampere\per\volt})$} \\
        \midrule
        0   & 0.0       & 0.0      & 2.2  &       14  & 90.0      & 436.0    & 4.5 \\
        1   & 5.0       & 11.0     & 3.2  &       15  & 100.0     & 481.0    & 2.9 \\
        2   & 10.0      & 27.0     & 3.6  &       16  & 110.0     & 510.0    & 1.5 \\
        3   & 15.0      & 45.0     & 3.8  &       17  & 120.0     & 525.0    & 0.9 \\
        4   & 20.0      & 64.0     & 5.4  &       18  & 130.0     & 534.0    & 1.2 \\
        5   & 25.0      & 91.0     & 4.8  &       19  & 140.0     & 546.0    & 0.8 \\
        6   & 30.0      & 115.0    & 5.0  &       20  & 150.0     & 554.0    & 0.7 \\
        7   & 35.0      & 140.0    & 5.8  &       21  & 160.0     & 561.0    & 0.5 \\
        8   & 40.0      & 169.0    & 4.6  &       22  & 180.0     & 571.0    & 0.35 \\
        9   & 45.0      & 192.0    & 6.0  &       23  & 200.0     & 578.0    & 0.25 \\
        10  & 50.0      & 222.0    & 6.1  &       24  & 220.0     & 583.0    & 0.25 \\
        11  & 60.0      & 283.0    & 5.2  &       25  & 240.0     & 588.0    & 0.3 \\
        12  & 70.0      & 335.0    & 4.8  &       26  & 250.0     & 591.0    & {-} \\
        13  & 80.0      & 383.0    & 5.3  & & & &  \\

        \bottomrule
    \end{tabular}
    \caption{Messdaten für $I_\text{f}= \qty{2.3}{\ampere}$}
    \label{tab:c02}
\end{table}
\begin{table}
    \centering
    \begin{tabular}{S S S S }
        \toprule
        {$\theta / \unit{\degree}$} & {$N / \frac{1}{5\unit{\s}}$} &
        {$\theta / \unit{\degree}$} & {$N / \frac{1}{5\unit{\s}}$} \\
        \midrule
        19.0    & 130.0    &     21.7    & 139.0  \\    
        19.1    & 126.0    &     21.8    & 150.0  \\    
        19.2    & 126.0    &     21.9    & 157.0  \\    
        19.3    & 129.0    &     22.0    & 181.0  \\    
        19.4    & 139.0    &     22.1    & 255.0  \\    
        19.5    & 128.0    &     22.2    & 597.0  \\    
        19.6    & 139.0    &     22.25   & 2343.0 \\    
        19.7    & 0.0      &     22.4    & 4380.0 \\    
        19.75   & 146.0    &     22.5    & 4633.0 \\    
        19.9    & 185.0    &     22.6    & 4155.0 \\    
        20.0    & 572.0    &     22.7    & 2632.0 \\    
        20.1    & 1187.0   &     22.75   & 1325.0 \\    
        20.2    & 1298.0   &     22.9    & 262.0  \\    
        20.25   & 1268.0   &     23.0    & 196.0  \\    
        20.4    & 909.0    &     23.1    & 160.0  \\    
        20.5    & 440.0    &     23.2    & 139.0  \\    
        20.6    & 206.0    &     23.3    & 128.0  \\    
        20.7    & 179.0    &     23.4    & 110.0  \\    
        20.75   & 176.0    &     23.5    & 112.0  \\    
        20.9    & 160.0    &     23.6    & 106.0  \\    
        21.0    & 144.0    &     23.7    & 102.0  \\    
        21.1    & 143.0    &     23.8    & 99.0   \\    
        21.2    & 138.0    &     23.9    & 96.0   \\    
        21.3    & 135.0    &     24.0    & 100.0  \\    
        21.4    & 131.0    &             &        \\
        21.5    & 131.0    &             &        \\
        21.6    & 129.0    &             &        \\
        \bottomrule
    \end{tabular}
    \caption{Genaueres Spektrum der Cu-Röhre, Integrationszeit = \qty{5}{\s}, Schrittgröße \qty{0.1}{\degree}}
    \label{tab:c03_Cu-Roere_fein}
\end{table}