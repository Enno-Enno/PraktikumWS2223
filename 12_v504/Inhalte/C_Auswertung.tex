\section{Auswertung}

\subsection{Kennlinien und Raumkurve}
In den folgenden Messungen wird der Sättigungsstrom und der Gültigkeitsbereich des
Langmuir-Schottkyschen Raumladungsgesetzes \eqref{eq:lang_schott} überprüft.
Dazu wird die Anodenspannung $U_A$ mit den an der Diode gemessenen Strömen $I$ dargestellt.
Zu jedem Wertepaar wird eine Steigung berechnet
\begin{align}
    \text{grad}_{i} = \frac{I_{i+1}-I_{i}}{U_{i+1}- U_{i}}.
\end{align}
Anhand dieser Steigundsdreiecke lässt sich ein Wendepunkt abschätzen, vor dem das Langmuir-Schottkysche Raumladungsgesetz gilt.
Für die Wertepaare wird eine Funktion der Form
\begin{align}
    a \cdot U^{b}
    \label{eq:raum_fit}
\end{align}
gewählt.
Mit dieser Form wird der Exponent $b$ dieses Zusammenhangs ermittelt.
Für die Sättigungskurve wird ein exponentieller Zusammenhang der Form
\begin{align}
    -a \cdot e^{b\cdot (U-d)} + c
    \label{eq:sat}
\end{align}
verwendet.
von besonderem Interesse ist hier nur der Wert für $c$, da er dem Sättigungsstrom $I_s$ entspricht.
In den Messungen bei $I_\text{f} = \qty{2.3}{\ampere}$ und $\qty{2.5}{\ampere}$ lässt sich diese Ausgleichsrechnung nicht mehr
durchführen, da die ströme keinen Sättigungswert erreichen.
Deshalb wird für diese Kurven der Sättigungsstrom anhand der graphischen Darstellung abgeschätzt.
Diese einschätzung ist ungenauer als die mit dem Curve fit.

\subsection{Messung 1}
\begin{table}
    \centering
    \sisetup{table-format=:.1}
    \begin{tabular}{S[table-format=:.0] S S S}
        \toprule
        {$i$} &
        {$U / \unit{\volt}$} &
        {$I/ \unit{\micro\ampere}$} &
        {$\text{grad}_i/ (\unit{\micro\ampere\per\volt})$} \\
        \midrule
        0   & 0.0       & 0.0    & 1.0 \\
        1   & 5.0       & 5.0    & 1.2 \\
        2   & 10.0      & 11.0    & 1.2 \\
        3   & 15.0      & 17.0    & 1.4 \\
        4   & 20.0      & 24.0    & 0.8 \\
        5   & 25.0      & 28.0    & 0.8 \\
        6   & 30.0      & 32.0    & 0.6 \\
        7   & 35.0      & 35.0    & 0.4 \\
        8   & 40.0      & 37.0    & 0.2 \\
        9   & 50.0      & 39.0    & 0.1 \\
        10   & 60.0     & 40.0    & 0.0 \\
        11   & 70.0     & 40.0    & 0.0 \\
        12   & 80.0     & 40.0    & {-} \\
        \bottomrule
    \end{tabular}
    \caption{}
    % \label{}
\end{table}% \label{tab:c01}
\begin{figure}
    \centering
    \includegraphics[width=0.75\textwidth]{12_v504/build/C01_plot.pdf}
    \caption{}
    \label{fig:c01}
\end{figure} % \label{fig:c01}
Bei dieser Messung wird ein Heizstrom von $I_\text{f} = \qty{2.0}{\ampere}$ verwendet.
Aus den Steigungsdreiecken in Tabelle \ref{tab:c01} ergibt sich ein Wendepunkt bei $U_A = \qty{15}{\volt}$.
Dementsprechend wird die Ausgleichsrechnung mit der Funktion \eqref{eq:raum_fit} für $i \in [0,3]$ durchgeführt
Es ergeben sich Parameter von 
\begin{align}
    a &= \num{0.862 \pm 0.034}, & b &= \num{1.102 \pm 0.015}.
\end{align}
Der Exponent dieses Zusammenhangs ist also $b = \num{1.102 \pm 0.015}$

Die Sättigungskurve \eqref{eq:sat} ergibt in ihrer Ausgleichsrechnung eine Verschiebung bzw. einen Sättigungsstrom von 
\begin{align}
    c= I_s = \qty{40.669 \pm 0.438}{\micro\ampere}.
\end{align}


\subsection{Messung 2}
\begin{table}[h]
    \centering
    \sisetup{table-format=:.1}
    \begin{tabular}{S[table-format=:.0] S S S S[table-format=:.0] S S S}
        \toprule
        {$i$} &
        {$U / \unit{\volt}$} &
        {$I/ \unit{\micro\ampere}$} &
        {$\text{grad}_i/ (\unit{\micro\ampere\per\volt})$}&
        {$i$} &
        {$U / \unit{\volt}$} &
        {$I/ \unit{\micro\ampere}$} &
        {$\text{grad}_i/ (\unit{\micro\ampere\per\volt})$} \\
        \midrule
        0   & 0.0       & 0.0      & 2.2  &       14  & 90.0      & 436.0    & 4.5 \\
        1   & 5.0       & 11.0     & 3.2  &       15  & 100.0     & 481.0    & 2.9 \\
        2   & 10.0      & 27.0     & 3.6  &       16  & 110.0     & 510.0    & 1.5 \\
        3   & 15.0      & 45.0     & 3.8  &       17  & 120.0     & 525.0    & 0.9 \\
        4   & 20.0      & 64.0     & 5.4  &       18  & 130.0     & 534.0    & 1.2 \\
        5   & 25.0      & 91.0     & 4.8  &       19  & 140.0     & 546.0    & 0.8 \\
        6   & 30.0      & 115.0    & 5.0  &       20  & 150.0     & 554.0    & 0.7 \\
        7   & 35.0      & 140.0    & 5.8  &       21  & 160.0     & 561.0    & 0.5 \\
        8   & 40.0      & 169.0    & 4.6  &       22  & 180.0     & 571.0    & 0.35 \\
        9   & 45.0      & 192.0    & 6.0  &       23  & 200.0     & 578.0    & 0.25 \\
        10  & 50.0      & 222.0    & 6.1  &       24  & 220.0     & 583.0    & 0.25 \\
        11  & 60.0      & 283.0    & 5.2  &       25  & 240.0     & 588.0    & 0.3 \\
        12  & 70.0      & 335.0    & 4.8  &       26  & 250.0     & 591.0    & {-} \\
        13  & 80.0      & 383.0    & 5.3  & & & &  \\

        \bottomrule
    \end{tabular}
    \caption{Messdaten für $I_\text{f}= \qty{2.3}{\ampere}$}
    \label{tab:c02}
\end{table}
\begin{figure}
    \centering
    \includegraphics[width=0.75\textwidth]{12_v504/build/C02_plot.pdf}
    \caption{Ausgleichskurve und Kennlinie bei $I_f = \qty{2.3}{\ampere}$}
    \label{fig:c02}
\end{figure}
Bei dieser Messung wird ein Heizstrom von $I_\text{f}= \qty{2.3}{\ampere}$ verwendet.
Der Wendepunkt in den Messdaten aus Tabelle \ref{tab:c02} befindet sich bei \qty{50}{\volt}.
Für die Ausgleichsrechnung der Raumkurve werden also die Messungen $i \in [0,10]$ herangezogen.
Es ergeben sich Parameter von
\begin{align}
    a &= \num{1.317 \pm 0.073}, & b &= \num{1.311 \pm 0.015}.
\end{align}
Für den Sättigungsstrom wird aus Grafik \ref{fig:c02} ein Wert von $I_s = \qty{600 \pm 10}{\micro\ampere}$ abgeschätzt.


\subsection{Messung 3}
\begin{table}
    \centering
    \begin{tabular}{S S S S }
        \toprule
        {$\theta / \unit{\degree}$} & {$N / \frac{1}{5\unit{\s}}$} &
        {$\theta / \unit{\degree}$} & {$N / \frac{1}{5\unit{\s}}$} \\
        \midrule
        19.0    & 130.0    &     21.7    & 139.0  \\    
        19.1    & 126.0    &     21.8    & 150.0  \\    
        19.2    & 126.0    &     21.9    & 157.0  \\    
        19.3    & 129.0    &     22.0    & 181.0  \\    
        19.4    & 139.0    &     22.1    & 255.0  \\    
        19.5    & 128.0    &     22.2    & 597.0  \\    
        19.6    & 139.0    &     22.25   & 2343.0 \\    
        19.7    & 0.0      &     22.4    & 4380.0 \\    
        19.75   & 146.0    &     22.5    & 4633.0 \\    
        19.9    & 185.0    &     22.6    & 4155.0 \\    
        20.0    & 572.0    &     22.7    & 2632.0 \\    
        20.1    & 1187.0   &     22.75   & 1325.0 \\    
        20.2    & 1298.0   &     22.9    & 262.0  \\    
        20.25   & 1268.0   &     23.0    & 196.0  \\    
        20.4    & 909.0    &     23.1    & 160.0  \\    
        20.5    & 440.0    &     23.2    & 139.0  \\    
        20.6    & 206.0    &     23.3    & 128.0  \\    
        20.7    & 179.0    &     23.4    & 110.0  \\    
        20.75   & 176.0    &     23.5    & 112.0  \\    
        20.9    & 160.0    &     23.6    & 106.0  \\    
        21.0    & 144.0    &     23.7    & 102.0  \\    
        21.1    & 143.0    &     23.8    & 99.0   \\    
        21.2    & 138.0    &     23.9    & 96.0   \\    
        21.3    & 135.0    &     24.0    & 100.0  \\    
        21.4    & 131.0    &             &        \\
        21.5    & 131.0    &             &        \\
        21.6    & 129.0    &             &        \\
        \bottomrule
    \end{tabular}
    \caption{Genaueres Spektrum der Cu-Röhre, Integrationszeit = \qty{5}{\s}, Schrittgröße \qty{0.1}{\degree}}
    \label{tab:c03_Cu-Roere_fein}
\end{table}
\begin{figure}[h]
    \centering
    \includegraphics[width=0.75\textwidth]{12_v504/build/C03_plot.pdf}
    \caption{Ausgleichskurve und Kennlinie bei $I_f = \qty{2.5}{\ampere}$}
    \label{fig:c03}
\end{figure}
Schließlich wird der maximale Heizstrom von $I_\text{f}= \qty{2.5}{\ampere}$ verwendet.
Für den Wendepunkt ergibt sich aus Tabelle \ref{tab:c03} eine Spannung von \qty{120}{\volt}.
Die Raumkurve, wie in Abbildung \ref{fig:c03} dargestellt, hat die Parameter
\begin{align}
    a &= \num{1.154 \pm 0.083}, & b &= \num{1.410 \pm 0.016}.
\end{align}
Der Sättigungsstrom wird auf $I_s= \qty{2300 \pm 250}{\micro\ampere}$ abgeschätzt.






\subsection{Der Anlaufstrom}
\begin{table}
    \centering
    \begin{tabular}{S S}
        {$U_A / \unit{\volt}$}{$I/\unit{\ampere}$}\\
        -0.0067   & 6.70 \\
        -0.0451   & 5.10 \\
        -0.0840   & 4.00 \\
        -0.1231   & 3.10 \\
        -0.1624   & 2.40 \\
        -0.2017   & 1.70 \\
        -0.2413   & 1.30 \\
        -0.2809   & 0.90 \\
        -0.3206   & 0.60 \\
        -0.3603   & 0.30 \\
        -0.4007   & 0.75 \\
        -0.4406   & 0.58 \\
        -0.4804   & 0.45 \\
        -0.5203   & 0.32 \\
        -0.5602   & 0.24 \\
        -0.6001   & 0.15 \\
        -0.7000   & 0.03 \\        
    \end{tabular}
    \caption{Messwerte für den Anlaufstrom}
    \label{tab:c04}
\end{table}
\begin{figure}
    \centering
    \includegraphics[width=0.75\textwidth]{12_v504/build/C04_plot.pdf}
    \caption{Messwerte und Ausgleichskurve für den Anlaufstrom}
    \label{fig:c03}
\end{figure}

Für die umgekehrt angeschlossene Stromquelle wird der Anlaufstrom gemessen.
Die eingestellte Spannung fällt nicht nur über der Diode sondern auch an dem Spannungsmessgerät ab.
Für den Widerstand des Messgerätes wird ein Spannungsabfall von $R = \qty{1}{\mega\ohm}$ angenommen
Es ergibt sich eine Korrektur von
\begin{align}
    U_A = U_\text{gemessen} + R \cdot I.
\end{align}
Der Spannungsabfall am Messgerät wird positiv angerechnet, da die angelegten Spannungen in Tabelle \ref{tab:c04} negativ notiert werden.
In Tabelle \ref{tab:c04} wird der korrigierte Anlaufstrom mit nur zwei Nachkommastellen dargestellt, da die Werte nicht genauer bestimmt wurden.
Die Korrekturen sind in einer Größenordnung von \qty{1e-3}{\volt} weshalb sie in der Rundung nur selten einen Unterschied ausmachen.

Mit den korrigierten Werten von $U_A$ wird eine Ausgleichsrechnung für ein exponentielles Wachstum aufgestellt.
\begin{align}
    a \cdot \text{e}^{b\cdot U_A} 
\end{align}
Es ergeben sich die Parameter
\begin{align}
    a &= \qty{1.154 \pm 0.083}{\nano\ampere}, & b &= \num{6.513 \pm 0.196}.
\end{align}
In Abbildung \ref{fig:c04} wird die Anlaufkurve dargestellt.
Nach Gleichung \eqref{eq:teo_anlauf} ergibt sich für die Temperatur eine Formel von
\begin{align}
    T = \frac{e_0}{k_B b}
\end{align}
Für die Temperatur ergibt sich so ein wert von T = \qty{1780 +- 50}{\kelvin}

\subsection{Temperatur des Glühdrahts}
Nach Gleichung \eqref{eq:leistung} ergibt sich für die Temperatur die Formel
\begin{align}
    T = \sqrt[4]{\frac{I_\text{Heiz} U_\text{Heiz}-N_\text{Leit}}{F \eta \sigma}}
\end{align}
Für die Heizspannung wird von einem Ablesefehler von \qty{\pm 0.5}{\volt} ausgegangen.
Der Fehler bei dem Strom wird auf \qty{\pm 0.05}{\ampere} abgeschätzt.
Für die Wärmeleitung wird der Wert $N_\text{Leit}= \qty{1}{\watt}$ gewählt.
Die weiteren Daten finden sich in Abschnitt \ref{sec:teo_temp}.
Die Fehler der Temperatur werden mit der gaußschen Fehlerforpflanzung berechnet.
\begin{align}
    \Delta f(x_i) = \sqrt{
    \left(\frac{\partial f}{\partial x_1} \Delta x_1\right)^2%
     + \left(\frac{\partial f}{\partial x_2} \Delta x_2\right)^2%
     + \dots%
     + \left(\frac{\partial f}{\partial x_k} \Delta x_k\right)^2%
    }.
    \label{eq:gauss}
\end{align}
Die Temperaturen sind in Tabelle \ref{tab:c05} angegeben
\begin{table}
    \centering
    \begin{tabular}{S S S}
        \toprule
        {$I_\text{Heiz}/\unit{\ampere}$} & {$U_\text{Heiz}/\unit{\volt}$} & {$T / \unit{\kelvin}$} \\
        \midrule
        2.00 \pm 0.05  &  4.0 \pm 0.5  &  1924 \pm  70 \\
        2.30 \pm 0.05  &  5.0 \pm 0.5  &  2129 \pm  60 \\
        2.50 \pm 0.05  &  5.0 \pm 0.5  &  2178 \pm  60 \\
        \bottomrule
    \end{tabular}
    \caption{Temperaturen des Glühdrahts}
    \label{tab:c05}
\end{table}

\section{Austrittsarbeit des Kathodenmaterials}

In Gleichung \eqref{eq:austritt_richardson} wurde bereits die Formel für die Austrittsarbeit der Kathodenoberfläche aufgestellt.
\begin{align*}
    W = e_0 \xi = - k_\text{B} T \cdot \ln\left(\frac{h^3 j_\text{S}}{4 \pi e_0 m_0 k_\text{B}^2 T^2}\right)
\end{align*}
Mit der Boltzmannkonstante $k_\text{B}$, 
Der Sättigungsstrom $I_S$ taucht in der Formel in dem elektrischen Fluss auf,  der aus der Kathode austritt.
Es gilt also
\begin{align*}
    j_s = \frac{I_S}{F}.  
\end{align*}

\begin{table}
    \centering
    \begin{tabular}{S S c}
        \toprule
        {$T/ \unit{\kelvin}$} & {$I_S / \unit{\micro\ampere}$} & {$W/ \unit{\joule}$}\\
        \midrule
        1924.1 +- 70.1         &   40.669 +-  0.438    &  $(7.68  \pm 0.299)\cdot 10^{-19}$ \\
        2129.4 +- 59.7         &   600.0  +-  10.0     &  $(7.76  \pm 0.234)\cdot 10^{-19}$ \\
        2178.4 +- 60.4         &   2300.0 +-  250.0    &  $(7.55  \pm 0.228)\cdot 10^{-19}$ \\        
        \bottomrule
    \end{tabular}
    \caption{Die gemessenen Werte mit }
\end{table}

\noindent
Der Mittelwert der Ergebnisse $\overline{W}$ wird aus den mittleren gemessenen Werten für die $W_i$ berechnet
\begin{align}
    \overline{W} = \frac{\sum_{i=1}^{3} W_i}{3}
\end{align} 
Die Standartabweichung $\Delta W$ der Werte wird über die Standartabweichungen der $W_i$ berechnet.
\begin{align}
    \Delta W  = \frac{ \sqrt{\sum_{i=1}^{3} (\Delta W_i)^2} }{3}
\end{align}
Es ergibt sich ein wert von
\begin{align}
    W = \qty{7.66(15)e-19}{\joule}.
\end{align} 