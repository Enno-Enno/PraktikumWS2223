\section{Diskussion}

\subsection{Kennlinienschar und Sättigungsstrom}
Es war zu erwarten, dass für steigende Heizströme auch der Anodenstrom steigt, da bei höheren Temperaturen mehr
Elektronen emittiiert werden.
Die Plots der Kennlinienscharen \ref{fig:c01}, \ref{fig:c02} und \ref{fig:c03} sowie die resultierenden 
Sättigungsströme $I_{s,1} = \qty{40.669 \pm 0.438}{\micro\ampere}$, $I_{s,2} = \qty{600 \pm 10}{\micro\ampere}$ und 
$I_{s,3}= \qty{2300 \pm 250}{\micro\ampere}$ können dieser Erwartung gerecht werden.
Die verhältnismäßig hohen und steigenden Abweichungen in Messreihe 2 und 3 kommen dabei durch Abschätzungenauigkeiten zustande.
Ferner sei erwähnt, dass sich die Stromstärke von $\qty[]{0}{\micro\ampere}$ bei $\qty[]{0}{\volt}$ durch die Genauigkeit des 
Messgeräts ergibt.
Dies entspricht allerdings nicht dem tatsächlichen Wert, wie man bei der Untersuchung des Anlaufstromgebiets sehen kann.
Die eigentlichen Werte im Nanoamperebereich sind für das in diesem Teil verwendete Messgerät nicht detektierbar.




\subsection{Das Raumladungsgesetz}


%Leitung ziwcshen nA-meter und anode kurz genug?
% Der exponent der Raumkurve ist b