\section{Diskussion}

\subsection{Kennlinienschar und Sättigungsstrom}
Es war zu erwarten, dass für steigende Heizströme auch der Anodenstrom steigt, da bei höheren Temperaturen mehr
Elektronen emittiert werden.
Die Plots der Kennlinienscharen \ref{fig:c01}, \ref{fig:c02} und \ref{fig:c03} sowie die resultierenden 
Sättigungsströme $I_{s,1} = \qty{40.669 \pm 0.438}{\micro\ampere}$, $I_{s,2} = \qty{600 \pm 10}{\micro\ampere}$ und 
$I_{s,3}= \qty{2300 \pm 250}{\micro\ampere}$ können dieser Erwartung gerecht werden.
Außerdem haben die Plots eine ähnliche Form wie in der Theorie, vgl. \ref{fig:kennlinie}.
Die verhältnismäßig hohen und steigenden Abweichungen in Messreihe 2 und 3 kommen dabei durch Abschätzungenauigkeiten zustande,
da der jeweilige Wert hier nicht mehr im Messbereich liegt.
Ferner sei erwähnt, dass sich die Stromstärke von $\qty[]{0}{\micro\ampere}$ bei $\qty[]{0}{\volt}$ durch die Genauigkeit des 
Messgeräts ergibt.
Dies entspricht allerdings nicht dem tatsächlichen Wert, wie man bei der Untersuchung des Anlaufstromgebiets sehen kann.
Die eigentlichen Werte im Nanoamperebereich sind für das in diesem Teil verwendete Messgerät nicht detektierbar.




\subsection{Das Raumladungsgesetz}
Die Gültigkeit des Raumladungsgesetzes konnte zu einem Bereich von \qty[]{0}{\volt} bis \qty{120}{\volt} bestimmt werden.
Der Exponent der Strom-Spannungsbeziehung beläuft sich auf $b = \num{1.410 \pm 0.016}$, was einer Abweichung von 
$\num{6.0+-1.1}\, \%$ vom Literaturwert $b_\text{Lit} = \num[]{1.5}$ entspricht.
Eine Erklärung für diese Abweichung ist die Schwankung der Werte für $grad_i$ in Tabelle \ref{tab:c03} im Bereich \qtyrange[]{60}{120}{\volt},
da anhand dieser der Wendepunkt nur grob abgeschätzt werden kann.
Außerdem könnten Messungenauigkeiten ein weiterer Faktor sein, der die Abweichung vom Literaturwert erzeugt.
%Abweichung b =  0.060+/-0.011 = 6.0+/-1.1 %



\subsection{Das Anlaufstromgebiet}
Anhand der Ausgleichsrechnung der Messdaten und des resultierenden Plots \ref{fig:c04} kann der exponentielle Zusammenhang gemäß 
Gleichung \eqref{eq:teo_anlauf} verifiziert werden.
Für die maximale Heizspannung ergibt sich eine Temperatur von T = \qty{1780 +- 50}{\kelvin}, die
mitsamt der Abweichungen im angegebenen Intervall von \qtyrange[]{1000}{3000}{\kelvin}
(vgl. Abschnitt \ref{sec:diode}) liegt.

%Leitung ziwcshen nA-meter und anode kurz genug?
% Der exponent der Raumkurve ist b


\subsection{Kathodentemperatur}
Die Temperaturen in Tabelle \ref{tab:c04} liegen allesamt im Intervall von \qtyrange[]{1000}{3000}{\kelvin} und steigen wie zu erwarten
mit dem Heizstrom an.
An dieser Stelle können mögliche systematische Abweichungen nicht genauer quantifiziert werden, da beispielsweise die Wärmeleitung der Fadenhalterung
in \cite[]{man:v504} sowie die Fläche nur abgeschätzt wurden.
Dennoch ist davon auszugehen, dass die tatsächlichen Werte immer noch im gegebenen Intervall liegen.


\subsection{Austrittsarbeit von Wolfram}
Nach \cite[]{austrittsarbeit} beträgt die Austrittsarbeit in Wolfram \qtyrange[]{4.54}{4.60}{\electronvolt}.
Der experimentell bestimmte Wert $W = \qty{7.66(15)e-19}{\joule} = \qty[]{4.783 +- 0.092}{\electronvolt}$ hat demnach eine 
Abweichung von $\num[]{4.7+-2.1} \, \%$, was hinreichend genau ist.
%4.783 \pm 0.092
%Abweichung :  -0.047+/-0.021