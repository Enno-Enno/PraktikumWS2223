\documentclass[
  bibliography=totoc,     % Literatur im Inhaltsverzeichnis
  captions=tableheading,  % Tabellenüberschriften
  titlepage=firstiscover, % Titelseite ist Deckblatt (Finnd ich nicht so gut)
]{scrartcl}

% Paket float verbessern
\usepackage{scrhack}

% Warnung, falls nochmal kompiliert werden muss
\usepackage[aux]{rerunfilecheck}

% unverzichtbare Mathe-Befehle
\usepackage{amsmath}
% viele Mathe-Symbole
\usepackage{amssymb}
% Erweiterungen für amsmath
\usepackage{mathtools}

%\usepackage{physics}

% Fonteinstellungen
\usepackage{fontspec}
% Latin Modern Fonts werden automatisch geladen
% Alternativ zum Beispiel:
%\setromanfont{Libertinus Serif}
%\setsansfont{Libertinus Sans}
%\setmonofont{Libertinus Mono}

% Wenn man andere Schriftarten gesetzt hat,
% sollte man das Seiten-Layout neu berechnen lassen
\recalctypearea{}

% deutsche Spracheinstellungen
\usepackage[ngerman]{babel}


\usepackage[
  math-style=ISO,    % ┐
  bold-style=ISO,    % │
  sans-style=italic, % │ ISO-Standard folgen
  nabla=upright,     % │
  partial=upright,   % ┘
  warnings-off={           % ┐
    mathtools-colon,       % │ unnötige Warnungen ausschalten
    mathtools-overbracket, % │
  },                       % ┘
]{unicode-math}

% traditionelle Fonts für Mathematik
\setmathfont{Latin Modern Math}
% Alternativ zum Beispiel:
%\setmathfont{Libertinus Math}

\setmathfont{XITS Math}[range={scr, bfscr}]
\setmathfont{XITS Math}[range={cal, bfcal}, StylisticSet=1]

% Zahlen und Einheiten
\usepackage[
  locale=DE,                   % deutsche Einstellungen
  separate-uncertainty=true,   % immer Unsicherheit mit \pm
  per-mode=symbol-or-fraction, % / in inline math, fraction in display math
]{siunitx}

% chemische Formeln
\usepackage[
  version=4,
  math-greek=default, % ┐ mit unicode-math zusammenarbeiten
  text-greek=default, % ┘
]{mhchem}

% richtige Anführungszeichen
\usepackage[autostyle]{csquotes}

% schöne Brüche im Text
\usepackage{xfrac}

% Standardplatzierung für Floats einstellen
\usepackage{float}
\floatplacement{figure}{htbp}
\floatplacement{table}{htbp}

% Floats innerhalb einer Section halten
\usepackage[
  section, % Floats innerhalb der Section halten
  below,   % unterhalb der Section aber auf der selben Seite ist ok
]{placeins}

% Seite drehen für breite Tabellen: landscape Umgebung
\usepackage{pdflscape}

% Captions schöner machen.
\usepackage[
  labelfont=bf,        % Tabelle x: Abbildung y: ist jetzt fett
  font=small,          % Schrift etwas kleiner als Dokument
  width=0.9\textwidth, % maximale Breite einer Caption schmaler
]{caption}
% subfigure, subtable, subref
\usepackage{subcaption}

% Grafiken können eingebunden werden
\usepackage{graphicx}
\usepackage{wrapfig}

% schöne Tabellen
\usepackage{booktabs}

% Verbesserungen am Schriftbild
\usepackage{microtype}

% Literaturverzeichnis
\usepackage[
  backend=biber,
  sorting=none
]{biblatex}
% Quellendatenbank
\addbibresource{lit.bib}
\addbibresource{programme.bib}

% Hyperlinks im Dokument
\usepackage[
  german,
  unicode,        % Unicode in PDF-Attributen erlauben
  pdfusetitle,    % Titel, Autoren und Datum als PDF-Attribute
  pdfcreator={},  % ┐ PDF-Attribute säubern
  pdfproducer={}, % ┘
]{hyperref}
% erweiterte Bookmarks im PDF
\usepackage{bookmark}

% Trennung von Wörtern mit Strichen
\usepackage[shortcuts]{extdash}

%\setcounter{tocdepth}{3} % + subsubsections



\author{%
  Clara Sondermann \\%
  \href{mailto:clara.sondermann@tu-dortmund.de}{clara.sondermann@tu-dortmund.de}%
  \and%
  Enno Wellmann \\%
  \href{mailto:enno.wellmann@tu-dortmund.de}{enno.wellmann@tu-dortmund.de}%
}
\publishers{TU Dortmund – Fakultät Physik}


\newcommand*\diff{\mathop{}\!\mathrm{d}}

\NewDocumentCommand \OverfullCenter {+m} {
\noindent\makebox[\linewidth]{#1} }

\usepackage{adjustbox}


\title{Versuch 207: Kugelfall-Viskosimeter nach Höppler}
\date{Durchführung: 22.11.2022, Abgabe: 29.11.22}

%Assistentin hat im letzten Protokoll "gerne Deckblatt und Inhaltsverzeichnis" geschrieben

\begin{document}
\maketitle

\tableofcontents
\newpage

\section{Ziel}
In diesem Versuch wird die dynamische Viskosität
\footnote{Nicht zu verwechseln mit der \textit{kinematischen} Viskosität. Im folgenden wird die \textit{dynamische} Viskosität mit Viskosität bezeichnet.} $\eta$ 
von destilliertem Wasser in Abhängigkeit der Temperatur mit Hilfe des Kugelfall-Viskosimeters nach Höppler ermittelt.

\section{Vorbereitungsaufgaben}
\subsection{laminare Strömung und Reynoldszahl}
Von einer laminaren Strömung spricht man, wenn sich in einem realen Stoff, d.h. einer Flüssigkeit oder einem Gas,
die einzelnen Schichten mit verchiedenen Geschwindigkeiten nicht vermischen. Das bedeutet, dass es keine Wirbel oder
Turbulenzen innerhalb des Stoffes gibt.
Die Reynoldszahl $Re$ ist ein Maß dafür, ob es sich um eine laminare oder turbulente Strömung handelt\cite*[]{geschke}.

\begin{align}
    Re = \frac{\rho \overline{v} R}{\eta}
\end{align}

Dabei ist $\rho$ die Dichte des Fluids, $\overline{v}$ die relative mittlere Geschwindigkeit zwischen Fluid und Körper und
$R$ die charakteristische Länge.
Wenn die Reynoldszahl unterhalb einer kritischen Grenze $Re_c$ liegt, kann man von einer laminaren Strömung sprechen.

\subsection{Temperaturabhängigkeiten Viskosität und Dichte von destilliertem Wasser}
Die Dichte $\rho$ von Wasser hat ihren Maximalwert bei $T_{max} = \qty[]{4}{\degreeCelsius}$ \cite*[]{geschke}.
Außerdem ist die Dichte von Wasser im festen Aggregatszustand höher als im flüssigen Zustand. 
Diese beiden Tatsachen werden auch als "Anomalie des Wassers" bezeichnet\cite*[]{demtroeder}.

Die Viskosität $\rho$ fällt exponentiell mit dem Kehrwert der Temperatur $\frac{1}{T}$.

\section{Grundlagen}
Die charakteristische Länge $R$ ist hier der Durchmesser der Kugel und die kritische Grenze der Reynoldszahl $Re_c$ beträgt ca. 2300.

\section{Durchführung}

\section{Messdaten}
Durchmesser kleine Kugel: $ d_k = \qty{15.61 \pm 0.01}{\mm}$\\
Masse kleine Kugel (gegeben): $ m_k = \qty{4.4531}{\g}$\\
Dichte kleine Kugel: $ \rho  = \left(\frac{4}{3}\Pi \left(\frac{d}{2}\right)^{3}\right)^{-1} m = \qty{2.2359e-3}{\g\per\mm}$\\
Durchmesser große Kugel: $ d_g = \qty{15.78 \pm 0.01}{\mm}$\\
Masse große Kugel (gegeben): $ m_g = \qty{4.9528}{\g}$

\begin{table}
    \caption{Kleine Kugel bei Zimmertemperatur; Fallhöhe = 10 cm}
    \label{tab:klKu_Zitemp}
    \centering
    \sisetup{table-format=2.2}
    \begin{tabular}{S S}
        \toprule
        {Fallzeit $ t / \unit{\s}$}\\
        {oben} & {unten}\\
        \midrule
        12.21 &  12.66 \\
        12.40 &  12.00 \\
        12.25 &  12.32 \\
        12.65 &  12.06 \\
        12.16 &  12.32 \\
        12.13 &  12.09 \\
        12.50 &  12.15 \\
        12.38 &  12.18 \\
        12.34 &  12.38 \\
        12.22 &  12.32 \\
        \bottomrule

    \end{tabular}
\end{table}

\section{Auswertung}

\section{Diskussion}

\section*{Anhang: Messdaten}

%\section*{Literatur}
\nocite{geschke}
\nocite{walcher}
\nocite{demtroeder}
\printbibliography

\end{document}