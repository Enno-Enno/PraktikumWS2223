\input{../header.tex}

\title{Versuch 207: Kugelfall-Viskosimeter nach Höppler}
\date{Durchführung: 22.11.2022, Abgabe: 29.11.22}

%Assistentin hat im letzten Protokoll "gerne Deckblatt und Inhaltsverzeichnis" geschrieben

\begin{document}
\maketitle

\tableofcontents
\newpage

\section{Ziel}
In diesem Versuch wird die dynamische Viskosität $\eta$ von destilliertem Wasser in Abhängigkeit der Temperatur mit Hilfe des 
Kugelfall-Viskosimeters nach Höppler ermittelt.

\section{Vorbereitungsaufgaben}
\subsection{laminare Strömung und Reynoldszahl}
Von einer laminaren Strömung spricht man, wenn sich in einem realen Stoff, d.h. einer Flüssigkeit oder einem Gas,
die einzelnen Schichten mit verchiedenen Geschwindigkeiten nicht vermischen. Das bedeutet, dass es keine Wirbel oder
Turbulenzen innerhalb des Stoffes gibt.
Die Reynoldszahl $Re$ ist ein Maß dafür, ob es sich um eine laminare oder turbulente Strömung handelt.

\begin{align}
    Re = \frac{\rho \overline{v} R}{\eta}
\end{align}

Dabei ist %...

\section{Grundlagen}

\section{Durchführung}

\section{Messdaten}
Durchmesser kleine Kugel: $ d_k = \qty{15.61 \pm 0.01}{\mm}$\\
Masse kleine Kugel (gegeben): $ m_k = \qty{4.4531}{\g}$\\
Dichte kleine Kugel: $ \rho  = \left(\frac{4}{3}\Pi \left(\frac{d}{2}\right)^{3}\right)^{-1} m = \qty{2.2359e-3}{\g\per\mm}$\\
Durchmesser große Kugel: $ d_g = \qty{15.78 \pm 0.01}{\mm}$\\
Masse große Kugel (gegeben): $ m_g = \qty{4.9528}{\g}$

\begin{table}
    \caption{Kleine Kugel bei Zimmertemperatur; Fallhöhe = 10 cm}
    \label{tab:klKu_Zitemp}
    \centering
    \sisetup{table-format=2.2}
    \begin{tabular}{S S}
        \toprule
        {Fallzeit $ t / \unit{\s}$}\\
        {oben} & {unten}\\
        \midrule
        12.21 &  12.66 \\
        12.40 &  12.00 \\
        12.25 &  12.32 \\
        12.65 &  12.06 \\
        12.16 &  12.32 \\
        12.13 &  12.09 \\
        12.50 &  12.15 \\
        12.38 &  12.18 \\
        12.34 &  12.38 \\
        12.22 &  12.32 \\
        \bottomrule

    \end{tabular}
\end{table}

\section{Auswertung}

\section{Diskussion}

\section*{Anhang: Messdaten}

\section*{Literatur}

\end{document}