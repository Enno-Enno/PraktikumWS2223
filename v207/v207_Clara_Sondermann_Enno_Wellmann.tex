\documentclass[
  bibliography=totoc,     % Literatur im Inhaltsverzeichnis
  captions=tableheading,  % Tabellenüberschriften
  titlepage=firstiscover, % Titelseite ist Deckblatt (Finnd ich nicht so gut)
]{scrartcl}

% Paket float verbessern
\usepackage{scrhack}

% Warnung, falls nochmal kompiliert werden muss
\usepackage[aux]{rerunfilecheck}

% unverzichtbare Mathe-Befehle
\usepackage{amsmath}
% viele Mathe-Symbole
\usepackage{amssymb}
% Erweiterungen für amsmath
\usepackage{mathtools}

%\usepackage{physics}

% Fonteinstellungen
\usepackage{fontspec}
% Latin Modern Fonts werden automatisch geladen
% Alternativ zum Beispiel:
%\setromanfont{Libertinus Serif}
%\setsansfont{Libertinus Sans}
%\setmonofont{Libertinus Mono}

% Wenn man andere Schriftarten gesetzt hat,
% sollte man das Seiten-Layout neu berechnen lassen
\recalctypearea{}

% deutsche Spracheinstellungen
\usepackage[ngerman]{babel}


\usepackage[
  math-style=ISO,    % ┐
  bold-style=ISO,    % │
  sans-style=italic, % │ ISO-Standard folgen
  nabla=upright,     % │
  partial=upright,   % ┘
  warnings-off={           % ┐
    mathtools-colon,       % │ unnötige Warnungen ausschalten
    mathtools-overbracket, % │
  },                       % ┘
]{unicode-math}

% traditionelle Fonts für Mathematik
\setmathfont{Latin Modern Math}
% Alternativ zum Beispiel:
%\setmathfont{Libertinus Math}

\setmathfont{XITS Math}[range={scr, bfscr}]
\setmathfont{XITS Math}[range={cal, bfcal}, StylisticSet=1]

% Zahlen und Einheiten
\usepackage[
  locale=DE,                   % deutsche Einstellungen
  separate-uncertainty=true,   % immer Unsicherheit mit \pm
  per-mode=symbol-or-fraction, % / in inline math, fraction in display math
]{siunitx}

% chemische Formeln
\usepackage[
  version=4,
  math-greek=default, % ┐ mit unicode-math zusammenarbeiten
  text-greek=default, % ┘
]{mhchem}

% richtige Anführungszeichen
\usepackage[autostyle]{csquotes}

% schöne Brüche im Text
\usepackage{xfrac}

% Standardplatzierung für Floats einstellen
\usepackage{float}
\floatplacement{figure}{htbp}
\floatplacement{table}{htbp}

% Floats innerhalb einer Section halten
\usepackage[
  section, % Floats innerhalb der Section halten
  below,   % unterhalb der Section aber auf der selben Seite ist ok
]{placeins}

% Seite drehen für breite Tabellen: landscape Umgebung
\usepackage{pdflscape}

% Captions schöner machen.
\usepackage[
  labelfont=bf,        % Tabelle x: Abbildung y: ist jetzt fett
  font=small,          % Schrift etwas kleiner als Dokument
  width=0.9\textwidth, % maximale Breite einer Caption schmaler
]{caption}
% subfigure, subtable, subref
\usepackage{subcaption}

% Grafiken können eingebunden werden
\usepackage{graphicx}
\usepackage{wrapfig}

% schöne Tabellen
\usepackage{booktabs}

% Verbesserungen am Schriftbild
\usepackage{microtype}

% Literaturverzeichnis
\usepackage[
  backend=biber,
  sorting=none
]{biblatex}
% Quellendatenbank
\addbibresource{lit.bib}
\addbibresource{programme.bib}

% Hyperlinks im Dokument
\usepackage[
  german,
  unicode,        % Unicode in PDF-Attributen erlauben
  pdfusetitle,    % Titel, Autoren und Datum als PDF-Attribute
  pdfcreator={},  % ┐ PDF-Attribute säubern
  pdfproducer={}, % ┘
]{hyperref}
% erweiterte Bookmarks im PDF
\usepackage{bookmark}

% Trennung von Wörtern mit Strichen
\usepackage[shortcuts]{extdash}

%\setcounter{tocdepth}{3} % + subsubsections



\author{%
  Clara Sondermann \\%
  \href{mailto:clara.sondermann@tu-dortmund.de}{clara.sondermann@tu-dortmund.de}%
  \and%
  Enno Wellmann \\%
  \href{mailto:enno.wellmann@tu-dortmund.de}{enno.wellmann@tu-dortmund.de}%
}
\publishers{TU Dortmund – Fakultät Physik}


\newcommand*\diff{\mathop{}\!\mathrm{d}}

\NewDocumentCommand \OverfullCenter {+m} {
\noindent\makebox[\linewidth]{#1} }

\usepackage{adjustbox}


\title{Versuch 207: Kugelfall-Viskosimeter nach Höppler}
\date{Durchführung: 22.11.2022, Abgabe: 29.11.22}


\begin{document}
\maketitle

\tableofcontents
\newpage



\section{Ziel}
In diesem Versuch wird die dynamische Viskosität\footnote{Nicht zu verwechseln mit der \textit{kinematischen} Viskosität.
Im folgenden wird die \textit{dynamische} Viskosität mit Viskosität bezeichnet.} $\eta$ 
von destilliertem Wasser in Abhängigkeit der Temperatur mit Hilfe des Kugelfall-Viskosimeters nach Höppler ermittelt.



\section{Grundlagen}
\label{sec:grundlagen}
Reale Stoffe (d.h. Flüssigkeiten oder Gase) sind \textit{strömend}, wenn sie einen makroskopischen Impuls haben.
Die thermische Bewegung der einzelnen Moleküle kann dabei vernachlässigt werden.
Die Strömung kann mit Hilfe der Strömungsgeschwindigkeit $\symbf{u}\left(\symbf{r},t\right)$ beschrieben werden.
Als \textit{Stromlinie} oder \textit{Stromfaden} bezeichnet man die Ortskurve, die ein Flüssigkeitselement, etwa ein Korkstück, durchläuft.
Eine Stromlinie wird durch Reibungskräfte im Medium beeinflusst \cite[]{demtroeder}.
Ein Maß für die Stärke der inneren Reibung bzw. die Zähigkeit des Stoffes ist die \textit{dynamische Viskosität} $\eta$.
Diese ist eine temperaturabhängige Materialkonstante \cite[]{geschke}.
Man unterscheidet hierbei zwischen idealen bzw. nicht-viskosen Stoffen, für die $\eta \simeq 0$ gilt, und viskosen Stoffen,
deren Strömungsverhalten in laminar und turbulent eingeteilt wird (vgl. Abschnitt \ref{sec:vorbereitung}).

Die Reibung, die eine Kugel mit Radius $r$ und Geschwindigkeit $v$ in einem Fluid erfährt wird mit Hilfe der Stokesschen Reibung 

\begin{align}
    F_R = 6 \pi \eta v r    
\end{align}

beschrieben.
Sie wirkt mit der Auftriebskraft $\symbf{F}_A$ der Schwerkraft $\symbf{F}_g$ entgegen. 
Mit steigender Geschwindigkeit $v$ wächst die Reibungskraft $\symbf{F}_R$, bis sich ein Kräftegleichgewicht einstellt
und $v$ einen konstanten Wert annimmt.
Dabei wird die Viskosität des Fluids mit der empirischen Formel 

\begin{align}
    \label{eq:empirisch}
    \eta = K \left(\rho_K -\rho_{Fl}\right) \cdot t
\end{align}

ermittelt, wobei $K$ eine Apparaturkonstante ist, die Fallhöhe und Kugelgeometrie berücksichtigt,
$\rho_K$ bzw. $\rho_{Fl}$ die Dichte der Kugel bzw. des Fluids sind und $t$ die Fallzeit darstellt.


Durch die \textit{Andradesche Gleichung} wird die Temperaturabhängigkeit der Viskosität von Flüssigkeiten dargestellt.
\begin{equation}
\label{eq:andrade}
    \eta \left(T\right) = A \exp{\left(\frac{B}{T}\right)}
\end{equation}
$A$ und $B$ sind dabei Konstanten \cite*[]{va207}.



%\section[]{Messaufgaben}
%Wollen wir das noch hier angeben?



\section{Vorbereitungsaufgaben}
\subsection{laminare Strömung und Reynoldszahl}
\label{sec:vorbereitung}
Von einer laminaren Strömung spricht man, wenn in einem realen Stoff
die einzelnen Stromfäden ohne sich zu beeinflussen neben einander liegen. Das bedeutet, dass es keine Wirbel oder
Turbulenzen innerhalb des Stoffes gibt und sich die einzelnen Schichten mit Strömungsgeschwindigkeiten nicht vermischen.
Die \textit{Reynoldszahl} $Re$ ist ein Maß dafür, ob es sich um eine laminare oder turbulente Strömung handelt.

\begin{align}
    Re = \frac{\rho \overline{v} R}{\eta}
\end{align}

Dabei ist $\rho$ die Dichte des Fluids, $\overline{v}$ die relative mittlere Geschwindigkeit zwischen Fluid und Körper und
$R$ die charakteristische Länge.
Wenn die Reynoldszahl unterhalb einer kritischen Grenze $Re_c$ liegt, kann man von einer laminaren Strömung sprechen.
Die charakteristische Länge $R$ ist hier der Durchmesser der Kugel und die kritische Grenze der Reynoldszahl $Re_c$ beträgt ca. 2300 \cite*[]{geschke}.


\subsection{Temperaturabhängige Viskosität und Dichte von destilliertem Wasser}
Die Dichte $\rho$ von Wasser hat ihren Maximalwert bei $T_{max} = \qty[]{4}{\degreeCelsius}$ \cite*[]{geschke}.
Außerdem ist $\rho$ im festen Aggregatszustand niedriger als im flüssigen Zustand. 
Diese beiden Tatsachen werden auch als \enquote{Dichteanomalie des Wassers} bezeichnet\cite*[]{demtroeder}.\\
Die Viskosität $\rho$ fällt exponentiell mit dem Kehrwert der Temperatur $\frac{1}{T}$ (vgl. \eqref{eq:andrade}).



\section[]{Versuchsaufbau}
Das Kugelfall-Viskosimeter nach Höppler besteht aus einem zylinderförmigen Glasrohr,
das mit einer Flüssigkeit gefüllt wird (hier destilliertes Wasser), deren Viskosität bestimmt werden soll. 
In dem befüllten Rohr lässt man eine Kugel fallen, deren Durchmesser nahe am Rohrdurchmeser liegt.
Es wurden hier zwei Kugeln mit unterschiedlichen Durchmessern verwendet. 
Um unkontrollierte Stöße mit der Rohrwand und Wirbelbildungen zu vermeiden, wird das Rohr etwas geneigt.
Am Rohr selbst befinden sich insgesamt drei Markierungen, die je einen Abstand von 5 Zentimetern zu einander haben.

Das Fallrohr befindet sich in einem Wasserbad, durch das mit Hilfe eines angeschlossenen Thermostates die Temperatur im Rohr geregelt werden kann.
Am Thermostat befindet sich ein Thermometer, sodass die Temperatur im Viskosimeter abgelesen werden kann.



\section{Durchführung}
Zu Beginn des Versuchs wurden die Durchmesser der Kugeln mit Hilfe einer Schieblehre bestimmt, damit man mit den vorgegebenen Massen die Dichten ermitteln kann.
Außerdem wurde die Zimmertemperatur erfasst.
Anschließend wurde der obere Verschluss geöffnet und das Fallrohr mit destilliertem Wasser befüllt. 
Danach wurde zunächst die kleinere Kugel in das Rohr gelassen.
Hierbei war darauf zu achten, dass sich keine Luftbläschen bildeten.
Die Bläschen, die nach dem Befüllen und Einlassen der Kugel vorhanden waren, wurden mit einer Bürste bzw. einem Glaskolben entfernt.
Anschließend wurde der Deckel verschlossen und darauf geachtet, dass sich keine weitere Luft im Fallrohr befand.

Bereits am Anfang des Versuchs war Wasser im Wasserbad vorhanden und die Temperatur wurde auf Zimmertemperatur reguliert,
damit für die ersten beiden Versuchsreihen die Temperatur im Rohr konstant gehalten werden konnte.

Die erste Versuchsreihe bestand darin, dass bei Zimmertemperatur die kleine Kugel fallen gelassen wurde.
Nachdem sich eine konstante Geschwindigkeit durch das in Abschnitt \ref{sec:grundlagen} genannte Kräftegleichgewicht eingestellt hatte,
wurde die Fallzeit bei einer Strecke von 10 Zentimetern (oberste bis unterste Markierung)
mit Hilfe einer Stoppuhr gemessen.
Nach jedem Fall wurde das Rohr um 180° und erneut gemessen.
Je nach Orientierung werden diese Messwerte im Folgenden mit \enquote{oben} bzw. \enquote{unten} betitelt.
% Anführungszeichen funktionieren nur mit \enquote{}
Dies wurde 10 Mal wiederholt, sodass sich insgesamt 20 Werte (je 10 oben und unten) ergaben.

Die zweite Messreihe verlief analog zur ersten mitsamt der Vorbereitungen, nur dass nun die größere Kugel auf einer Strecke von 5 Zentimetern betrachtet wurde.
Um die Kugeln auszutauschen, musste der Verschluss geöffnet werden, wodurch das destillierte Wasser aus dem Rohr entweichen konnte.
Auch hier mussten also beim erneuten befüllen etwaige Luftbläschen von Rohr und Kugel werden.
Außerdem wurden die Messungen hier nur 5 Mal wiederholt.

Die dritte Messreihe bestand darin, dass mit dem Thermostat die Temperatur langsam erhöht wurde
und dabei die Fallzeit der größeren Kugel auf 5 Zentimetern gemessen wurde.
Da die gleiche Kugel wie zuvor verwendet wurde, musste das Rohr nicht aufgeschraubt oder aufgefüllt werden.
Für jede Temperatur wurden je zwei Werte für oben und unten aufgenommen und es wurde für insgesamt 10 unterschiedliche Temperaturen gemessen.
Hierbei war zu beachten, dass sich erst die gewünschte Temperatur im Fallrohr einstellte, bevor man für die höhere Temperatur gemessen hat, 
um Messunsicherheiten zu vermeiden. Die Wartezeit betrug häufig ca. 3-5 Minuten. Maximal sollte eine Temperatur von 50°C erreicht werden.



\section{Messdaten}
\subsection{Kenngrößen der Kugeln}
%Durchmesser kleine Kugel: $ d_k = \qty{15.61 \pm 0.01}{\mm}$\\
%Masse kleine Kugel (gegeben): $ m_k = \qty{4.4531}{\g}$\\
%Dichte kleine Kugel: $ \rho  = \left(\frac{4}{3}\Pi \left(\frac{d}{2}\right)^{3}\right)^{-1} m = \qty{2.2359e-3}{\g\per\mm}$\\
%Durchmesser große Kugel: $ d_g = \qty{15.78 \pm 0.01}{\mm}$\\
%Masse große Kugel (gegeben): $ m_g = \qty{4.9528}{\g}$

Anfangs wurden die relevanten Kenngrößen der Kugeln bestimmt.
Die Massen waren vorgegeben.
Die Dichte $\rho$ errechnet sich aus dem Verhältnis zwischen Masse $m$ und Volumen
$V=\frac{4 \pi}{3} \left(\frac{d}{2}\right)^3$ und $\rho = \frac{m}{V}$.
Die Rechnung für die Fehlerfortpflanzung wird im Kapitel Auswertung noch einmal
ausführlicher behandelt.
Die Zimmertemperatur betrug stets $T_{Zt}=\qty[]{19}{\degreeCelsius}$.

\begin{table}[h!]
    \caption[]{Kenngrößen der kleinen und der großen Kugel. Masse $m$, Durchmesser $d$ und Dichte $\rho$.}
    \label{tab:kenngroessen}
    \centering
    \begin{tabular}[]{S S[table-format=1.4] S[table-format=2.2] @{${}\pm{}$} S[table-format=0.2] S[table-format=1.4]}
        \toprule
        {Kugel} & {$m / \unit{\kg}$} & \multicolumn{2}{c}{$d / \unit{\mm}$}  & {$\rho / \left( \unit{\mg \per \cubic\mm}\right)$} \\
        \midrule
        {klein} & 4.4531 & 15.61 & 0.01 & 2.2359 \\
        {groß}  & 2.9528 & 15.78 & 0.01 & 2.4073 \\
        \bottomrule 

    \end{tabular}
\end{table}
%ich fand das so als Tabelle etwas schöner hihi :D
%ich weiß nur nicht warum ziwschen Durchmesser und Fehler so eine Lücke ist...
%Lösung: Die Überschrift war zu breit. Mit nur d/mm passt es besser


\subsection{Messreihe 1: kleine Kugel, konstante Temperatur}

\begin{table}[h!]
    \caption{Kleine Kugel bei Zimmertemperatur; Fallhöhe = 10 cm}
    \label{tab:klKu_Zitemp}
    \centering
    \sisetup{table-format=2.2}
    \begin{tabular}{S S}
        \toprule
        \multicolumn{2}{c}{Fallzeit $ t / \unit{\s}$}\\
        {oben} & {unten}\\
        \midrule
        12.21 &  12.66 \\
        12.40 &  12.00 \\
        12.25 &  12.32 \\
        12.65 &  12.06 \\
        12.16 &  12.32 \\
        12.13 &  12.09 \\
        12.50 &  12.15 \\
        12.38 &  12.18 \\
        12.34 &  12.38 \\
        12.22 &  12.32 \\
        \bottomrule

    \end{tabular}
\end{table}





\subsection{Messreihe 2: große Kugel, konstante Temperatur}

\begin{table}[h!]
    \caption{Große Kugel bei Zimmertemperatur; Fallhöhe = 5 cm}
    \label{tab:grKu_Zitemp}
    \centering
    \sisetup{table-format=2.2}
    \begin{tabular}{S S}
        \toprule
        \multicolumn{2}{c}{Fallzeit $ t / \unit{\s}$}\\
        {oben} & {unten}\\
        \midrule
            34.78 &  34.75 \\
            34.65 &  34.94 \\
            35.00 &  36.90 \\
            35.22 &  35.63 \\
            35.47 &  35.78 \\
        \bottomrule

    \end{tabular}
\end{table}
%Durchschnittszeiten grosse Kugel:
%oben: 35.02400 mit Fehler:  0.29574
%unten: 35.60000 mit Fehler:  0.75913

\subsection{Messreihe 3: große Kugel, steigende Temperatur}


\begin{table}[h!]
    \caption{Große Kugel bei variabler Temperatur T; Fallhöhe = 5 cm}
    \label{tab:grKu_steigendeTemp}
    \centering
    \sisetup{table-format=2.2}
    \begin{tabular}{S[table-format=2.0] S S S[table-format=2.0] S S}
        \toprule
        & \multicolumn{2}{c}{Fallzeit $ t / \unit{\s}$} & & \multicolumn{2}{c}{Fallzeit $ t / \unit{\s}$} \\
        {T / \unit{\celsius}} & {oben} & {unten} & {T / \unit{\celsius}} & {oben} & {unten}\\
        \midrule
            26 & 31.37 &  31.22 &  38 & 24.69 &  24.81 \\
               & 30.66 &  31.19 &     & 24.93 &  24.97 \\
            27 & 30.43 &  31.12 &  40 & 23.81 &  23.78 \\
               & 30.37 &  30.63 &     & 23.31 &  23.09 \\
            30 & 29.03 &  29.78 &  43 & 22.28 &  22.37 \\
               & 28.82 &  28.53 &     & 22.35 &  22.15 \\
            32 & 27.63 &  27.96 &  48 & 20.57 &  21.00 \\
               & 28.10 &  28.10 &     & 20.32 &  20.78 \\
            35 & 25.82 &  25.81 &  52 & 19.65 &  19.09 \\
               & 25.68 &  26.16 &     & 19.46 &  19.16 \\   
        \bottomrule

    \end{tabular}
\end{table}

\section{Auswertung}

\subsection{Messreihe 1: kleine Kugel, konstante Temperatur}

Für die kleine Kugel oben ergibt sich mit Tabelle \ref{tab:klKu_Zitemp} 
also für die Fallzeit ein arithmetisches Mittel von

\begin{align*}
    \overline{t}_{\text{oben}} &= \frac{1}{10} \sum_{k=1}^{10} {t_{\text{oben}, k}} = \qty[]{12.324}{\s}\\%
%
\intertext{mit einer Standardabweichung von}%
%
    \Delta \overline{t}_{\text{oben}} &= 
    \sqrt[]{\frac{1}{N \left(N-1\right)} \sum_{k=1}^{10} \left(t_{\text{oben}, k} - \overline{t}_{\text{oben}}\right)^2}
    = \qty[]{0.155}{\s}\\%
    %
\intertext{Analog folgt für unten eine Fallzeit von der Mittelwert}%
%
    \overline{t}_{\text{unten}} &= \qty[]{12.248}{\s}%
\intertext{und die Standardabweichung}
    \Delta\overline{t}_{\text{unten}} &= \qty[]{0.18405}{\s}.
\end{align*}


%Durchschnittszeiten kleine Kugel:
%oben: 12.32400 mit Fehler:  0.15500
%unten: 12.24800 mit Fehler:  0.18405

Für die kleine Kugel beträgt nach \cite*[]{va207} die Apparaturkonstante $K_{kl} = \qty[]{0.07640}{\milli\Pa \cubic\cm}$.
Mit Gleichung \eqref{eq:empirisch} folgt somit für die Viskosität 



\section{Diskussion}

\section*{Anhang: Messdaten}


%\nocite{demtroeder}
%\nocite{geschke}
%\nocite{walcher}
\printbibliography

\end{document}